% Options for packages loaded elsewhere
\PassOptionsToPackage{unicode}{hyperref}
\PassOptionsToPackage{hyphens}{url}
%
\documentclass[
  12pt,
]{article}
\usepackage{amsmath,amssymb}
\usepackage{lmodern}
\usepackage{setspace}
\usepackage{ifxetex,ifluatex}
\ifnum 0\ifxetex 1\fi\ifluatex 1\fi=0 % if pdftex
  \usepackage[T1]{fontenc}
  \usepackage[utf8]{inputenc}
  \usepackage{textcomp} % provide euro and other symbols
\else % if luatex or xetex
  \usepackage{unicode-math}
  \defaultfontfeatures{Scale=MatchLowercase}
  \defaultfontfeatures[\rmfamily]{Ligatures=TeX,Scale=1}
  \setmainfont[Scale=MatchLowercase]{Merriweather}
  \setmathfont[Scale=MatchUppercase]{STIX Two Math}
\fi
% Use upquote if available, for straight quotes in verbatim environments
\IfFileExists{upquote.sty}{\usepackage{upquote}}{}
\IfFileExists{microtype.sty}{% use microtype if available
  \usepackage[]{microtype}
  \UseMicrotypeSet[protrusion]{basicmath} % disable protrusion for tt fonts
}{}
\makeatletter
\@ifundefined{KOMAClassName}{% if non-KOMA class
  \IfFileExists{parskip.sty}{%
    \usepackage{parskip}
  }{% else
    \setlength{\parindent}{0pt}
    \setlength{\parskip}{6pt plus 2pt minus 1pt}}
}{% if KOMA class
  \KOMAoptions{parskip=half}}
\makeatother
\usepackage{xcolor}
\IfFileExists{xurl.sty}{\usepackage{xurl}}{} % add URL line breaks if available
\IfFileExists{bookmark.sty}{\usepackage{bookmark}}{\usepackage{hyperref}}
\hypersetup{
  pdftitle={Indecisive Decision Theory},
  pdfauthor={Brian Weatherson},
  hidelinks,
  pdfcreator={LaTeX via pandoc}}
\urlstyle{same} % disable monospaced font for URLs
\usepackage[margin=1.38in]{geometry}
\usepackage{longtable,booktabs,array}
\usepackage{calc} % for calculating minipage widths
% Correct order of tables after \paragraph or \subparagraph
\usepackage{etoolbox}
\makeatletter
\patchcmd\longtable{\par}{\if@noskipsec\mbox{}\fi\par}{}{}
\makeatother
% Allow footnotes in longtable head/foot
\IfFileExists{footnotehyper.sty}{\usepackage{footnotehyper}}{\usepackage{footnote}}
\makesavenoteenv{longtable}
\usepackage{graphicx}
\makeatletter
\def\maxwidth{\ifdim\Gin@nat@width>\linewidth\linewidth\else\Gin@nat@width\fi}
\def\maxheight{\ifdim\Gin@nat@height>\textheight\textheight\else\Gin@nat@height\fi}
\makeatother
% Scale images if necessary, so that they will not overflow the page
% margins by default, and it is still possible to overwrite the defaults
% using explicit options in \includegraphics[width, height, ...]{}
\setkeys{Gin}{width=\maxwidth,height=\maxheight,keepaspectratio}
% Set default figure placement to htbp
\makeatletter
\def\fps@figure{htbp}
\makeatother
\setlength{\emergencystretch}{3em} % prevent overfull lines
\providecommand{\tightlist}{%
  \setlength{\itemsep}{0pt}\setlength{\parskip}{0pt}}
\setcounter{secnumdepth}{5}
\usepackage[italic]{mathastext}
\usepackage{caption}
\usepackage{booktabs}
\usepackage{longtable}
\usepackage{array}
\usepackage{multirow}
\usepackage{wrapfig}
\usepackage{float}
\usepackage{colortbl}
\usepackage{pdflscape}
\usepackage{tabu}
\usepackage{threeparttable}
\usepackage{threeparttablex}
\usepackage[normalem]{ulem}
\usepackage{makecell}
\usepackage{xcolor}
\ifluatex
  \usepackage{selnolig}  % disable illegal ligatures
\fi
\newlength{\cslhangindent}
\setlength{\cslhangindent}{1.5em}
\newlength{\csllabelwidth}
\setlength{\csllabelwidth}{3em}
\newenvironment{CSLReferences}[2] % #1 hanging-ident, #2 entry spacing
 {% don't indent paragraphs
  \setlength{\parindent}{0pt}
  % turn on hanging indent if param 1 is 1
  \ifodd #1 \everypar{\setlength{\hangindent}{\cslhangindent}}\ignorespaces\fi
  % set entry spacing
  \ifnum #2 > 0
  \setlength{\parskip}{#2\baselineskip}
  \fi
 }%
 {}
\usepackage{calc}
\newcommand{\CSLBlock}[1]{#1\hfill\break}
\newcommand{\CSLLeftMargin}[1]{\parbox[t]{\csllabelwidth}{#1}}
\newcommand{\CSLRightInline}[1]{\parbox[t]{\linewidth - \csllabelwidth}{#1}\break}
\newcommand{\CSLIndent}[1]{\hspace{\cslhangindent}#1}

\title{Indecisive Decision Theory}
\author{Brian Weatherson}
\date{2021-10-13}

\begin{document}
\maketitle

\setlength\heavyrulewidth{0ex}
\setlength\lightrulewidth{0.08ex}

\aboverulesep=0ex
\belowrulesep=0ex
\renewcommand{\arraystretch}{1.2}
\hypersetup{hidelinks}

\renewcommand\refname{References}

\captionsetup{labelformat=empty, font = small, font = bf, position = below}

\setstretch{1.05}
\hypertarget{decisiveness}{%
\section{Decisiveness}\label{decisiveness}}

Say a decision theory is \textbf{decisive} iff for any decision problem,
it says either:

\begin{enumerate}
\def\labelenumi{\arabic{enumi}.}
\tightlist
\item
  There is a uniquely best choice, and rationality requires choosing
  it.; or
\item
  There is a non-singleton set of choices each of which is tied for
  being best, and each of which can be permissibly chosen.
\end{enumerate}

A decision theory is \textbf{decisive over binary choices} iff it
satisfies this condition for all decision problems where there are just
two choices. Most decision theories in the literature are decisive, and
of those that are not, most of them are at least decisive over binary
choices. I'm going to argue that the correct decision theory, whatever
it is, is indecisive. It is not, I'll argue, even decisive over binary
choices.

My argument will focus on the following kind of problem. Player has a
binary choice. There is a Demon who is very good, arbitrarily good, at
predicting choices, and has already made a prediction about what Player
will choose. Player's payoff is a function of what they choose, and what
the Demon predicts. I'll use capital letters for Player's choices, and
matching lower case letters for the Demon's predictions. Here is one
very familiar example of the kind of case I have in mind.

\begin{table}[H]

\caption{\label{tab:unnamed-chunk-2}Demonic Prisoners' Dilemma}
\centering
\begin{tabular}[t]{>{}r|cc}

\textbf{} & \textbf{c} & \textbf{d}\\
\midrule
\textbf{C} & 4 & 0\\
\textbf{D} & 5 & 1\\

\end{tabular}
\end{table}

I've called this \textbf{Demonic Prisoners' Dilemma}. It's
\emph{Demonic} because there's a demon. And it's Prisoners' Dilemma
because Player's payouts are taken from the payout table for Prisoners'
Dilemma. I'll come back in the next section to the general strategy I'm
using for generating puzzles like this. If you're a philosopher reading
this, you're probably used to thinking of Demonic Prisoners' Dilemma as
Newcomb's Problem, but I want to use a productive naming convention, so
I'm calling it Demonic Prisoners' Dilemma.

Here is another decision problem of the same broad kind.

\begin{table}[H]

\caption{\label{tab:unnamed-chunk-3}Demonic Stag Hunt}
\centering
\begin{tabular}[t]{>{}r|cc}

\textbf{} & \textbf{g} & \textbf{h}\\
\midrule
\textbf{G} & 15 & 40\\
\textbf{H} & 0 & 50\\

\end{tabular}
\end{table}

It's called Demonic Stag Hunt because it's got a Demon, and Player's
payouts are like in a Stag Hunt.

The main decision problem I'm interested has the following structure:
Player plays Demonic Prisoners' Dilemma, then Demonic Stag Hunt, and
gets the sum of the payouts across the two games. More precisely, the
following six things happen in this order.

\begin{enumerate}
\def\labelenumi{\arabic{enumi}.}
\tightlist
\item
  Demon predicts what Player will do in Demonic Prisoners' Dilemma.
\item
  Player makes a choice in Demonic Prisoners' Dilemma.
\item
  Demon's prediction and Player's choices are revealed (to both Demon
  and Player), and Player receives their first payout.
\item
  Demon predicts what Player will do in Demonic Stag Hunt.
\item
  Player makes a choice in Demonic Stag Hunt.
\item
  Demon's prediction and Player's choices are revealed (to both Demon
  and Player), and Player receives their second payout.
\end{enumerate}

I'm going to argue for the following claims.

\begin{enumerate}
\def\labelenumi{\arabic{enumi}.}
\tightlist
\item
  If a decisive decision theory is correct, Player's choices at stages 2
  and 5 should be exactly the choices they would make if they were
  playing these games as one-shot games, independent of any other
  interaction with the Demon.
\item
  Player's choice dispositions over the two games should be consistent
  with the choice they would make if they were choosing a
  \textbf{strategy} for playing the two games. I'm going to call this
  constraint Backwards Dynamic Consistency.
\item
  If 1 is true, then all existing decisive decision theories either
  violate Backwards Dynamic Consistency, or are independently
  objectionable.
\end{enumerate}

So that's the plan, but before we start there are two pieces of
important housekeeping.

First, the definition of decisiveness referred to options being tied.
For the definition to be interesting, it can't just be that options are
tied if each is rationally permissible. Then a decisive theory would
just be one that either says one option is mandatory or many options are
permissible. To solve this problem, I'll borrow a technique from Ruth
\protect\hyperlink{ref-Chang2002}{Chang}
(\protect\hyperlink{ref-Chang2002}{2002}). Some options are
\textbf{tied} iff either is permissible, but this permissibility is
sensitive to sweetening. That is, if options \(X\) and \(Y\) are tied,
then for any positive \(\varepsilon\), the agent prefers
\(X + \varepsilon\) to \(Y\). If either choice is permissible even if
\(X\) is `sweetened,' i.e.., replaced in the list of choices by
\(X + \varepsilon\), we'll say they aren't tied. My thesis then is that
the correct decision theory says that sometimes there are multiple
permissible options, and each of them would still be permissible if one
of them was sweetened. Indeed, the argument will be that the very first
decision problem I've stated is like that.

Second, there is an important term in the definition of decisiveness
that I haven't clarified: \textbf{decision problem}. Informally, the
argument assumes that in setting out Demonic Prisoners' Dilemma (aka
Newcomb's Problem) and Demonic Stag Hunt I've described two decision
problems. More formally, I'm assuming it suffices to specify a decision
problem to describe the following four values.

\begin{itemize}
\tightlist
\item
  What choices the Player has;
\item
  What possible states of the world there are (where it is understood
  that the choices of Player make no causal impact on which state is
  actual)\footnote{My personal preference is to understand states
    historically. For any proposition relevant to the decision, a state
    determines its truth value if it is about the past, or its chance at
    the start of deliberation if it is about the future. And then causal
    independence comes in from a separate presupposition that there is
    no backwards causation. But I definitely won't assume this picture
    of states here.};
\item
  What the probability is of being in any state conditional on making
  each choice; and
\item
  What return Player gets for each choice-state pair.
\end{itemize}

Most recent papers on decision theory do not precisely specify what they
count as a decision problem, but they seem to implicitly share this
assumption, since they will often describe a vignette that settles
nothing beyond these four things as a decision problem. And that's what
I did as well! And you should understand this as being part of the
definition of decisiveness. And this implies that there are two ways to
reject decisiveness.

First, say that these four conditions underspecify a real decision
problem. In any real situation, decision theory has a decisive verdict,
but it rests on information above and beyond the setting of these four
values. Some versions of Causal Decision Theory go this route, though as
we'll see not all do. So on this model, for any person playing Demonic
Stag Hunt there is a decision theoretically correct thing for them to do
(or the options are tied), but it's potentially a different thing for
different people

Second, say that no matter how much one adds to the specification, there
will be cases where the correct decision theory is indecisive. This is
the view I ultimately want to defend, and this paper is a part of the
defence. But it's a proper part. In particular, it says nothing against
the theories that say that once a problem like Demonic Stag Hunt is
fully specified, there is a single correct option. That's an argument
for another day. Today, we have enough to be getting on with.

\hypertarget{demonic-games}{%
\section{Demonic Games}\label{demonic-games}}

Here is a general recipe for generating a philosophically interesting
decision problem. Take an interesting game, and replace one of the
players with a demon. A demon, in the relevant context, can be defined
in one of two more or less equivalent ways. One way is to say that the
demon's payouts are 1 if in some sense they make the `same' play as the
other player, and 0 otherwise. Another is to say that each of their
moves are predictions of what the other player will do, and conditional
on any choice that player makes, the conditional probability of the
demon making a correct prediction is arbitrarily high.

Let's illustrate this with a familiar game, Prisoners' Dilemma. I'm
going to use slightly different payouts from those that
\protect\hyperlink{ref-Axelrod1984}{Axelrod}
(\protect\hyperlink{ref-Axelrod1984}{1984}) uses in his classic
discussion, but not so different as to make the example unfamiliar.
First, here is the game version.

\begin{table}[H]

\caption{\label{tab:unnamed-chunk-4}Prisoners' Dilemma}
\centering
\begin{tabular}[t]{>{}r|cc}

\textbf{} & \textbf{c} & \textbf{d}\\
\midrule
\textbf{C} & 4, 4 & 0, 5\\
\textbf{D} & 5, 0 & 1,1\\

\end{tabular}
\end{table}

Now here's what happen if we demonize the payouts for column.

\begin{table}[H]

\caption{\label{tab:unnamed-chunk-5}Prisoners' Dilemma with a Demon}
\centering
\begin{tabular}[t]{>{}r|cc}

\textbf{} & \textbf{c} & \textbf{d}\\
\midrule
\textbf{C} & 4, 1 & 0, 0\\
\textbf{D} & 5, 0 & 1,1\\

\end{tabular}
\end{table}

And here is what happens if we remove the demon altogether.

\begin{table}[H]

\caption{\label{tab:unnamed-chunk-6}Demonic Prisoners' Dilemma}
\centering
\begin{tabular}[t]{>{}r|cc}

\textbf{} & \textbf{c} & \textbf{d}\\
\midrule
\textbf{C} & 4 & 0\\
\textbf{D} & 5 & 1\\

\end{tabular}
\end{table}

And that's both Newcomb's Problem, and what I was calling Demonic
Prisoners' Dilemma.

It isn't only Newcomb's Problem that can be generated in this way.
Demonic Matching Pennies\footnote{You can see examples of all these
  games, and all the game theoretic machinery I use throughout this
  paper, in any standard game theory textbook. My favorite such textbook
  is \protect\hyperlink{ref-Bonanno2018}{Bonanno}
  (\protect\hyperlink{ref-Bonanno2018}{2018}), which has the two
  advantages of being philosophically sophisticated and open access. I'm
  not going to include citations for every bit of textbook game theory I
  use; that seems about as appropriate as citing an undergrad logic
  textbook every time I use logic. But if you want more details on
  anything unfamiliar in this paper, that's where to look.} is Death in
Damascus (\protect\hyperlink{ref-GibbardHarper1978}{Gibbard and Harper
1978}). Demonic Battle of the Sexes is Asymmetric Death in Damascus
(\protect\hyperlink{ref-Richter1984}{Richter 1984}). Demonic Chicken is
Egan's Psychopath Button example
(\protect\hyperlink{ref-Egan2007-EGASCT}{Egan 2007}). A lot of the most
notable examples in modern decision theory are Demonic versions of
famous games.

When I'm introducing games in classes, I usually walk through five
instances of two player, two options each, games. Four of them have
already been mentioned in this section: Prisoners' Dilemma, Matching
Pennies, Battle of the Sexes, and Chicken. But the fifth is probably the
most interesting of the lot: Stag Hunt. Brian Sykrms has written
extensively on why Stag Hunt is philosophically important
(\protect\hyperlink{ref-Skyrms2001}{Skyrms 2001},
\protect\hyperlink{ref-Skyrms2004}{2004}). I'll really just focus on one
reason.

Here is an abstract form of a Stag Hunt game, where the options are G/g
for Gather or H/h for Hunt. Actually, this is a table for a generic
symmetric game; what makes it a Stag Hunt are the four constraints
listed below.

\begin{table}[H]

\caption{\label{tab:unnamed-chunk-7}Generic Stag Hunt}
\centering
\begin{tabular}[t]{>{}r|cc}

\textbf{} & \textbf{g} & \textbf{h}\\
\midrule
\textbf{G} & $x, x$ & $y, z$\\
\textbf{H} & $z, y$ & $w, w$\\

\end{tabular}
\end{table}

\begin{itemize}
\tightlist
\item
  \(x > z\)
\item
  \(w > y\)
\item
  \(w > x\)
\item
  \(x + y > z + w\)
\end{itemize}

The first two constraints imply that \(\langle G, g \rangle\) and
\(\langle H, h \rangle\) are both equilibria. This isn't like Prisoners'
Dilemma, that only has one equilibrium. But it is like Prisoners'
Dilemma in that there is a cooperative solution, in this case
\(\langle H, h \rangle\), but it isn't always easy to get to it. It
isn't easy because there are at least two kinds of reasons to play
\(G\).

First, one might play \(G\) because one wants to minimise regret. Each
play is a guess that the other player will do the same thing. If one
plays \(G\) and guesses wrong, one loses \(w - y\) compared to what one
could have received. If one plays \(H\) and guesses wrong, one loses
\(x - z\). And the last constraint entails that \(x - z > w - y\). So
playing \(G\) minimises possible regret.

Second, one might want to maximise expected utility, given uncertainty
about what the other player will do. Since one has no reason to think
the other player will prefer \(g\) to \(h\) or vice versa - both are
equilibria - maybe one should give each of them equal probability. And
then it will turn out that \(G\) is the option with highest expected
utility. Intuitively, \(H\) is a risky option and \(G\) is a safe
option, and when in doubt, perhaps one should go for the safe option.

To be clear, I am not endorsing either of these arguments. I think both
\(G\) and \(H\) are permissible moves in any version of Demonic Stag
Hunt. Since all instances of Demonic Stag Hunt stay being instances of
Demonic Stag Hunt under mild enough sweetening, just saying this about
Demonic Stag Hunt is enough to make a theory indecisive. So Demonic Stag
Hunt plays a few different roles in this story.

The arguments are distinct, even though in the two option game they are
basically both consequences of \(x + y > z + w\). In Stag Hunt type
games with 3 or more options, the regret based approaches and the
indifference over equilibria based approaches can lead to different
choices. And there are ways of mixing and matching these two principles
to get any number of other alternatives. In general, there are a lot of
ways to motivate \(G\), and these are not just conceptually distinct but
lead to distinct recommendations in more complex games. But the
arguments for choosing \(H\) all basically come down to \emph{Better
outcomes are better. Rocket emoji!}. So in this paper I'll spend a lot
more time on decisive theories that recommend \(G\) than on decisive
theories that recommend \(H\). This isn't because these theories are
necessarily better; just that they are more numerous.

\hypertarget{two-round-demonic-games}{%
\section{Two Round Demonic Games}\label{two-round-demonic-games}}

In a two round Demonic Game, Player plays one Demonic Game, the outcome
of that game is revealed to both Player and the Demon, then Player plays
another Demonic Game. Player's ultimate payout is the sum of their
payouts in the two individual games.

In all Demonic Games, we assume that the Demon is very reliable at
predicting Player's choices, perhaps arbitrarily reliable. In a two
round Demonic Game, we assume this, and we also assume that the Demon's
errors, as few as they are, are randomly distributed. By that I mean
that the Demon's probabilities of making a correct prediction in each of
the two games are independent of each other. What could count as good
evidence that the Demon's error probabilities are independent in this
way is a tricky question, but presumably some evidence could do it. And
we assume that Player has exactly that evidence.

This independence of the error probabilities has consequences for what
recommendations decisive theories make in two round games. In general,
one might not want to do the same thing in a two round game as one does
in the two games taken individually. But for that to be the case, one of
two things must happen.

One possibility is that there is a strategic benefit from some otherwise
sub-optimal play in the first round. If a decision problem is defined
the way we've defined it here, the only way that there could be such a
strategic benefit is if the conditional probability of states (i.e.,
predictions) given choices could change depending on what one did in the
first round. (The other three aspects of the second game clearly
couldn't change, so if there is some change it must be here.) But the
independence stipulation rules that out.

Another possibility is that something one learned from the first round
could provide a reason to do something different in the second round.
Call this an informational reason to change one's play in round two. But
just like the strategic reason for changing one's play in round one,
this could only happen if the conditional probability of states (i.e.,
predictions) given choices could change depending on what one did in the
first round. And independence once again rules that out.

Note that this argument does not carry across to indecisive theories.
One way to be indecisive is to think that there is some further
information, behind the four things we've taken to describe a game here,
that affect what choices one should make. And it is possible that there
could be causal or evidential links between what one does in round one,
and that `further information' in round two. So given indecisiveness, it
is possible there are strategic reasons for doing something in round 1
that one would not do in a one round game, or informational reasons for
doing something in round 2 that one would not do in a one round game.
And in fact I think that is the case - but it is something that only
indecisive theories can accept.

We are going to be interested primarily in the following two round game:
Demonic Prisoners' Dilemma plus Demonic Stag Hunt. And to make things
concrete, I'll use the following payouts for the two games. First,
Demonic Prisoners' Dilemma.

\begin{table}[H]

\caption{\label{tab:unnamed-chunk-8}Demonic Prisoners' Dilemma}
\centering
\begin{tabular}[t]{>{}r|cc}

\textbf{} & \textbf{c} & \textbf{d}\\
\midrule
\textbf{C} & 4 & 0\\
\textbf{D} & 5 & 1\\

\end{tabular}
\end{table}

Second, Demonic Stag Hunt.

\begin{table}[H]

\caption{\label{tab:unnamed-chunk-9}Demonic Stag Hunt}
\centering
\begin{tabular}[t]{>{}r|cc}

\textbf{} & \textbf{g} & \textbf{h}\\
\midrule
\textbf{G} & 15 & 40\\
\textbf{H} & 0 & 50\\

\end{tabular}
\end{table}

As you can see, I've made the stakes much higher in Demonic Stag Hunt
than in Demonic Prisoners' Dilemma. Whether Player can cause the Demon
to play \(h\) in round 2 is much more significant than whether they play
\(C\) or \(D\) in round 1.

There are 16 possible outcomes to this game, depending on what Player
and Demon do. They are given in this table.

\begin{table}[H]

\caption{\label{tab:unnamed-chunk-10}Possible Outcomes of DPD+DSH}
\centering
\begin{tabular}[t]{>{}r|cccc}

\textbf{} & \textbf{cg} & \textbf{ch} & \textbf{dg} & \textbf{dh}\\
\midrule
\textbf{CG} & 19 & 44 & 15 & 40\\
\textbf{CH} & 4 & 54 & 0 & 50\\
\textbf{DG} & 20 & 45 & 16 & 41\\
\textbf{DH} & 5 & 55 & 1 & 51\\

\end{tabular}
\end{table}

But note that Player doesn't choose one of these four rows. In fact, it
is wrong twice over to think that Player just makes one of these four
choices. Player makes choices at two times, not at one. And if we want
to model Player's choice at one time, we should model Player as choosing
a strategy, not just a pair of moves.

\hypertarget{strategies}{%
\section{Strategies}\label{strategies}}

So what are strategies? In general, a strategy is a plan for playing a
game that takes place over time. In the standard textbook treatment of
strategies, the specify a choice for every node that might be reached in
a game. And this turns out require a lot of information, since nodes in
games are individuated by their history.\footnote{To be more precise, a
  strategy specifies what to do at each information set, and information
  sets are individuated by the learning history of the player making the
  choice. The game we are playing is sort of a full information game, so
  this complication won't bother us.} This means there are many more
nodes than you might expect.

So think about a strategy for playing chess. There are two intuitive but
mistaken ways to think about what a strategy, in this textbook sense,
says a chess strategy does.

First, you might think that for any state of the board, it should
specify a move. But in fact strategies must do more than that; they have
to specify a move for each state of the board and each way of getting to
that state. So a strategy should have separate entries for what to do
after 1. e4, Nc6; 2. Nf3, e5 and what to do after 1. e4, e5; 2. Nf3,
Nc6.

Second, you might think that a strategy for white specifies what to do
from the initial position, then a strategy for how to respond to each
move black might make, then a further strategy for how to respond to
each subsequent move black might make, and so on until the game ends.
And that kind of strategy would be unimaginably large, and a tiny
fraction of what a real strategy does. A real strategy specifies what to
do at every possible state of the game, including what to do in states
that are excluded by earlier moves in this strategy. So if the strategy
says to play e4 at move 1, it doesn't just say what to do at move 2
after \ldots e5, it says what to do at move 2 if the game starts 1. d4,
d5.

In two stage Demonic Games, this means that a strategy has to specify
five binary choices, so there are \(2^5 = 32\) possible strategies. In
the game we're primarily interested in, a strategy has to specify

\begin{enumerate}
\def\labelenumi{\arabic{enumi}.}
\tightlist
\item
  What to do in Demonic Prisoners' Dilemma;
\item
  What to do in Demonic Stag Hunt if the first game ends
  \(\langle C, c \rangle\).
\item
  What to do in Demonic Stag Hunt if the first game ends
  \(\langle C, d \rangle\).
\item
  What to do in Demonic Stag Hunt if the first game ends
  \(\langle D, c \rangle\).
\item
  What to do in Demonic Stag Hunt if the first game ends
  \(\langle D, d \rangle\).
\end{enumerate}

I'm generally going to write strategies as a string of five letters,
which answer each of these five questions in order. So \(CGHHG\) is the
strategy which plays \(C\) in the first game, then plays \(G\) in the
second game if the Demon's prediction was correct in the first game, and
\(H\) if it is was incorrect.

Why do we want to specify strategies so finely? It turns out there are
two reasons, one relatively obvious, one less obvious.

The obvious reason is that in round 2, Player has information about what
the Demon has done in round 1, and we should at least in principle allow
them to use it. Decisive theories say that there is nothing you can
learn in round 1 that matters to round 2, but whether Decisive theories
are correct is part of what's at issue here. So let's leave that as an
open question.

The less obvious reason concerns off-path choices, i.e., what the
strategy says to do if you didn't do the thing you planned to do at move
1. It turns out that different predictions the Demon makes about one's
strategy can affect one's payout, even if those predictions only differ
in off-path choices. So imagine one's strategy is \(DGGGG\), i.e.,
Defect in Round 1, then Gather in Round 2 whatever happens in Round 1.
And compare the payout one gets if the Demon predicts one is playing
\(CGGGG\), to if the Demon predicts one is playing \(CGGHG\). In the
first case, one gets 20 (5 in round 1, 15 in round 2). In the second
case, the Demon predicts that one will play \(H\) in round 2, so trying
to make a correct prediction, they will play \(h\)h. And now the payout
will be 45.

If you think the Demon has probability 1 of making a correct prediction,
and to make the math easier that is what I'm going to assume from now
on, then it turns out that the strategies fall into 16 pairs. Within
each pair, the two strategies have the same payouts no matter what the
Demon does, and each other strategy has the same payout no matter which
of the pair the Demon predicts one will play. So there isn't any
practical difference between \(Cx_1x_2x_3G\) and \(Cx_1x_2x_3H\), or
between \(DGx_1x_2x_3\) and \(DHx_1x_2x_3\), for any \(x_1, x_2, x_3\).
For some purposes I'll treat these pairs as the `same' strategy, but
I'll make it clear when I'm doing this if anything at all turns on it.

To foreshadow a little bit, I'm going to be particularly interested in
this strategy: \(CHGGG\). This strategy Cooperates in round 1, then
Hunts in round 2 if Demon correctly predicted Cooperation, and Gathers
otherwise. This is a fairly interesting strategy for a few reasons. It
has a very good return - 54 when the highest is 55. And even though
\(C\) is strongly dominated by \(D\), this strategy is not dominated by
anything. Indeed, \(\langle CHGGG, chggg \rangle\) is Nash equilibrium
of the strategic form of the game. (And it's a subgame perfect
equilibrium and a perfect Bayesian equilibrium, and satisfies any number
of other solution concepts.) I'm not a Decisive decision theorist, so I
don't want to say it's the one and only correct play in Demonic
Prisoners' Dilemma followed by Demonic Stag Hunt. But I do think it's
what I personally would play if forced to choose. And I think it's a
rationally permissible choice. But no plausible decisive theory can find
it as the correct choice.

Now I'm not going to argue from the intuitive plausibility of \(CHGGG\)
to Indecisiveness. Instead, I'm going to use \(CHGGG\), and occasionally
\(DGGGH\), to argue that a huge range of Decisive theories are
dynamically inconsistent. And they are dynamically inconsistent in a
systematic enough way that we should be very sceptical that there is any
dynamically consistent Decisive theory. But this style of argument
raises an obvious question: what is dynamic consistency, and why should
we care about it?

\hypertarget{dynamic-consistency}{%
\section{Dynamic Consistency}\label{dynamic-consistency}}

Think about the following two ways to get to the end of a two round
Demonic Game. The natural way is that Player makes one choice, sees the
outcome of that choice, then makes another choice. I'll say that when
Player does this, they choose a *sequence* of moves. The less natural
way is that Player is asked in advance what strategy they want to play,
and then that strategy is carried out over the course of the game. I'll
call this choosing a *strategy*. The two seem closely related, close
enough that each of the following principles seem plausible.

\begin{description}
\tightlist
\item[\textbf{Backwards Dynamic Consistency}]
Any sequence that Player can rationally choose is part of some strategy
that they can rationally choose.
\item[\textbf{Forwards Dynamic Consistency}]
Any strategy that Player can rationally choose is such that choosing a
sequence by doing what that strategy requires is rational.
\end{description}

The dynamic consistency argument I'm going to use in this paper relies
on Backwards Dynamic Consistency. The argument is neutral on whether
Forwards Dynamic Consistency is a true principle. Personally, I think
it's false, and I'll say why in a bit. But the argument does not assume
that it's false. (Actually, the argument would probably be strengthened
if it were true - saying it is false is somewhat of a statement against
interests here.)

I think Backwards Dynamic Consistency is more intuitive than any
argument I could give for it, and certainly more intuitive than the
intuitions about cases that usually ground arguments in decision theory.
But there is one nice argument for Backwards Dynamic Consistency that
has some persuasive force, and helps explain what it says, and even
helps explain why it is a bit more plausible than Forwards Dynamic
Consistency. So that argument is worth working through.

One way to get Player to choose a strategy is to show them all the
possible strategies and ask them to pick one. But the forms you need to
use for that kind of questionnaire grow really quickly. A better way is
to ask Player several questions, the answers to which will collectively
determine a strategy. So we could tell Player Demonic Prisoners' Dilemma
plus Demonic Stag Hunt will be played tomorrow, but unfortunately they
won't be able to make choices in real time. So they now have to make
some choices that determine their strategy. In particular, they have to
fill out this form.

\begin{longtable}[]{@{}lcc@{}}
\caption{Form for choosing a strategy in DPD+DSH}\tabularnewline
\toprule
What will you do in & & \\
\midrule
\endfirsthead
\toprule
What will you do in & & \\
\midrule
\endhead
Round 1 & C \(\square\) & D \(\square\) \\
Round 2 if R1 ends \(Cc\) & G \(\square\) & H \(\square\) \\
Round 2 if R1 ends \(Cd\) & G \(\square\) & H \(\square\) \\
Round 2 if R1 ends \(Dc\) & G \(\square\) & H \(\square\) \\
Round 2 if R1 ends \(DD\) & G \(\square\) & H \(\square\) \\
\bottomrule
\end{longtable}

Here's what it would take for Backwards Dynamic Consistency to fail. All
the following would have to be true together.

\begin{enumerate}
\def\labelenumi{\arabic{enumi}.}
\tightlist
\item
  Player rationally makes move \(M_1\) in round 1.
\item
  Player rationally makes move \(M_2\) in round 2 after seeing how round
  1 ends.
\item
  There is no way to rationally fill in that form that includes saying
  \(M_1\) to the first question, and \(M_2\) to the conditional question
  that corresponds to how the game actually went in round 1.
\end{enumerate}

But it is implausible this is true. In some good sense, the questions
that we ask Player when we are finding what strategy they want just are
the questions we ask them when the game is being played. To reject
Backwards Dynamic Consistency is to say that there could be a situation
\emph{S} such that these two questions have different answers.

\begin{description}
\tightlist
\item[\textbf{Question S1}]
What instruction would you like carried out if \emph{S} happens?
\item[\textbf{Question S2}]
You're in \emph{S}. What instruction would you like carried out?
\end{description}

But this way of putting it, that these two questions are basically the
same question, makes it sound like both Forwards and Backwards Dynamic
Consistency should be true. Now as I noted, I'm not assuming Forwards
Dynamic Consistency is false. But I'm also not assuming that it is true.
How is this viable given how similar the tasks of choosing a strategy
and choosing a sequence look like?

Well, imagine that Player thinks that there is some situation that has
probability 0 of arising. To be concrete, let's say it is that the first
game ends \(Cd\). And they also believe that were the first game to end
\(Cd\), then \(G\) would have higher expected utility than \(H\). How
should Player answer the third question, what to do if the first game
ends \(Cd\) in the little survey above?

There are two arguments here that both seem plausible, but which point
in opposite directions. The first says that since each answer to the
survey is alike in expected utility, Player can choose whimsically. They
are alike in expected utility because what one does at probability 0
events cannot make a difference to the expected utility of one's plan.
The second argument says that choosing \(G\) here could be better - if
the first game does actually end \(Cd\) - and couldn't be worse, so
there is a weak dominance argument for choosing \(G\) that Player should
follow. Personally, I think the expected utility consideration is
stronger, but nothing in this paper turns on this particular question.

So there is a somewhat plausible argument that says there could be a
situation where Question S1 is answered whimsically, although Question
S2 has a uniquely correct answer. So maybe there is a permissible answer
to Question S1 which is not a permissible answer to Question S2. But the
theorist who denies Backwards Dynamic Consistency says that there is a
permissible answer to Question S2 which is not even a permissible answer
to Question S1. And there is no good reason to think that could be
possible. So Backwards Dynamic Consistency is correct. Whatever sequence
of moves you could rationally do, you could rationally give as part of
your answers to the survey, and then rationally complete the rest of the
survey.

But it turns out that a quite surprising variety of attempts to provide
a decisive decision theory end up violating Backwards Dynamic
Consistency. I'm going to go through a lot of examples in what follows.
I'm not going to prove there is no decisive theory that is plausible and
which doesn't violate Backwards Dynamic Consistency. But I hope you'll
agree by the end that we have strong inductive evidence that no such
theory exists.

At some level of generality, the argument I'm giving here is a variant
on the argument against ``Single Valued Solution Concepts'' that David
Pearce made in an old unpublished manuscript
(\protect\hyperlink{ref-Pearce1983}{Pearce 1983}). The manuscript is
cited in the paper where Pearce introduced the concept of a
rationalizable strategy (\protect\hyperlink{ref-Pearce1984}{Pearce
1984}). The argument I'm giving grew out of an attempt to simply
translate the argument of Pearce's manuscript into the language of
decision theory. And it is part of a larger project of applying the
lesson's of Pearce's famous paper to decision theory. But that larger
project will have to be carried out elsewhere; this paper is long enough
as it is. Instead, let's turn to seeing the conflict between various
decision theories and Backwards Dynamic Consistency.

\hypertarget{evidential-decision-theory-is-dynamically-inconsistent}{%
\section{Evidential Decision Theory is Dynamically
Inconsistent}\label{evidential-decision-theory-is-dynamically-inconsistent}}

As an illustration of these dynamic consistency principles, it's worth
walking through an argument that traditional Evidential Decision Theory
(Evidential Decision Theory) violates Backwards Dynamic Consistency.
This is not a new point. It is made in
\protect\hyperlink{ref-GibbardHarper1978}{Gibbard and Harper}
(\protect\hyperlink{ref-GibbardHarper1978}{1978}) using a variant of
Newcomb's Problem where the player has a chance to change their choice
(for a small fee) after the demon's predictions are announced. So I'm
not offering a new argument against Evidential Decision Theory. Instead,
I'm presenting this case because I find it helpful to see what the
Dynamic Consistency Principles say by first starting with a relatively
simple case.

Player A has the following interaction with a demon. Tomorrow, A will be
asked to choose the Left or Right box. The demon wants to predict A's
choice. The demon will put \$1000 into whatever box they predict A will
choose. And, A will be shown the contents of the boxes before they make
their choice. This seems like an easy game - every decision theorist I
know says that A should take the money. (And the demon will get their
wish of correctly predicting A's choice.)

Player B has an interaction like Player A's, but with the following
difference. The demon has the option of \emph{passing}, and not
predicting what choice B will make. The demon would prefer correct
prediction to passing, and passing to incorrect prediction. If the demon
passes, B will get \$2000, not just \$1000. But still, B will be shown
how much is in each box. So almost all theorists will say that B should
take the money, and since the demon knows this, the demon shouldn't
pass. (I'll come back to theories that say B should pass up the money in
the penultimate section of the paper.)

Player C is like Player B, except they will be at work all day tomorrow,
and unable to make a choice. So they are asked to record in advance
their strategy,. Now things get interesting. C has four options to
choose between - since they have to announce left or right for each
possible revelation of where the money is. And the demon has three
choices, left, right or pass. So here's the game table, with Player C as
Row and Demon as Column. (I'll write X-Y to mean for C to mean Do X if
the money is in the left box, and Do Y if the money is in the right box.
And I'll assume each \$1000 is worth 1 util to C.)

\begin{table}[H]

\caption{\label{tab:unnamed-chunk-11}Threat Game}
\centering
\begin{tabular}[t]{>{}r|ccc}

\textbf{} & \textbf{Left} & \textbf{Right} & \textbf{Pass}\\
\midrule
\textbf{Left-Left} & 1, 1 & 0, 0 & 2, 0.5\\
\textbf{Left-Right} & 1, 1 & 1, 1 & 2, 0.5\\
\textbf{Right-Left} & 0, 0 & 0, 0 & 2, 0.5\\
\textbf{Right-Right} & 0, 0 & 1, 1 & 2, 0.5\\

\end{tabular}
\end{table}

From what we've said about the demon, we can deduce the following
conditional probabilities for predictions given strategies. (The second
line is because the value for \(x\) is undetermined, but it's not going
to matter given what comes next.)

\begin{table}[H]

\caption{\label{tab:unnamed-chunk-12}Probabilities in Threat Game}
\centering
\begin{tabular}[t]{>{}r|ccc}

\textbf{} & \textbf{Left} & \textbf{Right} & \textbf{Pass}\\
\midrule
\textbf{Left-Left} & 1 & 0 & 0\\
\textbf{Left-Right} & $x$ & $1-x$ & 0\\
\textbf{Right-Left} & 0 & 0 & 1\\
\textbf{Right-Right} & 0 & 1 & 0\\

\end{tabular}
\end{table}

So the expected value of Right-Left is 2 and the expected value of the
other three options is 1. So Evidential Decision Theory says that Player
C should choose Right-Left.

And that's to say that the only rational strategy for Player B is
Right-Left. But that's clearly not a rational sequence of moves, by the
lights of Evidential Decision Theory, for Player B. The sequence that is
rational at each time it is made is Left-Right.

So Evidential Decision Theory violates both Forwards and Backwards
Dynamic Consistency. It violates Backwards Dynamic Consistency because
it says the only sequence that is rational is Left-Right, but it does
not say this is one of the rational strategies. And it violates Forwards
Dynamic Consistency because it says that Right-Left is among the
rational strategies, but it is not among the rational sequences.

And these violations seem like bad news for Evidential Decision Theory.
If you ask the evidential decision theorist to compile a strategy by
answering conditional questions, they will say they will turn the money
down whatever the demon does. But if you ask them what to do once they
see where the money is, they will take it. This looks like a bad
combination of attitudes to have.

Now as I said at the start of this section, this point was made about
Evidential Decision Theory over 40 years ago, and it clearly hasn't
convinced everyone. So I don't expect this re-presentation will change
many more minds. But I hope it's a helpful guide to what violations of
Backwards (and Forwards) Dynamic Consistency look like, and why they
look, at least to many of us, to be problems.

\hypertarget{regret-based-strategies}{%
\section{Regret Based Strategies}\label{regret-based-strategies}}

In this section I'll argue that two recent attempts to land between
Evidential and Causal Decision Theory are dynamically inconsistent.
These are the theories developed by Ralph
\protect\hyperlink{ref-Wedgwood2013}{Wedgwood}
(\protect\hyperlink{ref-Wedgwood2013}{2013}) and Dmitri
\protect\hyperlink{ref-Gallow2020}{Gallow}
(\protect\hyperlink{ref-Gallow2020}{2020}). Both theories handle two
option Demonic cases the same way, so let's start with what they have in
common.

For any options \(X, Y\), let \(V_Y(X)\) be the (evidentially) expected
value of \(X\), on the assumption that the demon will act as if one has
chosen \(Y\). So \(V_X(Y) - V_X(X)\) will be how much one subsequently
regrets not choosing \(Y\), when one actually chose \(X\). In the two
option case, but Wedgwood and Gallow say that one should choose \(X\)
over \(Y\) so as to minimise expected regret. So one should choose \(X\)
over \(Y\) if \(V_Y(X) - V_Y(Y) > V_X(Y) - V_X(X)\), and be indifferent
between the two options if they are equal. But the theories differ about
what to do when there are more than two options.

Wedgwood says that one should evaluate each option \(X\) by comparing
\(V_X(X)\) to the set of values \(\{V_X(Y): Y \in O\}\), where \(O\) is
the set of options. There are a few different comparisons one could make
here - how does \(V_X(X)\) compare to the maximum value of that set, or
the minimum value, or the mean value, or the median value? It turns out
it won't matter for our purposes what choice you make here. The general
picture is that one wants to maximise
\(V_X(X) - f(\{V_X(Y): Y \in O\})\), for one particular function \(f\).
Given a choice of \(f\), Wedgwood calls \(f(\{V_X(Y): Y \in O\})\) the
\textbf{benchmark} for \(X\), and the aim is to maximise the surplus
value of choosing \(X\) over its own benchmark. For simplicity, we'll
say the benchmark is the mean value, but actually it doesn't matter
which choice we make for what follows.

Given this setup, one should choose \(D\) in Demonic Prisoners' Dilemma,
and \(G\) in Stag Hunt. But if you built the 32-by-32 strategy table for
Demonic Prisoners' Dilemma plus Demonic Stag Hunt, the strategies with
the highest surplus value over their own benchmark are \(DGGGH\) and
\(DGGHH\). Just behind them are \(CHGGG\) and \(CHHGG\). And all the
other strategies are, by the lights of Wedgwood's theories, much worse.
So this theory violates Backwards Dynamic Consistency. If one is filling
in the form, saying what one plans to do in various situations, one must
say that one plans to play \(H\) in the situation one expects to find
oneself in. But one must not play \(H\) when that situation arises.

Gallow's theory takes off from the two option case in a separate
direction. Say that the measure
\((V_X(X) - V_X(Y)) - (V_Y(Y) - V_Y(X))\) is a measure of the prima
facie preferability of \(X\) over \(Y\). As we saw above, this will be
positive if \(X\) is strictly preferred to \(Y\) in the two option case;
we won't be interested in the case where it is negative.

Now a simple theory would be to say that in general, whether \(X\) is
preferred to \(Y\) is determined by whether this value is positive or
negative. But as Gallow notes, this will lead to intransitivities. It is
easy to come up with three option cases where, by this measure, \(X\) is
preferred to \(Y\), \(Y\) is preferred to \(Z\), and \(Z\) is preferred
to \(X\). So Gallow offers a more sophisticated theory of all things
considered preferability.

Here is Gallow's recipe for constructing a rank ordering of options. I'm
going to simplify a bit and ignore how he handles ties, which require
careful attention, but what I say about how to handle non-ties will be
enough to give you the spirit of the theory. If we are trying to
construct a rank ordering of all options, we just need to construct a
set of binary rankings \(X > Y\) that satisfies transitivity. Gallow
gives us a procedure for finding all those pairs. Start by taking all
the pairs of options \(\langle X, Y \rangle\) and rank them from highest
to lowest by \((V_X(X) - V_X(Y)) - (V_Y(Y) - V_Y(X))\). Then work in
steps down the list. At every step, find the pair
\(\langle X, Y \rangle\) with the highest such value you haven't
processed. Add \(X > Y\) to the master list of rankings. Then take the
transitive closure of the list; if there are some options
\(X_1, \dots X_n\) such that the list contains \(X_1 > X_2\) and
\(\dots\) and \(X_{n-1} > X_n\), add \(X_1 > X_n\) to the list. And if
\(X_n > X_1\) was on the list of prime facie preferences, delete it from
that list. Now return to the (possibly) modified list of prima facie
preferences, find the highest ranking preference that is neither added
to the master list, nor deleted by this deletion procedure, add it to
the master list, and continue. Eventually, one will have either
\(X > Y\) or \(Y > X\) on the master list for all \(X, Y\), and by
construction the master list will be transitive.

I'm not going to work through here how this works for the 32-by-32
strategy game. But it turns out that the best option is \(DGGGH\). And
just behind is \(CHGGG\). But when you apply the strategy to the two
games individually, you get the clear verdicts that one should choose
\(D\) in Demonic Prisoners' Dilemma, and \(G\) in Stag Hunt. So the
theory is dynamically inconsistent in just the same way that Wedgwood's
theory is. When filling in the form, one must say that one plans to play
\(H\) in the situation one expects to find oneself in at the second
game. But one must not play \(H\) when that situation arises.

\hypertarget{deliberational-strategies}{%
\section{Deliberational Strategies}\label{deliberational-strategies}}

So far I've argued, following
\protect\hyperlink{ref-GibbardHarper1978}{Gibbard and Harper}
(\protect\hyperlink{ref-GibbardHarper1978}{1978}), that evidential
decision theory is dynamically inconsistent, and that two recent
attempts to steer a middle path between evidential and causal decision
theory are dynamically inconsistent. It's time to turn to causal
decision theory.

At first glance, causal decision theories look like they can't pose a
problem to my thesis that the correct theory is indecisive. That's
because causal decision theories often take something beyond the acts,
the states, the payouts, and the conditional probabilities of the states
given the acts, to be crucial. This is even assuming that, as causal
decision theorists insist, that the problem requires a specification
that the states are causally independent of the acts. A lot of causal
decision theories will insist that we also need the unconditional
probabilities of the states in order to make a decision. And those
theories will be, in my technical sense, indecisive. But some broadly
causal theories do not require this. I'll look at two classes of such
theories, one in this section and one in the next section.

Theories from the class I'm going to discuss in this section are not, to
the best of my knowledge, actually defended anywhere in the literature.
But the point of this paper is not to argue that other philosophers are
mistaken, but to argue that there is no plausible decisive decision
theory. And to my mind, the biggest challenge to that conclusion comes
from theories in this class. So I'm going to spent more time on them
than on other theories. In this respect I'm following
\protect\hyperlink{ref-Gallow2020}{Gallow}
(\protect\hyperlink{ref-Gallow2020}{2020}), who also spends a fair bit
of time arguing against this version of causal decision theory. (And
indeed I'll appropriate one of Gallow's arguments in what follows.)

The theories I have in mind start with the important work of Brian
\protect\hyperlink{ref-Skyrms1990}{Skyrms}
(\protect\hyperlink{ref-Skyrms1990}{1990}) on the dynamics of
deliberation. The resulting view is not going to be Skyrms's own view.
Skyrms favors an indecisive view, and I'm not going to object to his
positive view.\footnote{It is an interesting question whether Skyrms's
  view is sufficiently indecisive, or whether it is too restrictive.
  That's a question for another, longer, paper.} But I will be
interested in how a decisive theorist might appropriate Skyrms's
machinery.

The part of that machinery we're going to focus on concerns updating
one's credences during deliberation. There is a huge literature on
updating credences during investigation. But that's not what we're
interested in. Think about the process a good detective goes through. (A
somewhat realistic one, not the idealizations of formal confirmation
theory.) They investigate, and during the investigation gather a bunch
of evidence. Then they reflect, and ask how the evidence fits together,
and what conclusions it supports. Most work on credal updating has
focussed on modeling the first step, the investigation. Skyrms is
interested in modeling the second step, the deliberation. Now we're not
doing detective stories here; we're trying to win money not solve a
crime. But the same basic idea holds. Given what one knows about the
case, one should reflect in a way that brings one's cognitive state into
a kind of equilibrium.

Here is how Skyrms thinks we reach that equilibrium. I'll illustrate
using Demonic Stag Hunt. (The one round game that is; I'll come back to
the two round game very soon.) Imagine Player starts out thinking that
it's 50/50 whether they'll Gather or Hunt. Since Player believes that
Demon will do whatever they do, Player also thinks it's 50/50 whether
Demon will play gather or hunt. Given that, the expected value to Player
of Gather is 27.5, and the expected value of Hunt is 25, and Player's
expected return is 26.25.

Now here's the crucial step. Since Gather has a return above
expectations, and Hunt has a return below expectations, Player should
adjust their credences about what they'll end up playing in the
direction of the more successful strategy. They should ``seek the
good.'' There are a lot of ways to do this; here's one that Skyrms
particularly likes.

For any strategy whose expected value is higher than Player's overall
expected value, say that its \textbf{covetability} is the difference
between those two expectations. For any other strategy, the covetability
is 0. So in this situation, Gather has a covetability of 1.25, and Hunt
has a covetability of 0. Now assume Player updates their credences using
the following rule. In this rule, \(O\) is an arbitrary strategy,
\(p_i\) is Player's probability at time \(t_i\) about what they'll play,
\(c_i\) is the function from strategies to covetabilities at \(t_i\),
the sum in the denominator ranges over all strategies, and \(r\) is a
measure of caution, that I'll say more about in a bit.

\[
p_{i+1}(O) = \frac{rp_i(O) + c_i(O)}{r + \sum c(X)}
\]

Let's set \(r\) to 2.5 to make the arithmetic easier. Then if \(p_0\) of
both Gather and Hunt was 0.5, we'll have \(p_1(G) = frac{2}{3}\), and
\(p_1(H) = \frac{1}{3}\). So at \(t_1\) Player will have credence
\(\frac{2}{3}\) than demon will play gather. So the expected return of
Gather will be \(\frac{70}{3}\), the expected return of Hunt will be
\(\frac{50}{3}\), and the overall expected return will be
\(\frac{190}{9}\). So at \(t_2\) the probability of Gather will rise a
little more, the probability of Hunt will go down a little more, and
this will continue until the Player is certain they will Gather.

This is what happens if Player is originally 50/50 about what they will
do. But it's very dependent on that initial assumption. If Player had
started off fairly confident that they would Hunt, and hence that demon
would play hunt, then the resulting equilibrium would have been that
they were certain they would Hunt. More precisely, that would happen if
they started off with credence greater than 0.6 that they would Hunt. If
they started off with credence exactly 0.6 that they would Hunt, they
would be in an equilibrium already, and their credences would never
change.

This is all well and good, but it doesn't feel like a decision theory
yet. But there are a few simple ways to turn it into one. For example we
could say (and Skyrms more or less does say) that any of these
equilibrium states are rational, and so could ground rational choices.
So in Demonic Stag Hunt, it's rational to Gather, rational to Hunt, and
rational to play the mixed strategy Gather with probability 0.4, Hunt
with probability 0.6. This last one is weird in a few ways; let's look
at one of these.

Although there are three equilibria, they are not symmetric. They have
different outcomes - one has an expectation of 50, one of 15, and the
mixed strategy has an expectation of 40. We'll come back to theories
that expectations to choose between the equilibria in the next section.
For now, focus on what starting points get to those equilibria. Each
equilibria has a `basin of attraction': a set of initial probabilities
that (given a choice of update function) lead to that equilibria. And
these basins are of different sizes. Using a natural measure on the
space of probability functions, the basin that leads to Gathering has
measure 0.6, the basin that leads to Hunting has measure 0.4, and the
basin that leads to the mixed strategy has measure 0. It's rather
tempting to exclude equilibria whose basin has measure 0 from the set of
permissible strategies. But ignore that for now, and focus on the pure
strategy equilibria. The basin that leads to Gather has two salient
features that the basin that leads to Hunt lacks.

\begin{enumerate}
\def\labelenumi{\arabic{enumi}.}
\tightlist
\item
  It is larger.
\item
  It includes the mid-point.
\end{enumerate}

In two option choices these features usually go together, though not
always, and they often come apart when there are more than two options.
Either of these could be used to generate a decisive decision theory.
That is, the following two Skyrms inspired theories are both broadly
causal decision theories, and are both decisive in my sense.

\begin{enumerate}
\def\labelenumi{\arabic{enumi}.}
\tightlist
\item
  In any decision problem, the rational choice is the equilibrium with
  the largest basin of attraction.
\item
  In any decision problem, the rational choice is the equilibrium whose
  basin of attraction includes the mid-point.
\end{enumerate}

Like Gallow, I'm going to largely focus on the second of these, though
I'll make some notes about the first from time to time. I'll call the
second theory, the one I'm focussing on, the \textbf{midpoint theory}.

Let's bring this back to our two round game, Demonic Prisoners' Dilemma
plus Demonic Stag Hunt. Midpoint theory says that one should Defect in
Demonic Prisoners' Dilemma, and Gather in Demonic Stag Hunt. What does
it say about the two stage game? Well, to answer that we need to
precisify midpoint theory further. In particular, we need to answer two
questions.

\begin{itemize}
\tightlist
\item
  What update function should we use?
\end{itemize}

It turns out this doesn't matter too much, as long as the function is
sufficiently cautious. That is, the rule described above is fine as long
as \(r\) is high enough. (A low value of \(r\) means that, at least for
some games, one ends up oscillating between states rather than reaching
equilibria; you need \(r\) low enough to make oscillations impossible.)
The biggest question is this one.

\begin{itemize}
\tightlist
\item
  What strategies should we take seriously enough to give any positive
  probability to?
\end{itemize}

You might be interested in this question for boringly pragmatic reasons.
The space of probability functions is an \(n-1\) dimension simplex,
where \(n\) is the number of strategies taken seriously. So if there are
32 strategies, and you're trying to find the one with the largest basin
of attraction, one needs to muck about with 31-dimensional geometry. And
that is non-trivial.

But you might be interested in it for more philosophical reasons. If one
can see that a strategy is sub-optimal, perhaps because it is dominated,
or because it is not a possible equilibrium, it seems odd to even start
the reflective process by having some probability one will end up there.
So let's say that midpoint theory will involve starting with probability
\(\frac{1}{n}\) in each of the \(n\) sensible strategies. And the theory
can be precisified in a few ways depending on what one counts as
sensible.

For any natural choice of update function, and of sensible strategies,
the resulting theory is dynamically inconsistent. There are a lot of
choices for each of these questions, and I'm not going to go through how
they play out one by one. But there is a pattern. In the strategic form
of the two round game, the resulting equilibrium is either \(CHGGG\) or
\(DGGGH\). That is, in the strategic form of the game, rationality
according to the midpoint theory requires one to Hunt in round two (or
at least to expect to Hunt with probability 1). But if one plays the
game in real time according to the midpoint theory, one must Gather in
round 2. This is dynamically inconsistent, and tells against midpoint
theory.\footnote{I would have liked to have a theory about just which
  combinations of update rule and sensible strategies led to \(CHGGG\)
  and which led to \(DGGGH\), but I couldn't see any pattern to them. I
  suspect there is something useful to say here, but I don't know what
  it is.}

But there is a loophole here; there is one way to make the midpoint
theory dynamically consistent. And it's a simple one. If we start by
saying every theory is `sensible,' that at \(t_0\) each theory has
probability \(\frac{1}{32}\) in being played, the midpoint theory says
that the rational choice in the strategic form of the game is \(DGGGG\).
(Assuming an appropriate choice of update function. But in fact given
this starting point, lots of update functions will do.) That's
dynamically consistent. And it's the first dynamically consistent
decisive theory we've seen. So is this the theory a philosopher who
wants a decisive theory should approve of?

There are three reasons for thinking this theory - midpoint theory that
starts with every strategy having equal probability no matter how absurd
the strategy is - can't be right. As you can possibly tell from the fact
that I'm going to offer three reasons, I'm not sure any one of them on
their own is decisive. And even after these reasons, I still think this
theory is the best decisive theory on offer. But I think it's flawed for
these three reasons.

The first reason is one that \protect\hyperlink{ref-Gallow2020}{Gallow}
(\protect\hyperlink{ref-Gallow2020}{2020}) offers as a criticism. The
theory handles clone options very badly. Imagine Player is a bit
ambidextrous; they are right handed, but they can draw straight lines
with their left hand. They can't, however, draw curves. And they have to
write `G' or `H' to play (one round) Demonic Stag Hunt. So now they have
three options in Demonic Stag Hunt.

\begin{itemize}
\tightlist
\item
  Write `G' with their right hand.
\item
  Write `H' with their left hand.
\item
  Write `H' with their right hand.
\end{itemize}

Midpoint theory says that in this game, Player should play one of the
latter two options. Roughly, that's because they will start with a
probability of \(\frac{2}{3}\) that they will play `H' one way or the
other, and as long as they start with a probability greater than 0.6
that they'll play `H,' that's what they should end up with. But it's
absurd that `cloning' one of the choices should change what they play.

The second reason relates to my term `sensible.' Think about what the
midpoint theory, with no restrictions on initial strategies, is saying.
It says one should start with this ludicrous probability, where one
thinks it is somewhat likely that one will play a strictly dominated
strategy, then advance from there by steps that are each sensible
movements from where one is. But there is nothing philosophically
significant about the fact that a certain state is the endpoint of
applying a rational process from an irrational starting point. It's like
saying I should believe the sky is green because this follows by the
rational rule of and-elimination from the starting point `grass is blue
and the sky is green.' So even if this theory is dynamically consistent,
it is not philosophically coherent.

Now I should note at this point that Skyrms himself is not particularly
sympathetic to this line of reasoning. He thinks it is fine (at least
for some purposes) to include some absurdities in the initial state.
After all, the point of the dynamic deliberative process is to weed out
absurdities. And he does have a point; there is some amount of
redundancy in the view that one should first delete the obviously bad
strategies, then go through some process to delete the less obviously
bad strategies. Why not just trust the process?

So let's turn to the third reason, which is that midpoint theory is also
arguably dynamically inconsistent in a different case. I say `arguably'
because the case involves mixed strategies, and just how to understand
dynamic consistency when mixed strategies are around is a tricky
question.

The example involves playing the following two Demonic Anti-Coordination
games in order, with the results of the first game being revealed to
Player and demon before the second game is played.

\begin{table}[H]

\caption{\label{tab:unnamed-chunk-13}First Anti-Coordination Game}
\centering
\begin{tabular}[t]{>{}r|cc}

\textbf{} & \textbf{u} & \textbf{d}\\
\midrule
\textbf{U} & 0 & 9\\
\textbf{D} & 1 & 0\\

\end{tabular}
\end{table}

\begin{table}[H]

\caption{\label{tab:unnamed-chunk-14}Second Anti-Coordination Game}
\centering
\begin{tabular}[t]{>{}r|cc}

\textbf{} & \textbf{u} & \textbf{d}\\
\midrule
\textbf{U} & 0 & 4\\
\textbf{D} & 1 & 0\\

\end{tabular}
\end{table}

Read \(U\) as Up and \(D\) as Down. Each game on its own has a unique
Skyrms equilibrium. We don't need to worry about update rules or basins
or anything - the approach of seeking the good will end in the first
game with having probability 0.9 of playing Up, and in the second game
with having probability 0.8 of playing Up. So the strategic form of the
game should have an equilibrium that consists of playing Up with
probability 0.9 then, whatever happens in the first game, playing Up
with probability 0.8 in the second game.

And that is one of the equilibria in the 32 strategy strategic form of
the game. But it isn't the only equilibrium. There is another
equilibrium that looks like this. (I'll use the same convention as above
for describing strategies, so the 2nd through 5th letters describe what
to do after the first game ends \(Uu\), \(Ud\), \(Du\) and \(Dd\)
respectively.)

\begin{longtable}[]{@{}cc@{}}
\caption{A mixed strategy for the two round anti-coordination
game}\tabularnewline
\toprule
Strategies & Probability \\
\midrule
\endfirsthead
\toprule
Strategies & Probability \\
\midrule
\endhead
\(UUUUU\), \(UUUUD\), \(UUUDU\), \(UUUDD\) & \(\frac{51}{310}\) \\
\(UUDUU\), \(UUDUD\), \(UUDDU\), \(UUDDD\) & \(0\) \\
\(UDUUU\), \(UDUUD\), \(UDDDU\), \(UDUDD\) & \(\frac{51}{1240}\) \\
\(UDDUU\), \(UDDUD\), \(UDDDU\), \(UDDDD\) & \(0\) \\
\(DUUUU\), \(DUDUU\), \(DDUUU\), \(DDDUU\) & \(\frac{11}{310}\) \\
\(DUUUD\), \(DUDUD\), \(DDUUD\), \(DDDUD\) & \(0\) \\
\(DUUDU\), \(DUDDU\), \(DDUDU\), \(DDDDU\) & \(\frac{11}{1240}\) \\
\(DUUDD\), \(DUDDD\), \(DDUDD\), \(DDDDD\) & \(0\) \\
\bottomrule
\end{longtable}

In words, here's what this strategy does.

\begin{itemize}
\tightlist
\item
  In round 1, play Up with probability \(\frac{51}{62}\).
\item
  If one gets a positive return in round 1, in round 2 play Up with
  probability 0.8.
\item
  If one gets a zero return, in round 2 always play Up.
\end{itemize}

It isn't that hard to verify that this is an equilibrium. If one has
this probability function over the demon's choices, then each strategy
with positive probability has an expected return of \(\frac{809}{310}\),
and the strategies with probability 0 have lower expected return. (Note
this is a marginally higher expected return than playing the equilibrium
mixed strategy each round, which returns 2.6, or \(\frac{806}{310}\).)

What's surprising is that this equilibrium is the one that midpoint
theory recommends in the strategic form of the game. Or, at least, if
\(r\) is low enough for an equilibrium to be found, it recommends this
equilibrium. (If \(r\) is too high, the process never reaches
equilibrium, and instead oscillates between two somewhat strange
strategies.) So midpoint theory says this is the one and only right
strategy in the game.\footnote{The basin of attraction for this
  equilibrium is very large, but I'm not quite sure how large. I think
  it has measure 1, but I haven't found a proof of this yet. If it does
  have measure 1, this has consequences for the view that says
  strategies with measure 0 are not rationally playable.}

But this seems to make midpoint theory dynamically inconsistent.
According to the strategic form of the game, midpoint theory says the
one and only rational play in the first round is to play Up with
probability \(\frac{51}{62}\). But when the game is being played in real
time, midpoint theory says the one and only rational play in the first
round is to play Up with probability 0.9. In the strategic form of the
game, midpoint theory says that it is never permissible to play Down in
round 2 after getting a zero return in round 1. But it also says that in
the real time version of the game, one should play Down in round 2 with
probability 0.2, whatever happens in round 1. Now it isn't perfectly
clear exactly how one should understand dynamic consistency when mixed
strategies are being considered. But one way or the other, these results
feel like they show midpoint theory is dynamically inconsistent.

So I don't think there is a way of converting the Skyrms dynamics to a
plausible and consistent decisive theory by looking at basins of
attractions. But obviously I haven't surveyed all of the ways that one
could build a dynamic theory this way. Perhaps some future decisive
theorist will find a way through the gaps in the argument above. And
there are meant to be gaps. The argument here is essentially inductive -
all these approaches fail, and that's probably because there isn't a
successful approach to be found. But rather than try and shore up this
inductive inference yet further, I'll turn to a different way that it
has been suggested we use the Skyrms approach to develop a decisive
theory - one that recommends Hunting in the one round Demonic Stag Hunt.

\hypertarget{choose-the-best-equilibrium}{%
\section{Choose the Best
Equilibrium}\label{choose-the-best-equilibrium}}

So far I've spent most of the time considering solutions to Demonic Stag
Hunt that recommend Gathering. It turns out we have to spend much less
time on solutions that recommend Hunting. As I noted earlier, that's
because the different approaches to getting to Hunting all have quite a
bit in common.

One way to motivate Hunting starts with Jeffrey's version of a broadly
evidential decision theory. As he noted in the second edition of
\emph{The Logic of Decision}
(\protect\hyperlink{ref-Jeffrey1983}{Jeffrey 1983}), to handle real life
Newcomb problems, it is natural to add side-constraints to evidential
decision theory. Those constraints will rule out one-boxing, and the
equivalent of one-boxing in various real life situations. Call such a
view evidential decision theory plus side constraints. It won't matter
for our purposes what the side constraints are, just that they don't
rule out anything in Stag Hunt. So the view recommends Hunting.

Or alternatively, one could start with the Skyrms approach to dynamic
deliberation that was the focus of the last section. But instead of
choosing between multiple equilibria by looking at properties of basins,
as midpoint theory did, one could just choose the equilibria with the
highest expected return. That's the broadly causal theory that Frank
\protect\hyperlink{ref-Arntzenius2008}{Arntzenius}
(\protect\hyperlink{ref-Arntzenius2008}{2008}) recommends. And it too
recommends Hunting.

In a lot of cases, these two approaches will recommend very similar
choices. That might be surprising if you think of one of them as broadly
evidential and one as broadly causal. But a better way to think about
them, I think, is to see them as reaching a natural compromise point
between the two big approaches (evidential and causal) from different
directions. And it's a good-making feature of a compromise that it can
be reached by multiple paths in this way. Unfortunately, the resulting
theory is dynamically inconsistent.

Go back to the two stage game, and consider the strategy \(CHGGG\). If
Player plays this strategy, and Demon predicts this, Player does very
well. They get 54, which is almost the highest return in the game. And
given Demon's prediction, they can't do better. Also, they can't do
better in any circumstance where Demon predicts correctly. So this is an
equilibrium, in Skyrms's sense, and ratifiable, in Jeffrey's sense. More
generally, it satisfies any plausible side constraints one could add on
to evidential decision theory. And it is the choice that evidential
decision theory recommends. So it is the strategy any such theory should
recommend. Strictly speaking, the theory says one should be indifferent
between \(CHGGG\) and \(CHGGH\); I won't fuss over this because what is
going to matter is what one does at the very first step.

These decision theories recommend the strategy \(CHGGG\). But they do
not recommend choosing \(C\) in round 1 of a real life Demonic
Prisoners' Dilemma plus Demonic Stag Hunt. They recommend choosing \(D\)
in a stand alone version of Demonic Prisoners' Dilemma. Indeed, they are
designed to give that recommendation. So they could only recommend
playing \(C\) in round 1 of the two round game if there was a strategic
benefit to playing \(C\). But there is no strategic benefit. Playing
\(C\) does not cause a change in any of the parameters that are relevant
to the game in round 2. Indeed, it doesn't even provide evidence that
changes one's best estimates of those parameters. (Here it is crucial
that we said the Demon's errors are probabilistically independent of
each other.) So the theories say, inconsistently, to mark \(C\) when
filling in the form, but to answer \(D\) when asked how one wants to
play round 1. So they too are dynamically inconsistent.

It's possibly a failure of imagination on my part, but I really can't
see any other motivation for Hunting than the fact that the Both Hunt
equilibrium has a higher payout than the Both Gather. I'm not saying
that's a bad motivation. If I personally was playing Demonic Stag Hunt,
I'd probably Hunt, and for just this reason. What I am saying is that we
can't, on pain of contradiction, say that it is a decisive reason to
Hunt. The approach to decision theory that says it is a decisive reason
offers inconsistent advice in the two stage game, and so is wrong. What
we should say, and what indecisive theories do say, is that it's
rationally permissible to Hunt for this reason, but it is also
rationally permissible to Gather because Gathering reduces regret, or
because Gathering has a larger Skyrmsian basin of attraction, or for any
number of other reasons.

\hypertarget{existentialist-decision-theory}{%
\section{Existentialist Decision
Theory}\label{existentialist-decision-theory}}

So far I've argued that decisive decision theories are dynamically
incoherent. In this section I want to note one assumption that I've been
making so far, and look at what happens if we drop that assumption. I
call the assumption \emph{existentialism}, but it takes a bit of
explaining to see why that's a good name.

An existentialist decision theory says that each choice should be judged
on its own, rather than as the contribution it makes to a strategy. Now
a decision should be sensitive to the evidence; sometimes \(X\) is a
better decision than \(Y\) because of what's happened earlier in the
game. And a decision should be sensitive to long-run consequences;
sometimes \(X\) is a better decision than \(Y\) because even though
\(Y\) would have better short run consequences, \(X\) will promote
valuable cooperation in the short run. But still, each decision should
be made, and be judged, on its own, and not as part of a larger
strategy.

The contrast to existentialism is what I'll call intellectualism. The
intellectualist says that what makes a decision rational is that it is a
manifestation of a rational long run strategy. The
existentialism/intellectualism debate is orthogonal to the debate
between evidential and causal decision theory, but it's very striking to
see how it plays out if you assume evidential decision theory. The
combination of intellectualism plus evidential decision theory says that
in the cases that show evidential decision theory is dynamically
inconsistent, one should take the choice that involves taking less
money. And that's true even though at the time of the choice, one knows
precisely how much one will get from each choice. That seems bad, and I
think it's a decisive reason to reject such theories.\footnote{I think
  that intellectualism plus evidential decision theory is what
  \protect\hyperlink{ref-LevinsteinSoares2020}{Levinstein and Soares}
  (\protect\hyperlink{ref-LevinsteinSoares2020}{2020}) call
  `functionalist decision theory.' I think, that is, that what they call
  choosing an algorithm is what I call choosing a strategy. But I'm not
  sure about this, since intellectualism plus evidential decision theory
  is a very strange theory. Note that it is strictly speaking indecisive
  in my sense. You can't tell what to do in such a theory given the
  setup of any decision problem. For any decision problem you like,
  including the one where the chooser simply has to choose more money
  rather than less, there is some possible pre-history of the problem
  where different choices will be rational. So if my interpretation is
  right, everything Levinstein and Soares say about what their theory
  recommends in one or another case is only correct given some
  substantive but unstated assumptions about the pre-history of the
  cases. That said, the broad approach they take does seem to be as
  anti-existentialist, in the ordinary sense of `existentialist,' as it
  is possible to be. That is some evidence I guess in favor of this
  reading.}

It's helpful to think of a strategy, a plan for what to do in all
situations, as the essence of the chooser. And the sequence of choices
the chooser makes in real time is their existence. So the
intellectualist thinks that essence precedes existence; a choice is made
good by its position in a grand strategy. And the existentialist denies
this. They think that existence precedes essence, or at least that it
does not proceed from essence.

The intellectualist position, whatever first order theory it is mixed
with, immediately entails Backwards Dynamic Consistency and Forwards
Dynamic Consistency. If what it is for a sequence of choices to be
rational just is for it to be a manifestation of a rational strategy,
then obviously there will be tight connections between rational
sequences and rational strategies. But the reverse entailment is not
true; the consistency norms do not entail intellectualism. One could, as
I'll argue in a minute, hold on to the norms within a thoroughly
existentialist framework.

At first glance, existentialism looks hostile to the very idea of
inter-temporal norms on agency. And there is a super strong form of
existentialism that denies all such norms. But there are three natural
ways to moderate one's existentialism to allow for the possibility of
such norms.

First, one could think that individual humans choose the units of time
over which their choices will be assessed. Intellectualism is the view
that the relevant unit is a life. The alternative to that need not be
that each instant is to be assessed anew. So the existentialist can
agree with \protect\hyperlink{ref-Holton2009}{Holton}
(\protect\hyperlink{ref-Holton2009}{2009}) that it is good to make
plans, that once a plan is made it is typically good to carry through
with it, and even that if one chooses a plan, that plan should be
assessed as a unit, instead of assessing the individual actions that
make up a plan. Existentialism, in the sense I'm using (or co-opting)
the expression, simply denies that planning in this sense is rationally
mandatory, and in particular that there need to be life-plans.

Second, one could think that there are norms about the appropriate
amount of fickleness in one's values and choices. It's consistent to say
that each choice should be assessed on its own merits, but that a choice
is bad in virtue of being excessively fickle. In that case one must say
that it is the second choice of the fickle pair that's the bad one; that
the chooser will later have different views is not a bad making feature
of a particular choice. An advantage of this approach over
intellectualism is that it allows that both stubbornness and fickleness
are vices. The intellectualist thinks that the ideal agent has one grand
plan and sticks to it through thick and thin. This doesn't sound great,
and the existentialist is not committed to it.

Third, and most importantly, one could think that failures of dynamic
consistency are not bad in themselves, but that they are evidence that
one of the individual choices within them is bad. (I'm borrowing ideas
from \protect\hyperlink{ref-Christensen1996}{Christensen}
(\protect\hyperlink{ref-Christensen1996}{1996}) and
\protect\hyperlink{ref-Kolodny2005}{Kolodny}
(\protect\hyperlink{ref-Kolodny2005}{2005}) here.) If you know a person
believes \(p\) and believes \(\neg p\), you know they've made a mistake,
even if you don't know when or where. And the mistake isn't (just) that
they are incoherent; one of these two beliefs was ill-formed. This is
the role that dynamic consistency norms play in the argument of this
paper. We can see that a view is wrong by seeing that it is dynamically
inconsistent. But the dynamic inconsistency is not constitutive of the
wrongness; it is just how we see that the view is wrong.

That's the position I'm adopting here. The problem with the decisive
views I've criticized is not that they are existentialist. Good theories
are existentialist. It's that when you ask them the same question in two
different guises, they give different answers. That's bad, and it's bad
even if one thinks that good choosers do not need an essence, or a
strategy, before they interact with the world.

\hypertarget{conclusions-and-further-research}{%
\section{Conclusions and Further
Research}\label{conclusions-and-further-research}}

I've argued that a whole bunch of decisive theories fail, in systematic
ways, to be dynamically consistent. I haven't offered anything like a
proof that there could not be a plausible theory that is both decisive
and dynamically consistent. Since some intellectualist theories are
decisive and dynamically consistent, maybe there is a way to make one of
them plausible. But by now the prospects should look grim.

So if decisive theories are bad, what should a good indecisive theory
look like? We've already seen one plausible indecisive theory: the
theory that says all Skyrms equilibria can be rationally chosen. But is
it the only plausible indecisive theory? Is it ultimately plausible at
all?

I think the way to finding a plausible indecisive theory goes via
answering the following five questions.

First, does decision theory start with what the chooser believes, or
with what they should believe? If Player is certain that the red box has
more money, but they have conclusive evidence that the blue box has more
money, which box does decision theory say that they should choose? David
\protect\hyperlink{ref-Lewis-Mellor-14101981}{Lewis}
(\protect\hyperlink{ref-Lewis-Mellor-14101981}{2020a}) says that the
answer is `the red box'; it is just the theory about how to turn beliefs
and desires into action. I'm more sympathetic to the arguments that Nomy
\protect\hyperlink{ref-Arpaly2003}{Arpaly}
(\protect\hyperlink{ref-Arpaly2003}{2003}) makes that the theory of
rational choice should not pay any special attention to the chooser's
beliefs. What's rational to choose in a situation is a function of
what's rational to believe in that situation, not what one actually
believes.

Second, in a given situation, how many different beliefs are rational?
The Uniqueness thesis says the answer is one. Permissivism says that
Uniquenes is false, and for some propositions in some situations, there
are multiple rational attitudes to have. See
\protect\hyperlink{ref-KopecTitelbaum2016}{Kopec and Titelbaum}
(\protect\hyperlink{ref-KopecTitelbaum2016}{2016}) for a good survey of
the issues, \protect\hyperlink{ref-Schultheis2018}{Schultheis}
(\protect\hyperlink{ref-Schultheis2018}{2018}) for a recent argument for
Uniqueness, and \protect\hyperlink{ref-Callahan2021}{Callahan}
(\protect\hyperlink{ref-Callahan2021}{2021}) for a recent argument for
Permissivism.\footnote{Interestingly, Callahan connects Permissivism to
  existentialism. I suspect there are deep and unexplored connections
  between the issues raised in the previous section and the
  Uniqueness/Permissivism debate.} I'm on the Permissivist side of this
debate.

Now if you think decision theory should be sensitive to rational beliefs
rather than actual beliefs, and you think Permissivism is true, you're
committed to indecisiveness. You won't even need demons. After all, any
situation where any credence in \(p\) between \(x\) and \(y\) is
permissible will mean there are multiple bets at distinct odds on \(p\)
that rationality neither requires taking nor requires passing. I think
this is a perfectly sound argument for indecisiveness, but I didn't lean
on it here because the premises are considerably less secure than
Backwards Dynamic Consistency.

But there is a third question that needs answering before we can offer a
plausible indecisive theory: what is a mixed strategy? Relatedly, what
role do mixed strategies have in the correct decision theory? This is a
rather vexed question, and an important one. Almost all recent arguments
against causal decision theory seem, to my eyes at least, to turn on
attributing a bad theory of mixed strategies to the causal decision
theorist. But that's a topic for another paper.\footnote{I haven't given
  a positive theory here because it's a big question, but I think the
  story must include the following two factors. First, playing a mixed
  strategy is just what
  \protect\hyperlink{ref-Lewis-Kavka-10071979}{Lewis}
  (\protect\hyperlink{ref-Lewis-Kavka-10071979}{2020b}) calls using a
  tie-breaking procedure. Second, the output of such a tie-breaking
  procedure is in principle unpredictable by anything that doesn't time
  travel.}

Note one thing I haven't said so far, and won't say in what follows. I
don't say that the way to find the correct indecisive theory is to come
up with a bunch of cases, consult our intuitions about them, and then
see which theory can match at least 80\% of those intuitions. (Or
whatever percentage we are working with this week.) That is a dubious
approach in general, but around here it is close to incoherent.

Most contemporary work in decision theory starts with the assumption
that when there are no demons around (or anything else vaguely demonic),
expected utility maximisation is the correct decision theory. And then
theorists will start rolling out fantastic cases involving demons or
predictors or lesions or genes or twins or triplets or whatever is in
fashion. And they will ask what extension of expected utility theory
best tracks intuitions about these cases. But this seems like a very
dubious strategy, since intuitions about cases will not lead one to
expected utility theory in the first place. Trying to match intuitions
about cases like the Allais or Ellsberg paradoxes will lead one to
prefer some non-standard theory like the one developed by John
\protect\hyperlink{ref-Quiggin1982}{Quiggin}
(\protect\hyperlink{ref-Quiggin1982}{1982}) or Lara
\protect\hyperlink{ref-BuchakRisk}{Buchak}
(\protect\hyperlink{ref-BuchakRisk}{2013}). It seems very unlikely that
the best way to extend a counterintuitive theory like expected utility
maximisation is by consulting intuitions about puzzle cases. It is much
better to ask what principles we want our theory to endorse, and work
towards a theory that satisfies those principles. And that is the
methodology I have adopted here.

I've relied heavily in this paper on one such principle: Backwards
Dynamic Consistency. I'll end by describing one more principle, and
noting two questions it raises. The principle is that a decider should
be a probabilist, and that they should maximise expected utility. More
precisely, it says that if the states are \(\{S_1, \dots, S_m\}\), and
the choices are \(\{O_1, \dots, O_n\}\), then \(O_i\) is a permissible
choice just in case there is some probability function \(Pr\) such that

\[
\sum_{k = 1}^m V(S_k \wedge O_i)Pr(S_k) \geq \sum_{k = 1}^m V(S_k \wedge O_j)Pr(S_k)
\]

for all \(j \in \{1, \dots, n\}\). Even if the subjective probability of
the state is affected by the choice one makes, there should be some
probability function that the chooser ends up with, and their choice
should make sense by the lights of that probability function. Note that
if we assume that the chooser can select any mixed strategy from among
their choices, there is guaranteed to be at least one strategy that
satisfies this requirement, even if one thinks the states are choices of
a demon who can predict one's strategy.\footnote{If the demon can
  predict what one will do on a given occasion while playing a mixed
  strategy, this guarantee may fail. But assuming what I said in the
  last footnote about mixed strategies, that would mean we're in the
  realm of backwards causation, and the states are not causally
  independent of the actions.}

So that seems to me like a minimal constraint on choices. As
\protect\hyperlink{ref-Pearce1984}{Pearce}
(\protect\hyperlink{ref-Pearce1984}{1984}) shows, it is equivalent to
the requirement that one not make a choice that is strictly dominated by
some other choice, or by some mixture of other choices. (This result is
hardly obvious, but it turns out to be a reasonably straightforward
consequence of the existence of Nash equilibria for all finite zero-sum
games.) That's hardly an uncontroversial principle, but it is also one
I'm happy to adopt. If you're still on board, there are two more
questions that we need to answer before we finish our decision theory.

Are all further constraints on rational decisions representable as
constraints on the \(Pr\) in this principle? There surely are some
further constraints on rational decisions. If you're offered a bet at
even money on whether I will become Canadian President next week, the
only rational thing to do is to decline it. And that's true even though
there is a \(Pr\) such that taking the bet maximises expected utility.
But that \(Pr\) is completely irrational given your evidence. So
\emph{Do something that maximises expected utility given some
probability} is too liberal a rule; we need to say something about the
\(Pr\). Do we need to say more than that? My answer is no, though I'm
not even going to start defending that here.\footnote{Note that if you
  say no to this question, and you think that probabilities have to be
  real-valued, then you're committed to weak dominance not having a role
  to play in decision theory. So this is a non-trivial question.}

The Canadian Presidency examples suggests that there are constraints on
\(Pr\) that are external to decision theory. You shouldn't take that bet
because you shouldn't have probability above 0.5 that I'll become
Canadian President next week. The order of explanation runs from the
(ir)rationality of the credal state to the (ir)rationality of the
decision. Our fifth and final question is, are there any cases where the
order of explanation goes the other way?
\protect\hyperlink{ref-Arntzenius2008}{Arntzenius}
(\protect\hyperlink{ref-Arntzenius2008}{2008}) argued that one should
have credences such that the highest value equilibrium was also the
choice that maximised expected utility. That's an example of a
constraint on \(Pr\) where the order of explanation runs from decisions
to beliefs. I argued against that principle, but not because of a
systematic reason to think that the order of explanation can't run that
way. Instead I argued that this particular principle was dynamically
incoherent. That leaves open the general question of whether any such
principles, where constraints on decisions explain constraints on
belief, are right.

The long term goal of the project behind this paper is to argue that
there are no such principles. The only constraints on rational decision
are that one should maximise expected utility given some \(Pr\), and
this \(Pr\) should satisfy independently motivated epistemic
requirements. Now I haven't come close to arguing for that here, and
it's a very strong claim. Given everything else I've said, it basically
amounts to the claim that the theory of equilibrium selection has no
role to play in normative decision theory. It may have a central role to
play in descriptive decision theory, in explaining why people end up at
a certain equilibrium. But it can't justify that equilibrium, since any
equilibrium could be rationally justified.

But all of this is for future work. The aim of this paper has been to
open up the possibility of an indecisive, i.e., permissive, decision
theory. Decisive decision theories have to take a stand on Demonic Stag
Hunt, but it seems surprisingly hard to find a plausible way to take a
stand on it in a dynamically coherent way. All the decisive theories
I've considered across a wide range of traditions, have ended up saying
that there is a clearly right thing do if \(p\) is correct, but it's not
the thing they will say if you tell them \(p\) is true and ask them what
to do. That's incoherent, so those theories are incorrect. The best
explanation of this range of indecisive theories being incorrect, I
suggest, is that no decisive theory is correct. The correct decision
theory is indecisive.

\newpage

\hypertarget{references}{%
\subsection*{References}\label{references}}
\addcontentsline{toc}{subsection}{References}

\hypertarget{refs}{}
\begin{CSLReferences}{1}{0}
\leavevmode\hypertarget{ref-Arntzenius2008}{}%
Arntzenius, Frank. 2008. {``No Regrets; or, Edith Piaf Revamps Decision
Theory.''} \emph{Erkenntnis} 68 (2): 277--97.
\url{https://doi.org/10.1007/s10670-007-9084-8}.

\leavevmode\hypertarget{ref-Arpaly2003}{}%
Arpaly, Nomy. 2003. \emph{Unprincipled Virtue}. Oxford: Oxford
University Press.

\leavevmode\hypertarget{ref-Axelrod1984}{}%
Axelrod, Robert. 1984. \emph{The Evolution of Cooperation}. New York:
Basic Books.

\leavevmode\hypertarget{ref-Bonanno2018}{}%
Bonanno, Giacomo. 2018. {``Game Theory.''} Davis, CA: CreateSpace
Independent Publishing Platform. 2018.
\url{http://faculty.econ.ucdavis.edu/faculty/bonanno/GT_Book.html}.

\leavevmode\hypertarget{ref-BuchakRisk}{}%
Buchak, Lara. 2013. \emph{Risk and Rationality}. Oxford: Oxford
University Press.

\leavevmode\hypertarget{ref-Callahan2021}{}%
Callahan, Laura Frances. 2021. {``Epistemic Existentialism.''}
\emph{Episteme}. \url{https://doi.org/10.1017/epi.2019.25}.

\leavevmode\hypertarget{ref-Chang2002}{}%
Chang, Ruth. 2002. {``The Possibility of Parity.''} \emph{Ethics} 112
(4): 659--88. \url{https://doi.org/10.1086/339673}.

\leavevmode\hypertarget{ref-Christensen1996}{}%
Christensen, David. 1996. {``Dutch-Book Arguments {D}e-Pragmatized:
Epistemic Consistency for Partial Believers.''} \emph{Journal of
Philosophy} 93 (9): 450--79. \url{https://doi.org/10.2307/2940893}.

\leavevmode\hypertarget{ref-Egan2007-EGASCT}{}%
Egan, Andy. 2007. {``{Some Counterexamples to Causal Decision
Theory}.''} \emph{Philosophical Review} 116 (1): 93--114.
\url{https://doi.org/10.1215/00318108-2006-023}.

\leavevmode\hypertarget{ref-Gallow2020}{}%
Gallow, J. Dmitri. 2020. {``The Causal Decision Theorist's Gudie to
Managing the News.''} \emph{The Journal of Philosophy} 117 (3): 117--49.

\leavevmode\hypertarget{ref-GibbardHarper1978}{}%
Gibbard, Allan, and William Harper. 1978. {``Counterfactuals and Two
Kinds of Expected Utility.''} In \emph{Foundations and Applications of
Decision Theory}, edited by C. A. Hooker, J. J. Leach, and E. F.
McClennen, 125--62. Dordrecht: Reidel.

\leavevmode\hypertarget{ref-Holton2009}{}%
Holton, Richard. 2009. \emph{Willing, Wanting, Waiting}. Oxford: Oxford
University Press.

\leavevmode\hypertarget{ref-Jeffrey1983}{}%
Jeffrey, Richard. 1983. {``Bayesianism with a Human Face.''} In
\emph{Testing Scientific Theories}, edited by J. Earman (ed.).
Minneapolis: University of Minnesota Press.

\leavevmode\hypertarget{ref-Kolodny2005}{}%
Kolodny, Niko. 2005. {``Why Be Rational?''} \emph{Mind} 114 (455):
509--63. \url{https://doi.org/10.1093/mind/fzi509}.

\leavevmode\hypertarget{ref-KopecTitelbaum2016}{}%
Kopec, Matthew, and Michael G. Titelbaum. 2016. {``The Uniqueness
Thesis.''} \emph{Philosophy Compass} 11 (4): 189--200.
\url{https://doi.org/10.1111/phc3.12318}.

\leavevmode\hypertarget{ref-LevinsteinSoares2020}{}%
Levinstein, Benjamin Anders, and Nate Soares. 2020. {``Cheating Death in
Damascus.''} \emph{Journal of Philosophy} 117 (5): 237--66.
\url{https://doi.org/10.5840/jphil2020117516}.

\leavevmode\hypertarget{ref-Lewis-Mellor-14101981}{}%
Lewis, David. 2020a. {``Letter to {D}. H. Mellor, 14 October 1981.''} In
\emph{Philosophical Letters of David {K}. Lewis}, edited by Helen Beebee
and A. R. J. Fisher, 2:432--34. Oxford: Oxford University Press.

\leavevmode\hypertarget{ref-Lewis-Kavka-10071979}{}%
---------. 2020b. {``Letter to Gregory Kavka, 10 July 1979.''} In
\emph{Philosophical Letters of David {K}. Lewis}, edited by Helen Beebee
and A. R. J. Fisher, 2:423--24. Oxford: Oxford University Press.

\leavevmode\hypertarget{ref-Pearce1983}{}%
Pearce, David G. 1983. {``A Problem with Single Valued Solution
Concepts.''} 1983.
\url{https://sites.google.com/a/nyu.edu/davidpearce/}.

\leavevmode\hypertarget{ref-Pearce1984}{}%
---------. 1984. {``Rationalizable Strategic Behavior and the Problem of
Perfection.''} \emph{Econometrica} 52 (4): 1029--50.
\url{https://doi.org/10.2307/1911197}.

\leavevmode\hypertarget{ref-Quiggin1982}{}%
Quiggin, John. 1982. {``A Theory of Anticipated Utility.''}
\emph{Journal of Economic Behavior \& Organization} 3 (4): 323--43.
\url{https://doi.org/10.1016/0167-2681(82)90008-7}.

\leavevmode\hypertarget{ref-Richter1984}{}%
Richter, Reed. 1984. {``Rationality Revisited.''} \emph{Australasian
Journal of Philosophy} 62 (4): 393--404.
\url{https://doi.org/10.1080/00048408412341601}.

\leavevmode\hypertarget{ref-Schultheis2018}{}%
Schultheis, Ginger. 2018. {``Living on the Edge: Against Epistemic
Permissivism.''} \emph{Mind} 127 (507): 863--79.
\url{https://doi.org/10.1093/mind/fzw065}.

\leavevmode\hypertarget{ref-Skyrms1990}{}%
Skyrms, Brian. 1990. \emph{The Dynamics of Rational Deliberation}.
Cambridge, MA: Harvard University Press.

\leavevmode\hypertarget{ref-Skyrms2001}{}%
---------. 2001. {``The Stag Hunt.''} \emph{Proceedings and Addresses of
the American Philosophical Association} 75 (2): 31--41.
\url{https://doi.org/10.2307/3218711}.

\leavevmode\hypertarget{ref-Skyrms2004}{}%
---------. 2004. \emph{The Stag Hunt and the Evolution of Social
Structure}. Cambridge: {C}ambridge {U}niversity {P}ress.

\leavevmode\hypertarget{ref-Wedgwood2013}{}%
Wedgwood, Ralph. 2013. {``A Priori Bootstrapping.''} In \emph{The a
Priori in Philosophy}, edited by Albert Casullo and Joshua C. Thurow,
225--46. Oxford: Oxford University Press.

\end{CSLReferences}

\end{document}
