% Options for packages loaded elsewhere
% Options for packages loaded elsewhere
\PassOptionsToPackage{unicode}{hyperref}
\PassOptionsToPackage{hyphens}{url}
\PassOptionsToPackage{dvipsnames,svgnames,x11names}{xcolor}
%
\documentclass[
  10.5pt,
  twoside]{article}
\usepackage{xcolor}
\usepackage[paperheight=10in,,paperwidth=7in,,top=1in,bottom=1in,inner=0.8in,outer=0.8in,headsep=0.25in,headheight=1in,footskip=0.35in]{geometry}
\usepackage{amsmath,amssymb}
\setcounter{secnumdepth}{5}
\usepackage{iftex}
\ifPDFTeX
  \usepackage[T1]{fontenc}
  \usepackage[utf8]{inputenc}
  \usepackage{textcomp} % provide euro and other symbols
\else % if luatex or xetex
  \usepackage{unicode-math} % this also loads fontspec
  \defaultfontfeatures{Scale=MatchLowercase}
  \defaultfontfeatures[\rmfamily]{Ligatures=TeX,Scale=1}
\fi
\usepackage{lmodern}
\ifPDFTeX\else
  % xetex/luatex font selection
  \setmainfont[ItalicFont=EB Garamond Italic,BoldFont=EB Garamond
SemiBold]{EB Garamond Math}
  \setsansfont[]{EB Garamond SemiBold}
  \setmathfont[]{EB Garamond Math}
\fi
% Use upquote if available, for straight quotes in verbatim environments
\IfFileExists{upquote.sty}{\usepackage{upquote}}{}
\IfFileExists{microtype.sty}{% use microtype if available
  \usepackage[]{microtype}
  \UseMicrotypeSet[protrusion]{basicmath} % disable protrusion for tt fonts
}{}
\usepackage{setspace}
\makeatletter
\@ifundefined{KOMAClassName}{% if non-KOMA class
  \IfFileExists{parskip.sty}{%
    \usepackage{parskip}
  }{% else
    \setlength{\parindent}{0pt}
    \setlength{\parskip}{6pt plus 2pt minus 1pt}}
}{% if KOMA class
  \KOMAoptions{parskip=half}}
\makeatother
% Make \paragraph and \subparagraph free-standing
\makeatletter
\ifx\paragraph\undefined\else
  \let\oldparagraph\paragraph
  \renewcommand{\paragraph}{
    \@ifstar
      \xxxParagraphStar
      \xxxParagraphNoStar
  }
  \newcommand{\xxxParagraphStar}[1]{\oldparagraph*{#1}\mbox{}}
  \newcommand{\xxxParagraphNoStar}[1]{\oldparagraph{#1}\mbox{}}
\fi
\ifx\subparagraph\undefined\else
  \let\oldsubparagraph\subparagraph
  \renewcommand{\subparagraph}{
    \@ifstar
      \xxxSubParagraphStar
      \xxxSubParagraphNoStar
  }
  \newcommand{\xxxSubParagraphStar}[1]{\oldsubparagraph*{#1}\mbox{}}
  \newcommand{\xxxSubParagraphNoStar}[1]{\oldsubparagraph{#1}\mbox{}}
\fi
\makeatother


\usepackage{longtable,booktabs,array}
\usepackage{calc} % for calculating minipage widths
% Correct order of tables after \paragraph or \subparagraph
\usepackage{etoolbox}
\makeatletter
\patchcmd\longtable{\par}{\if@noskipsec\mbox{}\fi\par}{}{}
\makeatother
% Allow footnotes in longtable head/foot
\IfFileExists{footnotehyper.sty}{\usepackage{footnotehyper}}{\usepackage{footnote}}
\makesavenoteenv{longtable}
\usepackage{graphicx}
\makeatletter
\newsavebox\pandoc@box
\newcommand*\pandocbounded[1]{% scales image to fit in text height/width
  \sbox\pandoc@box{#1}%
  \Gscale@div\@tempa{\textheight}{\dimexpr\ht\pandoc@box+\dp\pandoc@box\relax}%
  \Gscale@div\@tempb{\linewidth}{\wd\pandoc@box}%
  \ifdim\@tempb\p@<\@tempa\p@\let\@tempa\@tempb\fi% select the smaller of both
  \ifdim\@tempa\p@<\p@\scalebox{\@tempa}{\usebox\pandoc@box}%
  \else\usebox{\pandoc@box}%
  \fi%
}
% Set default figure placement to htbp
\def\fps@figure{htbp}
\makeatother


% definitions for citeproc citations
\NewDocumentCommand\citeproctext{}{}
\NewDocumentCommand\citeproc{mm}{%
  \begingroup\def\citeproctext{#2}\cite{#1}\endgroup}
\makeatletter
 % allow citations to break across lines
 \let\@cite@ofmt\@firstofone
 % avoid brackets around text for \cite:
 \def\@biblabel#1{}
 \def\@cite#1#2{{#1\if@tempswa , #2\fi}}
\makeatother
\newlength{\cslhangindent}
\setlength{\cslhangindent}{1.5em}
\newlength{\csllabelwidth}
\setlength{\csllabelwidth}{3em}
\newenvironment{CSLReferences}[2] % #1 hanging-indent, #2 entry-spacing
 {\begin{list}{}{%
  \setlength{\itemindent}{0pt}
  \setlength{\leftmargin}{0pt}
  \setlength{\parsep}{0pt}
  % turn on hanging indent if param 1 is 1
  \ifodd #1
   \setlength{\leftmargin}{\cslhangindent}
   \setlength{\itemindent}{-1\cslhangindent}
  \fi
  % set entry spacing
  \setlength{\itemsep}{#2\baselineskip}}}
 {\end{list}}
\usepackage{calc}
\newcommand{\CSLBlock}[1]{\hfill\break\parbox[t]{\linewidth}{\strut\ignorespaces#1\strut}}
\newcommand{\CSLLeftMargin}[1]{\parbox[t]{\csllabelwidth}{\strut#1\strut}}
\newcommand{\CSLRightInline}[1]{\parbox[t]{\linewidth - \csllabelwidth}{\strut#1\strut}}
\newcommand{\CSLIndent}[1]{\hspace{\cslhangindent}#1}



\setlength{\emergencystretch}{3em} % prevent overfull lines

\providecommand{\tightlist}{%
  \setlength{\itemsep}{0pt}\setlength{\parskip}{0pt}}



 


% Custom title page mimicking Ergo class style
% Add this to your Quarto YAML with:
% format:
%   pdf:
%     include-in-header:
%       - maketitle.tex

\makeatletter

% Redefine \maketitle to match Ergo style
\renewcommand\maketitle{
  \begingroup
  \setlength{\parskip}{0pt plus 0pt minus 0pt}
  \long\def\@makefntext##1{\parindent 1em\noindent
          \hb@xt@1.8em{%
             \hss\@textsuperscript{\normalfont\@thefnmark}}##1}%
  \newpage
  \null
  \noindent{}\begin{minipage}[l]{4in}
    \vskip 53.3pt plus 0pt minus 0pt
    {\LARGE \textsc{\@title}\par}% Title in large small caps
    \vskip 15pt plus 0pt minus 0pt
    {\Huge\MakeUppercase{\@author}\par}% Author in huge uppercase
    \vskip 5pt plus 0pt minus 0pt
    {\normalsize\textit{University of Michigan}\par}% Date in normal size italics (or affiliation)
  \end{minipage}
  \vskip 12pt plus 0pt minus 0pt
  \thispagestyle{plain}
  \endgroup
  \setcounter{footnote}{0}%
  \global\let\thanks\relax
  \global\let\@thanks\@empty
  \global\let\@date\@empty
  \global\let\title\relax
  \global\let\author\relax
  \global\let\date\relax
  \global\let\and\relax
}

% Redefine abstract to match Ergo style with margins
\renewenvironment{abstract}
 {\list{}{
    \setlength{\leftmargin}{.25in}
    \setlength{\rightmargin}{\leftmargin}
  }
  \item\relax
  \small}
 {\vskip -3pt plus 0pt minus 0pt\null\endlist}

% Font size adjustments to match Ergo
\renewcommand\normalsize{\@setfontsize\normalsize{10.5pt}{14pt}}
\renewcommand\small{\@setfontsize\small{9pt}{13pt}}
\renewcommand\LARGE{\@setfontsize\LARGE{18pt}{22pt}}
\renewcommand\Huge{\@setfontsize\Huge{9.5pt}{13.5pt}}

\makeatother
% Body text and heading styles mimicking Ergo class
% Add this to your Quarto YAML with:
% format:
%   pdf:
%     include-in-header:
%       - body-style.tex

\usepackage[automark]{scrlayer-scrpage}
\clearpairofpagestyles
\cehead{
  Brian Weatherson
  }
\cohead{
  Humberstone on Possibility Frames
  }
\ohead{\bfseries \pagemark}
\cfoot{}

\makeatletter

% Paragraph indentation and spacing
\setlength{\parindent}{.25in}
\setlength{\parskip}{0pt plus 0pt minus 0pt}

% Section numbering with periods and spacing
\renewcommand*{\@seccntformat}[1]{%
  \csname the#1\endcsname.$\:$%
}

% Section styles (matching Ergo)
\renewcommand\section{\@startsection{section}{1}{\z@}%
  {-4.6ex \@plus 0ex \@minus 0ex}%        % space before
  {2.3ex \@plus 0ex}%                      % space after
  {\normalfont\Large\bfseries}}            % Large, bold

\renewcommand\subsection{\@startsection{subsection}{2}{\z@}%
  {-2.3ex \@plus 0ex \@minus 0ex}%        % space before
  {2.3ex \@plus 0ex \@minus 0ex}%         % space after
  {\normalfont\large\itshape}}             % large, italic

\renewcommand\subsubsection{\@startsection{subsubsection}{3}{\z@}%
  {-2.3ex \@plus 0ex \@minus 0ex}%        % space before
  {2.3ex \@plus 0ex \@minus 0ex}%         % space after
  {\normalfont\large}}                     % large, normal

% Font sizes (matching Ergo)
\renewcommand\normalsize{\@setfontsize\normalsize{10.5pt}{14pt}}
\renewcommand\footnotesize{\@setfontsize\footnotesize{9pt}{11.5pt}}
\renewcommand\small{\@setfontsize\small{9pt}{13pt}}
\renewcommand\large{\@setfontsize\large{11.5pt}{12pt}}
\renewcommand\Large{\@setfontsize\Large{12pt}{14pt}}

% Quotation and quote environments with Ergo spacing
\renewenvironment{quotation}
  {\list{}{
    \setlength{\listparindent}{.25in}%
    \setlength{\leftmargin}{.25in}
    \setlength{\rightmargin}{\leftmargin}
    \setlength{\parsep}{0in plus 0in minus 0in}
    \item\relax
    \let\item\relax}
  }
  {\endlist}

\renewenvironment{quote}
  {\vskip 5pt%
   \list{}{
    \setlength{\listparindent}{.25in}
    \setlength{\leftmargin}{.25in}
    \setlength{\rightmargin}{\leftmargin}
    \setlength{\parsep}{0in plus 0in minus 0in}
    }
    \item\relax
    \let\item\relax}
  {\endlist\vskip 5pt}

% Footnote rule styling
\renewcommand{\footnoterule}{%
  \kern -3pt
  \hrule width 1in height .4pt
  \kern 2.5pt
}

% Non-superscript numerals for footnote text numbering
\renewcommand\@makefntext[1]{%
  \parindent .25in%
  \@thefnmark.~#1}%

% Figure and table captions with period separator
\renewcommand\@makecaption[2]{%
  \vskip\abovecaptionskip
  \sbox\@tempboxa{#1. #2}%
  \ifdim \wd\@tempboxa >\hsize
    #1. #2\par
  \else
    \global \@minipagefalse
    \hb@xt@\hsize{\hfil\box\@tempboxa\hfil}%
  \fi
  \vskip\belowcaptionskip}

% Enumerate environment with extra vertical space
\let\oldenumerate\enumerate
\let\endoldenumerate\endenumerate
\renewenvironment{enumerate}
  {\vskip 5pt\oldenumerate}
  {\endoldenumerate\vskip 5pt}

\let\olddescription\description
\let\endolddescription\enddescription
\renewenvironment{description}
  {\vskip 5pt\olddescription}
  {\endolddescription\vskip 5pt}

\let\olditemize\itemize
\let\endolditemize\enditemize
\renewenvironment{itemize}
  {\vskip 5pt\olditemize}
  {\endolditemize\vskip 5pt}

\newcommand*\NoIndentAfterEnv[1]{%
  \AfterEndEnvironment{#1}{\par\@afterindentfalse\@afterheading}}
\makeatother
\NoIndentAfterEnv{itemize}
\NoIndentAfterEnv{enumerate}
\NoIndentAfterEnv{description}
\NoIndentAfterEnv{quote}
\NoIndentAfterEnv{equation}
\NoIndentAfterEnv{longtable}

\makeatother
% Simple aggressive line breaking for DOIs
\usepackage{xurl}
\PassOptionsToPackage{hyphens}{url}

% Allow breaks at many characters
\def\UrlBreaks{\do\a\do\b\do\c\do\d\do\e\do\f\do\g\do\h\do\i\do\j%
\do\k\do\l\do\m\do\n\do\o\do\p\do\q\do\r\do\s\do\t\do\u\do\v\do\w%
\do\x\do\y\do\z\do\A\do\B\do\C\do\D\do\E\do\F\do\G\do\H\do\I\do\J%
\do\K\do\L\do\M\do\N\do\O\do\P\do\Q\do\R\do\S\do\T\do\U\do\V\do\W%
\do\X\do\Y\do\Z\do\0\do\1\do\2\do\3\do\4\do\5\do\6\do\7\do\8\do\9%
\do\.\do\-\do\/\do\:\do\=\do\?\do\&\do\_}

% Emergency line breaking
\sloppy
\newcommand{\nmodels}{\mathrel{\ooalign{$\models$\cr\raisebox{-0.001ex}{\hss$\mkern-1mu/\hss$}\cr}}}
\newcommand{\llbracket}{[\![}
\newcommand{\rrbracket}{]\!]}
\setlength\heavyrulewidth{0ex}
\setlength\lightrulewidth{0ex}
\usepackage[lines=2]{lettrine}
\cehead{Draft of February 22, 2026}
\DeclareSymbolFont{symbolsC}{U}{txsyc}{m}{n}
\DeclareMathSymbol{\boxright}{\mathrel}{symbolsC}{128}
\makeatletter
\@ifpackageloaded{caption}{}{\usepackage{caption}}
\AtBeginDocument{%
\ifdefined\contentsname
  \renewcommand*\contentsname{Table of contents}
\else
  \newcommand\contentsname{Table of contents}
\fi
\ifdefined\listfigurename
  \renewcommand*\listfigurename{List of Figures}
\else
  \newcommand\listfigurename{List of Figures}
\fi
\ifdefined\listtablename
  \renewcommand*\listtablename{List of Tables}
\else
  \newcommand\listtablename{List of Tables}
\fi
\ifdefined\figurename
  \renewcommand*\figurename{Figure}
\else
  \newcommand\figurename{Figure}
\fi
\ifdefined\tablename
  \renewcommand*\tablename{Table}
\else
  \newcommand\tablename{Table}
\fi
}
\@ifpackageloaded{float}{}{\usepackage{float}}
\floatstyle{ruled}
\@ifundefined{c@chapter}{\newfloat{codelisting}{h}{lop}}{\newfloat{codelisting}{h}{lop}[chapter]}
\floatname{codelisting}{Listing}
\newcommand*\listoflistings{\listof{codelisting}{List of Listings}}
\makeatother
\makeatletter
\makeatother
\makeatletter
\@ifpackageloaded{caption}{}{\usepackage{caption}}
\@ifpackageloaded{subcaption}{}{\usepackage{subcaption}}
\makeatother
\usepackage{bookmark}
\IfFileExists{xurl.sty}{\usepackage{xurl}}{} % add URL line breaks if available
\urlstyle{same}
\hypersetup{
  pdftitle={Humberstone on Possibility Frames},
  pdfauthor={Brian Weatherson},
  colorlinks=true,
  linkcolor={blue},
  filecolor={Maroon},
  citecolor={Blue},
  urlcolor={blue},
  pdfcreator={LaTeX via pandoc}}


\title{Humberstone on Possibility Frames}
\author{Brian Weatherson}
\date{2026-02-22}
\begin{document}
\maketitle
\begin{abstract}
Insert abstract here
\end{abstract}


\setstretch{1}
In his 1981 paper, ``From Worlds to Possibilities'', Lloyd Humberstone
shows a way to do modal logic without the apparatus of possible worlds.
Instead of worlds he uses \emph{possibilities}, which may, unlike
worlds, be incomplete. The non-modal parts of the view are discussed
again in section 6.44 of \emph{The Connectives}, though the differences
between the view there and the 1981 view are largely presentational. In
this paper I'll set out this \emph{possibility frame} approach to modal
logic, make some notes about its logic, and end with a survey of the
many possible applications it has.

Mathematically, possibilities are just points in a model, just like
possible worlds are points in different kinds of models. But it helps to
have a mental picture of what kind of thing they are. In ``From Worlds
to Possibilities'', Humberstone notes that one picture you could have is
that they are sets of possible worlds. This isn't a terrible picture,
but it's not perfect for a couple of reasons. For one thing, as
Humberstone notes, part of the point of developing possibilities is to
do without the machinery of possible worlds. Understanding possibilities
as sets of possible worlds wouldn't help with that project. For another,
as Wesley Holliday (\citeproc{ref-Holliday2025}{2025, 271--72}) notes,
the natural way to generate modal accessibility relations on sets of
worlds from accessibility on the worlds themselves doesn't always work
the way Humberstone wants accessibility to work. So let's start with a
different picture.

Possibilities, as I'll think of them, are \emph{stories}. To make things
concrete, let's focus on a particular story: \emph{A Study in Scarlet}
(Conan Doyle (\citeproc{ref-ConanDoyle1995}{1995})), the story where
Sherlock Holmes was introduced. That story settles some questions, both
explicitly, e.g., that Holmes is a detective, and implicitly, e.g., that
Holmes has never set foot on the moon. But it leaves several other
questions open, e.g., how many (first) cousins Holmes has. It's not that
\emph{A Study in Scarlet} is a story. It has proper parts which are
stories. The first chapter is a story, one which tells of the first
meeting between Holmes and Watson. And arguably it is a proper part of
larger story, made up of all of Conan Doyle's stories of Holmes and
Watson. When a story \(x\) is a proper part of story \(y\), what that
means is that everything settled in \(x\) is still true in \(y\), and
more things besides are settled. When this happens, we'll call \(y\) a
proper \emph{refinement} of \(x\). For most purposes it will be more
convenient to use the more general notion of \emph{refinement}, where
each story counts as an improper refinement of itself.

Following Humberstone, I'll write \(x \leqslant y\) to mean that \(y\)
is a refinement of \(x\). As he notes, this notation can be confusing if
one things of \(x\) and \(y\) as sets, because in that case the
refinement will typically be \emph{smaller}.\footnote{Holliday
  (\citeproc{ref-Holliday2025}{2025}) writes \(y \sqsubseteq x\) when
  \(y\) is a refinement of \(x\), mirroring this way of thinking about
  possibilities.} But if we think of possibilities as stories, the
notation becomes more intuitive. We have \(x \leqslant y\) when \(y\) is
created by adding new content to \(x\). Keeping with this theme, I'll
typically model stories not as worlds, but as finite sequences. (In the
main example in Section~\ref{sec-proof}, they will be sequences of 0s
and 1s.) In these models, \(x \leqslant y\) means that \(x\) is an
initial segment of \(y\).

\section{The Formal Structure}\label{sec-formal}

\subsection*{The Basic Language}\label{the-basic-language}
\addcontentsline{toc}{subsection}{The Basic Language}

To start with, assume we're working in a simple language that just has a
countable set \(\mathcal{P}\) countable infinity of propositional
variables, and three connectives: \(\neg\), \(\wedge\) and \(\vee\). We
have a set of possibilities \(W\), and a transitive refinement relation
\(\geqslant\) on them. The following rules show how to build what I'll
call a \emph{Humberstone possibility model} on
\(\langle W, \leqslant \rangle\). (I'll call this a \emph{possibility
frame} in most contexts, but a \emph{Humberstone frame} when I'm
comparing it to similar structures, especially in the context of
discussing Holliday (\citeproc{ref-Holliday2025}{2025}).)

A Humberstone possibility model \(\mathcal{M}\) is a triple
\(\langle W, \leqslant, V \rangle\), where \(V\) is a function from
\(\mathcal{P}\) to \(W\), intuitively saying where each atomic
proposition is true, satisfying these two constraints:

\begin{itemize}
\tightlist
\item
  For all \(x\), if \(x \in V(p)\) and \(y \geqslant x\), then
  \(y \in V(p)\). Intuitively, truth for atomics is \textbf{persistent}
  across refinements.
\item
  For all \(x\), if
  \(\forall y \geqslant x \exists z \geqslant y: z \in V(p)\), then
  \(x \in V(p)\). This is what Humberstone
  (\citeproc{ref-Humberstone2011}{2011, 900}) calls
  \textbf{refinability}, and it means that \(p\) only fails to be true
  at \(x\) if there is some refinement of \(x\) where it is settled as
  being untrue.
\end{itemize}

Given these constraints, Humberstone suggests the following theory of
truth at a possibility for all sentences in this language. (We'll treat
\(\rightarrow\) as a defined connective, with
\(A \rightarrow B =_{df} A \vee \neg B\).)

\begin{align*}
[\text{Vbls}] \quad & \mathcal{M} \models_x p_i \text{ iff } x \in V(p_i); \\
[\neg] \quad & \mathcal{M} \models_x \neg A \text{ iff } \forall y \geqslant x, \, \mathcal{M} \nmodels_y  A; \\
[\wedge] \quad & \mathcal{M} \models_x A \wedge B \text{ iff } \mathcal{M} \models_x A \text{ and } \mathcal{M} \models_x B; \\
[\vee] \quad & \mathcal{M} \models_x A \vee B \text{ iff } \forall y \geqslant x \, \exists z \geqslant y \, : \, \mathcal{M} \models_z A \text{ or } \mathcal{M} \models_z B.
\end{align*}

Given these definitions, it's possible to prove three things. First,
every sentence in the language is persistent. If
\(\mathcal{M} \models_x A\) and \(x \leqslant y\), then
\(\mathcal{M} \models_x A\). For any sentence, truth is always preserved
when moving to a refinement. Second, refinability holds for all
sentences in the language. This is, as Humberstone notes, easier to
state using this definition of \(\neg\). It now becomes the claim, for
arbitrary \(A\), that if \(\mathcal{M} \nmodels_x A\), there is some
refinement \(y\) of \(x\) such that \(\mathcal{M} \models_y \neg A\).
Third, for any set of sentences \(\Gamma\) and sentence \(A\), the truth
at a point of all sentences in \(\Gamma\) guarantees the truth of \(A\)
iff the sequent \(\Gamma\) entails \(A\) in classical propositional
logic.

In this paper, I'm going to discuss three extensions of this language.
I'll introduce them in reverse order of how much they are discussed in
Humberstone, starting with one he does not discuss at all: infinitary
disjunction.

\subsection*{Infinitary Disjunction}\label{infinitary-disjunction}
\addcontentsline{toc}{subsection}{Infinitary Disjunction}

We'll add to the language a new symbol \(\bigvee\), which forms a new
sentence out of any countable set of sentences not containing
\(\bigvee\). Intuitively, it is true when one of the sentences in the
set is true. More formally, its definition of truth at a possibility is:

\begin{align*}
[\bigvee] \quad & \mathcal{M} \models_x \bigvee ({A_1, A_2, \dots})  \text{ iff } \forall y \geqslant x \, \exists z \geqslant y \, : \,\text{ for some } i \, \mathcal{M} \models_z A_i.
\end{align*}

Again, it's fairly simple to show that this addition to the language
will preserve persistence and refinability. But while this is simple, it
is significant, because things could easily have been otherwise.

\subsection*{Quantifiers}\label{quantifiers}
\addcontentsline{toc}{subsection}{Quantifiers}

The second extension will be to add quantifiers, following a suggestion
in Humberstone (\citeproc{ref-Humberstone1981a}{1981}, xxxx). Assume, as
usual, that the language has a stock of names \(c_1, \dots\), and for
each \(n\), a stock of \(n\)-place predicates \(F^n_1, F^n_2, \dots\). A
\emph{first-order (Humberstone) possibility model} is a structure
\(\langle W, \leqslant, D, V \rangle\), where \(D\) assigns a non-empty
domain of objects to each point, and \(V\) interprets the non-logical
vocabulary. More precisely:

\begin{itemize}
\tightlist
\item
  \(D\) is a function assigning to each \(x \in W\) a non-empty set
  \(D(x)\), the \textbf{domain} at \(x\).
\item
  \(V\) assigns to each name \(c_i\) and each \(x \in W\) either a
  designated element \(V(c_i, x) \in D(x)\), or is undefined at \(x\).
\item
  \(V\) assigns to each \(n\)-place predicate \(F^n_j\) and each
  \(x \in W\) a set \(V(F^n_j, x) \subseteq D(x)^n\), the
  \textbf{extension} of \(F^n_j\) at \(x\).
\end{itemize}

These must satisfy the following constraints:

\begin{description}
\tightlist
\item[Domain monotonicity]
If \(x \leqslant y\), then \(D(x) \subseteq D(y)\).
\item[Name coverage]
For each name \(c_i\) and each \(x \in W\), there exists some
\(y \geqslant x\) such that \(V(c_i, y)\) is defined.
\item[Persistence for names]
If \(V(c_i, x)\) is defined and \(x \leqslant y\), then \(V(c_i, y)\) is
defined and \(V(c_i, y) = V(c_i, x)\).
\item[Persistence for predicate extensions]
If \(\langle o_1, \dots, o_n \rangle \in V(F^n_j, x)\) and
\(x \leqslant y\), then
\(\langle o_1, \dots, o_n \rangle \in V(F^n_j, y)\).
\item[Refinability for predicate extensions]
If \(\langle o_1, \dots, o_n \rangle \notin V(F^n_j, x)\), then there
exists some \(y \geqslant x\) such that for all \(z \geqslant y\),
\(\langle o_1, \dots, o_n \rangle \notin V(F^n_j, z)\).
\end{description}

Given a model and a variable assignment \(g\) mapping variables to
objects, truth at a point is defined as follows. Write \(g[v/o]\) for
the assignment that maps variable \(v\) to object \(o\) and otherwise
agrees with \(g\). For a term \(t\), write
\(\llbracket t \rrbracket^{g,x}\) for the denotation of \(t\) under
\(g\) at \(x\): for a variable \(v\) this is \(g(v)\), and for a name
\(c_i\) this is \(V(c_i, x)\) when defined, and undefined otherwise.

\begin{align*}
[=] \quad & \mathcal{M}, g \models_x t_1 = t_2 \text{ iff } \forall y \geqslant x \, \exists z \geqslant y : \llbracket t_1 \rrbracket^{g,z} \text{ and } \llbracket t_2 \rrbracket^{g,z} \text{ are both defined and equal}; \\
[F^n] \quad & \mathcal{M}, g \models_x F^n_j(t_1, \dots, t_n) \text{ iff } \forall y \geqslant x \, \exists z \geqslant y : \langle \llbracket t_1 \rrbracket^{g,z}, \dots, \llbracket t_n \rrbracket^{g,z} \rangle \in V(F^n_j, z); \\
[\forall] \quad & \mathcal{M}, g \models_x \forall v \, A \text{ iff } \forall y \geqslant x \, \forall o \in D(y) : \mathcal{M}, g[v/o] \models_y A; \\
[\exists] \quad & \mathcal{M}, g \models_x \exists v \, A \text{ iff } \forall y \geqslant x \, \exists z \geqslant y \, \exists o \in D(z) : \mathcal{M}, g[v/o] \models_z A.
\end{align*}

The Boolean connectives are handled exactly as in the propositional
case.

The \(\forall\exists\) pattern in the atomic clauses is necessary to
ensure that persistence and refinability hold for all sentences,
including atomic ones. Consider \(c_i = c_i\): if a name has no
denotation at \(x\) but acquires one at some refinement, then a simple
``check the denotation at \(x\)'' condition would leave \(c_i = c_i\)
neither true nor false at \(x\), and no refinement of \(x\) could settle
it as false either, violating refinability. The \(\forall\exists\)
condition handles this correctly: \(c_i = c_i\) is true at \(x\)
whenever \(c_i\) is covered at \(x\) (i.e., every refinement has a
further refinement where \(c_i\) gets a referent), since once \(c_i\)
gets a referent \(o\), persistence of names ensures \(o = o\) at all
further refinements.

The atomic clauses simplify when names are fully defined. If \(t_1\) and
\(t_2\) are variables, or names that already have denotations at \(x\),
then by persistence of names and predicate extensions the
\(\forall\exists\) quantifier prefix collapses:
\(\mathcal{M}, g \models_x t_1 = t_2\) iff
\(\llbracket t_1 \rrbracket^{g,x} = \llbracket t_2 \rrbracket^{g,x}\),
and \(\mathcal{M}, g \models_x F^n_j(t_1, \dots, t_n)\) iff
\(\langle \llbracket t_1 \rrbracket^{g,x}, \dots, \llbracket t_n \rrbracket^{g,x} \rangle \in V(F^n_j, x)\).
The more complex clauses above are needed only to handle the case where
some name occurring in the formula lacks a denotation at \(x\) but is
guaranteed to acquire one.

This is a possibilist treatment of the universal quantifier, in contrast
to the actualist quantifiers discussed in Harrison-Trainor
(\citeproc{ref-HarrisonTrainor2019}{2019}). I'll return in
Section~\ref{sec-quant} to the reasons we are best off using possibilist
quantifiers, and the difficulties this raises for talking about just
what's true in a possibility.

\subsection*{Modal Operators}\label{modal-operators}
\addcontentsline{toc}{subsection}{Modal Operators}

The third extension will be the introduction of modal operators. Here
I'll follow Humberstone (\citeproc{ref-Humberstone1981a}{1981}) very
closely, save just that I'll have a plurality of modal operators rather
than just one. So I'll use these structures to define (as Holliday
(\citeproc{ref-Holliday2025}{2025}) does) multi-modal logics. But I'll
follow Humberstone, and not Holliday, in defining modal operators in
terms of accessibility relations \(R_i\) satisfying these three
conditions\footnote{I'm using the names for these that Holliday uses,
  which are more evocative than Humberstone's original names.}:

\begin{description}
\tightlist
\item[\textbf{UpR}:]
If \(x \leqslant x'\) and \(x' R_i y\), then \(x R_i y\).
\item[\textbf{RDown}:]
If \(x R_i y\) and \(y \leqslant y'\), then \(x R_i y'\).
\item[\textbf{RRef++}:]
If \(x R_i y\), then there exists \(x' \geqslant x\) such that for all
\(x'' \geqslant x'\), \(x'' R_i y\).
\end{description}

\textbf{UpR} says that if a refinement of \(x\) can access \(y\), then
\(x\) itself can already access \(y\): accessibility is not something
that can be gained by adding detail to the source. \textbf{RDown} is a
converse of this; it says that accessibility cannot be gained by adding
detail to the target. \textbf{RRef++} says that if \(x\) can access
\(y\), there is some refinement \(x'\) of \(x\) where it is settled that
\(x'\) can access \emph{y}. This last access can't be overturned by
further refinement of \(x'\).

Given these constraints, the truth conditions for the box operator is:

\begin{align*}
[\Box_i] \quad & \mathcal{M} \models_x \Box_i A \text{ iff } \forall y \, (x R_i y \Rightarrow \mathcal{M} \models_y A); \\
\end{align*}

This should be familiar: \(\Box_i A\) is true at \(x\) iff \(A\) is true
at every \(R_i\)-accessible possibility.

Humberstone treats \(\Diamond\) as a defined connective,
\(\Diamond_i A\) just means \(\neg \Box_i \neg A\), and I'll follow
suit. If we just spell out what it means for \(\neg \Box_i \neg A\) to
be true, we get the rule {[}\(\Diamond_i\){]}\textsubscript{Official}.
But if we impose the above three constraints on \(R_i\), we can see that
this is equivalent to the more familiar
{[}\(\Diamond_i\){]}\textsubscript{Simple}.

\begin{align*}
[\Diamond_i]_{\text{Official}} \quad & \mathcal{M} \models_x \Diamond_i A \text{ iff } \forall y \geqslant x \, \exists z \geqslant y \, \exists w \, (z R_i w \text{ and } \mathcal{M} \models_w A). \\
[\Diamond_i]_{\text{Simple}} \quad & \mathcal{M} \models_x \Diamond_i A \text{ iff } \exists y \, (x R_i y \Rightarrow \mathcal{M} \models_y A);
\end{align*}

If \(R_i\) obeys \textbf{RDown}, then
{[}\(\Diamond_i\){]}\textsubscript{Official} will imply
{[}\(\Diamond_i\){]}\textsubscript{Simple}. For \(\Diamond_i A\) to be
true at \(x\) according to {[}\(\Diamond_i\){]}\textsubscript{Official},
there must be some refinement which can access a point where \(A\) is
true, and so by \textbf{RDown}, \(x\) itself can access that point. If
\(R_i\) obeys \textbf{RRef++}, then
{[}\(\Diamond_i\){]}\textsubscript{Simple} will imply
{[}\(\Diamond_i\){]}\textsubscript{Official}. If there is some \(y\)
such that \(xR_iy\) and \(A\) is true at \(y\), then by \textbf{RRef++},
there is some refinement of \(x\) such that every refinement of it can
access \(y\), and hence can access a point where \(A\) is true. So these
are equivalent.

From now on, I'll use {[}\(\Diamond_i\){]}\textsubscript{Simple} when
working out what's true at points in particular models. But when we are
proving general facts about the language, it will help to remember that
\(\Diamond_i\) is a defined connective, so we don't need an extra part
of inductive arguments to cover it.

\subsection*{Modal Constraints}\label{modal-constraints}
\addcontentsline{toc}{subsection}{Modal Constraints}

Why should we impose the three constraints Humberstone proposes? It is
not hard to show that they guarantee that persistence and refinability
hold for sentences generated using these new modal connectives. At
least, it isn't hard as long as we remember that \(\Diamond\) is being
treated as defined, so the only new step in the inductive proofs
involves \(\Box\). And \textbf{UpR} guarantees persistence for \(\Box\)
sentences, while \textbf{RRef++} guarantees refinability.

But this is overkill. As Humberstone points out, we haven't used
\textbf{RDown} in the proof, so this doesn't explain why we'd impose
\textbf{RDown}. As Holliday points out {[}note to Claude, we need page
number for this{]} \textbf{UpR} is stronger than we need for
persistence. We could weaken it by making greater use of the fact that
\(A\) is persistent. All we need is that if \(x \leqslant x'\) and
\(xRy\) then there is some \(z \geqslant y\) such that \(xRy\). That
will guarantee the key fact if \(x'\) can access a world where \(A\),
then so can \(x\).

So we need other arguments for these constraints other than their role
in ensuring persistence and refinability. Humberstone offers two other
arguments here. One anticipates the multi-modal setting that is being
used here. It's that if we want to use this system for tense logic, then
we want \(R_i^{-1}\) to satisfy all the constraints, so if we impose
\textbf{UpR}, we should also impose \textbf{RDown}. This isn't
convincing on its own though. For one thing, as already noted, we might
not need \textbf{UpR}. For another, by these lights we should worry that
the system is incomplete because we haven't put in a converse of
\textbf{RRef++}.

The argument that Humberstone spends more time on, and which I think is
more compelling, comes from rethinking the relationship between \(R_i\)
and \(\Box_i\). It's very tempting to read those truth conditions as
being explanatory from right-to-left. On this way of thinking, facts
about which \(\Box_i\) sentences are true at a point are grounded in
which non-modal sentences are true and which \(R_i\) relations obtain.
But while this is tempting, it isn't compulsory.

We could instead take the modal facts as given, and ask what
accessibility relations must obtain to be consistent with them. The
process here is familiar from the construction of canonical models. We
take the sets of consistent sentences as given, and say \(s_1R_is_2\)
iff whenever \(\Box_i A \in s_1\), then \(A \in s_2\). Humberstone's
approach is, I think, similar. Start with the idea that some sentences
are true in some model \(\mathcal{M}\) at possibilities \(x\) and \(y\),
say \(xR_iy\) iff \(\mathcal{M} \models_y A\) whenever
\(\mathcal{M} \models_x \Box_i A\), and ask what constraints \(R_i\)
will thereby satisfy.

The answer is that, if all sentences are persistent and refinable, then
\(R_i\) will satisfy these three constraints. If
\(\mathcal{M} \models_y A\) whenever
\(\mathcal{M} \models_x' \Box_i A\), then by the persistence of
\(\Box_i\), we know that \(\mathcal{M} \models_y A\) whenever
\(\mathcal{M} \models_x \Box_i A\). A similar argument, using the
persistence of \(A\), justifies \textbf{RDown}. Finally, \textbf{RRef++}
follows from the fact that \(\Box_i A\) is refinable. If \(\Box_i A\) is
not true at \(x\), that means that it is determinately not true at some
refinement. So if \(\mathcal{M} \nmodels_x A\), there must be some
refinement \(x'\) such that for all further refinements \(x''\),
\(\mathcal{M} \nmodels_x'' A\). If \(\mathcal{M} \nmodels_x \Box_i A\),
then there must be some possibility \(y\) such that \(xR_iy\) and
\(\mathcal{M} \nmodels_y A\). So as long as \(x''R_iy\), refinability
will be satisfied. I don't see how to prove there isn't a weaker
condition that would also work, it's possible we could use the
refinability of A to find some weaker condition, but I don't quite see
how that would work. So I think \textbf{RRef++} also follows from this
way of thinking about accessibility.

In the next section I'll discuss what logics can be defined using frames
that satisfy all of these conditions.

\section{Logics Determinable on Humberstone Frames}\label{sec-proof}

Holliday (\citeproc{ref-Holliday2025}{2025, sec. 8.2}) raises an
interesting question. As well as the familiar Kripke frames most
commonly used as a semantics for modal logic, and the Humberstone frames
defined above, he introduces a class of `full possibility' frames, which
weaken some of Humberstone's constraints. It won't matter here exactly
what these weakenings are, but what does matter is that he shows that
using these weakened frames, we can determine logics which are not
determinable on any class of Kripke frames. To state this more
precisely, for any class of frames \(\mathsf{F}\), let
\(\mathrm{L}(\mathsf{F})\) be the set of sentences true at all points in
all models definable some member of \(\mathsf{F}\). Then let
\(\mathrm{ML}(\mathsf{F})\) be the set
\(\{\mathrm{L}(\mathsf{X}) : \mathsf{X} \subseteq \mathsf{F}\}\). That
is \(\mathrm{ML}(\mathsf{F})\) will be the class of logics which can be
determined using just \(\mathsf{F}\).

If we let \(\mathsf{K}\) denote the class of Kripke frames, and
\(\mathsf{FP}\) denote the class of full possibility frames, Holliday
(\citeproc{ref-Holliday2025}{2025, sec. 2.5}) constructs a very clever
argument to show that
\(\mathrm{ML}(\mathsf{K}) \subsetneq \mathrm{ML}(\mathsf{FP})\). But, if
we let \(\mathsf{H}\) denote the class of Humberstone frames, he notes
that while the fact that every Kripke frame is a Humberstone frame and
every Humberstone frame is a full possibility frame,
\(\mathrm{ML}(\mathsf{K}) \subseteq \mathrm{ML}(\mathsf{H}) \subseteq \mathrm{ML}(\mathsf{FP})\),
and while
\(\mathrm{ML}(\mathsf{K}) \subsetneq \mathrm{ML}(\mathsf{FP})\) implies
that one of those inclusions is strict, it isn't clear which one. He
leaves the question of which one it is, and of course it could be both,
as an open question.

I don't have an answer to that question as asked, since it is asked
about languages with sentences with finite lengths. I do have a proof
that if we allow infinite disjunction, as discussed above, then
\(\mathrm{ML}(\mathsf{H}) \neq \mathrm{ML}(\mathsf{K})\). If we expand
the language like that, at least the first inequality is strict. I will
show this by constructing a single Humberstone frame that, in the
infinitary language, defines a logic with no Kripke equivalent. The
construction will follow Holliday's construction very closely, but
differ just enough to ensure compliance with Humberstone's conditions.

\subsection*{The Frame}\label{the-frame}
\addcontentsline{toc}{subsection}{The Frame}

The frame is built from two copies of the set of finite binary
sequences---sequences of 0s and 1s of any finite length, including the
empty sequence. Call one copy the \textbf{left-handed} sequences and the
other the \textbf{right-handed} sequences. The refinement relation is:
\(x \leqslant y\) iff \(x\) and \(y\) have the same handedness and \(x\)
is an initial segment of \(y\). So within each copy the frame is just
the binary tree ordered by extension, and no left-handed sequence
refines a right-handed sequence or vice versa. It will help to have some
notation for referring to points in this model. When \(s\) is a finite
binary sequence, I'll write \(s^L\) for the left-handed version of
\(s\), and \(s^R\) for the right-handed version.

\subsection*{The Accessibility
Relations}\label{the-accessibility-relations}
\addcontentsline{toc}{subsection}{The Accessibility Relations}

Next I'll define an accessibility relation and a separate infinite
family of accessibility relations. The single relation, which I'll write
\(R^{\rightarrow}\) is such that \(xR^{\rightarrow}y\) iff \(x\) is
left-handed and \(y\) is right-handed. The family of relations, each of
which I'll write as \(R^{\leftarrow}_i\), for \(i \in \mathbb{N}\), is
such that \(xR^{\leftarrow}_iy\) iff \(x\) is right-handed, \(x\) does
not have a \(0\) in its \(i\)-th position (either because \(x\) has
length less than \(i\), or because it has a \(1\) in position \(i\)),
and \(y\) is left-handed.

That \(R^{\rightarrow}\) satisfies \textbf{UpR}, \textbf{RDown}, and
\textbf{RRef++} is obvious. It is also obvious that for each \(i\),
\(R^{\leftarrow}_i\) satisfies \textbf{UpR} and \textbf{RDown}. It's
only a little harder to show that it satisfies \textbf{RRef++}. Assume
\(xR^{\leftarrow}_iy\). So \(x\) is right-handed, and \(y\) is
left-handed. If \(x\) is of length at least \(i\), then \(x\) itself can
be the refinement such that every further refinement can access \(y\).
If \(x\)'s length is less than \(i\), extend \(x\) with enough 1s to
create a sequence of length \(i\). The result will be a refinement such
that every refinement can access \(y\), as required.

\subsection*{The Example}\label{the-example}
\addcontentsline{toc}{subsection}{The Example}

Now consider the proposition (\ref{Splitting}); a minor variant on
Holliday's example (\textsc{Split}). (I'm using \(\mathsf{T}\) for an
arbitrary tautology.)

\begin{equation}
\neg \Diamond^{\rightarrow} p 
  \vee
\bigvee_{i \in \mathbb{N}}(
  \Diamond^{\rightarrow}
    (
      p
        \wedge
      \Diamond_i^{\leftarrow} \mathsf{T}
    )
  \wedge
  \Diamond^{\rightarrow}
    (
      p
        \wedge
      \neg \Diamond_i^{\leftarrow} \mathsf{T}
    )
  )
\tag{\textsc{Splitting}}
\label{Splitting}
\end{equation}

I'm going to make three claims about (\ref{Splitting}). First, it is
true throughout the frame I just described. Second,
\(\neg \Diamond^{rightarrow} p\) is not true on all models on that
frame. Third, there is no class of Kripke frames throughout which
(\ref{Splitting}) is always true and \(\neg \Diamond^{rightarrow} p\) is
not always true. From this it follows that
\(\mathrm{ML}(\mathsf{K}) \neq \mathrm{ML}(\mathsf{K})\).

For the first claim, I'll show something slightly stronger, namely that
at each point one or other disjunct in (\ref{Splitting}) is true. If the
point is right-handed, then the first disjunct is true, since each
right-handed point is a dead-end with respect to \(R^{\rightarrow}\). So
we just have to look at the left-handed points. Let \(x\) be an
arbitrary left-handed point. If there is no \(y\) such that
\(xR^{\rightarrow}y\) and \(p\) is true at \(y\), then again the first
disjunct is true.

Now consider the cases where there are points \(y\) such that
\(xR^{\rightarrow}y\) and \(p\) is true at \(y\). Here is will be
helpful to write \(x\) as \(s_x^L\), and \(y\) as \(s_y^R\). Consider
those \(s_y^R\), for which \(s_x^LR^{\rightarrow}s_y^R\), where
\textbar s\_y\^{}R\textbar, the length of \textbar s\_y\^{}R\textbar, is
minimal. There must be some such points, since sequence lengths are
positive integers. Let \$i = \textbar s\_y\^{}R\textbar{} + 1. Let
\(s_y^R \oplus \langle 0 \rangle\) and
\(s_y^R \oplus \langle 1 \rangle\) be the right-handed sequences
generated by adding either a 0 or a 1, respectively, to \(s_y\). Since
\(p\) is true at \(s_y^R\) and truth is persistent, \(p\) will be true
at both \(s_y^R \oplus \langle 0 \rangle\) and
\(s_y^R \oplus \langle 1 \rangle\). Since
\(s_y^R \oplus \langle 0 \rangle\) has a 0 at it's \(i\)'th position, it
is a dead-end with respect to \(R^{\leftarrow}_i\). So
\(\neg \Diamond_i^{\leftarrow} \mathsf{T}\) is true there. But since
\(R^{\rightarrow}\) satisfies \textbf{RDown},
\(s_x^LR^{\rightarrow}s_y^R\oplus \langle 0 \rangle\). So
\(\Diamond^{\rightarrow}(p \wedge \neg \Diamond_i^{\leftarrow} \mathsf{T})\)
is true at \(s_x^L\). A similar argument shows that
\(p \wedge \Diamond_i^{\leftarrow} \mathsf{T}\) is true at
\(s_y^R\oplus \langle 1 \rangle\), so
\(\Diamond^{\rightarrow}(p \wedge \Diamond_i^{\leftarrow} \mathsf{T})\)
is true at \(s_x^L\). And that implies that the right-hand disjunct of
(\ref{Splitting}) is true at \(s_x^L\), as required.

The second point, that \(\neg \Diamond^{rightarrow} p\) is not true on
all models on the frame, is trivial, since that will fail at some
left-handed points whenever \(p\) is true at some right-handed point.

For the third point, we follow Holliday's argument particularly closely.
Consider any class of Kripke frames such that
\(\neg \Diamond^{\rightarrow} p\) is not valid on that class. Look at
the class of models of those frames where \(p\) is true at exactly one
world. For any disjunct of the right-hand disjunct of (\ref{Splitting})
to be true at a point, it must access a world where \(p\) is true that's
a dead-end with respect to \(R^{\leftarrow}_i\), and a world where \(p\)
is true that's not a dead-end with respect to \(R^{\leftarrow}_i\).
That's impossible if \(p\) is true at just one world. So throughout this
class of models, that right-hand disjunct of (\ref{Splitting}) will be
false. But if \(\neg \Diamond^{\rightarrow} p\) is not valid on the
frame, there will also be points in this class of models where
\(\Diamond^{\rightarrow} p\) is true. So the disjunction will not be
false somewhere, and hence not valid on the frame.

Putting these together, there is no class of Kripke frames which
validate exactly those the sentences valid on this particular
Humberstone frame. So
\(\mathrm{ML}(\mathsf{K}) \neq \mathrm{ML}(\mathsf{H})\), and hence
\(\mathrm{ML}(\mathsf{H}) \subsetneq \mathrm{ML}(\mathsf{K})\).

\section{Quantifiers and Necessitism}\label{sec-quant}

\section{Conditionals}\label{sec-conditionals}

The only discussion of possibilities (as opposed to worlds) in \emph{The
Connectives} is in the chapter on disjunction. But there are several
potential connections to conditionals, and in this section I'll go over
a couple of them.

One connection concerns Conditional Excluded Middle (as discussed on
pages 1008-1013 of \emph{The Connectives}), and more generally the
relationship between \(A > (B \vee C)\) and
\((A \boxright B) \vee (A \boxright C)\). On a Stalnaker-Lewis style
approach to conditionals, these are equivalent iff there is a nearest
possible world in which \(A\) is true, for any possble \(A\). It is
natural to think about whether that is true in largely metaphysical
terms, asking whether there really is guaranteed to be a single nearest
world where \(A\) is true. And as Lewis
(\citeproc{ref-Lewis1973a}{1973}) argued, it is natural to answer that
question negatively.

To take one striking example of that, consider the discussion by Jeremy
Goodman (\citeproc{ref-Goodmanms}{2018}) of the example, originally due
to Max Black (\citeproc{ref-Black1952a}{1952}), of the two spheres alone
in space. Black says that the spheres are really two, so this is a
counterexample to the Principle of Identity of Indiscernibles. Let's
assume, for now, that Black is right, and we can call one sphere \(a\)
and the other sphere \(b\). Goodman asks what we should say about the
counterfactual possibility that one of the spheres is heavier. On
Lewis's picture, rejecting Conditional Excluded Middle, both (1) and (2)
are false.

\begin{enumerate}
\def\labelenumi{(\arabic{enumi})}
\tightlist
\item
  If one of the spheres were heavier, it would be \(a\).
\item
  If one of the spheres were heavier, it would be \(b\).
\end{enumerate}

A common thought at this point is that this verdict really does follow
from Lewis's `nearest possible world' semantics for conditionals, but
that data about the inferential role of conditionals shows that
Conditional Excluded Middle must be correct.\footnote{For a recent
  statement of this last view, with many more citations to similar
  statements, see Cariani and Goldstein
  (\citeproc{ref-CarianiGoldstein2020}{2020}).} This is, many think, a
problem for the Lewisian view.

One move here, discussed by Humberstone
(\citeproc{ref-Humberstone2011}{2011, 1011}), is to use supervaluations.
Perhaps it is in some sense indeterminate whether the world where \(a\)
is heavier or the world where \(b\) is heavier is more like actuality. A
related, but I think, preferable, move is to analyse conditionals not in
terms of possible worlds, but in terms of possibilities.

Here is one possible way to analyse conditionals, mixing Stalnaker's
approach with Humberstone's possibilities. (The particular formulation
I'm going to use draws heavily on the theory presented by Andrew Bacon
(\citeproc{ref-Bacon2023}{2023, 382}). The four conditions are directly
quotes from his paper, though I mean something different by them since
on my version the variables pick out possibilities not worlds. I'll have
more to say about Bacon's paper presently.) Extend a possibility model
\(\langle W, \geqslant, V \rangle\) to a conditional possibility model
by adding a selection function \(f\). This is a function
\(\mathcal{P}(W) \times W \rightarrow \mathcal{P}(W)\), intuitively
picking out the `nearest' possibilities to a world where a particular
proposition is true, satisfying these constraints.\footnote{Humberstone
  uses \(R\) rather than \emph{f} for the same idea; I'm going to follow
  Bacon, who in turn follows Stalnaker, to highlight the connection to
  theories which validate Conditional Excluded Middle.}

\begin{description}
\tightlist
\item[\textbf{MP}]
\(x \in f(A, x)\) whenever \(x \in A\)
\item[\textbf{ID}]
\(f(A, x) \subseteq A\)
\item[\textbf{CEM}]
\(|f(A, x)| \leq 1\)
\item[\textbf{AB}]
If \(f(A, x) \subseteq B\) and \(f(B, x) = \emptyset\) then
\(f(A, x) = \emptyset\)
\end{description}

The cardinality constraint \textbf{CEM} guarantees that Conditional
Excluded Middle will hold. But we don't have to make invidious choices
about whether the nearest possibility where one of the spheres is
heavier makes \(a\) or \(b\) heavier. Rather, we just say that the
nearest possibility is an unrefined possibility that makes the
disjunctive proposition \emph{One of them is heavier} true, without
making either disjunct true. It will have refinements where each is
true, but the nearest possibility will not validate either disjunct.
This seems like an intuitive treatment of the case.

Goodman uses this example to argue that Black was incorrect, and the
spheres are in fact discernible. His argument is that one but not the
other will have the property of being the one which would be heavier if
they were different. I don't think this argument goes through on the
possibilities framework, but settling that would require saying more
about how higher-order quantification works on the possibilities
framework, and that would take us too far afield. Instead I'll turn to
the puzzle that Bacon introduces them to solve. That is a puzzle,
introduced by Kit Fine (\citeproc{ref-Fine2012a}{2012b},
\citeproc{ref-Fine2012b}{2012a}), which is a counterfactual version of a
paradox from José Bernadete (\citeproc{ref-Bernadete1964}{1964}).

There is a room that is very dangerous to cross. A man is thinking of
crossing it, but he is warned off when he learns that it contains an
infinity of gods. God\textsubscript{1} will kill him if he makes it
half-way across the room. God\textsubscript{2} will kill him if he makes
it a quarter of the way across, \textsubscript{3} will kill him if he
makes it one-eighth of the way across. More generally,
God\textsubscript{\emph{n}} will kill him if he makes it
(1/2)\textsuperscript{\emph{n}} of the way across the room. Does he
enter? Of course not; he'd be killed! But who would kill him? Presumably
not God\textsubscript{1}; how would he make it that far? This
generalises. God\textsubscript{\emph{n}} can't kill him, because
God\textsubscript{\emph{n}+1} would already have done the job. So he
would be killed by the gods, but not by any God. This doesn't sound very
plausible.

The case looks like the kind that motivated Lewis to reject what he
called the \emph{Limit Assumption}. This says that if \(A\) is possible,
then relative to any world \(w\) there are some closest worlds where
\(A\) is true. Humberstone (\citeproc{ref-Humberstone2011}{2011,
1014--15}) discusses Lewis's rejection of the Limit Assumption, and
adopts the position that we shouldn't impose it in general, but can
freely talk as if it is true, because it doesn't make a difference to
the logic. This is right in the context Humberstone is writing in, but
possibly misleading. The Limit Assumption does make a big difference to
the logic if we have either quantifiers or infinitary connectives in the
language. This fact is what Fine's puzzle turns on.

Stated without the Limit Assumption, Lewis's view is that
\(A \boxright B\) is true at \(w\) if there is some world where \(A\) is
true such that there is no closer world where \(A \wedge \neg B\) is
true. If we assume that for any \emph{n}, the world where the man enters
and God\textsubscript{\emph{n}+1} kills him is closer than the world
where he enters and God\textsubscript{\emph{n}} kills him, then Lewis is
committed to both (3) and (4).

\begin{enumerate}
\def\labelenumi{(\arabic{enumi})}
\setcounter{enumi}{2}
\tightlist
\item
  If the man were to enter the room, he would be killed by either
  God\textsubscript{1} or God\textsubscript{2} or \(\dots\).
\item
  For each \emph{n}, it is not the case that if the man were to enter
  the room, he would be killed by God\textsubscript{\emph{n}}.
\end{enumerate}

As Fine notes, Lewis's theory of counterfactuals is committed to denying
a principle he calls \textbf{Infinite Conjunction}.

\begin{description}
\tightlist
\item[Infinite Conjunction]
If \(A \boxright C_i\) is true for each \(i\), then
\(A \boxright (C_1 \wedge C_2 \wedge \dots)\) is true.
\end{description}

Without the Limit Assumption, Lewis's semantics would endorse
\textbf{Infinite Conjunction}. But it would also have a problem. Which
of the gods would kill the man? Any choice seems not only arbitrary, but
mistaken. Let's spell this out a bit more carefully. Fine's way of
spelling out the paradox makes heavy use of a principle he calls
\textbf{Disjunction}. (This is called \textbf{Subj. Dilemma} by
Humberstone (\citeproc{ref-Humberstone2011}{2011, 1015}).)

\begin{description}
\tightlist
\item[Disjunction]
If \(A \boxright C\) and \(B \boxright C\) are true, so is
\((A \vee B) \boxright C\).
\end{description}

Then both (5) and (6) seem like they should be true.

\begin{enumerate}
\def\labelenumi{(\arabic{enumi})}
\setcounter{enumi}{4}
\tightlist
\item
  If the man had entered the room, and been killed by one of
  God\textsubscript{1} through God\textsubscript{\emph{k}+1}, he
  wouldn't have been killed by God\textsubscript{\emph{k}} (because
  God\textsubscript{\emph{k}+1} would have killed him first).
\item
  If the man had entered the room, and not been killed by one of
  God\textsubscript{1} through God\textsubscript{\emph{k}+1}, he
  wouldn't have been killed by God\textsubscript{\emph{k}}.
\end{enumerate}

Putting these together using \textbf{Disjunction}, we get the following
sentence. It's a mouthful, but it's important to spell it out for what
comes next.

\begin{enumerate}
\def\labelenumi{(\arabic{enumi})}
\setcounter{enumi}{6}
\tightlist
\item
  If either the man had entered the room, and been killed by one of
  God\textsubscript{1} through God\textsubscript{\emph{k}+1}, or he had
  entered the room, and not been killed by one of God\textsubscript{1}
  through God\textsubscript{\emph{k}+1}, he wouldn't have been killed by
  God\textsubscript{\emph{k}}.
\end{enumerate}

To finish off the paradox, let's add a new principle \textbf{Antecedent
Substitution}.

\begin{description}
\tightlist
\item[Antecedent Substitution]
If \(A\) and \(B\) are provably equivalent in classical logic, and
\(A \boxright C\) is true, so is \(B \boxright C\).
\end{description}

Then using \textbf{Antecedent Substitution} we can get from (7) to (8).

\begin{enumerate}
\def\labelenumi{(\arabic{enumi})}
\setcounter{enumi}{7}
\tightlist
\item
  If the man had entered the room, he would not have been killed by
  God\textsubscript{\emph{k}}.
\end{enumerate}

And since \emph{k} is arbitrary in (8), we can derive (4), without any
appeal to the metaphysics of counterfactuals. It looks like our only
options, short of abandoning classical logic, are to give up one
\textbf{Infinite Conjunction}, \textbf{Disjunction}, or
\textbf{Antecedent Substitution}. In Fine's original discussion of the
paradox he introduces several more principles that could in theory be
given up, but Brian Embry (\citeproc{ref-Embry2014}{2014}) convincingly
argues that really it has to be one of these three that go, and I'm
following his lead.

If these are the three options, it isn't obvious which one to take. All
three paths forward have their proponents, or at least are consequences
of otherwise plausible views. I've already noted that Lewis
(\citeproc{ref-Lewis1973a}{1973}) is committed to rejecting
\textbf{Infinite Conjunction}, because he does not endorse the Limit
Assumption. This does not look great; it's crucial to reasoning with
counterfactuals that if some things would each be true were \(A\) the
case, then were \(A\) the case they would each be true.

Fine (\citeproc{ref-Fine2012a}{2012b}) recommends giving up
\textbf{Antecedent Substitution}. He develops a theory of conditionals
that doesn't use possible worlds, but instead uses incomplete states.
These are not entirely unlike Humberstone's possibilities, but the
resulting theory is quite different. I suspect the key distinction, the
one that drives all of the rest of the results, is that Fine takes
disjunctions to be true at a state only if a disjunct is true at that
state. Anyway, Fine thinks that the misstep in the trilemma above is the
derivation of (8) from (7). That step requires substituting \(A\) for
\((A \wedge B) \vee (A \wedge \neg B)\) in an antecedent, which Fine
takes to be illegitimate.

There are a couple of reasons to be unhappy with this way of getting out
of the problem. One is that this did not feel like the most
controversial step when we were developing the problem. But a bigger one
is that Fine's resolution of the trilemma ends up endorsing not just
\textbf{Disjunction}, but also its converse. This is the principle that
Humberstone (\citeproc{ref-Humberstone2011}{2011, 1016}) calls
\emph{Conv. Subj. Dilemma}, the key being that from
\((A \vee B) \boxright C)\) one can infer \(A \boxright C\). The
criticisms of Fine in Embry (\citeproc{ref-Embry2014}{2014}) largely
centre on this aspect of Fine's view, and its consequences. But the key
problems with the principle are already pretty clear in Humberstone
(\citeproc{ref-Humberstone2011}{2011, 1016--22}). So I think we
shouldn't go that way.

So that leaves \textbf{Disjunction}. This is the step that Andrew Bacon
(\citeproc{ref-Bacon2023}{2023}) rejects, and it's what I'll reject as
well. I think there are two key reasons to worry about giving up
\textbf{Disjunction}, and they are pretty hard worries to address in the
possible worlds framework. But they both seem more or less manageable in
the possibilities framework.

The first worry is that without \textbf{Disjunction}, we have to give up
the idea that the selection function \emph{f} is in any sense a measure
of similarity, or really any kind of nearness. Here is how Bacon puts
it,

\begin{quote}
The second consideration in favour of \textbf{Disjunction} is that its
validity is predicted by the now dominant account of counterfactuals,
prominently defended by Lewis and Stalnaker, based on similarity
semantics. For roughly, if the closest \(A\) worlds are \(C\) worlds,
and the closest \(B\) worlds are \(C\) worlds, then the closest
\(A \vee B\) worlds are \(C\) worlds. (\citeproc{ref-Bacon2023}{Bacon
2023, 374})
\end{quote}

Now for Bacon, this isn't a worry, because he thinks there are
independent reasons to reject the picture of similarity or nearness as
being foundational to counterfactuals. I don't find those reasons
convincing, and basically agree with the response to them that Fine
(\citeproc{ref-Fine2023}{2023}) gives. But it's not just critics of
\textbf{Disjunction} who connect it to the similarity picture. The same
connection is drawn by Humberstone, who does endorse
\textbf{Disjunction}. He first notes that, by analogy with other maximal
relations, we should expect that the nearest \(A \vee B\)-world will be
either the nearest \(A\)-world or the nearest \(B\)-world
(\citeproc{ref-Humberstone2011}{Humberstone 2011, 1015}). He then argues
more formally that if the set of nearest \(A\)-worlds is generated by an
underlying three place similarity relation \(S_wxy\), meaning \(x\) is
at least as similar to \(w\) as \(y\) is, and if for any \(w, S_w\) is a
total preorder on worlds, then \textbf{Disjunction} is guaranteed to
hold (\citeproc{ref-Humberstone2011}{Humberstone 2011, 1025--26}).

That argument seems irrefutable if, \emph{but only if}, we're working in
a possible worlds framework. If we're in a possibilities framework, it
doesn't look right. It could be that the nearest possibility in which
\(A \vee B\) is not identical to either the nearest possibility in which
\(A\), or the nearest possibility in which \(B\), but is instead a
coarsening of one of those possibilities.

How could \textbf{Disjunction} fail on the possible worlds picture? We
must fail to have the nearest \(A \vee B\)-world be a \(C\)-world. But
that's impossible if the nearest \(A\)-world is a \(C\)-world, and the
nearest \(B\)-world is a \(C\)-world, and all \(A \vee B\)-worlds are
\(A\)-worlds or \(B\)-worlds. On the possibilities picture, that last
clause fails. It might be that the nearest possibility which makes
\(A \vee B\) true does not make either \(A\) true or \(B\) true, it just
guarantees that sequence of refinements will eventually make one or the
other true. Also note that we don't require that the nearest
\(A \vee B\) possibility makes \(\neg C\) true; it could be that
\textbf{Disjunction} fails because both \(C\) and \(\neg C\) are true at
different refinements of the nearest \(A \vee B\) possibility.

That's what happens with our man who wisely doesn't enter the room.
What's the nearest possibility in which he does enter the room? It's the
incomplete possibility where he is killed by one of the gods. For each
\emph{i, j: i \textgreater{} j}, the possibility where he is killed by
God\textsubscript{\emph{i}} is closer than the one where he is killed by
God\textsubscript{\emph{j}}. But the indeterminate possibility where he
is killed, but the possibility does not specify which god he is killed
by, is closer to actuality than any complete possibility which specifies
the homicidal divinity.

So rejecting \textbf{Disjunction} is compatible with the similarity
approach to counterfactuals, as long as we use possibilities. The other
worry with rejecting disjunction, one Fine
(\citeproc{ref-Fine2023}{2023}) stresses in his response to Bacon, is
that we use \textbf{Disjunction} a lot in ordinary counterfactual
reasoning. We should be cautious about giving it up. Of course, there
are plenty of rules that we use in ordinary reasoning that work in all
but a few edge cases. If we could show that \textbf{Disjunction} was
truth-preserving in all but some rare exception cases, we could justify
using it as a rule of inference. After all, we don't think that the
failure of Axiom V in full generality means that it's always a mistake
to infer from the existence of some things to the existence of a set
containing all and only them.

As Fine stresses, it's hard to see how on Bacon's view
\textbf{Disjunction} would even count as typically fine. The reason
Bacon says that it fails in the paradoxical case from Bernadete
generalises to more humdrum uses.

But that's not true for the possibilities model. You need to have some
very unusual relationships between Humberstone's family of similarity
relations \(S_w\) and \(\geqslant\) in order for \textbf{Disjunction} to
fail. For one thing, you need the nearest possibility \(x\) where
\(A \vee B\) holds to be an incomplete possibility where neither \(A\)
nor \(B\) holds. Clearly \(x\) has to have refinements where \(A\) is
true, and refinements where \(B\) is true. To get a \textbf{Disjunction}
failure, you need one of the refinements of \(x\) where one of the
disjuncts is true to not be (a refinement of) the closest possibility to
actuality where that disjunct obtains. I can't construct an intuitive
case where that happens that doesn't involve infinite sequences like in
Bernadete's case, though I also don't have a proof that no such case can
be constructed. This is all far from conclusive, but it seems plausible
that on the possibilities model, failures of \textbf{Disjunction} will
be rare. And that would be enough to explain the fact Fine appeals to,
that we are usually happy using it in everyday inferences.

\section{Conclusion}\label{sec-conclusion}

I've gone over two more uses of the possibilties framework, but there
are many more things that we could imagine using it for. I'll end by
briefly mentioning two of them.

As I briefly alluded to in Section~\ref{sec-conditionals}, possibilities
can do a lot of work that philosophers have tried to make
supervaluations do. As well as using possibilities instead of
supervaluations in preserving Conditional Excluded Middle, it's worth
exploring whether they are useful in thinking about vagueness, or about
open future.

Humberstone (\citeproc{ref-Humberstone1981a}{1981}) mentions that
possibilities do a better job than possible worlds at making sense of
talk about `belief worlds'. We could say the same thing about the worlds
of fiction. David Lewis (\citeproc{ref-Lewis1978b}{1978}) ends up
treating the operator \emph{In this fiction} as box-like, because he
thinks otherwise we'd have to make arbitrary choices about some details
of how the story is to be filled out. Using possibilities here seems
smoother. Any (coherent) fiction, I conjecture, picks out a particular
possibility. That will always be a less than fully refined possibility,
but a possibility nonetheless. On this approach we don't get left with
the unfortunate triple, which Lewis is committed to, of it being true in
the story that \(A \vee B\), but false that it's true in the story that
\(A\), and false that it's true in the story that \(B\). I also think
there might be some uses of possibilities in characterising the
distinctive relationship between a story and its sequel.

But I'll leave those tasks for another day. The main point of this paper
is to remind the reader how many uses Humberstone's notion of a
possibility has, and to explore what happens to the logic of
possibilities when we add quantifiers or infinitary connectives.

\section*{References}\label{references}
\addcontentsline{toc}{section}{References}

\phantomsection\label{refs}
\begin{CSLReferences}{1}{0}
\bibitem[\citeproctext]{ref-Bacon2023}
Bacon, Andrew. 2023. {``Counterfactuals, Infinity and Paradox.''} In
\emph{Kit Fine on Truthmakers, Relevance, and Non-Classical Logic},
edited by Federico L. G. Faroldi and Frederik Van De Putte, 349--88.
Cham: Springer International Publishing.
https://doi.org/10.1007/978-3-031-29415-0\_17.

\bibitem[\citeproctext]{ref-Bernadete1964}
Bernadete, Jose. 1964. \emph{Infinity: An Essay in Metaphysics}. Oxford:
Clarendon Press.

\bibitem[\citeproctext]{ref-Black1952a}
Black, Max. 1952. {``The Identity of Indiscernibles.''} \emph{Mind} 61
(242): 153--64. https://doi.org/10.1093/mind/LXI.242.153.

\bibitem[\citeproctext]{ref-CarianiGoldstein2020}
Cariani, Fabrizio, and Simon Goldstein. 2020. {``Conditional
Heresies.''} \emph{Philosophy and Phenomenological Research} 101 (2):
251--82. https://doi.org/doi.org/10.1111/phpr.12565.

\bibitem[\citeproctext]{ref-ConanDoyle1995}
Conan Doyle, Arthur. 1995. \emph{A Study in Scarlet}. Urbana, Illinois:
Project Gutenberg.

\bibitem[\citeproctext]{ref-Embry2014}
Embry, Brian. 2014. {``Counterfactuals Without Possible Worlds? A
Difficulty for Fine?s Exact Semantics for Counterfactuals.''}
\emph{Journal of Philosophy}, no. 5: 276--87.
https://doi.org/10.5840/jphil2014111522.

\bibitem[\citeproctext]{ref-Fine2012b}
Fine, Kit. 2012a. {``A Difficulty for the Possible Worlds Analysis of
Counterfactuals.''} \emph{Synthese} 189 (1).
https://doi.org/10.1007/s11229-012-0094-y.

\bibitem[\citeproctext]{ref-Fine2012a}
---------. 2012b. {``Counterfactuals Without Possible Worlds.''}
\emph{Journal of Philosophy} 109 (3): 221--46.
https://doi.org/10.5840/jphil201210938.

\bibitem[\citeproctext]{ref-Fine2023}
---------. 2023. {``{`Defense of a Truthmaker Approach to
Counterfactuals'}: Response to Andrew Bacon's {`Counterfactuals,
Infinity and Paradox'}.''} In \emph{Kit Fine on Truthmakers, Relevance,
and Non-Classical Logic}, edited by Federico L. G. Faroldi and Frederik
Van De Putte, 389--406. Cham: Springer International Publishing.
https://doi.org/10.1007/978-3-031-29415-0\_18.

\bibitem[\citeproctext]{ref-Goodmanms}
Goodman, Jeremy. 2018. {``Consequences of Conditional Excluded
Middle.''}

\bibitem[\citeproctext]{ref-HarrisonTrainor2019}
Harrison-Trainor, Matthew. 2019. {``First-Order Possibility Models and
Finitary Completeness Proofs.''} \emph{Review of Symbolic Logic} 12 (4):
637--62. https://doi.org/10.1017/S1755020319000418.

\bibitem[\citeproctext]{ref-Holliday2025}
Holliday, Wesley H. 2025. {``Possibility Frames and Forcing for Modal
Logic.''} \emph{Australasian Journal of Logic} 22 (2): 44--288.
https://doi.org/10.26686/ajl.v22i2.5680.

\bibitem[\citeproctext]{ref-Humberstone1981a}
Humberstone, Lloyd. 1981. {``From Worlds to Possibilities.''}
\emph{Journal of Philosophical Logic} 10 (3): 313--39.
https://doi.org/10.1007/BF00293423.

\bibitem[\citeproctext]{ref-Humberstone2011}
---------. 2011. \emph{The Connectives}. Cambridge, MA: MIT Press.

\bibitem[\citeproctext]{ref-Lewis1973a}
Lewis, David. 1973. \emph{Counterfactuals}. Oxford: Blackwell
Publishers.

\bibitem[\citeproctext]{ref-Lewis1978b}
---------. 1978. {``Truth in Fiction.''} \emph{American Philosophical
Quarterly} 15 (1): 37--46.

\end{CSLReferences}




\end{document}
