% Options for packages loaded elsewhere
% Options for packages loaded elsewhere
\PassOptionsToPackage{unicode}{hyperref}
\PassOptionsToPackage{hyphens}{url}
\PassOptionsToPackage{dvipsnames,svgnames,x11names}{xcolor}
%
\documentclass[
  10.5pt,
  twoside]{article}
\usepackage{xcolor}
\usepackage[paperheight=10in,,paperwidth=7in,,top=1in,bottom=1in,inner=0.8in,outer=0.8in,headsep=0.25in,headheight=1in,footskip=0.35in]{geometry}
\usepackage{amsmath,amssymb}
\setcounter{secnumdepth}{-\maxdimen} % remove section numbering
\usepackage{iftex}
\ifPDFTeX
  \usepackage[T1]{fontenc}
  \usepackage[utf8]{inputenc}
  \usepackage{textcomp} % provide euro and other symbols
\else % if luatex or xetex
  \usepackage{unicode-math} % this also loads fontspec
  \defaultfontfeatures{Scale=MatchLowercase}
  \defaultfontfeatures[\rmfamily]{Ligatures=TeX,Scale=1}
\fi
\usepackage{lmodern}
\ifPDFTeX\else
  % xetex/luatex font selection
  \setmainfont[ItalicFont=EB Garamond Italic,BoldFont=EB Garamond
SemiBold]{EB Garamond Math}
  \setsansfont[]{EB Garamond SemiBold}
  \setmathfont[]{EB Garamond Math}
\fi
% Use upquote if available, for straight quotes in verbatim environments
\IfFileExists{upquote.sty}{\usepackage{upquote}}{}
\IfFileExists{microtype.sty}{% use microtype if available
  \usepackage[]{microtype}
  \UseMicrotypeSet[protrusion]{basicmath} % disable protrusion for tt fonts
}{}
\usepackage{setspace}
\makeatletter
\@ifundefined{KOMAClassName}{% if non-KOMA class
  \IfFileExists{parskip.sty}{%
    \usepackage{parskip}
  }{% else
    \setlength{\parindent}{0pt}
    \setlength{\parskip}{6pt plus 2pt minus 1pt}}
}{% if KOMA class
  \KOMAoptions{parskip=half}}
\makeatother
% Make \paragraph and \subparagraph free-standing
\makeatletter
\ifx\paragraph\undefined\else
  \let\oldparagraph\paragraph
  \renewcommand{\paragraph}{
    \@ifstar
      \xxxParagraphStar
      \xxxParagraphNoStar
  }
  \newcommand{\xxxParagraphStar}[1]{\oldparagraph*{#1}\mbox{}}
  \newcommand{\xxxParagraphNoStar}[1]{\oldparagraph{#1}\mbox{}}
\fi
\ifx\subparagraph\undefined\else
  \let\oldsubparagraph\subparagraph
  \renewcommand{\subparagraph}{
    \@ifstar
      \xxxSubParagraphStar
      \xxxSubParagraphNoStar
  }
  \newcommand{\xxxSubParagraphStar}[1]{\oldsubparagraph*{#1}\mbox{}}
  \newcommand{\xxxSubParagraphNoStar}[1]{\oldsubparagraph{#1}\mbox{}}
\fi
\makeatother


\usepackage{longtable,booktabs,array}
\usepackage{calc} % for calculating minipage widths
% Correct order of tables after \paragraph or \subparagraph
\usepackage{etoolbox}
\makeatletter
\patchcmd\longtable{\par}{\if@noskipsec\mbox{}\fi\par}{}{}
\makeatother
% Allow footnotes in longtable head/foot
\IfFileExists{footnotehyper.sty}{\usepackage{footnotehyper}}{\usepackage{footnote}}
\makesavenoteenv{longtable}
\usepackage{graphicx}
\makeatletter
\newsavebox\pandoc@box
\newcommand*\pandocbounded[1]{% scales image to fit in text height/width
  \sbox\pandoc@box{#1}%
  \Gscale@div\@tempa{\textheight}{\dimexpr\ht\pandoc@box+\dp\pandoc@box\relax}%
  \Gscale@div\@tempb{\linewidth}{\wd\pandoc@box}%
  \ifdim\@tempb\p@<\@tempa\p@\let\@tempa\@tempb\fi% select the smaller of both
  \ifdim\@tempa\p@<\p@\scalebox{\@tempa}{\usebox\pandoc@box}%
  \else\usebox{\pandoc@box}%
  \fi%
}
% Set default figure placement to htbp
\def\fps@figure{htbp}
\makeatother


% definitions for citeproc citations
\NewDocumentCommand\citeproctext{}{}
\NewDocumentCommand\citeproc{mm}{%
  \begingroup\def\citeproctext{#2}\cite{#1}\endgroup}
\makeatletter
 % allow citations to break across lines
 \let\@cite@ofmt\@firstofone
 % avoid brackets around text for \cite:
 \def\@biblabel#1{}
 \def\@cite#1#2{{#1\if@tempswa , #2\fi}}
\makeatother
\newlength{\cslhangindent}
\setlength{\cslhangindent}{1.5em}
\newlength{\csllabelwidth}
\setlength{\csllabelwidth}{3em}
\newenvironment{CSLReferences}[2] % #1 hanging-indent, #2 entry-spacing
 {\begin{list}{}{%
  \setlength{\itemindent}{0pt}
  \setlength{\leftmargin}{0pt}
  \setlength{\parsep}{0pt}
  % turn on hanging indent if param 1 is 1
  \ifodd #1
   \setlength{\leftmargin}{\cslhangindent}
   \setlength{\itemindent}{-1\cslhangindent}
  \fi
  % set entry spacing
  \setlength{\itemsep}{#2\baselineskip}}}
 {\end{list}}
\usepackage{calc}
\newcommand{\CSLBlock}[1]{\hfill\break\parbox[t]{\linewidth}{\strut\ignorespaces#1\strut}}
\newcommand{\CSLLeftMargin}[1]{\parbox[t]{\csllabelwidth}{\strut#1\strut}}
\newcommand{\CSLRightInline}[1]{\parbox[t]{\linewidth - \csllabelwidth}{\strut#1\strut}}
\newcommand{\CSLIndent}[1]{\hspace{\cslhangindent}#1}



\setlength{\emergencystretch}{3em} % prevent overfull lines

\providecommand{\tightlist}{%
  \setlength{\itemsep}{0pt}\setlength{\parskip}{0pt}}



 


% Custom title page mimicking Ergo class style
% Add this to your Quarto YAML with:
% format:
%   pdf:
%     include-in-header:
%       - maketitle.tex

\makeatletter

% Redefine \maketitle to match Ergo style
\renewcommand\maketitle{
  \begingroup
  \setlength{\parskip}{0pt plus 0pt minus 0pt}
  \long\def\@makefntext##1{\parindent 1em\noindent
          \hb@xt@1.8em{%
             \hss\@textsuperscript{\normalfont\@thefnmark}}##1}%
  \newpage
  \null
  \noindent{}\begin{minipage}[l]{4in}
    \vskip 53.3pt plus 0pt minus 0pt
    {\LARGE \textsc{\@title}\par}% Title in large small caps
    \vskip 15pt plus 0pt minus 0pt
    {\Huge\MakeUppercase{\@author}\par}% Author in huge uppercase
    \vskip 5pt plus 0pt minus 0pt
    {\normalsize\textit{University of Michigan}\par}% Date in normal size italics (or affiliation)
  \end{minipage}
  \vskip 12pt plus 0pt minus 0pt
  \thispagestyle{plain}
  \endgroup
  \setcounter{footnote}{0}%
  \global\let\thanks\relax
  \global\let\@thanks\@empty
  \global\let\@date\@empty
  \global\let\title\relax
  \global\let\author\relax
  \global\let\date\relax
  \global\let\and\relax
}

% Redefine abstract to match Ergo style with margins
\renewenvironment{abstract}
 {\list{}{
    \setlength{\leftmargin}{.25in}
    \setlength{\rightmargin}{\leftmargin}
  }
  \item\relax
  \small}
 {\vskip -3pt plus 0pt minus 0pt\null\endlist}

% Font size adjustments to match Ergo
\renewcommand\normalsize{\@setfontsize\normalsize{10.5pt}{14pt}}
\renewcommand\small{\@setfontsize\small{9pt}{13pt}}
\renewcommand\LARGE{\@setfontsize\LARGE{18pt}{22pt}}
\renewcommand\Huge{\@setfontsize\Huge{9.5pt}{13.5pt}}

\makeatother
% Body text and heading styles mimicking Ergo class
% Add this to your Quarto YAML with:
% format:
%   pdf:
%     include-in-header:
%       - body-style.tex

\usepackage[automark]{scrlayer-scrpage}
\clearpairofpagestyles
\cehead{
  Brian Weatherson
  }
\cohead{
  Humberstone on Possibility Frames
  }
\ohead{\bfseries \pagemark}
\cfoot{}

\makeatletter

% Paragraph indentation and spacing
\setlength{\parindent}{.25in}
\setlength{\parskip}{0pt plus 0pt minus 0pt}

% Section numbering with periods and spacing
\renewcommand*{\@seccntformat}[1]{%
  \csname the#1\endcsname.$\:$%
}

% Section styles (matching Ergo)
\renewcommand\section{\@startsection{section}{1}{\z@}%
  {-4.6ex \@plus 0ex \@minus 0ex}%        % space before
  {2.3ex \@plus 0ex}%                      % space after
  {\normalfont\Large\bfseries}}            % Large, bold

\renewcommand\subsection{\@startsection{subsection}{2}{\z@}%
  {-2.3ex \@plus 0ex \@minus 0ex}%        % space before
  {2.3ex \@plus 0ex \@minus 0ex}%         % space after
  {\normalfont\large\itshape}}             % large, italic

\renewcommand\subsubsection{\@startsection{subsubsection}{3}{\z@}%
  {-2.3ex \@plus 0ex \@minus 0ex}%        % space before
  {2.3ex \@plus 0ex \@minus 0ex}%         % space after
  {\normalfont\large}}                     % large, normal

% Font sizes (matching Ergo)
\renewcommand\normalsize{\@setfontsize\normalsize{10.5pt}{14pt}}
\renewcommand\footnotesize{\@setfontsize\footnotesize{9pt}{11.5pt}}
\renewcommand\small{\@setfontsize\small{9pt}{13pt}}
\renewcommand\large{\@setfontsize\large{11.5pt}{12pt}}
\renewcommand\Large{\@setfontsize\Large{12pt}{14pt}}

% Quotation and quote environments with Ergo spacing
\renewenvironment{quotation}
  {\list{}{
    \setlength{\listparindent}{.25in}%
    \setlength{\leftmargin}{.25in}
    \setlength{\rightmargin}{\leftmargin}
    \setlength{\parsep}{0in plus 0in minus 0in}
    \item\relax
    \let\item\relax}
  }
  {\endlist}

\renewenvironment{quote}
  {\vskip 5pt%
   \list{}{
    \setlength{\listparindent}{.25in}
    \setlength{\leftmargin}{.25in}
    \setlength{\rightmargin}{\leftmargin}
    \setlength{\parsep}{0in plus 0in minus 0in}
    }
    \item\relax
    \let\item\relax}
  {\endlist\vskip 5pt}

% Footnote rule styling
\renewcommand{\footnoterule}{%
  \kern -3pt
  \hrule width 1in height .4pt
  \kern 2.5pt
}

% Non-superscript numerals for footnote text numbering
\renewcommand\@makefntext[1]{%
  \parindent .25in%
  \@thefnmark.~#1}%

% Figure and table captions with period separator
\renewcommand\@makecaption[2]{%
  \vskip\abovecaptionskip
  \sbox\@tempboxa{#1. #2}%
  \ifdim \wd\@tempboxa >\hsize
    #1. #2\par
  \else
    \global \@minipagefalse
    \hb@xt@\hsize{\hfil\box\@tempboxa\hfil}%
  \fi
  \vskip\belowcaptionskip}

% Enumerate environment with extra vertical space
\let\oldenumerate\enumerate
\let\endoldenumerate\endenumerate
\renewenvironment{enumerate}
  {\vskip 5pt\oldenumerate}
  {\endoldenumerate\vskip 5pt}

\let\olddescription\description
\let\endolddescription\enddescription
\renewenvironment{description}
  {\vskip 5pt\olddescription}
  {\endolddescription\vskip 5pt}

\let\olditemize\itemize
\let\endolditemize\enditemize
\renewenvironment{itemize}
  {\vskip 5pt\olditemize}
  {\endolditemize\vskip 5pt}

\newcommand*\NoIndentAfterEnv[1]{%
  \AfterEndEnvironment{#1}{\par\@afterindentfalse\@afterheading}}
\makeatother
\NoIndentAfterEnv{itemize}
\NoIndentAfterEnv{enumerate}
\NoIndentAfterEnv{description}
\NoIndentAfterEnv{quote}
\NoIndentAfterEnv{equation}
\NoIndentAfterEnv{longtable}

\makeatother
% Simple aggressive line breaking for DOIs
\usepackage{xurl}
\PassOptionsToPackage{hyphens}{url}

% Allow breaks at many characters
\def\UrlBreaks{\do\a\do\b\do\c\do\d\do\e\do\f\do\g\do\h\do\i\do\j%
\do\k\do\l\do\m\do\n\do\o\do\p\do\q\do\r\do\s\do\t\do\u\do\v\do\w%
\do\x\do\y\do\z\do\A\do\B\do\C\do\D\do\E\do\F\do\G\do\H\do\I\do\J%
\do\K\do\L\do\M\do\N\do\O\do\P\do\Q\do\R\do\S\do\T\do\U\do\V\do\W%
\do\X\do\Y\do\Z\do\0\do\1\do\2\do\3\do\4\do\5\do\6\do\7\do\8\do\9%
\do\.\do\-\do\/\do\:\do\=\do\?\do\&\do\_}

% Emergency line breaking
\sloppy
\newcommand{\nmodels}{\mathrel{\ooalign{$\models$\cr\raisebox{-0.001ex}{\hss$\mkern-1mu/\hss$}\cr}}}
\newcommand{\llbracket}{[\![}
\newcommand{\rrbracket}{]\!]}
\setlength\heavyrulewidth{0ex}
\setlength\lightrulewidth{0ex}
\usepackage[lines=2]{lettrine}
\cehead{Draft of February 11, 2026}
\makeatletter
\@ifpackageloaded{caption}{}{\usepackage{caption}}
\AtBeginDocument{%
\ifdefined\contentsname
  \renewcommand*\contentsname{Table of contents}
\else
  \newcommand\contentsname{Table of contents}
\fi
\ifdefined\listfigurename
  \renewcommand*\listfigurename{List of Figures}
\else
  \newcommand\listfigurename{List of Figures}
\fi
\ifdefined\listtablename
  \renewcommand*\listtablename{List of Tables}
\else
  \newcommand\listtablename{List of Tables}
\fi
\ifdefined\figurename
  \renewcommand*\figurename{Figure}
\else
  \newcommand\figurename{Figure}
\fi
\ifdefined\tablename
  \renewcommand*\tablename{Table}
\else
  \newcommand\tablename{Table}
\fi
}
\@ifpackageloaded{float}{}{\usepackage{float}}
\floatstyle{ruled}
\@ifundefined{c@chapter}{\newfloat{codelisting}{h}{lop}}{\newfloat{codelisting}{h}{lop}[chapter]}
\floatname{codelisting}{Listing}
\newcommand*\listoflistings{\listof{codelisting}{List of Listings}}
\makeatother
\makeatletter
\makeatother
\makeatletter
\@ifpackageloaded{caption}{}{\usepackage{caption}}
\@ifpackageloaded{subcaption}{}{\usepackage{subcaption}}
\makeatother
\usepackage{bookmark}
\IfFileExists{xurl.sty}{\usepackage{xurl}}{} % add URL line breaks if available
\urlstyle{same}
\hypersetup{
  pdftitle={Humberstone on Possibility Frames},
  pdfauthor={Brian Weatherson},
  colorlinks=true,
  linkcolor={blue},
  filecolor={Maroon},
  citecolor={Blue},
  urlcolor={blue},
  pdfcreator={LaTeX via pandoc}}


\title{Humberstone on Possibility Frames}
\author{Brian Weatherson}
\date{2026-02-11}
\begin{document}
\maketitle
\begin{abstract}
Insert abstract here
\end{abstract}


\setstretch{1}
In his 1981 paper, ``From Worlds to Possibilities'', Lloyd Humberstone
shows a way to do modal logic without the apparatus of possible worlds.
Instead of worlds he uses \emph{possibilities}, which may, unlike
worlds, be incomplete. The non-modal parts of the view are discussed
again in section 6.44 of \emph{The Connectives}, though the differences
between the view there and the 1981 view are largely presentational. In
this paper I'll set out this \emph{possibility frame} approach to modal
logic, make some notes about its logic, and end with a survey of the
many possible applications it has.

Mathematically, possibilities are just points in a model, just like
possible worlds are points in different kinds of models. But it helps to
have a mental picture of what kind of thing they are. In ``From Worlds
to Possibilities'', Humberstone notes that one picture you could have is
that they are sets of possible worlds. This isn't a terrible picture,
but it's not perfect for a couple of reasons. For one thing, as
Humberstone notes, part of the point of developing possibilities is to
do without the machinery of possible worlds. Understanding possibilities
as sets of possible worlds wouldn't help with that project. For another,
as Wesley Holliday (\citeproc{ref-Holliday2025}{2025, 271--72}) notes,
the natural way to generate modal accessibility relations on sets of
worlds from accessibility on the worlds themselves doesn't always work
the way Humberstone wants accessibility to work. So let's start with a
different picture.

Possibilities, as I'll think of them, are \emph{stories}. To make things
concrete, let's focus on a particular story: \emph{A Study in Scarlet}
(Conan Doyle (\citeproc{ref-ConanDoyle1995}{1995})), the story where
Sherlock Holmes was introduced. That story settles some questions, both
explicitly, e.g., that Holmes is a detective, and implicitly, e.g., that
Holmes has never set foot on the moon. But it leaves several other
questions open, e.g., how many (first) cousins Holmes has. It's not that
\emph{A Study in Scarlet} is a story. It has proper parts which are
stories. The first chapter is a story, one which tells of the first
meeting between Holmes and Watson. And arguably it is a proper part of
larger story, made up of all of Conan Doyle's stories of Holmes and
Watson. When a story \(x\) is a proper part of story \(y\), what that
means is that everything settled in \(x\) is still true in \(y\), and
more things besides are settled. When this happens, we'll call \(y\) a
proper \emph{refinement} of \(x\). For most purposes it will be more
convenient to use the more general notion of \emph{refinement}, where
each story counts as an improper refinement of itself.

Following Humberstone, I'll write \(x \leqslant y\) to mean that \(y\)
is a refinement of \(x\). As he notes, this notation can be confusing if
one things of \(x\) and \(y\) as sets, because in that case the
refinement will typically be \emph{smaller}.\footnote{Holliday
  (\citeproc{ref-Holliday2025}{2025}) writes \(y \sqsubseteq x\) when
  \(y\) is a refinement of \(x\), mirroring this way of thinking about
  possibilities.} But if we think of possibilities as stories, the
notation becomes more intuitive. We have \(x \leqslant y\) when \(y\) is
created by adding new content to \(x\). Keeping with this theme, I'll
typically model stories not as worlds, but as finite sequences. (In the
main example in Section~\ref{sec-proof}, they will be sequences of 0s
and 1s.) In these models, \(x \leqslant y\) means that \(x\) is an
initial segment of \(y\).

\section{Formal Structure}\label{formal-structure}

To start with, assume we're working in a simple language that just has a
countable set \(\mathcal{P}\) countable infinity of propositional
variables, and three connectives: \(\neg\), \(\wedge\) and \(\vee\). We
have a set of possibilities \(W\), and a transitive refinement relation
\(\geqslant\) on them. The following rules show how to build what I'll
call a \emph{Humberstone possibility model} on
\(\langle W, \leqslant \rangle\). (I'll call this a \emph{possibility
frame} in most contexts, but a \emph{Humberstone frame} when I'm
comparing it to similar structures, especially in the context of
discussing Holliday (\citeproc{ref-Holliday2025}{2025}).)

A Humberstone possibility model \(\mathcal{M}\) is a triple
\(\langle W, \leqslant, V \rangle\), where \(V\) is a function from
\(\mathcal{P}\) to \(W\), intuitively saying where each atomic
proposition is true, satisfying these two constraints:

\begin{itemize}
\tightlist
\item
  For all \(x\), if \(x \in V(p)\) and \(y \geqslant x\), then
  \(y \in V(p)\). Intuitively, truth for atomics is \textbf{persistent}
  across refinements.
\item
  For all \(x\), if
  \(\forall y \geqslant x \exists z \geqslant y: z \in V(p)\), then
  \(x \in V(p)\). This is what Humberstone
  (\citeproc{ref-Humberstone2011}{2011, 900}) calls
  \textbf{refinability}, and it means that \(p\) only fails to be true
  at \(x\) if there is some refinement of \(x\) where it is settled as
  being untrue.
\end{itemize}

Given these constraints, Humberstone suggests the following theory of
truth at a possibility for all sentences in this language.

\begin{align*}
[\text{Vbls}] \quad & \mathcal{M} \models_x p_i \text{ iff } x \in V(p_i); \\
[\neg] \quad & \mathcal{M} \models_x \neg A \text{ iff } \forall y \geqslant x, \, \mathcal{M} \nmodels_y  A; \\
[\wedge] \quad & \mathcal{M} \models_x A \wedge B \text{ iff } \mathcal{M} \models_x A \text{ and } \mathcal{M} \models_x B; \\
[\vee] \quad & \mathcal{M} \models_x A \vee B \text{ iff } \forall y \geqslant x \, \exists z \geqslant y \, : \, \mathcal{M} \models_z A \text{ or } \mathcal{M} \models_z B.
\end{align*}

Given these definitions, it's possible to prove three things. First,
every sentence in the language is persistent. If
\(\mathcal{M} \models_x A\) and \(x \leqslant y\), then
\(\mathcal{M} \models_x A\). For any sentence, truth is always preserved
when moving to a refinement. Second, refinability holds for all
sentences in the language. This is, as Humberstone notes, easier to
state using this definition of \(\neg\). It now becomes the claim, for
arbitrary \(A\), that if \(\mathcal{M} \nmodels_x A\), there is some
refinement \(y\) of \(x\) such that \(\mathcal{M} \models_y \neg A\).
Third, for any set of sentences \(\Gamma\) and sentence \(A\), the truth
at a point of all sentences in \(\Gamma\) guarantees the truth of \(A\)
iff the sequent \(\Gamma\) entails \(A\) in classical propositional
logic.

In this paper, I'm going to discuss three extensions of this language.
I'll introduce them in reverse order of how much they are discussed in
Humberstone, starting with one he does not discuss at all: infinitary
disjunction.

We'll add to the language a new symbol \(\bigvee\), which forms a new
sentence out of any countable set of sentences not containing
\(\bigvee\). Intuitively, it is true when one of the sentences in the
set is true. More formally, its definition of truth at a possibility is:

\begin{align*}
[\bigvee] \quad & \mathcal{M} \models_x \bigvee ({A_1, A_2, \dots})  \text{ iff } \forall y \geqslant x \, \exists z \geqslant y \, : \,\text{ for some } i \, \mathcal{M} \models_z A_i.
\end{align*}

Again, it's fairly simple to show that this addition to the language
will preserve persistence and refinability. But while this is simple, it
is significant, because things could easily have been otherwise.

The second extension will be to add quantifiers, following a suggestion
in Humberstone (\citeproc{ref-Humberstone1981a}{1981}, xxxx). Assume, as
usual, that the language has a stock of names \(c_1, \dots\), and for
each \(n\), a stock of \(n\)-place predicates \(F^n_1, F^n_2, \dots\). A
\emph{first-order (Humberstone) possibility model} is a structure
\(\langle W, \leqslant, D, V \rangle\), where \(D\) assigns a non-empty
domain of objects to each point, and \(V\) interprets the non-logical
vocabulary. More precisely:

\begin{itemize}
\tightlist
\item
  \(D\) is a function assigning to each \(x \in W\) a non-empty set
  \(D(x)\), the \textbf{domain} at \(x\).
\item
  \(V\) assigns to each name \(c_i\) and each \(x \in W\) either a
  designated element \(V(c_i, x) \in D(x)\), or is undefined at \(x\).
\item
  \(V\) assigns to each \(n\)-place predicate \(F^n_j\) and each
  \(x \in W\) a set \(V(F^n_j, x) \subseteq D(x)^n\), the
  \textbf{extension} of \(F^n_j\) at \(x\).
\end{itemize}

These must satisfy the following constraints:

\begin{description}
\tightlist
\item[Domain monotonicity]
If \(x \leqslant y\), then \(D(x) \subseteq D(y)\).
\item[Name coverage]
For each name \(c_i\) and each \(x \in W\), there exists some
\(y \geqslant x\) such that \(V(c_i, y)\) is defined.
\item[Persistence for names]
If \(V(c_i, x)\) is defined and \(x \leqslant y\), then \(V(c_i, y)\) is
defined and \(V(c_i, y) = V(c_i, x)\).
\item[Persistence for predicate extensions]
If \(\langle o_1, \dots, o_n \rangle \in V(F^n_j, x)\) and
\(x \leqslant y\), then
\(\langle o_1, \dots, o_n \rangle \in V(F^n_j, y)\).
\item[Refinability for predicate extensions]
If \(\langle o_1, \dots, o_n \rangle \notin V(F^n_j, x)\), then there
exists some \(y \geqslant x\) such that for all \(z \geqslant y\),
\(\langle o_1, \dots, o_n \rangle \notin V(F^n_j, z)\).
\end{description}

Given a model and a variable assignment \(g\) mapping variables to
objects, truth at a point is defined as follows. Write \(g[v/o]\) for
the assignment that maps variable \(v\) to object \(o\) and otherwise
agrees with \(g\). For a term \(t\), write
\(\llbracket t \rrbracket^{g,x}\) for the denotation of \(t\) under
\(g\) at \(x\): for a variable \(v\) this is \(g(v)\), and for a name
\(c_i\) this is \(V(c_i, x)\) when defined, and undefined otherwise.

\begin{align*}
[=] \quad & \mathcal{M}, g \models_x t_1 = t_2 \text{ iff } \forall y \geqslant x \, \exists z \geqslant y : \llbracket t_1 \rrbracket^{g,z} \text{ and } \llbracket t_2 \rrbracket^{g,z} \text{ are both defined and equal}; \\
[F^n] \quad & \mathcal{M}, g \models_x F^n_j(t_1, \dots, t_n) \text{ iff } \forall y \geqslant x \, \exists z \geqslant y : \langle \llbracket t_1 \rrbracket^{g,z}, \dots, \llbracket t_n \rrbracket^{g,z} \rangle \in V(F^n_j, z); \\
[\forall] \quad & \mathcal{M}, g \models_x \forall v \, A \text{ iff } \forall y \geqslant x \, \forall o \in D(y) : \mathcal{M}, g[v/o] \models_y A; \\
[\exists] \quad & \mathcal{M}, g \models_x \exists v \, A \text{ iff } \forall y \geqslant x \, \exists z \geqslant y \, \exists o \in D(z) : \mathcal{M}, g[v/o] \models_z A.
\end{align*}

The Boolean connectives are handled exactly as in the propositional
case.

The \(\forall\exists\) pattern in the atomic clauses is necessary to
ensure that persistence and refinability hold for all sentences,
including atomic ones. Consider \(c_i = c_i\): if a name has no
denotation at \(x\) but acquires one at some refinement, then a simple
``check the denotation at \(x\)'' condition would leave \(c_i = c_i\)
neither true nor false at \(x\), and no refinement of \(x\) could settle
it as false either, violating refinability. The \(\forall\exists\)
condition handles this correctly: \(c_i = c_i\) is true at \(x\)
whenever \(c_i\) is covered at \(x\) (i.e., every refinement has a
further refinement where \(c_i\) gets a referent), since once \(c_i\)
gets a referent \(o\), persistence of names ensures \(o = o\) at all
further refinements.

The atomic clauses simplify when names are fully defined. If \(t_1\) and
\(t_2\) are variables, or names that already have denotations at \(x\),
then by persistence of names and predicate extensions the
\(\forall\exists\) quantifier prefix collapses:
\(\mathcal{M}, g \models_x t_1 = t_2\) iff
\(\llbracket t_1 \rrbracket^{g,x} = \llbracket t_2 \rrbracket^{g,x}\),
and \(\mathcal{M}, g \models_x F^n_j(t_1, \dots, t_n)\) iff
\(\langle \llbracket t_1 \rrbracket^{g,x}, \dots, \llbracket t_n \rrbracket^{g,x} \rangle \in V(F^n_j, x)\).
The more complex clauses above are needed only to handle the case where
some name occurring in the formula lacks a denotation at \(x\) but is
guaranteed to acquire one.

This is a possibilist treatment of the universal quantifier, in contrast
to the actualist quantifiers discussed in
(\citeproc{ref-HarrisonTrainor2019}{\textbf{HarrisonTrainor2019?}}).
I'll return in \textbf{?@sec-quant} to the reasons we are best off using
possibilist quantifiers, and the difficulties this raises for talking
about just what's true in a possibility.

The third extension will be the introduction of modal operators. Here
I'll follow Humberstone (\citeproc{ref-Humberstone1981a}{1981}) very
closely, save just that I'll have a plurality of modal operators rather
than just one. So I'll use these structures to define (as Holliday
(\citeproc{ref-Holliday2025}{2025}) does) multi-modal logics. But I'll
follow Humberstone, and not Holliday, in defining modal operators in
terms of accessibility relations \(R_i\) satisfying these three
conditions\footnote{I'm using the names for these that Holliday uses,
  which are more evocative than Humberstone's original names.}:

\begin{description}
\tightlist
\item[\textbf{UpR}:]
If \(x \leqslant x'\) and \(x' R_i y\), then \(x R_i y\).
\item[\textbf{RDown}:]
If \(x R_i y\) and \(y \leqslant y'\), then \(x R_i y'\).
\item[\textbf{RRef++}:]
If \(x R_i y\), then there exists \(x' \geqslant x\) such that for all
\(x'' \geqslant x'\), \(x'' R_i y\).
\end{description}

\textbf{UpR} says that if a refinement of \(x\) can access \(y\), then
\(x\) itself can already access \(y\): accessibility is not something
that can be gained by adding detail to the source. \textbf{RDown} is a
converse of this; it says that accessibility cannot be gained by adding
detail to the target. \textbf{RRef++} says that if \(x\) can access
\(y\), there is some refinement \(x'\) of \(x\) where it is settled that
\(x'\) can access \emph{y}. This last access can't be overturned by
further refinement of \(x'\).

Given these constraints, the truth conditions for the box and diamond
operators are:

\begin{align*}
[\Box_i] \quad & \mathcal{M} \models_x \Box_i A \text{ iff } \forall y \, (x R_i y \Rightarrow \mathcal{M} \models_y A); \\
[\Diamond_i] \quad & \mathcal{M} \models_x \Diamond_i A \text{ iff } \forall y \geqslant x \, \exists z \geqslant y \, \exists w \, (z R_i w \text{ and } \mathcal{M} \models_w A).
\end{align*}

The clause for \(\Box\) is the standard one: \(\Box_i A\) is true at
\(x\) iff \(A\) is true at every \(R_i\)-accessible possibility.
Officially, Humberstone treats \(\Diamond\) as a defined connective,
\(\Diamond_i A\) just means \(\neg \Box_i \neg A\). I've spelled out
what that means using the truth condition for \(\neg\). It says that no
refinement of \(x\) can permanently rule out there being an accessible
point where \(A\) holds. Equivalently, every refinement of \(x\) has a
further (possibly improper) refinement that accesses some point where
\(A\) is true.

Why should we impose these constraints? It is not hard to show that they
guarantee that persistence and refinability hold for sentences generated
using these new modal connectives. At least, it isn't hard as long as we
remember that \(\Diamond\) is being treated as defined, so the only new
step in the inductive proofs involves \(\Box\). And \textbf{UpR}
guarantees persistence for \(\Box\) sentences, while \textbf{RRef++}
guarantees refinability.

But this is overkill. As Humberstone points out, we haven't used
\textbf{RDown} in the proof, so this doesn't explain why we'd impose
\textbf{RDown}. As Holliday points out {[}note to Claude, we need page
number for this{]} \textbf{UpR} is stronger than we need for
persistence. We could weaken it by making greater use of the fact that
\(A\) is persistent. All we need is that if \(x \leqslant x'\) and
\(xRy\) then there is some \(z \geqslant y\) such that \(xRy\). That
will guarantee the key fact if \(x'\) can access a world where \(A\),
then so can \(x\).

So we need other arguments for these constraints other than their role
in ensuring persistence and refinability. Humberstone offers two other
arguments here. One anticipates the multi-modal setting that is being
used here. It's that if we want to use this system for tense logic, then
we want \(R_i^{-1}\) to satisfy all the constraints, so if we impose
\textbf{UpR}, we should also impose \textbf{RDown}. This isn't
convincing on its own though. For one thing, as already noted, we might
not need \textbf{UpR}. For another, by these lights we should worry that
the system is incomplete because we haven't put in a converse of
\textbf{RRef++}.

The argument that Humberstone spends more time on, and which I think is
more compelling, comes from rethinking the relationship between \(R_i\)
and \(\Box_i\). It's very tempting to read those truth conditions as
being explanatory from right-to-left. On this way of thinking, facts
about which \(\Box_i\) sentences are true at a point are grounded in
which non-modal sentences are true and which \(R_i\) relations obtain.
But while this is tempting, it isn't compulsory.

We could instead take the modal facts as given, and ask what
accessibility relations must obtain to be consistent with them. The
process here is familiar from the construction of canonical models. We
take the sets of consistent sentences as given, and say \(s_1R_is_2\)
iff whenever \(\Box_i A \in s_1\), then \(A \in s_2\). Humberstone's
approach is, I think, similar. Start with the idea that some sentences
are true in some model \(\mathcal{M}\) at possibilities \(x\) and \(y\),
say \(xR_iy\) iff \(\mathcal{M} \models_y A\) whenever
\(\mathcal{M} \models_x \Box_i A\), and ask what constraints \(R_i\)
will thereby satisfy.

The answer is that, if all sentences are persistent and refinable, then
\(R_i\) will satisfy these three constraints. If
\(\mathcal{M} \models_y A\) whenever
\(\mathcal{M} \models_x' \Box_i A\), then by the persistence of
\(\Box_i\), we know that \(\mathcal{M} \models_y A\) whenever
\(\mathcal{M} \models_x \Box_i A\). A similar argument, using the
persistence of \(A\), justifies \textbf{RDown}. Finally, \textbf{RRef++}
follows from the fact that \(\Box_i A\) is refinable. If \(\Box_i A\) is
not true at \(x\), that means that it is determinately not true at some
refinement. So if \(\mathcal{M} \nmodels_x A\), there must be some
refinement \(x'\) such that for all further refinements \(x''\),
\(\mathcal{M} \nmodels_x'' A\). If \(\mathcal{M} \nmodels_x \Box_i A\),
then there must be some possibility \(y\) such that \(xR_iy\) and
\(\mathcal{M} \nmodels_y A\). So as long as \(x''R_iy\), refinability
will be satisfied. I don't see how to prove there isn't a weaker
condition that would also work, it's possible we could use the
refinability of A to find some weaker condition, but I don't quite see
how that would work. So I think \textbf{RRef++} also follows from this
way of thinking about accessibility.

In the next section I'll discuss what logics can be defined using frames
that satisfy all of these conditions.

\section{Logics Determinable on Humberstone Frames}\label{sec-proof}

Holliday (\citeproc{ref-Holliday2025}{2025}) raises an interesting
question. As well as the familiar Kripke frames most commonly used as a
semantics for modal logic, and the Humberstone frames defined above, he
introduces a class of `full possibility' frames, which weaken some of
Humberstone's constraints. It won't matter here exactly what these
weakenings are, but what does matter is that he shows that using these
weakened frames, we can determine logics which are not determinable on
any class of Kripke frames. To state this more precisely, for any class
of frames \(\mathsf{F}\), let \(\mathrm{L}(\$\mathsf{F})\) be the set of
sentences true at all points in all models definable some member of
\(\mathsf{F}\). Then let \(\mathrm{ML}(\mathsf{F})\) be the set
\(\{\mathrm{L}(\mathsf{X}) : \mathsf{X} \subseteq \mathsf{F}\}\)

Since every Kripke frame is a Humberstone frame and every Humberstone
frame is a full possibility frame, we have

\[\mathrm{ML}(\mathsf{K}) \subseteq \mathrm{ML}(\mathsf{H}) \subseteq \mathrm{ML}(\mathsf{FP}),\]

where \(\mathsf{K}\) is the class of Kripke frames, \(\mathsf{H}\) is
the class of Humberstone frames, \(\mathsf{FP}\) is the class of full
possibility frames, and \(\mathrm{ML}(\mathsf{F})\) denotes the set of
modal formulas valid over every frame in the class \(\mathsf{F}\)---the
modal logic determined by that class. We know that at least one of these
inclusions is strict---but which one?

We will not answer Holliday's question exactly as posed. But we can
answer a closely related question. If we expand the language to include
infinitary disjunction \(\bigvee\), then the first inclusion is strict:
there are logics definable on Humberstone frames that are not definable
on Kripke frames. We will show this by constructing a single Humberstone
frame that, in the infinitary language, defines a logic with no Kripke
equivalent.

\subsection{The Frame}\label{the-frame}

The frame is built from two copies of the set of finite binary
sequences---sequences of 0s and 1s of any finite length, including the
empty sequence. We call one copy the \textbf{left-handed} sequences and
the other the \textbf{right-handed} sequences. We write \(x^L\) for the
left-handed copy of a sequence \(x\) and \(x^R\) for the right-handed
copy.

The refinement relation is: \(x \leqslant y\) iff \(x\) and \(y\) have
the same handedness and \(x\) is an initial segment of \(y\). So within
each copy the frame is just the binary tree ordered by extension, and no
left-handed sequence refines a right-handed sequence or vice versa.

\subsection{The Accessibility
Relations}\label{the-accessibility-relations}

We define two infinite families of accessibility relations. As we go, we
verify that each relation satisfies the three Humberstone constraints
\textbf{UpR}, \textbf{RDown}, and \textbf{RRef++}.

\textbf{The left-to-right relations.} For each \(k \geqslant 0\), define
\(R^\rightarrow_k\) by: \(x R^\rightarrow_k y\) iff \(x\) is
left-handed, \(y\) is right-handed, \(y\) has length at least \(k\), and
either \(x\) is an initial segment of \(y\), or the first \(k\) elements
of \(x\) are an initial segment of \(y\).

The special case \(k = 0\) is simply: \(x\) is left-handed and \(y\) is
right-handed (the length condition and the initial-segment disjunction
are trivially satisfied).

It helps to picture \(R^\rightarrow_k\) as follows. Each of the two sets
of sequences forms a binary tree in the usual way. Imagine bridges
connecting left-handed nodes to right-handed nodes whenever the two
nodes carry the same sequence of length exactly \(k\). Then
\(x R^\rightarrow_k y\) holds precisely when there is a top-to-bottom
path in the left tree starting at \(x\), crossing one of these bridges,
and then continuing along (and possibly beyond) \(y\) in the right tree.

We verify the three constraints for \(R^\rightarrow_k\).

\begin{itemize}
\item
  \textbf{UpR}: Suppose \(x \leqslant x'\) and \(x' R^\rightarrow_k y\).
  Then \(x\) and \(x'\) are both left-handed and \(x\) is an initial
  segment of \(x'\). Since \(x'\) is an initial segment of \(y\) or its
  first \(k\) elements are an initial segment of \(y\), the same holds
  for \(x\) (being a shorter initial segment of \(x'\)). So
  \(x R^\rightarrow_k y\).
\item
  \textbf{RDown}: Suppose \(x R^\rightarrow_k y\) and
  \(y \leqslant y'\). Then \(y'\) is right-handed and is a refinement
  (extension) of \(y\), so \(y'\) has length at least \(k\), and \(x\)
  is still an initial segment of \(y'\) or its first \(k\) elements are
  still an initial segment of \(y'\). So \(x R^\rightarrow_k y'\).
\item
  \textbf{RRef++}: Suppose \(x R^\rightarrow_k y\). Let \(x'\) be the
  left-handed sequence consisting of the first \(k\) elements of \(x\)
  (or all of \(x\) if \(|x| < k\)). Then \(x \leqslant x'\) and for
  every \(x'' \geqslant x'\), the first \(k\) elements of \(x''\) are an
  initial segment of \(y\) (since \(x'\) is an initial segment of
  \(x''\) and the first \(k\) elements of \(x'\) are the first \(k\)
  elements of \(x\), which by hypothesis form an initial segment of
  \(y\)). Hence \(x'' R^\rightarrow_k y\) for all \(x'' \geqslant x'\).
\end{itemize}

\textbf{The right-to-left relations.} For each \(k > 0\), define
\(R^\leftarrow_k\) by: \(x R^\leftarrow_k y\) iff \(x\) is right-handed,
\(x\) does not have a \(0\) in its \(k\)-th position (either because
\(x\) has length less than \(k\), or because it has a \(1\) in position
\(k\)), and \(y\) is left-handed.

We verify the three constraints for \(R^\leftarrow_k\).

\begin{itemize}
\item
  \textbf{UpR}: Suppose \(x \leqslant x'\) and \(x' R^\leftarrow_k y\).
  Then \(x\) and \(x'\) are both right-handed and \(x\) is an initial
  segment of \(x'\). Since \(x'\) does not have a \(0\) in position
  \(k\), either \(|x'| < k\) or position \(k\) of \(x'\) is \(1\). If
  \(|x'| < k\) then also \(|x| < k\), so \(x\) does not have a \(0\) in
  position \(k\). If position \(k\) of \(x'\) is \(1\), then since \(x\)
  is an initial segment of \(x'\), either \(|x| < k\) or position \(k\)
  of \(x\) is also \(1\). In either case \(x\) does not have a \(0\) in
  position \(k\), so \(x R^\leftarrow_k y\).
\item
  \textbf{RDown}: Suppose \(x R^\leftarrow_k y\) and \(y \leqslant y'\).
  Then \(y'\) is left-handed, and the condition on \(x\) is unchanged.
  So \(x R^\leftarrow_k y'\).
\item
  \textbf{RRef++}: Suppose \(x R^\leftarrow_k y\). We need some
  \(x' \geqslant x\) such that for all \(x'' \geqslant x'\),
  \(x'' R^\leftarrow_k y\). Take \(x' = x\). For any
  \(x'' \geqslant x'\), \(x''\) is right-handed and \(x'\) is an initial
  segment of \(x''\). Since \(x'\) does not have a \(0\) in position
  \(k\), neither does \(x''\) (if \(|x'| \geqslant k\) then position
  \(k\) of \(x''\) equals position \(k\) of \(x'\), which is \(1\); if
  \(|x'| < k\) then either \(|x''| < k\) also, or position \(k\) of
  \(x''\) is some bit which has nothing to do with \(x'\)---but wait,
  \(x'\) is an initial segment of \(x''\), so if
  \(|x''| \geqslant k > |x'|\), position \(k\) of \(x''\) is freely
  determined and may be \(0\)).

  To handle this correctly, take \(x'\) to be the left-handed sequence
  obtained by extending \(x\) to length \(k\) with \(1\)s if \(|x| < k\)
  (i.e., append \(1\)s until we reach length \(k\), keeping \(x\)'s bits
  for the first \(|x|\) positions). Then for all \(x'' \geqslant x'\),
  \(x''\) has length at least \(k\) and its \(k\)-th element is \(1\)
  (since \(x'\) is an initial segment of \(x''\) and the \(k\)-th
  element of \(x'\) is \(1\)). So \(x'' R^\leftarrow_k y\) for all
  \(x'' \geqslant x'\). This establishes \textbf{RRef++}.
\end{itemize}

\subsection*{References}\label{references}
\addcontentsline{toc}{subsection}{References}

\phantomsection\label{refs}
\begin{CSLReferences}{1}{0}
\bibitem[\citeproctext]{ref-ConanDoyle1995}
Conan Doyle, Arthur. 1995. \emph{A Study in Scarlet}. Urbana, Illinois:
Project Gutenberg.

\bibitem[\citeproctext]{ref-Holliday2025}
Holliday, Wesley H. 2025. {``Possibility Frames and Forcing for Modal
Logic.''} \emph{Australasian Journal of Logic} 22 (2): 44--288.
https://doi.org/10.26686/ajl.v22i2.5680.

\bibitem[\citeproctext]{ref-Humberstone1981a}
Humberstone, Lloyd. 1981. {``From Worlds to Possibilities.''}
\emph{Journal of Philosophical Logic} 10 (3): 313--39.
https://doi.org/10.1007/BF00293423.

\bibitem[\citeproctext]{ref-Humberstone2011}
---------. 2011. \emph{The Connectives}. Cambridge, MA: MIT Press.

\end{CSLReferences}




\end{document}
