% Options for packages loaded elsewhere
% Options for packages loaded elsewhere
\PassOptionsToPackage{unicode}{hyperref}
\PassOptionsToPackage{hyphens}{url}
\PassOptionsToPackage{dvipsnames,svgnames,x11names}{xcolor}
%
\documentclass[
  10.5pt,
  twoside]{article}
\usepackage{xcolor}
\usepackage[paperheight=10in,,paperwidth=7in,,top=1in,bottom=1in,inner=0.8in,outer=0.8in,headsep=0.25in,headheight=1in,footskip=0.35in]{geometry}
\usepackage{amsmath,amssymb}
\setcounter{secnumdepth}{-\maxdimen} % remove section numbering
\usepackage{iftex}
\ifPDFTeX
  \usepackage[T1]{fontenc}
  \usepackage[utf8]{inputenc}
  \usepackage{textcomp} % provide euro and other symbols
\else % if luatex or xetex
  \usepackage{unicode-math} % this also loads fontspec
  \defaultfontfeatures{Scale=MatchLowercase}
  \defaultfontfeatures[\rmfamily]{Ligatures=TeX,Scale=1}
\fi
\usepackage{lmodern}
\ifPDFTeX\else
  % xetex/luatex font selection
  \setmainfont[ItalicFont=EB Garamond Italic,BoldFont=EB Garamond
SemiBold]{EB Garamond Math}
  \setsansfont[]{EB Garamond SemiBold}
  \setmathfont[]{EB Garamond Math}
\fi
% Use upquote if available, for straight quotes in verbatim environments
\IfFileExists{upquote.sty}{\usepackage{upquote}}{}
\IfFileExists{microtype.sty}{% use microtype if available
  \usepackage[]{microtype}
  \UseMicrotypeSet[protrusion]{basicmath} % disable protrusion for tt fonts
}{}
\usepackage{setspace}
\makeatletter
\@ifundefined{KOMAClassName}{% if non-KOMA class
  \IfFileExists{parskip.sty}{%
    \usepackage{parskip}
  }{% else
    \setlength{\parindent}{0pt}
    \setlength{\parskip}{6pt plus 2pt minus 1pt}}
}{% if KOMA class
  \KOMAoptions{parskip=half}}
\makeatother
% Make \paragraph and \subparagraph free-standing
\makeatletter
\ifx\paragraph\undefined\else
  \let\oldparagraph\paragraph
  \renewcommand{\paragraph}{
    \@ifstar
      \xxxParagraphStar
      \xxxParagraphNoStar
  }
  \newcommand{\xxxParagraphStar}[1]{\oldparagraph*{#1}\mbox{}}
  \newcommand{\xxxParagraphNoStar}[1]{\oldparagraph{#1}\mbox{}}
\fi
\ifx\subparagraph\undefined\else
  \let\oldsubparagraph\subparagraph
  \renewcommand{\subparagraph}{
    \@ifstar
      \xxxSubParagraphStar
      \xxxSubParagraphNoStar
  }
  \newcommand{\xxxSubParagraphStar}[1]{\oldsubparagraph*{#1}\mbox{}}
  \newcommand{\xxxSubParagraphNoStar}[1]{\oldsubparagraph{#1}\mbox{}}
\fi
\makeatother


\usepackage{longtable,booktabs,array}
\usepackage{calc} % for calculating minipage widths
% Correct order of tables after \paragraph or \subparagraph
\usepackage{etoolbox}
\makeatletter
\patchcmd\longtable{\par}{\if@noskipsec\mbox{}\fi\par}{}{}
\makeatother
% Allow footnotes in longtable head/foot
\IfFileExists{footnotehyper.sty}{\usepackage{footnotehyper}}{\usepackage{footnote}}
\makesavenoteenv{longtable}
\usepackage{graphicx}
\makeatletter
\newsavebox\pandoc@box
\newcommand*\pandocbounded[1]{% scales image to fit in text height/width
  \sbox\pandoc@box{#1}%
  \Gscale@div\@tempa{\textheight}{\dimexpr\ht\pandoc@box+\dp\pandoc@box\relax}%
  \Gscale@div\@tempb{\linewidth}{\wd\pandoc@box}%
  \ifdim\@tempb\p@<\@tempa\p@\let\@tempa\@tempb\fi% select the smaller of both
  \ifdim\@tempa\p@<\p@\scalebox{\@tempa}{\usebox\pandoc@box}%
  \else\usebox{\pandoc@box}%
  \fi%
}
% Set default figure placement to htbp
\def\fps@figure{htbp}
\makeatother


% definitions for citeproc citations
\NewDocumentCommand\citeproctext{}{}
\NewDocumentCommand\citeproc{mm}{%
  \begingroup\def\citeproctext{#2}\cite{#1}\endgroup}
\makeatletter
 % allow citations to break across lines
 \let\@cite@ofmt\@firstofone
 % avoid brackets around text for \cite:
 \def\@biblabel#1{}
 \def\@cite#1#2{{#1\if@tempswa , #2\fi}}
\makeatother
\newlength{\cslhangindent}
\setlength{\cslhangindent}{1.5em}
\newlength{\csllabelwidth}
\setlength{\csllabelwidth}{3em}
\newenvironment{CSLReferences}[2] % #1 hanging-indent, #2 entry-spacing
 {\begin{list}{}{%
  \setlength{\itemindent}{0pt}
  \setlength{\leftmargin}{0pt}
  \setlength{\parsep}{0pt}
  % turn on hanging indent if param 1 is 1
  \ifodd #1
   \setlength{\leftmargin}{\cslhangindent}
   \setlength{\itemindent}{-1\cslhangindent}
  \fi
  % set entry spacing
  \setlength{\itemsep}{#2\baselineskip}}}
 {\end{list}}
\usepackage{calc}
\newcommand{\CSLBlock}[1]{\hfill\break\parbox[t]{\linewidth}{\strut\ignorespaces#1\strut}}
\newcommand{\CSLLeftMargin}[1]{\parbox[t]{\csllabelwidth}{\strut#1\strut}}
\newcommand{\CSLRightInline}[1]{\parbox[t]{\linewidth - \csllabelwidth}{\strut#1\strut}}
\newcommand{\CSLIndent}[1]{\hspace{\cslhangindent}#1}



\setlength{\emergencystretch}{3em} % prevent overfull lines

\providecommand{\tightlist}{%
  \setlength{\itemsep}{0pt}\setlength{\parskip}{0pt}}



 


% Custom title page mimicking Ergo class style
% Add this to your Quarto YAML with:
% format:
%   pdf:
%     include-in-header:
%       - maketitle.tex

\makeatletter

% Redefine \maketitle to match Ergo style
\renewcommand\maketitle{
  \begingroup
  \setlength{\parskip}{0pt plus 0pt minus 0pt}
  \long\def\@makefntext##1{\parindent 1em\noindent
          \hb@xt@1.8em{%
             \hss\@textsuperscript{\normalfont\@thefnmark}}##1}%
  \newpage
  \null
  \noindent{}\begin{minipage}[l]{4in}
    \vskip 53.3pt plus 0pt minus 0pt
    {\LARGE \textsc{\@title}\par}% Title in large small caps
    \vskip 15pt plus 0pt minus 0pt
    {\Huge\MakeUppercase{\@author}\par}% Author in huge uppercase
    \vskip 5pt plus 0pt minus 0pt
    {\normalsize\textit{University of Michigan}\par}% Date in normal size italics (or affiliation)
  \end{minipage}
  \vskip 12pt plus 0pt minus 0pt
  \thispagestyle{plain}
  \endgroup
  \setcounter{footnote}{0}%
  \global\let\thanks\relax
  \global\let\@thanks\@empty
  \global\let\@date\@empty
  \global\let\title\relax
  \global\let\author\relax
  \global\let\date\relax
  \global\let\and\relax
}

% Redefine abstract to match Ergo style with margins
\renewenvironment{abstract}
 {\list{}{
    \setlength{\leftmargin}{.25in}
    \setlength{\rightmargin}{\leftmargin}
  }
  \item\relax
  \small}
 {\vskip -3pt plus 0pt minus 0pt\null\endlist}

% Font size adjustments to match Ergo
\renewcommand\normalsize{\@setfontsize\normalsize{10.5pt}{14pt}}
\renewcommand\small{\@setfontsize\small{9pt}{13pt}}
\renewcommand\LARGE{\@setfontsize\LARGE{18pt}{22pt}}
\renewcommand\Huge{\@setfontsize\Huge{9.5pt}{13.5pt}}

\makeatother
% Body text and heading styles mimicking Ergo class
% Add this to your Quarto YAML with:
% format:
%   pdf:
%     include-in-header:
%       - body-style.tex

\usepackage[automark]{scrlayer-scrpage}
\clearpairofpagestyles
\cehead{
  Brian Weatherson
  }
\cohead{
  Humberstone on Possibility Frames
  }
\ohead{\bfseries \pagemark}
\cfoot{}

\makeatletter

% Paragraph indentation and spacing
\setlength{\parindent}{.25in}
\setlength{\parskip}{0pt plus 0pt minus 0pt}

% Section numbering with periods and spacing
\renewcommand*{\@seccntformat}[1]{%
  \csname the#1\endcsname.$\:$%
}

% Section styles (matching Ergo)
\renewcommand\section{\@startsection{section}{1}{\z@}%
  {-4.6ex \@plus 0ex \@minus 0ex}%        % space before
  {2.3ex \@plus 0ex}%                      % space after
  {\normalfont\Large\bfseries}}            % Large, bold

\renewcommand\subsection{\@startsection{subsection}{2}{\z@}%
  {-2.3ex \@plus 0ex \@minus 0ex}%        % space before
  {2.3ex \@plus 0ex \@minus 0ex}%         % space after
  {\normalfont\large\itshape}}             % large, italic

\renewcommand\subsubsection{\@startsection{subsubsection}{3}{\z@}%
  {-2.3ex \@plus 0ex \@minus 0ex}%        % space before
  {2.3ex \@plus 0ex \@minus 0ex}%         % space after
  {\normalfont\large}}                     % large, normal

% Font sizes (matching Ergo)
\renewcommand\normalsize{\@setfontsize\normalsize{10.5pt}{14pt}}
\renewcommand\footnotesize{\@setfontsize\footnotesize{9pt}{11.5pt}}
\renewcommand\small{\@setfontsize\small{9pt}{13pt}}
\renewcommand\large{\@setfontsize\large{11.5pt}{12pt}}
\renewcommand\Large{\@setfontsize\Large{12pt}{14pt}}

% Quotation and quote environments with Ergo spacing
\renewenvironment{quotation}
  {\list{}{
    \setlength{\listparindent}{.25in}%
    \setlength{\leftmargin}{.25in}
    \setlength{\rightmargin}{\leftmargin}
    \setlength{\parsep}{0in plus 0in minus 0in}
    \item\relax
    \let\item\relax}
  }
  {\endlist}

\renewenvironment{quote}
  {\vskip 5pt%
   \list{}{
    \setlength{\listparindent}{.25in}
    \setlength{\leftmargin}{.25in}
    \setlength{\rightmargin}{\leftmargin}
    \setlength{\parsep}{0in plus 0in minus 0in}
    }
    \item\relax
    \let\item\relax}
  {\endlist\vskip 5pt}

% Footnote rule styling
\renewcommand{\footnoterule}{%
  \kern -3pt
  \hrule width 1in height .4pt
  \kern 2.5pt
}

% Non-superscript numerals for footnote text numbering
\renewcommand\@makefntext[1]{%
  \parindent .25in%
  \@thefnmark.~#1}%

% Figure and table captions with period separator
\renewcommand\@makecaption[2]{%
  \vskip\abovecaptionskip
  \sbox\@tempboxa{#1. #2}%
  \ifdim \wd\@tempboxa >\hsize
    #1. #2\par
  \else
    \global \@minipagefalse
    \hb@xt@\hsize{\hfil\box\@tempboxa\hfil}%
  \fi
  \vskip\belowcaptionskip}

% Enumerate environment with extra vertical space
\let\oldenumerate\enumerate
\let\endoldenumerate\endenumerate
\renewenvironment{enumerate}
  {\vskip 5pt\oldenumerate}
  {\endoldenumerate\vskip 5pt}

\let\olddescription\description
\let\endolddescription\enddescription
\renewenvironment{description}
  {\vskip 5pt\olddescription}
  {\endolddescription\vskip 5pt}

\let\olditemize\itemize
\let\endolditemize\enditemize
\renewenvironment{itemize}
  {\vskip 5pt\olditemize}
  {\endolditemize\vskip 5pt}

\newcommand*\NoIndentAfterEnv[1]{%
  \AfterEndEnvironment{#1}{\par\@afterindentfalse\@afterheading}}
\makeatother
\NoIndentAfterEnv{itemize}
\NoIndentAfterEnv{enumerate}
\NoIndentAfterEnv{description}
\NoIndentAfterEnv{quote}
\NoIndentAfterEnv{equation}
\NoIndentAfterEnv{longtable}

\makeatother
% Simple aggressive line breaking for DOIs
\usepackage{xurl}
\PassOptionsToPackage{hyphens}{url}

% Allow breaks at many characters
\def\UrlBreaks{\do\a\do\b\do\c\do\d\do\e\do\f\do\g\do\h\do\i\do\j%
\do\k\do\l\do\m\do\n\do\o\do\p\do\q\do\r\do\s\do\t\do\u\do\v\do\w%
\do\x\do\y\do\z\do\A\do\B\do\C\do\D\do\E\do\F\do\G\do\H\do\I\do\J%
\do\K\do\L\do\M\do\N\do\O\do\P\do\Q\do\R\do\S\do\T\do\U\do\V\do\W%
\do\X\do\Y\do\Z\do\0\do\1\do\2\do\3\do\4\do\5\do\6\do\7\do\8\do\9%
\do\.\do\-\do\/\do\:\do\=\do\?\do\&\do\_}

% Emergency line breaking
\sloppy
\newcommand{\nmodels}{\mathrel{\ooalign{$\models$\cr\raisebox{-0.001ex}{\hss$\mkern-1mu/\hss$}\cr}}}
\newcommand{\llbracket}{[\![}
\newcommand{\rrbracket}{]\!]}
\setlength\heavyrulewidth{0ex}
\setlength\lightrulewidth{0ex}
\usepackage[lines=2]{lettrine}
\cehead{Draft of February 11, 2026}
\makeatletter
\@ifpackageloaded{caption}{}{\usepackage{caption}}
\AtBeginDocument{%
\ifdefined\contentsname
  \renewcommand*\contentsname{Table of contents}
\else
  \newcommand\contentsname{Table of contents}
\fi
\ifdefined\listfigurename
  \renewcommand*\listfigurename{List of Figures}
\else
  \newcommand\listfigurename{List of Figures}
\fi
\ifdefined\listtablename
  \renewcommand*\listtablename{List of Tables}
\else
  \newcommand\listtablename{List of Tables}
\fi
\ifdefined\figurename
  \renewcommand*\figurename{Figure}
\else
  \newcommand\figurename{Figure}
\fi
\ifdefined\tablename
  \renewcommand*\tablename{Table}
\else
  \newcommand\tablename{Table}
\fi
}
\@ifpackageloaded{float}{}{\usepackage{float}}
\floatstyle{ruled}
\@ifundefined{c@chapter}{\newfloat{codelisting}{h}{lop}}{\newfloat{codelisting}{h}{lop}[chapter]}
\floatname{codelisting}{Listing}
\newcommand*\listoflistings{\listof{codelisting}{List of Listings}}
\makeatother
\makeatletter
\makeatother
\makeatletter
\@ifpackageloaded{caption}{}{\usepackage{caption}}
\@ifpackageloaded{subcaption}{}{\usepackage{subcaption}}
\makeatother
\usepackage{bookmark}
\IfFileExists{xurl.sty}{\usepackage{xurl}}{} % add URL line breaks if available
\urlstyle{same}
\hypersetup{
  pdftitle={Humberstone on Possibility Frames},
  pdfauthor={Brian Weatherson},
  colorlinks=true,
  linkcolor={blue},
  filecolor={Maroon},
  citecolor={Blue},
  urlcolor={blue},
  pdfcreator={LaTeX via pandoc}}


\title{Humberstone on Possibility Frames}
\author{Brian Weatherson}
\date{2026-02-11}
\begin{document}
\maketitle
\begin{abstract}
Insert abstract here
\end{abstract}


\setstretch{1}
In his 1981 paper, ``From Worlds to Possibilities'', Lloyd Humberstone
shows a way to do modal logic without the apparatus of possible worlds.
Instead of worlds he uses \emph{possibilities}, which may, unlike
worlds, be incomplete. The non-modal parts of the view are discussed
again in section 6.44 of \emph{The Connectives}, though the differences
between the view there and the 1981 view are largely presentational. In
this paper I'll set out this \emph{possibility frame} approach to modal
logic, make some notes about its logic, and end with a survey of the
many possible applications it has.

Mathematically, possibilities are just points in a model, just like
possible worlds are points in different kinds of models. But it helps to
have a mental picture of what kind of thing they are. In ``From Worlds
to Possibilities'', Humberstone notes that one picture you could have is
that they are sets of possible worlds. This isn't a terrible picture,
but it's not perfect for a couple of reasons. For one thing, as
Humberstone notes, part of the point of developing possibilities is to
do without the machinery of possible worlds. Understanding possibilities
as sets of possible worlds wouldn't help with that project. For another,
as Wesley Holliday (\citeproc{ref-Holliday2025}{2025, 271--72}) notes,
the natural way to generate modal accessibility relations on sets of
worlds from accessibility on the worlds themselves doesn't always work
the way Humberstone wants accessibility to work. So let's start with a
different picture.

Possibilities, as I'll think of them, are \emph{stories}. To make things
concrete, let's focus on a particular story: \emph{A Study in Scarlet}
(Conan Doyle (\citeproc{ref-ConanDoyle1995}{1995})), the story where
Sherlock Holmes was introduced. That story settles some questions, both
explicitly, e.g., that Holmes is a detective, and implicitly, e.g., that
Holmes has never set foot on the moon. But it leaves several other
questions open, e.g., how many (first) cousins Holmes has. It's not that
\emph{A Study in Scarlet} is a story. It has proper parts which are
stories. The first chapter is a story, one which tells of the first
meeting between Holmes and Watson. And arguably it is a proper part of
larger story, made up of all of Conan Doyle's stories of Holmes and
Watson. When a story \(x\) is a proper part of story \(y\), what that
means is that everything settled in \(x\) is still true in \(y\), and
more things besides are settled. When this happens, we'll call \(y\) a
proper \emph{refinement} of \(x\). For most purposes it will be more
convenient to use the more general notion of \emph{refinement}, where
each story counts as an improper refinement of itself.

Following Humberstone, I'll write \(x \leqslant y\) to mean that \(y\)
is a refinement of \(x\). As he notes, this notation can be confusing if
one things of \(x\) and \(y\) as sets, because in that case the
refinement will typically be \emph{smaller}.\footnote{Holliday
  (\citeproc{ref-Holliday2025}{2025}) writes \(y \sqsubseteq x\) when
  \(y\) is a refinement of \(x\), mirroring this way of thinking about
  possibilities.} But if we think of possibilities as stories, the
notation becomes more intuitive. We have \(x \leqslant y\) when \(y\) is
created by adding new content to \(x\). Keeping with this theme, I'll
typically model stories not as worlds, but as finite sequences. (In the
main example in \textbf{?@sec-proof}, they will be sequences of 0s and
1s.) In these models, \(x \leqslant y\) means that \(x\) is an initial
segment of \(y\).

\section{Formal Structure}\label{formal-structure}

To start with, assume we're working in a simple language that just has a
countable set \(\mathcal{P}\) countable infinity of propositional
variables, and three connectives: \(\neg\), \(\wedge\) and \(\vee\). We
have a set of possibilities \(R\), and a transitive refinement relation
\(\geqslant\) on them. The following rules show how to build what I'll
call a \emph{Humberstone possibility model} on
\(\langle R, \leqslant \rangle\). (I'll call this a \emph{possibility
frame} in most contexts, but a \emph{Humberstone frame} when I'm
comparing it to similar structures, especially in the context of
discussing Holliday (\citeproc{ref-Holliday2025}{2025}).)

A Humberstone possibility model \(\mathcal{M}\) is a triple
\(\langle R, \leqslant, V \rangle\), where \(V\) is a function from
\(\mathcal{P}\) to \(R\), intuitively saying where each atomic
proposition is true, satisfying these two constraints:

\begin{itemize}
\tightlist
\item
  For all \(x\), if \(x \in V(p)\) and \(y \geqslant x\), then
  \(y \in V(p)\). Intuitively, truth for atomics is \textbf{persistent}
  across refinements.
\item
  For all \(x\), if
  \(\forall y \geqslant x \exists z \geqslant y: z \in V(p)\), then
  \(x \in V(p)\). This is what Humberstone
  (\citeproc{ref-Humberstone2011}{2011, 900}) calls
  \textbf{refinability}, and it means that \(p\) only fails to be true
  at \(x\) if there is some refinement of \(x\) where it is settled as
  being untrue.
\end{itemize}

Given these constraints, Humberstone suggests the following theory of
truth at a possibility for all sentences in this language.

\begin{align*}
[\text{Vbls}] \quad & \mathcal{M} \models_x p_i \text{ iff } x \in V(p_i); \\
[\neg] \quad & \mathcal{M} \models_x \neg A \text{ iff } \forall y \geqslant x, \, \mathcal{M} \nmodels_y  A; \\
[\wedge] \quad & \mathcal{M} \models_x A \wedge B \text{ iff } \mathcal{M} \models_x A \text{ and } \mathcal{M} \models_x B; \\
[\vee] \quad & \mathcal{M} \models_x A \vee B \text{ iff } \forall y \geqslant x \, \exists z \geqslant y \, : \, \mathcal{M} \models_z A \text{ or } \mathcal{M} \models_z B.
\end{align*}

Given these definitions, it's possible to prove three things. First,
every sentence in the language is persistent. If
\(\mathcal{M} \models_x A\) and \(x \leqslant y\), then
\(\mathcal{M} \models_x A\). For any sentence, truth is always preserved
when moving to a refinement. Second, refinability holds for all
sentences in the language. This is, as Humberstone notes, easier to
state using this definition of \(\neg\). It now becomes the claim, for
arbitrary \(A\), that if \(\mathcal{M} \nmodels_x A\), there is some
refinement \(y\) of \(x\) such that \(\mathcal{M} \models_y \neg A\).
Third, for any set of sentences \(\Gamma\) and sentence \(A\), the truth
at a point of all sentences in \(\Gamma\) guarantees the truth of \(A\)
iff the sequent \(\Gamma\) entails \(A\) in classical propositional
logic.

In this paper, I'm going to discuss three extensions of this language.
I'll introduce them in reverse order of how much they are discussed in
Humberstone, starting with one he does not discuss at all: infinitary
disjunction.

We'll add to the language a new symbol \(\bigvee\), which forms a new
sentence out of any countable set of sentences not containing
\(\bigvee\). Intuitively, it is true when one of the sentences in the
set is true. More formally, its definition of truth at a possibility is:

\begin{align*}
[\bigvee] \quad & \mathcal{M} \models_x \bigvee ({A_1, A_2, \dots})  \text{ iff } \forall y \geqslant x \, \exists z \geqslant y \, : \,\text{ for some } i \, \mathcal{M} \models_z A_i.
\end{align*}

Again, it's fairly simple to show that this addition to the language
will preserve persistence and refinability. But while this is simple, it
is significant, because things could easily have been otherwise.

The second extension will be to add quantifiers, following a suggestion
in Humberstone (\citeproc{ref-Humberstone1981a}{1981}, xxxx). Assume, as
usual, that the language has a stock of names \(c_1, \dots\), and for
each \(n\), a stock of \(n\)-place predicates \(F^n_1, F^n_2, \dots\). A
\emph{first-order (Humberstone) possibility model} is a structure
\(\langle R, \leqslant, D, V \rangle\), where \(D\) assigns a non-empty
domain of objects to each point, and \(V\) interprets the non-logical
vocabulary. More precisely:

\begin{itemize}
\tightlist
\item
  \(D\) is a function assigning to each \(x \in R\) a non-empty set
  \(D(x)\), the \textbf{domain} at \(x\).
\item
  \(V\) assigns to each name \(c_i\) and each \(x \in R\) either a
  designated element \(V(c_i, x) \in D(x)\), or is undefined at \(x\).
\item
  \(V\) assigns to each \(n\)-place predicate \(F^n_j\) and each
  \(x \in R\) a set \(V(F^n_j, x) \subseteq D(x)^n\), the
  \textbf{extension} of \(F^n_j\) at \(x\).
\end{itemize}

These must satisfy the following constraints:

\begin{description}
\tightlist
\item[Domain monotonicity]
If \(x \leqslant y\), then \(D(x) \subseteq D(y)\).
\item[Name coverage]
For each name \(c_i\) and each \(x \in R\), there exists some
\(y \geqslant x\) such that \(V(c_i, y)\) is defined.
\item[Persistence for names]
If \(V(c_i, x)\) is defined and \(x \leqslant y\), then \(V(c_i, y)\) is
defined and \(V(c_i, y) = V(c_i, x)\).
\item[Persistence for predicate extensions]
If \(\langle o_1, \dots, o_n \rangle \in V(F^n_j, x)\) and
\(x \leqslant y\), then
\(\langle o_1, \dots, o_n \rangle \in V(F^n_j, y)\).
\item[Refinability for predicate extensions]
If \(\langle o_1, \dots, o_n \rangle \notin V(F^n_j, x)\), then there
exists some \(y \geqslant x\) such that for all \(z \geqslant y\),
\(\langle o_1, \dots, o_n \rangle \notin V(F^n_j, z)\).
\end{description}

Given a model and a variable assignment \(g\) mapping variables to
objects, truth at a point is defined as follows. Write \(g[v/o]\) for
the assignment that maps variable \(v\) to object \(o\) and otherwise
agrees with \(g\). For a term \(t\), write
\(\llbracket t \rrbracket^{g,x}\) for the denotation of \(t\) under
\(g\) at \(x\): for a variable \(v\) this is \(g(v)\), and for a name
\(c_i\) this is \(V(c_i, x)\) when defined, and undefined otherwise.

\begin{align*}
[=] \quad & \mathcal{M}, g \models_x t_1 = t_2 \text{ iff } \forall y \geqslant x \, \exists z \geqslant y : \llbracket t_1 \rrbracket^{g,z} \text{ and } \llbracket t_2 \rrbracket^{g,z} \text{ are both defined and equal}; \\
[F^n] \quad & \mathcal{M}, g \models_x F^n_j(t_1, \dots, t_n) \text{ iff } \forall y \geqslant x \, \exists z \geqslant y : \langle \llbracket t_1 \rrbracket^{g,z}, \dots, \llbracket t_n \rrbracket^{g,z} \rangle \in V(F^n_j, z); \\
[\forall] \quad & \mathcal{M}, g \models_x \forall v \, A \text{ iff } \forall y \geqslant x \, \forall o \in D(y) : \mathcal{M}, g[v/o] \models_y A; \\
[\exists] \quad & \mathcal{M}, g \models_x \exists v \, A \text{ iff } \forall y \geqslant x \, \exists z \geqslant y \, \exists o \in D(z) : \mathcal{M}, g[v/o] \models_z A.
\end{align*}

The Boolean connectives are handled exactly as in the propositional
case.

The \(\forall\exists\) pattern in the atomic clauses is necessary to
ensure that persistence and refinability hold for all sentences,
including atomic ones. Consider \(c_i = c_i\): if a name has no
denotation at \(x\) but acquires one at some refinement, then a simple
``check the denotation at \(x\)'' condition would leave \(c_i = c_i\)
neither true nor false at \(x\), and no refinement of \(x\) could settle
it as false either, violating refinability. The \(\forall\exists\)
condition handles this correctly: \(c_i = c_i\) is true at \(x\)
whenever \(c_i\) is covered at \(x\) (i.e., every refinement has a
further refinement where \(c_i\) gets a referent), since once \(c_i\)
gets a referent \(o\), persistence of names ensures \(o = o\) at all
further refinements.

The atomic clauses simplify when names are fully defined. If \(t_1\) and
\(t_2\) are variables, or names that already have denotations at \(x\),
then by persistence of names and predicate extensions the
\(\forall\exists\) quantifier prefix collapses:
\(\mathcal{M}, g \models_x t_1 = t_2\) iff
\(\llbracket t_1 \rrbracket^{g,x} = \llbracket t_2 \rrbracket^{g,x}\),
and \(\mathcal{M}, g \models_x F^n_j(t_1, \dots, t_n)\) iff
\(\langle \llbracket t_1 \rrbracket^{g,x}, \dots, \llbracket t_n \rrbracket^{g,x} \rangle \in V(F^n_j, x)\).
The more complex clauses above are needed only to handle the case where
some name occurring in the formula lacks a denotation at \(x\) but is
guaranteed to acquire one.

\phantomsection\label{refs}
\begin{CSLReferences}{1}{0}
\bibitem[\citeproctext]{ref-ConanDoyle1995}
Conan Doyle, Arthur. 1995. \emph{A Study in Scarlet}. Urbana, Illinois:
Project Gutenberg.

\bibitem[\citeproctext]{ref-Holliday2025}
Holliday, Wesley H. 2025. {``Possibility Frames and Forcing for Modal
Logic.''} \emph{Australasian Journal of Logic} 22 (2): 44--288.
https://doi.org/10.26686/ajl.v22i2.5680.

\bibitem[\citeproctext]{ref-Humberstone1981a}
Humberstone, Lloyd. 1981. {``From Worlds to Possibilities.''}
\emph{Journal of Philosophical Logic} 10 (3): 313--39.
https://doi.org/10.1007/BF00293423.

\bibitem[\citeproctext]{ref-Humberstone2011}
---------. 2011. \emph{The Connectives}. Cambridge, MA: MIT Press.

\end{CSLReferences}




\end{document}
