% Options for packages loaded elsewhere
\PassOptionsToPackage{unicode}{hyperref}
\PassOptionsToPackage{hyphens}{url}
%
\documentclass[
  12pt,
]{article}
\usepackage{amsmath,amssymb}
\usepackage{lmodern}
\usepackage{setspace}
\usepackage{ifxetex,ifluatex}
\ifnum 0\ifxetex 1\fi\ifluatex 1\fi=0 % if pdftex
  \usepackage[T1]{fontenc}
  \usepackage[utf8]{inputenc}
  \usepackage{textcomp} % provide euro and other symbols
\else % if luatex or xetex
  \usepackage{unicode-math}
  \defaultfontfeatures{Scale=MatchLowercase}
  \defaultfontfeatures[\rmfamily]{Ligatures=TeX,Scale=1}
  \setmainfont[Scale=MatchLowercase]{Lato}
\fi
% Use upquote if available, for straight quotes in verbatim environments
\IfFileExists{upquote.sty}{\usepackage{upquote}}{}
\IfFileExists{microtype.sty}{% use microtype if available
  \usepackage[]{microtype}
  \UseMicrotypeSet[protrusion]{basicmath} % disable protrusion for tt fonts
}{}
\makeatletter
\@ifundefined{KOMAClassName}{% if non-KOMA class
  \IfFileExists{parskip.sty}{%
    \usepackage{parskip}
  }{% else
    \setlength{\parindent}{0pt}
    \setlength{\parskip}{6pt plus 2pt minus 1pt}}
}{% if KOMA class
  \KOMAoptions{parskip=half}}
\makeatother
\usepackage{xcolor}
\IfFileExists{xurl.sty}{\usepackage{xurl}}{} % add URL line breaks if available
\IfFileExists{bookmark.sty}{\usepackage{bookmark}}{\usepackage{hyperref}}
\hypersetup{
  pdftitle={Insulting Orders: Reply to Mandelkern},
  pdfauthor={Anon},
  hidelinks,
  pdfcreator={LaTeX via pandoc}}
\urlstyle{same} % disable monospaced font for URLs
\usepackage[margin=1.38in]{geometry}
\usepackage{graphicx}
\makeatletter
\def\maxwidth{\ifdim\Gin@nat@width>\linewidth\linewidth\else\Gin@nat@width\fi}
\def\maxheight{\ifdim\Gin@nat@height>\textheight\textheight\else\Gin@nat@height\fi}
\makeatother
% Scale images if necessary, so that they will not overflow the page
% margins by default, and it is still possible to overwrite the defaults
% using explicit options in \includegraphics[width, height, ...]{}
\setkeys{Gin}{width=\maxwidth,height=\maxheight,keepaspectratio}
% Set default figure placement to htbp
\makeatletter
\def\fps@figure{htbp}
\makeatother
\setlength{\emergencystretch}{3em} % prevent overfull lines
\providecommand{\tightlist}{%
  \setlength{\itemsep}{0pt}\setlength{\parskip}{0pt}}
\setcounter{secnumdepth}{5}
\usepackage[italic]{mathastext}
\ifluatex
  \usepackage{selnolig}  % disable illegal ligatures
\fi
\newlength{\cslhangindent}
\setlength{\cslhangindent}{1.5em}
\newlength{\csllabelwidth}
\setlength{\csllabelwidth}{3em}
\newenvironment{CSLReferences}[2] % #1 hanging-ident, #2 entry spacing
 {% don't indent paragraphs
  \setlength{\parindent}{0pt}
  % turn on hanging indent if param 1 is 1
  \ifodd #1 \everypar{\setlength{\hangindent}{\cslhangindent}}\ignorespaces\fi
  % set entry spacing
  \ifnum #2 > 0
  \setlength{\parskip}{#2\baselineskip}
  \fi
 }%
 {}
\usepackage{calc}
\newcommand{\CSLBlock}[1]{#1\hfill\break}
\newcommand{\CSLLeftMargin}[1]{\parbox[t]{\csllabelwidth}{#1}}
\newcommand{\CSLRightInline}[1]{\parbox[t]{\linewidth - \csllabelwidth}{#1}\break}
\newcommand{\CSLIndent}[1]{\hspace{\cslhangindent}#1}

\title{Insulting Orders: Reply to Mandelkern}
\author{Anon}
\date{2021-02-21}

\begin{document}
\maketitle

\setstretch{1.1}
Matthew \protect\hyperlink{ref-Mandelkern2021}{Mandelkern}
(\protect\hyperlink{ref-Mandelkern2021}{2021}) raises a fascinating pair
of puzzles about sentences like this one.

\begin{enumerate}
\def\labelenumi{(\arabic{enumi})}
\tightlist
\item
  \# Clean your room! I doubt you'll clean your room.
\end{enumerate}

One puzzle is that this speech seems defective in some way, even in a
context where each part of it seems acceptable. Parent could properly
want Child to clean their room, have the authority to order Child to
clean their room, utter the first sentence as a proper exercise of this
authority, know that Child is unlikely to follow the order, and utter
the second sentence as an expression of their rationally held pessimism
about child. Every step in this seems fine, but the overall speech is
somehow defective. The first puzzle is to explain why this is so.

A related puzzle, or perhaps a constraint on any solution to the first
puzzle, is that the two sentences in (1) can be uttered in as much
proximity as one likes, as long as they are to different audiences.

\begin{enumerate}
\def\labelenumi{(\arabic{enumi})}
\setcounter{enumi}{1}
\tightlist
\item
  Parent (to Child): Clean your room!\\
  Parent (to Co-parent, resignedly): She won't clean her room.
\end{enumerate}

Mandelkern notes that is fine, in a way that (1) is not. What's going on
here? Note already that (2) should make us suspicious that these are
similar to Moore's paradox. Parent can't say \emph{p} to Child and
immediately say \emph{Maybe not p} to Co-Parent. Moore paradoxical
assertions remain defective if the two parts have different audiences.
But whatever the problem is with (1), it only arises in the case where
there is one audience for the two parts. This will become important in a
bit, but first let's note what Mandelkern himself suggests is the
problem.

The problem with (1), Mandelkern says, is that it violates a principle
he calls \emph{Posturing}.

\emph{Posturing}: When you order someone to \(\varphi\), you must act
towards them as if you believe that they will \(\varphi\).
(\protect\hyperlink{ref-Mandelkern2021}{Mandelkern 2021, 52})

But this principle is too general, and is subject to counterexamples.
Let's look at two of them.

It is sometimes fine to follow up an order to \(\varphi\) with
instructions on how to \(\varphi\). It isn't always fine - often it's
insulting - but sometimes it is fine. So there is nothing wrong with
(3), if Child habitually forgets to brush their teeth.

\begin{enumerate}
\def\labelenumi{(\arabic{enumi})}
\setcounter{enumi}{2}
\tightlist
\item
  Get ready for school! Don't forget to brush your teeth!
\end{enumerate}

Brushing teeth is a constituent part of getting ready for school. By
\emph{Posturing}, Parent should already be acting as if they believe
that Child will get ready for school, and hence brush their teeth, by
the time the second sentence is uttered. But then the second sentence
would feel redundant and hence defective. But it doesn't, so
\emph{Posturing} fails in this case. And indeed it fails in any case
where it is acceptable to order someone to \(\varphi\), and immediately
give them help, or guidance, or instruction, on how to \(\varphi\). And
while many such speeches are annoying, these are at least sometimes
acceptable.

\emph{Posturing} also fails when it comes to omissions. Imagine Parent
has the following dispositions. If Child does clean their room, they
will take them out for ice cream. But Child's preferences over ice cream
shops change in unpredictable ways. So Parent is disposed to ask Child
where they would like to go for ice cream as soon as they believe Child
will actually clean the room. They won't wait until the room is clean,
since they want Child to enjoy the anticipation of going out for ice
cream as soon as they are clearly being cooperative. This all seems like
a coherent set of dispositions. But if Parent has these dispositions,
and \emph{Posturing} is correct, they should immediately follow
\emph{Clean your room!} with \emph{Where would you like to go for ice
cream?}. After all, given Parent's dispositions, the way to act towards
Child as if Parent believes Child will clean their room is to ask them
where they would like to go for ice cream. But if Parent is sceptical
Child will actually follow the order, there is no reason to ask that
question straight away. And more generally, ordering someone to
\(\varphi\) does not automatically licence, let alone require, futher
actions one is disposed to take on believing that they will \(\varphi\).
The omission of those actions is completely appropriate. That is,
\emph{Posturing} does not rule out those omissions.

So if \emph{Posturing} isn't the right story, what is? In the rest of
this note I'll sketch a rival explanation. To motivate this explanation,
consider this variant on the second puzzle.

\begin{enumerate}
\def\labelenumi{(\arabic{enumi})}
\setcounter{enumi}{3}
\tightlist
\item
  Parent (to Child): Clean your room!\\
  Parent (to Co-parent, in a stage whisper): She won't clean her room.\\
  Child (to both parents): I heard that!
\end{enumerate}

Child's response is appropriate, both in tone and content. The angry
tone is appropriate because the stage whisper is insulting. (It might be
a deserved insult, if Child never cleans their room, but it's still an
insult, and fine to get angry about.) But the content is more
interesting for our purposes. Why is it right to complain that the
second sentence was audible? It's because it's much more insulting to
make a negative evaluation of someone in a speech directed to them, than
it is to make it to a third party. Normally how insulting some behavior
is is positively correlated with how disrespectful it is. But this is an
exception; it's more insulting to tell someone they are wrong to their
face, but more disrespectful to say it behind their back.

These features of insults are interesting because they are structurally
the right kind of feature to explain the two puzzles. If the problem
with (1) is that it is impermissibly insulting, that could explain why
it is bad but (2) is not. But here you might worry that we've just
pushed the problem back a stage. Why is (1) insulting? Here we need to
take a little detour into speech act theory.

There are two big ways in which speech acts can go wrong: they can fail
to be authoritative, or they can fail to be apt. If they fail to be
authoritative, then the speaker doesn't really perform the speech act
they purport to perform. If my neighbor's cat walks up to me in the
backyard, and I say \emph{I hereby name you Roaring Kitty}, this goes
wrong because I've purported to perform a naming, but it fact I haven't.
I don't have the authority to rename my neighbor's cat. But other speech
acts that I do have the authority to perform can fail to be apt. Think
about what's at stake in the proper class many papers on norms of
assertion. A defender of the knowledge norm says that I violate a
standard when I assert something I don't know. They don't say that I
fail to even make an assertion; they say I make a defective assertion.

As well as these primary norms on speech acts, there are secondary
norms. As well as making speech acts that are authoritative and apt, one
must talk as if one's speech acts are authoritative and apt. Talking as
if one's speech acts are not authoritative and apt is bad, even if they
in fact are. One theory of what goes wrong in Moore paradoxical
assertions is that they violate this kind of secondary norm. If one says
\emph{p but maybe not p}, the second clause implies the first is not
apt. I'll say that the general secondary norm is that speech acts must
not be self-undermining.

What is it for an order to be authoritative and apt? The authority
condition for orders just is authority. One can only order someone else
to \(\varphi\) if one has sufficient authority. In (5), Child's
utterance fails this condition.

\begin{enumerate}
\def\labelenumi{(\arabic{enumi})}
\setcounter{enumi}{4}
\tightlist
\item
  Parent: Clean your room!\\
  Child: ?? You clean your room!
\end{enumerate}

As \protect\hyperlink{ref-Maitra2012}{Maitra}
(\protect\hyperlink{ref-Maitra2012}{2012}) notes, this authority
requirement is an ex post condition not ex ante. One doesn't need prior
authority to make an order. Instead, one can acquire authority by
purporting to make the order and not being challenged. Our cases are all
ones where Parent does have prior authority, but this ex ante/ex post
distinction will be very important to what comes next.

There isn't any literature that I know of that's directly on the aptness
condition for orders. Here's my hypothesis as to what the aptness
condition is.

\textbf{Must Rule}: One must (Order X to \(\varphi\) only if X must
\(\varphi\))

Like the authority condition, this is an ex post rule. (The ex ante
version is an interesting hypothesis about the aptness condition for
advice.) Orders are apt only if once the order is made, they must be
followed. This is related to the authority condition, since having
authority means being able to create reasons for the receiver to act.
But the aptness condition is in general stronger than the authority
condition. If Parent orders child to go to bed, and Child instead rushes
to the bathroom to avoid an accident, Parent's order was defective, but
it was still an order. In virtue of their authority, and the fact they
made an order, Parent gave Child a reason to go to bed. But Nature gave
them a stronger reason to go to the bathroom.

As evidence for the Must rule, note that declaring that one's own orders
violate it sounds bad, even if the declaration is to a third party.

\begin{enumerate}
\def\labelenumi{(\arabic{enumi})}
\setcounter{enumi}{5}
\tightlist
\item
  Parent (to Child): Clean your room!\\
  Parent (to Co-Parent): ?? She doesn't have to clean her room.
\end{enumerate}

I'm assuming that in context `must' and `have to' are close enough in
meaning for this to be a good test. Note that the Must Rule is, like
other rules for speech acts, a rule that's internal to the speech act,
and not an all-things-considered rule. It has the same status as the
rule that one must only assert truths. This is a correct rule governing
assertion, but not an all-things-considered rule. Contra Kant, one need
not tell the truth to the Terrorist at the door. And sometimes there are
orders that it is all-things-considered good to give, perhaps for the
sake of going on the record, even if the hearer does not in fact have to
do what you order. But that's just to say, sometimes there are reasons
to make defective assertions, or to give defective orders.

The Must Rule looks like it could be part of the solution to the puzzle
because it's bad to say someone might not do what they must do, even
when this is not part of an order. Sometimes saying that someone must
\(\varphi\) is reporting on a pre-existing norm, not in any sense
ordering them. Imagine that A is a member of the Sea Shanty Singers
Union (SSSU), and considers themselves bound by its norms. B is very
knowledgeable about the SSSU, but is not a member, and does not regard
its norms as binding. All of this is common knowledge in the
conversation. A has forgotten what their singing obligations are as a
member.

\begin{enumerate}
\def\labelenumi{(\arabic{enumi})}
\setcounter{enumi}{6}
\tightlist
\item
  A: How often do I have to sing sea shanties?\\
  B: You have to sing every morning when you wake up. \# But you
  probably won't.
\end{enumerate}

Note that B is not ordering A here. Rather, B is reporting on prior
norms that are binding on A. We can see this by noting that B could
continue ``But I think these are absurd rules; you should only sing when
you're on camera.'' Still, we get the same kind of phenomena here that
Mandelkern noted, which is a sign that it's not strictly speaking a
phenomena to do with orders.
(\protect\hyperlink{ref-Mandelkern2021}{Mandelkern}
(\protect\hyperlink{ref-Mandelkern2021}{2021, 43}) notes that if the
norms are not treated as binding on A, then B's utterance is fine. The
point I want to make is that B's utterance is bad even if the norms are
not treated as binding on B, and B thinks they should not be binding on
A.)

So this is the first part of a solution. It's incompatible to say that
hearer must \(\varphi\) and yet that one is unsure about whether they
will. And so it's incompatible to perform a speech act whose aptness
condition is that hearer must \(\varphi\), and express doubt about
whether they will. But this might not look like much progress - it's not
that far yet from Mandelkern's own statement of the problem. (He takes
declaratives like ``You must \(\varphi\)'' rather than imperatives like
``Do \(\varphi\)'' as his primary examples.) To finish the solution, we
need to think about insults a bit more. Consider the following three
exchanges.

\begin{enumerate}
\def\labelenumi{(\arabic{enumi})}
\setcounter{enumi}{7}
\item
  C: I'm going to try to read \emph{War and Peace}.\\
  D: ?? I doubt you'll finish it.
\item
  C: I'm going to try to read \emph{War and Peace}.\\
  (C leaves)\\
  D (to E): ? I doubt she'll finish it.
\item
  C: I'm going to try to read \emph{War and Peace}.\\
  (C leaves)\\
  E (to D): Do you think she'll finish it?\\
  D (to E): I doubt it.
\end{enumerate}

In (8), D has insulted C. In (10) they have not. I'm not entirely sure
what to say about (9); it's at least disrespectful to bring this up
first thing after C leaves, but whether it is insulting is tricky. Stil
the contrast between (8) and (10) is instructive. It's in general
insulting to say to someone's face that they probably won't succeed in
something they are trying to do. But if you reasonably believe they are
unlikely to succeed, it isn't insulting to say this to a third party (if
the topic is appropriate).

In fact, the non-insulting responses to C line up fairly well with the
non-defective continuants of an order like \emph{Read War and Peace!}.
It's not fine to say to them you doubt they will do it. It is fine to
say that to a third party. (One necessary condition: these doubts are
reasonable. One desiderata: the question of whether they will succeed is
already salient.) It's fine to respond with helpful advice about how to
succeed. (Note this is a thoroughly externalist norm. The advice must
actually be helpful, and helpful to the advisee, not just to the goal.)
And it's fine to not take further actions that would be appropriate if
you thought they'd succeed. (You don't have to offer suggestions for
follow up reading, even if you would make those offers were you to
believe they'd succeed.) In general, there is a surprising correlation
between what D can say to C without insult, and what one can say
following up an order.

Furthermore, it's disrespectful to believe of someone that they won't
even try to do a thing they must do. And it is insulting to say this, or
to imply it, or to say something that presupposes it.

Now we have all the pieces together to explain (1). One of the following
three things must be true when (1) is uttered.

\begin{enumerate}
\def\labelenumi{\roman{enumi}.}
\tightlist
\item
  Child does not have to clean their room.
\item
  Child has to clean their room, but they won't even try to clean their
  room.
\item
  Child will try to clean their room, but Parent doubts they will
  succeed.
\end{enumerate}

If i is true, then the initial order is inapt. Since it is bad to
undermine your own speech acts, Parent can't be read as taking i to be
the case. So ii or iii is true. But saying either of these things, or
even their disjunction, is insulting. And it's bad to be insulting in
just this way.

This story explains the contrast between (1) and (2). If iii is true,
Parent can order Child to clean their room, and not insult Child by
expressing doubts to Co-Parent that Child will do what they have to do.

I'll end with one objection, the objection that is most troubling to me.
I've argued that (1) is either defective or insulting. Since it's bad to
have a defective utterance, and bad to be insulting, it sounds bad. But,
says the objector, we know what insults sound like. They don't always
sound bad. Nothing wrong with saying to a colleague ``You're the Ted
Cruz of this department'' if they have warranted such a comparison. And
when they do sound bad, like when you say this to a colleague who is
actually good, they don't sound anything like Moore paradoxical
sentences. But (1) doesn't just sound bad, it sounds bad in something
like the way Moore paradoxical sentences do. How do we explain this?

I have three replies, in increasing order of responsiveness. The last is
the most speculative, but would be most persuasive if it works. So let's
build to it.

First, we should be a little sceptical that intuition can tell us not
just that an utterance has gone wrong, but how it has gone wrong. One of
the lessons of Gricean pragmatics is that intuition is a bad guide to
what errors there are in a sentence. It leads smart people to think
utterances which are true but misleading are actually false. So worst
case scenario, we could just reject the intuition behind this objection
here as yet another one that had to be corrected.

Second, we know that (1) can't be exactly like other Moore paradoxical
sentences because, as Mandelkern noted in the very setup of the problem,
the problems with it go away if you change audiences between the two
parts. And this is not the case with regular Moore paradoxical
sentences. So whatever solution we offer, it couldn't and shouldn't make
(1) be exactly like other Moore paradoxical sentences.

Third, it is possible that the principle of charity is tripping us up
here. If we hear someone uttering (1), we know they are either being
self-undermining or insulting. Linguistically, it seems worse to be
self-undermining. But morally, it's worse to be insulting. And if we
have to make an all-things-considered judgment, it's probably better to
violate linguistic norms than moral norms. Bad to be a fool, but worse
to be a knave. So we charitably interpret (1) as self-undermining,
because that's a more charitable interpretation than taking it to be an
insult. That's why it sounds like (1) is self-undermining; it is
self-undermining on its most charitable interpretation.

If this theory sounds far-fetched, and offhand it does sound far-fetched
to me, note that (4) provides some evidence for the theory. Parent's
second utterance there sounds not nearly as bad as the second sentence
in (1). If that's right, it's mysterious. It is a really blatant
counterexample to \emph{Posturing}, and otherwise the counterexamples to
\emph{Posturing} involve roundabout things like relevance conditions and
omissions. What could be going on? My guess is that (4) sounds better
because it is so obviously insulting. We don't hear it as defective
because it's so clear that Parent invites us to hear it as insulting.
But in (1), no such invitation is on offer, and so we look for an
alternative explanation, even if it involves the attribution of a
linguistic error.

\hypertarget{references}{%
\subsection*{References}\label{references}}
\addcontentsline{toc}{subsection}{References}

\hypertarget{refs}{}
\begin{CSLReferences}{1}{0}
\leavevmode\hypertarget{ref-Maitra2012}{}%
Maitra, Ishani. 2012. {``Subordinating Speech.''} In \emph{Speech and
Harm: Controversies over Free Speech}, edited by Ishani Maitra and Mary
Kate McGowan, 94--120. Oxford: Oxford University Press.
\url{https://doi.org/10.1093/acprof:oso/9780199236282.003.0005}.

\leavevmode\hypertarget{ref-Mandelkern2021}{}%
Mandelkern, Matthew. 2021. {``Practical Moore Sentences.''}
\emph{No{û}s} 55 (1): 39--61. \url{https://doi.org/10.1111/nous.12287}.

\end{CSLReferences}

\end{document}
