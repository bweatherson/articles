% Options for packages loaded elsewhere
\PassOptionsToPackage{unicode}{hyperref}
\PassOptionsToPackage{hyphens}{url}
%
\documentclass[
  12pt,
]{article}
\usepackage{amsmath,amssymb}
\usepackage{lmodern}
\usepackage{setspace}
\usepackage{iftex}
\ifPDFTeX
  \usepackage[T1]{fontenc}
  \usepackage[utf8]{inputenc}
  \usepackage{textcomp} % provide euro and other symbols
\else % if luatex or xetex
  \usepackage{unicode-math}
  \defaultfontfeatures{Scale=MatchLowercase}
  \defaultfontfeatures[\rmfamily]{Ligatures=TeX,Scale=1}
  \setmainfont[Scale=MatchLowercase]{Lato}
  \setmathfont[]{Fira Math}
\fi
% Use upquote if available, for straight quotes in verbatim environments
\IfFileExists{upquote.sty}{\usepackage{upquote}}{}
\IfFileExists{microtype.sty}{% use microtype if available
  \usepackage[]{microtype}
  \UseMicrotypeSet[protrusion]{basicmath} % disable protrusion for tt fonts
}{}
\makeatletter
\@ifundefined{KOMAClassName}{% if non-KOMA class
  \IfFileExists{parskip.sty}{%
    \usepackage{parskip}
  }{% else
    \setlength{\parindent}{0pt}
    \setlength{\parskip}{6pt plus 2pt minus 1pt}}
}{% if KOMA class
  \KOMAoptions{parskip=half}}
\makeatother
\usepackage{xcolor}
\usepackage[margin=1.4in]{geometry}
\usepackage{longtable,booktabs,array}
\usepackage{calc} % for calculating minipage widths
% Correct order of tables after \paragraph or \subparagraph
\usepackage{etoolbox}
\makeatletter
\patchcmd\longtable{\par}{\if@noskipsec\mbox{}\fi\par}{}{}
\makeatother
% Allow footnotes in longtable head/foot
\IfFileExists{footnotehyper.sty}{\usepackage{footnotehyper}}{\usepackage{footnote}}
\makesavenoteenv{longtable}
\usepackage{graphicx}
\makeatletter
\def\maxwidth{\ifdim\Gin@nat@width>\linewidth\linewidth\else\Gin@nat@width\fi}
\def\maxheight{\ifdim\Gin@nat@height>\textheight\textheight\else\Gin@nat@height\fi}
\makeatother
% Scale images if necessary, so that they will not overflow the page
% margins by default, and it is still possible to overwrite the defaults
% using explicit options in \includegraphics[width, height, ...]{}
\setkeys{Gin}{width=\maxwidth,height=\maxheight,keepaspectratio}
% Set default figure placement to htbp
\makeatletter
\def\fps@figure{htbp}
\makeatother
\setlength{\emergencystretch}{3em} % prevent overfull lines
\providecommand{\tightlist}{%
  \setlength{\itemsep}{0pt}\setlength{\parskip}{0pt}}
\setcounter{secnumdepth}{5}
\newlength{\cslhangindent}
\setlength{\cslhangindent}{1.5em}
\newlength{\csllabelwidth}
\setlength{\csllabelwidth}{3em}
\newlength{\cslentryspacingunit} % times entry-spacing
\setlength{\cslentryspacingunit}{\parskip}
\newenvironment{CSLReferences}[2] % #1 hanging-ident, #2 entry spacing
 {% don't indent paragraphs
  \setlength{\parindent}{0pt}
  % turn on hanging indent if param 1 is 1
  \ifodd #1
  \let\oldpar\par
  \def\par{\hangindent=\cslhangindent\oldpar}
  \fi
  % set entry spacing
  \setlength{\parskip}{#2\cslentryspacingunit}
 }%
 {}
\usepackage{calc}
\newcommand{\CSLBlock}[1]{#1\hfill\break}
\newcommand{\CSLLeftMargin}[1]{\parbox[t]{\csllabelwidth}{#1}}
\newcommand{\CSLRightInline}[1]{\parbox[t]{\linewidth - \csllabelwidth}{#1}\break}
\newcommand{\CSLIndent}[1]{\hspace{\cslhangindent}#1}
\hypersetup{hidelinks}
\ifLuaTeX
  \usepackage{selnolig}  % disable illegal ligatures
\fi
\IfFileExists{bookmark.sty}{\usepackage{bookmark}}{\usepackage{hyperref}}
\IfFileExists{xurl.sty}{\usepackage{xurl}}{} % add URL line breaks if available
\urlstyle{same} % disable monospaced font for URLs
\hypersetup{
  pdftitle={The Sporting Attitude},
  pdfauthor={Brian Weatherson},
  hidelinks,
  pdfcreator={LaTeX via pandoc}}

\title{The Sporting Attitude}
\author{Brian Weatherson}
\date{2022-08-23}

\begin{document}
\maketitle

\setstretch{1.1}
Steffen Borge's \emph{The Philosophy of Football} (Borge 2019) is a really great contribution to philosophy of sport. More than that, it shows how questions in metaphysics, aesthetics, and philosophy of mind can be illuminated by looking at them through the perspective of sport. I'm going to focus on one particular question he raises, primarily in chapter 3. What attitude towards a game should players take? What is, to use Suits's terminology (Suits 1978), the lusory attitude that goes along with playing a sport.

There are actually three distinct questions here that are worth separating. I'm going to start with the first, but as we'll see, I'm going to end up having more to say about the second and the third.

\begin{enumerate}
\def\labelenumi{\arabic{enumi}.}
\tightlist
\item
  What attitude to the game must players have if they are to play the game?
\item
  What attitude to the game should players have if they are playing the game?
\item
  What attitude must players generally have if the game is to be the game that it is?
\end{enumerate}

Borge argues that Suits's answer to question 1 is much too strong, and I'm mostly inclined to agree. But I think the answers he gives to 1 and 3 are too weak. And thinking about question 2 will help us see why.

To start, let's think about why we might be interested in question one in the first place. Imagine someone, call him George, who stands on a football field, but doesn't act like a footballer. When the ball comes to him, he catches it, or picks it up, and runs towards the opposite goal line. If he makes it there, he places the ball over the goal line in celebration. George isn't playing football - he's playing rugby. (Or, perhaps, he's trying to play rugby and not really succeeding at playing anything, since there clearly isn't a rugby game going on.) Now imagine someone else, call him Webb, who tries to be like George, but fails. He really wants to pick the ball up and run with it. And he tries to do this repeatedly. But he fails every time. He never lays a hand on the ball in fact. Is Webb playing football?

I think he's not, or at least that there is an important sense in which he is not. And this is a hard thing to capture. Webb isn't playing football well, since he isn't ever involved with the play. But actually kicking the ball, or even doing any kind of football like move, isn't essential to playing football. (See, for example, some of the less impressive performances of Mesut Özil's Arsenal career.) The problem with Webb is that he's trying to play a completely distinct game. It was easy to say why George was not playing football; he was gratuitously breaking the rules. But Webb is not breaking the rules. What's wrong with Webb, what makes him a non-footballer, is something mental.

Now at this stage you might be tempted to say that the problem with Webb is that he isn't trying to follow the rules. But, as Borge points out, this can't be the story\footnote{See the `Fistful of Fouls' example on page 139.}. A defender who grabs an opponent's jersey - just hard enough to not get penalised - is still playing football. A winger who drags an opponent back to prevent a counterattack - and knows that a yellow card will follow - is still playing football. To use an example we'll come back to a bit, Luis Suárez was playing football when he pulled off that impressive, but totally illegal, save in the 2010 World Cup. You can play football while deliberately, knowingly, breaking the rules of football. So if the problem with Webb is that he has the wrong attitude, what attitude that he lacks must you have to count as a football player?

I'm going to argue that playing a sport requires taking the rules of that sport as providing reasons against certain actions. To play football is, among other things, to regard oneself as having a reason to not handle the ball. (Except as a goalkeeper, or during a throw-in, etc.) This reason can be outweighed, but never defeated. Even if handling the ball is the right thing to do all things considered, one has an outweighed but undefeated reason to not do it. That's the attitude that's essential to playing football. Or, more precisely, playing football requires being part of a game where almost all the players have that attitude almost all the time.

To get to this conclusion, I'm going to start by looking at Suits's view that being a player requires treating the rules as binding; that one is not playing the game is the rules are broken. This requires reconsidering what the rules are, and I'm going to broadly agree with Borge's critique of this reconsideration. Then I'm going to go over my positive view, that being a player requires treating rules as providing reasons. Then I'll compare this view to Borge's view; the views might not be that far apart. And finally I'll talk about the view of rules as reasons can be strengthened by incorporating D'Agastino's view that games have an ethos, and playing the game requires upholding that ethos.

A simple way to relate rules to player attitudes is to say that playing the game requires treating the rules as binding. On the face of it this is absurd; players commit fouls, even intentional fouls, in every game. A way to make it plausible is to reinterpret the rules so that they are more or less never broken. Now this might seem absurd - a defender grabbing an attacker who is running by is breaking the rules. But as Borge discusses\footnote{See the ideas for how to flesh out a Suitsian view on pages 154ff.}, you don't have to think about rules this way. You cold say the rule is not \emph{Don't grab other players}. Instead, the rule is \emph{If you grab another player then (ceteris paribus), the other team gets a free kick}. The defender isn't breaking that rule. It's true that they do something that leads to the other team being awarded something by the referee. But a defender who kicks the ball into touch to stop an attack also does something that leads to the other team being awarded something by the referee, and they aren't breaking any rules. On this way of thinking, all rules are like the rules about what happens when the ball goes out of play, and players do not deliberately break those rules.

We can put the same point in Kantian terms. We ordinarily think of rules as being categorical imperatives, like \emph{Don't grab other players}, that players break. The view I'm interested in here is that rules are hypothetical imperatives, like \emph{If you grab another player then (ceteris paribus), the other team gets a free kick}. And while these might be broken too, some fouls are never called, players do not intend to break them. Indeed, it's not clear that a player could intend to break them. The few categorical imperatives there are, like \emph{Don't use a sword while playing football}, are clearly followed by the players.

This is a plausible model for some sports. In particular, it seems like a not absurd model for cricket. At first it might look like cricket has a number of rules for proper bowling, like that you must not overstep the crease when bowling, and you must not straighen your arm when bowling. But on closer look, it is plausible that some of these are hypothetical imperatives. The overstepping rule is really a conditional - if you overstep then the batting team is awarded a run (and some other things). The bowler isn't breaking a rule when they overstep, they are just doing something that results by rule in good results for the other team. In that respect, they are just like a fielder whose overthrow goes to the boundary. On the other hand, the rule about straightening your arm is a categorical imperative: you must not do that. And we can see that from the fact that the match officials' duty is not to penalise this kind of bowling, but to prevent it. So it's plausible that in cricket, most rules should be understood as conditionals, and players intend to conform to them.

But this is not a particularly plausible model for football. We can see this by considering a pair of cases. In each case, an attacker has the ball at the corner flag, and is about to cross to an unmarked teammate near the penalty box. In the first case, defender Ellie prevents the cross by sliding in and cleanly kicking the ball over the goal line. In the second case, attacker Sam, who isn't as good at this, prevents the cross by sliding in and bundling the ball, the attacker, the corner flag and the watching sideline official over the goal line. In both cases, the immediate thing the officials should do is award an unobstructed kick to the attacking team by the corner flag. If rules are hypothetical imperatives, then in an important sense Sam and Ellie did the same thing. They triggered a hypothetical that leads to the other team getting a reward of an unobstructed kick by the corner flag. But surely that leaves something out. What Ellie did was great defending, and what Sam did was foul play. This suggests that we want some notion of rules in which Sam was breaking the rules, and Ellie was not. If the Suitsian model says otherwise, it is wrong. But the Suitsian can't go on to say that what Sam did was against the rules, because then it would imply that being a player means intending to not do what Sam did, and that is clearly wrong.

Borge discusses some examples like this one, and gives two further arguments as to why the view I'm discussing gets the case wrong. I'm sympathetic to Borge's conclusions, but both the arguments seem to need further refinement. And working through is interesting because it reveals how hard it is to put one's finger on what distinguishes the cases.

Borge's first argument\footnote{Again, I'm focussing on the discussion on pages 143ff.} is that we need to say Sam broke the rules to explain why we added extra penalties in the 1980s and 1990s against this kind of foul play. But I suspect this is easier to explain than Borge thinks. Sports, especially football codes, change rules all the time to discourage behaviour they want to see less of. In Australian football, if a defender carries a ball into their own goal, the other team gets one point, and traditionally the defending team got a goal kick. Since the alternative might be giving up a goal, worth six points, this was often a sensible play. It was so sensible, and became so prevalent, that the rules were changed to discourage it, replacing the goal kick with a jump ball near the goal. The fact that administrators of the game changed the penalties for certain tackles doesn't show that those tackles were against the rules, it might just show they wanted less of those kinds of things. (Compare too the change to the rules of football to ban handling back passes.) And the fact that Sam might get a red card for this tackle doesn't even show it is against the rules. Dangerous play can get a red card even if it isn't a rule violation. There isn't a rule against kicking the ball hard and straight at an annoying fan pitchside, but it could be a red card if the kick was too hard and straight.

Borge's second argument is that the I'm imagining is too revisionist. We talk as if there are laws of the game, rules, that Sam broke and Ellie did not. And while this is true, I don't think we should be too concerned about this. That's in part because there are sports like cricket where this kind of revisionism seems on reflection plausible. But it is in part because of things internal to football. We talk about tackles like Sam's being against the Laws of the game. But we also talk about being offside as against the Laws. It certainly triggers the exception clause (unless there has been a violation of the Laws) in the clause about a definition of a goal. And the story I'm telling seems fine, and perhaps quite plausible, for off side. You can't get carded for repeatedly being offside, even if like Inzaghi you were born in an offside position. If there is a distinction between Sam's case and Ellie's case, it doesn't just feel like we talk as if Sam's action was against the rules (or the Laws), and Ellie's was not.

Still, I think is a key difference between the cases. And I suspect it does cause a problem for this view. Here is one way to see the difference between the cases. Imagine Sam gets away with just a yellow card for her tackle, so both versions of the story continue with the defending team gathering in the penalty box to defend a set piece. In a normal football game, the reactions of the defending team would be different in the two cases. Ellie would be getting fist bumps or other signs of appreciation at a job well done. But it would be very poor form for Sam's teammates to react in the same way. That's true even though doing what Sam did,triggering the condition of a hypothetical imperative, improved her team's position just as much as what Ellie did. Being a football player involves taking a certain attitude towards actions, and that attitude requires distinguishing Sam and Ellie's attitude.

There is a famous real life example of this: Luis Suárez's handball on the goal line against Ghana in the 2010 World Cup. On the rules as hypothetical imperatives model, the rules played out to perfection in this case. A penalty was awarded against Suárez's team, and he was given a red card and a suspension. But this benefited his team, since the penalty was missed, and his team went on to win a game they surely would have lost otherwise. On the view that rules are hypothetical imperatives, then what Suárez did was great football, just like Ellie in the fictional example. But that all seems wrong. A lot of people in the game thought that it was unseemly of Suárez to be so proud of what he did. Yet why shouldn't he have been proud?

Both of these cases can be explained if we understand rules as categorical imperatives, and the players' attitude towards them not as binding constraints, but as providing reasons. Football includes a rule against handling the ball, and a rule against kicking other players. It also provides penalties for breaching these rules. But the force of the rules is not exhausted by penalties. The rules provide reasons that can be outweighed by other considerations, but never defeated. That's why we don't celebrate tackles like Sam's, or saves like Suárez's. They have done something that may have increased the team's win probability, but which they had reason not to do. And their teammates share those reasons. Celebrating the action is a kind of complicity in wrongdoing.

Borge's view about the lusory attitude is similar to this, but I think a little different. He says that football players have to ``endure, obey or accept the arbitration of the rules of football'' (150). Or, as he'd put it previously, the players have to ``defer to the referee and \ldots{} respect his decisions''\footnote{Borge (2010, 164), as cited on page 150}. (I'm simplifying a bit here, not least by blurring the participant/practitioner distinction.)

Now there is an obvious objection to this view. Players clearly do not respect the authority of the referee. It is a commonplace to see them surrounding the referee after an adverse decision complaining about it, and trying to cajole the referee to change their mind. If a defendant in a criminal trial reacted to a judge's verdict this way, they'd be held in contempt of court. And it is hard to square respect with contempt.

Borge should, I think, say that respecting the authority of the referee is better understood not in its ordinary usage, but just in the sense that the players do what the referee says. Maybe they complain about the mistaken award of a corner, but they don't just take a goal kick if the referee is unmoved. That's to say, the term `endure' in the first quote above is important; it's what players most often do.

But even this would be too strong a claim. Let me give just one amusing example. In 2002, I was watching the Germany-Ireland World Cup game in a bar in London. It ended with a stoppage time equaliser by Robbie Keane which brought the house down. But before that the most striking moment was an otherwise routine Ireland free kick. Germany lined up a wall, and the referee clearly said where they were supposed to stand. The camera operator, in a moment of genius, focused on the feet of the German players as the referee walked away. And as soon as his back was turned, four pairs of feet started shuffling forward in unison. The bar erupted in laughter. The lesson for us is that the players don't have to respect the referee in the sense of doing what he says, or even endure his decisions; if they can get away with it they will just do something else.

A better idea, not far from Borge's I think, is to say that the players don't have to respect the referee, but they do have to respect the rules of the game. Now this might seem absurd, in light of the examples of gratuitous rule breaking that we've used. But I think we can see why something like it is right if we step away from Germans and Uruguayans at World Cups, and imagine a park game. Thinking about games that are low stakes, and so the incentive to win at all costs is reduced, will help us get a better sense of what's permissible.

Imagine Lisa is playing a game where there is a wall running down one sideline not far from the field of play. At one stage, Lisa is trapped with the ball near that sideline. She realises that a clever little bounce pass to herself off the wall will let her get out of the trap, and she executes it with aplomb. Now this might be a fun thing to do in practice, but it's really not compatible with playing. When she does this, she has to some extent ceased to be a football player, and instead become someone who likes to show off football skills.

Of course, Lisa won't get any advantage from this, because the referee will simply award a throw in to the opposition. At least, the referee will probably do that. But maybe the referee will be unsighted, or incompetent, and will not award the throw. Still, it was wrong for Lisa to do that. It's part of football that walls are not in play, and being a player requires acting as if that's true.

If we imagine an incompetent referee, then we can push intuitions about cases like this even further. Imagine that Lisa goes on to notice that the referee either can't or won't penalise players for using their arms to control passes that come in at chest height. So every time she receives a pass to her chest, she uses her arm to help cushion the ball. The opposition are infuriated, she isn't being subtle about it, but the referee doesn't stop her, so she keeps on doing it. And eventually she gets a goal.

I think she's doing something wrong here. And I suspect, though perhaps cultural norms will vary a bit on this point, that if it is too blatant and the stakes are low enough, her teammates won't be impressed either. They came to play a football game, and she's making a mockery of it. Maybe they won't celebrate the goal she gets by cheating this way, or maybe they will join in the opposition's remonstration. Why don't they just applaud her contribution to winning? The picture of rules as reasons explains this nicely I think. The rule against handball provides a reason for every player to not handle the ball. Maybe in a game with a huge amount at stake, the stakes override that reason. But in a park game, where the benefit of rule breaking is merely that Lisa gets a bit better control over the ball, that reason should be decisive. To the extent that she doesn't treat it as decisive, she is undermining the sense in which they are playing football. And this can be true even if the referee won't call this kind of foul.

I think, and again I could be wrong, that the players would react very differently to Lisa than they'd react to the kind of ordinary shirt pulling and soft fouling that goes on at most corners. There is something particularly disrespectful about what Lisa is doing that doesn't extend to fouls that everyone does all the time. And this is true even if Lisa would, were the referee to call her for a foul, be willing to shrug and hand the ball to the opposition for a free kick. (And then stand a foot closer than the referee said was allowed.) This is a puzzle, and I am not convinced Borge's theory of what it is to play football can account for it.

The right thing to say here draws on a view of Fred D'Agostino's that Borge discusses\footnote{See D'Agostino (1981) for the original, and pages 144-148 of Borge's book for the discussion.}. A sport has an ethos. This can't be derived from the written rules of the game, but is something like the collective spirit in which it is played. In D'Agostino's version, this provides the unbreakable rules of the game. The ethos says that if you do this or this, you're no longer playing the game. This is too strong, as Borge points out. But something like it is right. My preferred version is that the ethos of the game provides the strength of reasons that go along with each rule. Currently the ethos says that shirt tugging at corners is something one has little reason to avoid, handball is something one has strong reason to avoid, and tackles from behind one has stronger reason still to avoid. But these aren't essential to playing football; it was the same game when the strength of reasons were different.

In most cases in football, the strength of reasons is just what you might expect from a minimal familiarity with the game, combined with the fact that player safety is in everyone's interests. But in other sports you need something like an ethos to explain a lot of what we see. In both cricket and baseball, a player on the batting side is out if they hit the ball and it is caught by a fielder before touching the ground. And in both sports there are hard cases where the ball, the fielder, and the ground come together almost simultaneously. But the sports treat these cases very differently. In cricket it is very poor form to appeal for a catch unless you are confident you caught the ball, and if you believe you did not catch it, you should say so to the officials. In baseball, you appeal for everything and leave it up to the officials to make the decisions. These principles are followed from the lowest levels of the game to the highest. You couldn't derive them from the rules of the game, or from the idea that players should respect the rules and the officials. You need to appeal to something like D'Agostino's idea of ethos to explain the difference between the sports.

But if an ethos is so essential to a sport, does that mean that players in communities with a different ethos are literally playing different games? As Borge points out, this would be an absurd result\footnote{I'm drawing here on his discussion of Ghafoor Jahani at the 1978 World Cup, on page 148.}. His example involves a World Cup team not used to the stricter refereeing in international games. But you don't need to go that far afield. I've heard that in England it can be a debacle when a Premier League referee takes charge of a Championship game, because the players just one level down are used to getting away with much heavier tackles than a referee who has to look after superstars in the top flight will allow. Now here's the objection. If the ethos is essential to the game, and the ethos is different in the Premiership and the Championship, then it follows they are playing a different sport in the Premiership and the Championship. And that's a reductio of the position.

This criticism relies on reading too much into the notion of an ethos. It's true that in a colloquial sense, the game has a different ethos in a place where a certain tackle is commonplace to what it has in a place where that tackle is routinely penalised. But this isn't what D'Agastino meant by ``ethos'', and it isn't what I mean. In D'Agastino's version, it concerned what was simply not to be done. The teams who are used to lighter refereeing typically won't do things that teams used to stricter refereeing simply won't do. The things they get penalised for all the time are part of the repetoire of the more mannered teams; it's just that those teams don't do them as often. So I'm not sure these are examples of difference in ethos in D'Agastino's sense. And they need not be differences in my sense either. As I'm using the term, the ethos of a game tells you what reasons you have to not do certain actions beyond what penalties will be applied to those actions. Changing the penalties doesn't even look like something that changes the non-penalty reasons.

But the bigger point to make in reply turns on the fact, much stressed by Borge, that football is social. Indeed, it is social twice over. Whether one is playing football at a given moment is a social fact. Whether I am reading a book at a moment is largely up to me. But there is literally nothing I could do right now, sitting at my computer with no one around, that would make it the case that I was playing a game of football. For that I would need teammates, and opponents (and for that matter a field) and none of them are to hand. But that doesn't exhaust how social football is. As Borge argues in chapter two, what makes it the case that various token games are tokens of the kind \emph{football} consists largely of social facts as well. Once we take these things into account, we can see that appeal to something like an ethos of football won't make it the case that people with different attitudes are playing different games.

There is an objection to the whole project of this paper that you might have been considering, and which it is finally time to address. I've been asking what attitude is required to play football. And at some level the answer is that literally anything goes. If there is a field of the right kind, and 22 other people on it - 10 of them your teammates, 11 of them opponents, and a referee - and they are doing paradigmatically football type things, then as long as you're in uniform you're playing football. Short of pulling out a weapon and assaulting people with the weapon, there is little you could do that would count as not playing football, as opposed to playing badly. So how can we talk about the attitude that is necessary for playing football?

Well, we can still talk generically about what the players in general must think and feel in order for there to be a game. Exceptions can be tolerated. It is easy to come up with extreme cases. Imagine an East German player playing in France in the 1960s, and spending the whole game looking for the safest moment to defect. Or imagine a girl from an area where scouts never venture, finally getting a chance to play in front of a scout, and for this game only caring about how impressive her play is. It will be hard to come up with any plausible story about the attitude of football players that covers their attitudes, yet they are still playing football. But those exceptions can be tolerated, as long as they are exceptions. If everyone is looking for a chance to defect, it isn't really a game, it's an escape attempt. If everyone is just looking to impress the scouts, it's an exhibition or a scrimmage, not a game. What we're after here is what must be true in general.

Because to a pretty close approximation, all it takes to be playing football is to be part of a football game. And being part of it might literally just mean wearing the right kit, and being on the right field. And it being a football game is a matter of this game standing in the right social relations to games of football across space and time. Neither requires any player have attitudes of any kind. But we can ask what attitudes, if any, are necessary to be generic across the players in this game for it to stand in the right relations to the class of all football games. And we can ask what attitudes, if any, are necessary to be generic across the players in all games if those games are to be, collectively, football.

And like as above, I think an account in terms of reasons is basically right. What makes the players across all the football games the world over players of the same game? I think it's because they are, generally, taking the rules to provide reasons to act, and not act, in certain ways. There is massive variation within these. At a junior enough level, they are barely cognisant of the rules, and so cannot take them as reasons. At a high enough level, they might be so focused on winning that they care little for the rules beyond the fact that rule violations might lead to penalties. But it would have to be a very jaded team that celebrates Sam just as much as they celebrate Ellie; even at the highest level, players' reactive attitudes tend to generally acknowledge the reason-giving force of the rules.

I'll close by considering two related problems. If the games are associated with the way players take the rules to be reasons, that suggests that games are individuated much too finely. If here we regard the rule against handball as having just this strength as a reason, and over there they regard it as having a little less strength, then are we not both playing football? That would be absurd. And if little kids don't understand the rules as having reason-giving force, because perhaps they don't understand the rules at all, are they playing a different game? This seems wrong, since we can talk about someone having played football since they were four.

There are two points to note about `football' that are relevant to both of these objections. The term is vague. Whether these five year olds kicking a ball around a small field, with no throw ins, goal keepers, headers, or offside rules, are playing football is a bit vague. There is a sense in which they are, and a sense in which they are not. And even given a precisification, the question of whether two people are playing the same sport, or the same game, doesn't always correspond to the meaning of the name of the game, or games, they play. The same thing happens with language. How widely is English spoken? Do folks speak the same language in Glasgow, Pittsburgh and Sydney? There is a sense in which they are speaking different languages - they certain have a very different lexicon. But there is a perhaps more important sense in which they are speaking the same language, and that language is English. Is 50-over cricket the same sport, or the same game, as 20-over cricket? There is a sense in which the answer is yes, and a sense in which the answer is no. (I'm actually kind of surprised at how much the infrastructure around cricket supposes the no answer - the games are more similar to each other than either is to junior cricket.) The same happens here. It's true on my view that there is a sense in which players who differ in what strength they give to the rules of football are playing different games, just like 50-over and 20-over cricket might be different games. But just like those are both games of cricket, and like the folks in Glasgow, Pittsburgh and Sydney are all speaking English, the players might all be playing football.

While I've disagreed, at least on points of emphasis, with Borge, I want to close by expressing again my appreciation for his book. Philosophy is richer when it engages with real life, especially with those aspects of real life that make less sense the more you think about them. And his book is a great example of this kind of engagement, and is rewarding reading for anyone who cares about either football or philosophy, and especially for those of us who care about both.

\emph{References}

\hypertarget{refs}{}
\begin{CSLReferences}{1}{0}
\leavevmode\vadjust pre{\hypertarget{ref-Borge2010}{}}%
Borge, Steffen. 2010. {``In Defence of Maradona's Hand of God.''} In \emph{Philosophy of Sport: International Perspectives}, edited by A. Hardman and C. Jones, 154--79. Cambridge Scholars Publishing.

\leavevmode\vadjust pre{\hypertarget{ref-Borge2019}{}}%
---------. 2019. \emph{The Philosophy of Football}. London: Routledge.

\leavevmode\vadjust pre{\hypertarget{ref-DAgostino1981}{}}%
D'Agostino, Fred. 1981. {``The Ethos of Games.''} \emph{Journal of the Philosophy of Sport}, 7--18.

\leavevmode\vadjust pre{\hypertarget{ref-Suits1978}{}}%
Suits, Bernard. 1978. \emph{The Grasshopper: Games, Life and Utopia}. Toronto: University of Toronto Press.

\end{CSLReferences}

\end{document}
