% Options for packages loaded elsewhere
\PassOptionsToPackage{unicode}{hyperref}
\PassOptionsToPackage{hyphens}{url}
%
\documentclass[
  12pt,
]{article}
\usepackage{lmodern}
\usepackage{setspace}
\usepackage{amssymb,amsmath}
\usepackage{ifxetex,ifluatex}
\ifnum 0\ifxetex 1\fi\ifluatex 1\fi=0 % if pdftex
  \usepackage[T1]{fontenc}
  \usepackage[utf8]{inputenc}
  \usepackage{textcomp} % provide euro and other symbols
\else % if luatex or xetex
  \usepackage{unicode-math}
  \defaultfontfeatures{Scale=MatchLowercase}
  \defaultfontfeatures[\rmfamily]{Ligatures=TeX,Scale=1}
  \setmainfont[Scale=MatchLowercase]{Fira Sans}
\fi
% Use upquote if available, for straight quotes in verbatim environments
\IfFileExists{upquote.sty}{\usepackage{upquote}}{}
\IfFileExists{microtype.sty}{% use microtype if available
  \usepackage[]{microtype}
  \UseMicrotypeSet[protrusion]{basicmath} % disable protrusion for tt fonts
}{}
\makeatletter
\@ifundefined{KOMAClassName}{% if non-KOMA class
  \IfFileExists{parskip.sty}{%
    \usepackage{parskip}
  }{% else
    \setlength{\parindent}{0pt}
    \setlength{\parskip}{6pt plus 2pt minus 1pt}}
}{% if KOMA class
  \KOMAoptions{parskip=half}}
\makeatother
\usepackage{xcolor}
\IfFileExists{xurl.sty}{\usepackage{xurl}}{} % add URL line breaks if available
\IfFileExists{bookmark.sty}{\usepackage{bookmark}}{\usepackage{hyperref}}
\hypersetup{
  pdftitle={How Not to Manage the News},
  pdfauthor={Anon},
  hidelinks,
  pdfcreator={LaTeX via pandoc}}
\urlstyle{same} % disable monospaced font for URLs
\usepackage[margin=1.4in]{geometry}
\usepackage{graphicx,grffile}
\makeatletter
\def\maxwidth{\ifdim\Gin@nat@width>\linewidth\linewidth\else\Gin@nat@width\fi}
\def\maxheight{\ifdim\Gin@nat@height>\textheight\textheight\else\Gin@nat@height\fi}
\makeatother
% Scale images if necessary, so that they will not overflow the page
% margins by default, and it is still possible to overwrite the defaults
% using explicit options in \includegraphics[width, height, ...]{}
\setkeys{Gin}{width=\maxwidth,height=\maxheight,keepaspectratio}
% Set default figure placement to htbp
\makeatletter
\def\fps@figure{htbp}
\makeatother
\setlength{\emergencystretch}{3em} % prevent overfull lines
\providecommand{\tightlist}{%
  \setlength{\itemsep}{0pt}\setlength{\parskip}{0pt}}
\setcounter{secnumdepth}{-\maxdimen} % remove section numbering
\usepackage{booktabs}
\usepackage{longtable}
\usepackage{array}
\usepackage{multirow}
\usepackage{wrapfig}
\usepackage{float}
\usepackage{colortbl}
\usepackage{pdflscape}
\usepackage{tabu}
\usepackage{threeparttable} 
\usepackage{threeparttablex} 
\usepackage[normalem]{ulem} 
\usepackage{makecell}
\usepackage{xcolor}
\usepackage{ulem}

\setlength\heavyrulewidth{0ex}
\setlength\lightrulewidth{0.08ex}

\aboverulesep=0ex
\belowrulesep=0ex
\renewcommand{\arraystretch}{1.2}
\hypersetup{hidelinks}

\renewcommand\refname{References}
\usepackage{booktabs}
\usepackage{longtable}
\usepackage{array}
\usepackage{multirow}
\usepackage{wrapfig}
\usepackage{float}
\usepackage{colortbl}
\usepackage{pdflscape}
\usepackage{tabu}
\usepackage{threeparttable}
\usepackage{threeparttablex}
\usepackage[normalem]{ulem}
\usepackage{makecell}
\usepackage{xcolor}

\title{How Not to Manage the News}
\author{Anon}
\date{2020-12-21}

\begin{document}
\maketitle

\setstretch{1.1}
J. Dmitri Gallow (2020) proposed an adjustment to causal decision theory
to handle cases like Death in Damascus. The adjustment is ingenious, but
it creates problems that are bigger than those it aimed to solve.

Gallow's theory has two main parts, the first dealing with choice
between two options, and the second extending the theory to choice
between more than two options.

The part of the theory dealing with binary choice is easiest to
understand in terms of regret.\footnote{I'm simplifying a bit here; see
  section 2 of Gallow's paper for a more detailed presentation.} \(A\)
is preferable to \(B\) iff the chooser regrets choosing \(B\) when they
could have chosen \(A\) more than they regret choosing \(A\) when they
could have chosen \(B\). More formally, let \(I\) be an
\emph{improvement} function, in the following sense. (`Improvement' here
is basically the converse of regret.) \(I_C(A, B)\) is the weighted
average of \(D(AK) - D(BK)\), where \(K\) is a possible state of the
world that's causally independent of the choice, \(D\) measures the
desirability of choice-state pairs, and the weights are given by
\(\Pr(K | C)\). That is, the weights are conditional probabilities of
states given choices. Very very roughly, \(I_C(A, B)\) measures how much
better off you would have been choosing \(A\) rather than \(B\),
assuming you did actually choose \(C\). Gallow is primarily interested
in the special case where \(A = C\); only that special case will play a
role in what follows. The news value \(N\) of \(A\) over \(B\) is
defined as \(I_A(A, B) - I_B(B, A)\). In the case where there are only
two options, \(A\) is strictly preferred to \(B\) iff \(N(A, B) > 0\).

\hypertarget{refs}{}
\leavevmode\hypertarget{ref-Gallow2020}{}%
Gallow, J. Dmitri. 2020. ``The Causal Decision Theorist's Gudie to
Managing the News.'' \emph{The Journal of Philosophy} 117 (3): 117--49.

\end{document}
