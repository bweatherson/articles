% Options for packages loaded elsewhere
\PassOptionsToPackage{unicode}{hyperref}
\PassOptionsToPackage{hyphens}{url}
%
\documentclass[
  12pt,
]{article}
\usepackage{lmodern}
\usepackage{setspace}
\usepackage{amssymb,amsmath}
\usepackage{ifxetex,ifluatex}
\ifnum 0\ifxetex 1\fi\ifluatex 1\fi=0 % if pdftex
  \usepackage[T1]{fontenc}
  \usepackage[utf8]{inputenc}
  \usepackage{textcomp} % provide euro and other symbols
\else % if luatex or xetex
  \usepackage{unicode-math}
  \defaultfontfeatures{Scale=MatchLowercase}
  \defaultfontfeatures[\rmfamily]{Ligatures=TeX,Scale=1}
  \setmainfont[Scale=MatchLowercase]{Lato}
  \setmathfont[Scale=MatchUppercase]{Fira Math}
\fi
% Use upquote if available, for straight quotes in verbatim environments
\IfFileExists{upquote.sty}{\usepackage{upquote}}{}
\IfFileExists{microtype.sty}{% use microtype if available
  \usepackage[]{microtype}
  \UseMicrotypeSet[protrusion]{basicmath} % disable protrusion for tt fonts
}{}
\makeatletter
\@ifundefined{KOMAClassName}{% if non-KOMA class
  \IfFileExists{parskip.sty}{%
    \usepackage{parskip}
  }{% else
    \setlength{\parindent}{0pt}
    \setlength{\parskip}{6pt plus 2pt minus 1pt}}
}{% if KOMA class
  \KOMAoptions{parskip=half}}
\makeatother
\usepackage{xcolor}
\IfFileExists{xurl.sty}{\usepackage{xurl}}{} % add URL line breaks if available
\IfFileExists{bookmark.sty}{\usepackage{bookmark}}{\usepackage{hyperref}}
\hypersetup{
  pdftitle={How Not to Manage the News},
  pdfauthor={Anon},
  hidelinks,
  pdfcreator={LaTeX via pandoc}}
\urlstyle{same} % disable monospaced font for URLs
\usepackage[margin=1.38in]{geometry}
\usepackage{graphicx,grffile}
\makeatletter
\def\maxwidth{\ifdim\Gin@nat@width>\linewidth\linewidth\else\Gin@nat@width\fi}
\def\maxheight{\ifdim\Gin@nat@height>\textheight\textheight\else\Gin@nat@height\fi}
\makeatother
% Scale images if necessary, so that they will not overflow the page
% margins by default, and it is still possible to overwrite the defaults
% using explicit options in \includegraphics[width, height, ...]{}
\setkeys{Gin}{width=\maxwidth,height=\maxheight,keepaspectratio}
% Set default figure placement to htbp
\makeatletter
\def\fps@figure{htbp}
\makeatother
\setlength{\emergencystretch}{3em} % prevent overfull lines
\providecommand{\tightlist}{%
  \setlength{\itemsep}{0pt}\setlength{\parskip}{0pt}}
\setcounter{secnumdepth}{-\maxdimen} % remove section numbering
\usepackage{booktabs}
\usepackage{longtable}
\usepackage{array}
\usepackage{multirow}
\usepackage{wrapfig}
\usepackage{float}
\usepackage{colortbl}
\usepackage{pdflscape}
\usepackage{tabu}
\usepackage{threeparttable}
\usepackage{threeparttablex}
\usepackage[normalem]{ulem}
\usepackage{makecell}
\usepackage{xcolor}

\title{How Not to Manage the News}
\author{Anon}
\date{2020-12-22}

\begin{document}
\maketitle

\setlength\heavyrulewidth{0ex}
\setlength\lightrulewidth{0.08ex}

\aboverulesep=0ex
\belowrulesep=0ex
\renewcommand{\arraystretch}{1.2}
\hypersetup{hidelinks}

\renewcommand\refname{References}

\setstretch{1.55}
J. Dmitri Gallow (2020) has proposed an adjustment to causal decision
theory to handle cases like Death in Damascus. The adjustment is
ingenious, but it creates problems that are bigger than those it aimed
to solve.

Gallow's theory has two main parts, the first dealing with choice
between two options, and the second extending the theory to choice
between more than two options.

The part of the theory dealing with binary choice is easiest to
understand in terms of regret.\footnote{I am simplifying a bit here; see
  section 2 of Gallow's paper for a more detailed presentation.} \(A\)
is preferable to \(B\) iff the chooser regrets choosing \(B\) when they
could have chosen \(A\) more than they regret choosing \(A\) when they
could have chosen \(B\). More formally, let \(I\) be an
\emph{improvement} function, in the following sense. (`Improvement' here
is basically the converse of regret.) \(I_C(A, B)\) is the weighted
average of \(D(AK) - D(BK)\), where \(K\) is a possible state of the
world that is causally independent of the choice, \(D\) measures the
desirability of choice-state pairs, and the weights are given by
\(\Pr(K | C)\). That is, the weights are conditional probabilities of
states given choices. Very very roughly, \(I_C(A, B)\) measures how much
better off you would have been choosing \(A\) rather than \(B\),
assuming you did actually choose \(C\). Gallow is primarily interested
in the special case where \(A = C\); only that special case will play a
role in what follows. The news value \(N\) of \(A\) over \(B\) is
defined as \(I_A(A, B) - I_B(B, A)\). In the case where there are only
two options, \(A\) is strictly preferred to \(B\) iff \(N(A, B) > 0\).

This is not an implausible view about binary choice. To position it
against more familiar views, consider the special case where the
possible states are predictions about choices made by a very good
predictor. How good? To make our lives easier, we shall simplify the
math a lot and say that for any \(X\), the probability that the
predictor predicted \(X\) given that \(X\) is chosen is 1. If we write
\(PX\) for \(X\) is predicted, the simplifying assumption is
\(\Pr(PX | X) = 1\) for all \(X\). And I will make this simplifying
assumption for the rest of this paper. (It could be dropped if you do
not like infallible predictors; it would just make the math a tiny
amount more complicated.) Now imagine a very special case of this, where
\(D(PA \wedge A) > D(PA \wedge B)\) and
\(D(PB \wedge B) > D(PB \wedge A)\). In this case, both \(A\) and \(B\)
are self-ratifying. If you imagine the chooser is playing a game with
the predictor, where the chooser wants to maximise \(D\) and the
predictor wants to make correct predictions, then both
\(\langle A, PA\rangle\) and \(\langle B, PB\rangle\) are equilibria of
the game. (This idea of treating the predictor as a player in a game is
from Harper (1986).) In this case, Gallow's view will prefer \(A\) iff
it is the risk-dominant equilibria in the sense of Harsanyi and Selten
(1988). And it is plausible, as Harsanyi (1995) argues, that this is the
right choice in such a case.

But what is novel about Gallow's view is not what he says about binary
choice, but what he says about choice among a larger class of options.
There have been a flurry of papers in recent years proposing something
like this as the correct rule for binary choice.\footnote{See, for
  example, Wedgwood (2013), Robert (2018), Podgorski (2020), Barnett
  (n.d.) for similar views, and Bassett (2015) for earlier criticisms.
  My criticisms are distinct from Bassett's, but I think complementary;
  his arguments appear to be motivated by the same kind of
  considerations that are behind the arguments offered here.} Gallow
offers a distinctive view, drawing on sophisticated work in voting
theory, about how to extend this view to the general case. On the one
hand, Gallow's extension is better justified, and in many cases more
plausible, than the other extensions that have been offered. On the
other hand, we shall soon see that it faces some serious
counterexamples. I suspect this is bad news for all of these regret
based theories, but that is a story for another day.

In order to see the counterexamples I am going to develop, we do not
need to understand the full details of Gallow's theory. Indeed, for two
of the examples we just need to see how it applies in the case of
three-way choice. A first pass thing to say is that the preference
ordering over the choices includes \(X > Y\) iff \(N(X, Y) > 0\). But
this will not quite do for a couple of reasons. One of these has to do
with cases where \(N(X, Y) = N(W, Z)\), but that does not particularly
matter for what will follow. The bigger reason is that this `first pass'
leads to cyclic preferences. It is possible that \(N(A, B), N(B, C)\)
and \(N(C, A)\) are all positive. And Gallow, rightly, wants to avoid
that. So in that case, he says we should igore the smallest of the
three. If \(N(A, B), N(B, C)\) and \(N(C, A)\) are all positive, but
\(N(C, A)\) is the smallest of the lot, then we accept the first pass
judgment that \(A > B\) and \(B > C\), and use transitivity to conclude
\(A > C\). That is the feature of his theory that will do a bit of work
in what follows.

There is another point that will be important when we turn to an example
with more than three choices. If \(N(X, Y)\) is positive for all
\(Y \neq X\), then \(X\) is the Condorcet winner of the contest and is
chosen overall. This is not something that is forced into the theory -
he shows how it naturally falls out of other independently motivated
principles. But it is true in the theory, and it will be relevant in the
final example.

Our first counterexample starts with something like Stag Hunt, and adds
an option that dominates one of the choices.

\begin{table}[!h]
\centering
\begin{tabular}[t]{>{}r|ccc}
\toprule
 & PA & PB & PC\\
\midrule
A & 200 & 400 & 0\\
B & 0 & 500 & 11\\
C & 210 & 410 & 10\\
\bottomrule
\end{tabular}
\end{table}

Remember that \(PX\) just means that the predictor predicts that \(X\)
will be chosen, and a standing assumption is that for all
\(X, \Pr(PX | X) = 1\). Given that standing assumpption, it is easy
enough to calculate the \(N\) values. \(N(X, Y)\) is
\((D(X \wedge PX) + D(X \wedge PY)) - (D(Y \wedge PX) + D(Y \wedge PY))\).
So from the table we can read off:

\begin{itemize}
\tightlist
\item
  \(N(A, B) = (200 + 400) - (0 + 500) = 100\), and hence
  \(N(B, A) = -100\).
\item
  \(N(B, C) = (500 + 11) - (410 + 10) = 91\), and hence
  \(N(C, B) = -91\).
\item
  \(N(C, A) = (210 + 10) - (200 + 0) = 20\), and hence
  \(N(A, C) = -20\).
\end{itemize}

This would generate a cycle if we just looked at which values were
positive. So we ignore the smallest positive \(N\) value, and the
resulting ranking of the choices is \(A > B > C\).

But this is absurd. For one thing, \(A\) is strongly dominated by \(C\).
For another, both evidential decision theory and (most plausible
versions of) causal decision theory would agree on \(B\). For yet
another, consider the `gamified' version of this problem where the
numbers here are Row's payouts, and Column's payouts are 1 in each pair
\(\langle X, PX\rangle\), and 0 otherwise. The only equilibrium of the
game is \(\langle B, PB\rangle\). Even more strongly, the only pair that
is even rationalizable\footnote{I'm using `rationalizable' in the sense
  defined by Bernheim (1984) and by Pearce (1984). Gallow uses the same
  term for a different notion in a part of his paper I'm not discussing.}
is \(\langle B, PB\rangle\). That pair is the only strategy pair that
survives iterated deletion of dominated strategies.\footnote{There is a
  small wrinkle here. If we are deleting strategies, then we first
  delete \(A\). After that, we have two possible grounds to delete
  \(PA\). One is that it is weakly dominated by \(PB\), and for that
  matter \(PC\). Another is that it is strongly dominated by any proper
  mixture of \(PB\) and \(PC\). Some people are suspicious of appeals to
  weak dominance, and others are suspicious of appeals to dominance by
  mixed strategies. So the argument here is not completely watertight.
  But the fact that there are two quite different ways to get this step
  to work makes it more credible.} The player who selects \(B\) will
(correctly) believe that they are not just maximising expected value,
but getting the best possible outcome in the game. Intuition and theory
agree that this is an easy case: the right choice is \(B\). But Gallow's
theory says otherwise: he says to choose \(A\).

Now to be fair, Gallow does note that his theory sometimes recommends
dominated options like \(A\). But the example he gives of this phenomena
is a case where no option seems particularly plausible. The gamified
version of the case has no pure strategy equilibria, and every choice
seems regrettable.\footnote{The restriction to pure strategies is
  important here. There are ratifiable mixed strategies in the game in
  question, and personally I think one of them is the optimal choice.
  The way that contemporary decision theorists handle mixed strategies
  is a topic for a much longer paper.} In that case he argues that being
dominated is bad, but since every choice in the game is bad in one way
or another, it might be that the dominated choice is the least bad
option. That response is not available here. There is nothing wrong with
choosing \(B\). It produces, with probability 1, the best outcome on the
table, and is endorsed (for different reasons) by both evidential and
causal decision theory. There would need to be a very good reason to
prefer a dominated option to it, and I rather doubt such a reason is
forthcoming.

Onto the second example. Gallow notes that his theory does a good job of
handling `clone' cases: adding another option that duplicates an
existing option does not change anything. That is a nice result. But the
theory does less well handling dominated options in cases like the
following.

\begin{table}[!h]
\centering
\begin{tabular}[t]{>{}r|ccc}
\toprule
 & PA & PB & PC\\
\midrule
A & 11 & 1 & -500\\
B & 1 & 10 & 1\\
C & 0 & 0 & 0\\
\bottomrule
\end{tabular}
\end{table}

In this game, we have

\begin{itemize}
\tightlist
\item
  \(N(A, B) = 1\), and hence \(N(B, A) = -1\).
\item
  \(N(B, C) = 11\), and hence \(N(C, B) = -11\).
\item
  \(N(C, A) = 489\), and hence \(N(A, C) = -489\).
\end{itemize}

Again we would have a cycle, and cycles are bad, so we ignore the
smallest positive \(N\) value. And the result is that the theory says
that the ordering over the options is \(B > C > A\).

But again, this is implausible. \(A\) is the natural choice according to
both evidential decision theory and the best versions of causal decision
theory. More importantly, look at the game we get if we just treat \(C\)
as something that obviously will not be chosen, and hence the predictor
knows we will not choose. (Why is \(C\) obviously bad? Its best case
scenario is worse than \(A\)'s worst case scenario. That is as strong a
form of domination as you can find.)

\begin{table}[!h]
\centering
\begin{tabular}[t]{>{}r|cc}
\toprule
 & PA & PB\\
\midrule
A & 11 & 1\\
B & 1 & 10\\
\bottomrule
\end{tabular}
\end{table}

That is very similar to Gallow's Cake in Damascus example. In that case
he argues, very plausibly, that the right choice is \(A\). But adding
the absurd choice \(C\), along with the possibility that the predictor
will predict \(C\), leads to a change in choice. That should not happen.
It is at least as bad as the complaints he makes about how causal
decision theory handles clone choices.

Finally, look at a case that Gallow handles, I believe, the same way as
other people who share his view on binary choice. Remember that I noted
that on his view, if \(A\) is a Condorcet winner, if it is pairwise
preferred to all other choices, then t is the overall best choice. This
part of his view is not novel - though the elegant way he derives it is.
Still, it leads to bad results. Note that for \(A\) to be the Condorcet
winner in a situation where the predictor is perfect, we only have to
look at three things: the values in the first row, the values in the
first column, and the values along the main diagonal. But intuitively,
the other values on the table might be relevant to which choice is best.
So consider this case.

\begin{table}[!h]
\centering
\begin{tabular}[t]{>{}r|cccc}
\toprule
 & PA & PB & PC & PD\\
\midrule
A & 0 & 3 & 3 & 3\\
B & 1 & 1 & 1000 & 1000\\
C & 1 & 1000 & 1 & 1000\\
D & 1 & 1000 & 1000 & 1\\
\bottomrule
\end{tabular}
\end{table}

If \(X \neq A, N(A, X) = 1\). So \(A\) is the Condorcet winner, and is
the best option according to Gallow's theory. (And, for that matter,
most theories that use something like regret to make pairwise choices.)

But again, this is very unintuitive. Evidential decision theory rejects
this conclusion; it says you should be indifferent between \(B, C\) and
\(D\). And causal decision theory rejects it. There is no probability
distribution over the four states that makes \(A\) utility maximising.
So what is there to say in favor of \(A\)? Well, if you choose one of
\(B, C\) and \(D\), you will regret not choosing \(A\). But you will
regret the other two options even more. And if you do choose \(A\), you
will wish you had done literally anything else, since you will probably
get the worst outcome on the board.

So Gallow's strategy for extending this view about binary choice to a
general theory runs into a number of problems. I have not really argued
for it here, but I suspect a more general conclusion can be drawn from
this. This part of Gallow's theory was inventive, sophisticated and
carefully motivated - if it fails, we should be a little suspicious that
there is any good way of turning this theory of binary choice into a
general theory. And since this theory of binary choice is getting rather
popular in the last couple of years, that would be a striking result.
But proving that is for a longer paper.

\hypertarget{references}{%
\subsection*{References}\label{references}}
\addcontentsline{toc}{subsection}{References}

\hypertarget{refs}{}
\leavevmode\hypertarget{ref-Barnettnd}{}%
Barnett, David James. n.d. ``Graded Ratifiability.''
\url{http://www.davidjamesbar.net/wp-content/uploads/2018/03/Barnett-Graded-Ratifiability-March-2018.pdf}.

\leavevmode\hypertarget{ref-Bassett2015}{}%
Bassett, Robert. 2015. ``A Critique of Benchmark Theory.''
\emph{Synthese} 192: 241--67.
\url{https://doi.org/10.1007/s11229-014-0566-3}.

\leavevmode\hypertarget{ref-Bernheim1984}{}%
Bernheim, B. Douglas. 1984. ``Rationalizable Strategic Behavior.''
\emph{Econometrica} 52 (4): 1007--28.
\url{https://doi.org/10.2307/1911196}.

\leavevmode\hypertarget{ref-Gallow2020}{}%
Gallow, J. Dmitri. 2020. ``The Causal Decision Theorist's Gudie to
Managing the News.'' \emph{The Journal of Philosophy} 117 (3): 117--49.

\leavevmode\hypertarget{ref-Harper1986}{}%
Harper, William. 1986. ``Mixed Strategies and Ratifiability in Causal
Decision Theory.'' \emph{Erkenntnis} 24 (1): 25--36.
\url{https://doi.org/10.1007/BF00183199}.

\leavevmode\hypertarget{ref-Harsanyi1995}{}%
Harsanyi, John C. 1995. ``A New Theory of Equilibrium Selection for
Games with Complete Information.'' \emph{Games and Economic Behavior} 8
(1): 91--122. \url{https://doi.org/10.1016/S0899-8256(05)80018-1}.

\leavevmode\hypertarget{ref-HarsanyiSelten1988}{}%
Harsanyi, John C., and Reinhard Selten. 1988. \emph{A General Theory of
Equilibrium Selection in Games}. Cambridge, MA: MIT Press.

\leavevmode\hypertarget{ref-Pearce1984}{}%
Pearce, David G. 1984. ``Rationalizable Strategic Behavior and the
Problem of Perfection.'' \emph{Econometrica} 52 (4): 1029--50.
\url{https://doi.org/10.2307/1911197}.

\leavevmode\hypertarget{ref-Podgorski2020}{}%
Podgorski, Aberlard. 2020. ``Tournament Decision Theory.'' \emph{Noûs}
tbc (tbc): xx--xx. \url{https://doi.org/10.1111/nous.12353}.

\leavevmode\hypertarget{ref-Robert2018}{}%
Robert, David. 2018. ``Expected Comparative Utility Theory: A New Theory
of Rational Choice.'' \emph{The Philosophical Forum} 49 (1): 19--37.
\url{https://doi.org/10.1111/phil.12178}.

\leavevmode\hypertarget{ref-Wedgwood2013}{}%
Wedgwood, Ralph. 2013. ``A Priori Bootstrapping.'' In \emph{The a Priori
in Philosophy}, edited by Albert Casullo and Joshua C. Thurow, 225--46.
Oxford: Oxford University Press.

\end{document}
