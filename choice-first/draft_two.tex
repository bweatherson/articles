% Options for packages loaded elsewhere
% Options for packages loaded elsewhere
\PassOptionsToPackage{unicode}{hyperref}
\PassOptionsToPackage{hyphens}{url}
%
\documentclass[
  12pt,
  letterpaper,
  DIV=11,
  numbers=noendperiod]{scrartcl}
\usepackage{xcolor}
\usepackage[left=1in,right=1in,top=1in,bottom=1in]{geometry}
\usepackage{amsmath,amssymb}
\setcounter{secnumdepth}{5}
\usepackage{iftex}
\ifPDFTeX
  \usepackage[T1]{fontenc}
  \usepackage[utf8]{inputenc}
  \usepackage{textcomp} % provide euro and other symbols
\else % if luatex or xetex
  \usepackage{unicode-math} % this also loads fontspec
  \defaultfontfeatures{Scale=MatchLowercase}
  \defaultfontfeatures[\rmfamily]{Ligatures=TeX,Scale=1}
\fi
\usepackage{lmodern}
\ifPDFTeX\else
  % xetex/luatex font selection
  \setmainfont[ItalicFont=EB Garamond Italic,BoldFont=EB Garamond
SemiBold]{EB Garamond Math}
  \setsansfont[]{EB Garamond SemiBold}
  \setmathfont[]{EB Garamond Math}
\fi
% Use upquote if available, for straight quotes in verbatim environments
\IfFileExists{upquote.sty}{\usepackage{upquote}}{}
\IfFileExists{microtype.sty}{% use microtype if available
  \usepackage[]{microtype}
  \UseMicrotypeSet[protrusion]{basicmath} % disable protrusion for tt fonts
}{}
\usepackage{setspace}
\makeatletter
\@ifundefined{KOMAClassName}{% if non-KOMA class
  \IfFileExists{parskip.sty}{%
    \usepackage{parskip}
  }{% else
    \setlength{\parindent}{0pt}
    \setlength{\parskip}{6pt plus 2pt minus 1pt}}
}{% if KOMA class
  \KOMAoptions{parskip=half}}
\makeatother
% Make \paragraph and \subparagraph free-standing
\makeatletter
\ifx\paragraph\undefined\else
  \let\oldparagraph\paragraph
  \renewcommand{\paragraph}{
    \@ifstar
      \xxxParagraphStar
      \xxxParagraphNoStar
  }
  \newcommand{\xxxParagraphStar}[1]{\oldparagraph*{#1}\mbox{}}
  \newcommand{\xxxParagraphNoStar}[1]{\oldparagraph{#1}\mbox{}}
\fi
\ifx\subparagraph\undefined\else
  \let\oldsubparagraph\subparagraph
  \renewcommand{\subparagraph}{
    \@ifstar
      \xxxSubParagraphStar
      \xxxSubParagraphNoStar
  }
  \newcommand{\xxxSubParagraphStar}[1]{\oldsubparagraph*{#1}\mbox{}}
  \newcommand{\xxxSubParagraphNoStar}[1]{\oldsubparagraph{#1}\mbox{}}
\fi
\makeatother


\usepackage{longtable,booktabs,array}
\usepackage{calc} % for calculating minipage widths
% Correct order of tables after \paragraph or \subparagraph
\usepackage{etoolbox}
\makeatletter
\patchcmd\longtable{\par}{\if@noskipsec\mbox{}\fi\par}{}{}
\makeatother
% Allow footnotes in longtable head/foot
\IfFileExists{footnotehyper.sty}{\usepackage{footnotehyper}}{\usepackage{footnote}}
\makesavenoteenv{longtable}
\usepackage{graphicx}
\makeatletter
\newsavebox\pandoc@box
\newcommand*\pandocbounded[1]{% scales image to fit in text height/width
  \sbox\pandoc@box{#1}%
  \Gscale@div\@tempa{\textheight}{\dimexpr\ht\pandoc@box+\dp\pandoc@box\relax}%
  \Gscale@div\@tempb{\linewidth}{\wd\pandoc@box}%
  \ifdim\@tempb\p@<\@tempa\p@\let\@tempa\@tempb\fi% select the smaller of both
  \ifdim\@tempa\p@<\p@\scalebox{\@tempa}{\usebox\pandoc@box}%
  \else\usebox{\pandoc@box}%
  \fi%
}
% Set default figure placement to htbp
\def\fps@figure{htbp}
\makeatother


% definitions for citeproc citations
\NewDocumentCommand\citeproctext{}{}
\NewDocumentCommand\citeproc{mm}{%
  \begingroup\def\citeproctext{#2}\cite{#1}\endgroup}
\makeatletter
 % allow citations to break across lines
 \let\@cite@ofmt\@firstofone
 % avoid brackets around text for \cite:
 \def\@biblabel#1{}
 \def\@cite#1#2{{#1\if@tempswa , #2\fi}}
\makeatother
\newlength{\cslhangindent}
\setlength{\cslhangindent}{1.5em}
\newlength{\csllabelwidth}
\setlength{\csllabelwidth}{3em}
\newenvironment{CSLReferences}[2] % #1 hanging-indent, #2 entry-spacing
 {\begin{list}{}{%
  \setlength{\itemindent}{0pt}
  \setlength{\leftmargin}{0pt}
  \setlength{\parsep}{0pt}
  % turn on hanging indent if param 1 is 1
  \ifodd #1
   \setlength{\leftmargin}{\cslhangindent}
   \setlength{\itemindent}{-1\cslhangindent}
  \fi
  % set entry spacing
  \setlength{\itemsep}{#2\baselineskip}}}
 {\end{list}}
\usepackage{calc}
\newcommand{\CSLBlock}[1]{\hfill\break\parbox[t]{\linewidth}{\strut\ignorespaces#1\strut}}
\newcommand{\CSLLeftMargin}[1]{\parbox[t]{\csllabelwidth}{\strut#1\strut}}
\newcommand{\CSLRightInline}[1]{\parbox[t]{\linewidth - \csllabelwidth}{\strut#1\strut}}
\newcommand{\CSLIndent}[1]{\hspace{\cslhangindent}#1}



\setlength{\emergencystretch}{3em} % prevent overfull lines

\providecommand{\tightlist}{%
  \setlength{\itemsep}{0pt}\setlength{\parskip}{0pt}}



 


\setlength\heavyrulewidth{0ex}
\setlength\lightrulewidth{0ex}
\KOMAoption{captions}{tableheading}
\makeatletter
\@ifpackageloaded{caption}{}{\usepackage{caption}}
\AtBeginDocument{%
\ifdefined\contentsname
  \renewcommand*\contentsname{Table of contents}
\else
  \newcommand\contentsname{Table of contents}
\fi
\ifdefined\listfigurename
  \renewcommand*\listfigurename{List of Figures}
\else
  \newcommand\listfigurename{List of Figures}
\fi
\ifdefined\listtablename
  \renewcommand*\listtablename{List of Tables}
\else
  \newcommand\listtablename{List of Tables}
\fi
\ifdefined\figurename
  \renewcommand*\figurename{Figure}
\else
  \newcommand\figurename{Figure}
\fi
\ifdefined\tablename
  \renewcommand*\tablename{Table}
\else
  \newcommand\tablename{Table}
\fi
}
\@ifpackageloaded{float}{}{\usepackage{float}}
\floatstyle{ruled}
\@ifundefined{c@chapter}{\newfloat{codelisting}{h}{lop}}{\newfloat{codelisting}{h}{lop}[chapter]}
\floatname{codelisting}{Listing}
\newcommand*\listoflistings{\listof{codelisting}{List of Listings}}
\makeatother
\makeatletter
\makeatother
\makeatletter
\@ifpackageloaded{caption}{}{\usepackage{caption}}
\@ifpackageloaded{subcaption}{}{\usepackage{subcaption}}
\makeatother
\usepackage{bookmark}
\IfFileExists{xurl.sty}{\usepackage{xurl}}{} % add URL line breaks if available
\urlstyle{same}
\hypersetup{
  pdftitle={Choice First},
  pdfauthor={Brian Weatherson},
  hidelinks,
  pdfcreator={LaTeX via pandoc}}


\title{Choice First}
\author{Brian Weatherson}
\date{2025-05-28}
\begin{document}
\maketitle
\begin{abstract}
A decider usually has several options to choose from, but philosophical
decision theory has very little to say about choices involving more than
two options. The standard approach is to describe in great detail either
a value function or a preference relation, and then say that they
determine what is to be done when choosing from a larger set. This paper
suggests that we should start with choices, or at least beliefs about
hypothetical choices, and construct values and preferences from them.
The resulting theory lets us solve, or perhaps dissolve, several
problems about so-called incomplete preferences.
\end{abstract}


\setstretch{1.75}
\section{Introduction}\label{sec-intro}

Subjective decision theory concerns itself with three things, as they
appear to the decider.

\begin{enumerate}
\def\labelenumi{\arabic{enumi}.}
\tightlist
\item
  The value of different options.
\item
  Which options are better than others.
\item
  Which options are choiceworthy from an option set.
\end{enumerate}

One way to taxonomise decision theories is to ask which of these they
take to be primary.\footnote{I'll assume throughout that the decider is
  coherent and logically omniscient, so we can attribute to them a state
  whenever its content is entailed by the contents of states we know
  they have.}

What I'll call \textbf{value-first} theories take the first to be
primary. There is some function \emph{v} from options to a range,
typically real numbers, which includes an intrinsic order. Given
\emph{v}, we can define the other states easily. Option \emph{a} is
better than \emph{b} iff
\emph{v}(\emph{a})~\textgreater~\emph{v}(\emph{b}), and option \emph{a}
is choiceworthy from a set S iff there is no option in S with a higher
value than \emph{v}(\emph{a}).

Value-first approaches are simple and intuitive, but they don't appear
to be popular.\footnote{I say `appear' because decision theorists
  typically don't express their views using the language of priority
  that I'm adopting here. But it's often clear enough which theses about
  priority are consistent with their views.} There are at least three
problems with them. One is that the choice of range seems arbitrary.
There is no simple reason why values should be real numbers, and the
best arguments that they should be require taking preferences to be
prior to values. Another is that there is no scale attached to these
numbers, and no obvious way to generate a scale without once again
taking preferences to be prior to values. And a third, more amorphous
but ultimately more significant, is that values seem deeply comparative.
It isn't clear what it even means to say that one good, in isolation,
has value 17. What is meaningful is to say that it is more valuable than
something else, or most valuable among some choices. The value-first
approach doesn't respect the essentially comparative nature of value.

The most popular view is what I'll call the \textbf{preference first}
view. There are various ways to express this view, both in terms of what
kinds of preferences we take as primary, and what notation we use. In
both respects, I'll follow the approach of Amartya Sen
(\citeproc{ref-Sen1970sec}{{[}1970{]} 2017, 55ff}). (I'll often be
following Sen in this paper.) Start with a relation \emph{R}, where
\emph{aRb} means \emph{a} is at least as good as \emph{b}.\footnote{Some
  might prefer to understand this as meaning \emph{b} is not better than
  \emph{a}.} We then define strict preference, \emph{P}, and
indifference, \emph{I}, as follows.

\begin{enumerate}
\def\labelenumi{(\arabic{enumi})}
\tightlist
\item
  \emph{xPy} =\textsubscript{df} \emph{xRy} ∧ ¬\emph{yRx}
\item
  \emph{xIy} =\textsubscript{df} \emph{xRy} ∧ \emph{yRx}
\end{enumerate}

There is a fourth relation, of preferential equality, that we might want
to later define. But for now these three will do.

If we impose enough constraints on \emph{R}, we can prove a
representation theorem relating preferences to values. Such a proof was
sketched by Ramsey (\citeproc{ref-RamseyTruthProb}{1926}), and worked
out in detail by Neumann and Morgenstern (\citeproc{ref-vNM1944}{1944}).
The proof shows that such a \emph{v} exists, is a function to reals, and
is defined up to positive affine transformation. This happily settles
several of the worries about values that I raised earlier. The fact that
values are reals falls out of the theorem, and the fact that values are
only defined up to positive affine transformation explains why things
like \emph{v}(\emph{a})~=~17 on their own don't make sense. But the
``enough constraints'' are non-trivial. One of these constraints that
we'll come back to frequently is that \emph{I} is transitive. Another
constraint, one which Sen calls PI, is in (3).

\begin{enumerate}
\def\labelenumi{(\arabic{enumi})}
\setcounter{enumi}{2}
\tightlist
\item
  (\emph{xPy} ∧ \emph{yIz}) → \emph{xPz}
\end{enumerate}

It's perhaps misleading to say that this is another constraint; as Sen
shows, (3) is equivalent to the transitivity of \emph{I} if we assume
that \emph{P} is transitive. As we'll see, there are cases where (3) is
intuitively false, and finding something sensible to say about these
cases will be a focus of this paper.

If \emph{R} is complete\footnote{Note that \emph{R} being complete is
  not the same thing as philosophers normally mean by preferences being
  complete; it's more like asymmetry of strict preference} and
acyclical, we can also define a choice function \emph{c}. We'll be using
these choice functions a lot. In general, \emph{c}(\emph{S})~=~X, where
\emph{S} is a set of options, means that the members of X are all and
only the choiceworthy members of \emph{S}. We're interested in a
subjective version of this, so we'll typically treat this as meaning
that decider believes that these are the choiceworthy options; a more
common interpretation is in terms of behavioral dispositions. We will
assume throughout that whenever \emph{S} is non-empty, so is
\emph{c}(\emph{S}). That is, there are always choiceworthy options;
there are no dilemmas.

Given complete and acyclical \emph{R}, we can generate a choice function
\emph{c} via (4).\footnote{Compare Sen
  (\citeproc{ref-Sen1970sec}{{[}1970{]} 2017}) 55.}

\begin{enumerate}
\def\labelenumi{(\arabic{enumi})}
\setcounter{enumi}{3}
\tightlist
\item
  \emph{c}(\emph{S}) = \{\emph{x} ∈ \emph{S}: ∀\emph{y}(\emph{y} ∈
  \emph{S} → \emph{xRy})\}
\end{enumerate}

Since completeness (in this sense) and acyclicality are much weaker than
the assumptions needed to generate utility functions, it isn't too
controversial that we can get from preference relations to choice
functions. What I want to controvert is whether we should go in this
direction.

What I'm going to argue, instead, is that we should start with start
with choice functions. The first argument for that will be that the
choice functions generated by (4) are too restrictive. So we'll start by
looking at properties of choice functions, to get a sense of what
restrictions might be in place.

\section{Properties of Choice
Functions}\label{properties-of-choice-functions}

We're going to be interested in five properties of choice functions. For
the first four, we'll use Sen's names (and indeed terminology).

\begin{description}
\tightlist
\item[Property α]
(x ∈ C(S) ∧ x ∈ T ∧ T ⊆ S) → x ∈ C(T)

If x is chosen from S and T is a subset of S that contains x, then x is
also chosen from T.
\end{description}

β: (x ∈ C(T) ∧ y ∈ C(T) ∧ T ⊆ S) → (x ∈ C(S) ↔ y ∈ C(S))

\begin{itemize}
\tightlist
\item
  If x and y are both chosen from T, and T is a subset of S, then x is
  chosen from S if and only if y is chosen from S.
\end{itemize}

γ: (x ∈ C(S) ∧ x ∈ C(T)) → (x ∈ C(S ∪ T))

\begin{itemize}
\tightlist
\item
  If x is chosen from both S and T, then x is chosen from the union of S
  and T.
\end{itemize}

δ: (x ∈ C(T) ∧ y ∈ C(T) ∧ T ⊆ S) → (\{y\} ≠ C(S))

\begin{itemize}
\tightlist
\item
  If x and y are both chosen from T, and T is a subset of S, then y is
  not the only option chosen from S.
\end{itemize}

\subsection*{References}\label{references}
\addcontentsline{toc}{subsection}{References}

\phantomsection\label{refs}
\begin{CSLReferences}{1}{0}
\bibitem[\citeproctext]{ref-vNM1944}
Neumann, John von, and Oskar Morgenstern. 1944. \emph{Theory of Games
and Economic Behavior}. Princeton, NJ: Princeton University Press.

\bibitem[\citeproctext]{ref-RamseyTruthProb}
Ramsey, Frank. 1926. {``Truth and Probability.''} In \emph{Philosophical
Papers}, edited by D. H. Mellor, 52--94. Cambridge: Cambridge University
Press.

\bibitem[\citeproctext]{ref-Sen1970sec}
Sen, Amartya. (1970) 2017. \emph{Collective Choice and Social Welfare:}
An expanded edition. Cambridge, MA: Harvard University Press.

\end{CSLReferences}




\end{document}
