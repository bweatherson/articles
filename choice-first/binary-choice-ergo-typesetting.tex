% Options for packages loaded elsewhere
% Options for packages loaded elsewhere
\PassOptionsToPackage{unicode}{hyperref}
\PassOptionsToPackage{hyphens}{url}
\PassOptionsToPackage{dvipsnames,svgnames,x11names}{xcolor}
%
\documentclass[
  10.5pt,
  twoside]{article}
\usepackage{xcolor}
\usepackage[paperheight=10in,,paperwidth=7in,,top=1in,bottom=1in,inner=0.8in,outer=0.8in,headsep=0.25in,headheight=1in,footskip=0.35in]{geometry}
\usepackage{amsmath,amssymb}
\setcounter{secnumdepth}{5}
\usepackage{iftex}
\ifPDFTeX
  \usepackage[T1]{fontenc}
  \usepackage[utf8]{inputenc}
  \usepackage{textcomp} % provide euro and other symbols
\else % if luatex or xetex
  \usepackage{unicode-math} % this also loads fontspec
  \defaultfontfeatures{Scale=MatchLowercase}
  \defaultfontfeatures[\rmfamily]{Ligatures=TeX,Scale=1}
\fi
\usepackage{lmodern}
\ifPDFTeX\else
  % xetex/luatex font selection
  \setmainfont[ItalicFont=EB Garamond Italic,BoldFont=EB Garamond
SemiBold]{EB Garamond Math}
  \setsansfont[]{EB Garamond SemiBold}
  \setmathfont[]{EB Garamond Math}
\fi
% Use upquote if available, for straight quotes in verbatim environments
\IfFileExists{upquote.sty}{\usepackage{upquote}}{}
\IfFileExists{microtype.sty}{% use microtype if available
  \usepackage[]{microtype}
  \UseMicrotypeSet[protrusion]{basicmath} % disable protrusion for tt fonts
}{}
\usepackage{setspace}
\makeatletter
\@ifundefined{KOMAClassName}{% if non-KOMA class
  \IfFileExists{parskip.sty}{%
    \usepackage{parskip}
  }{% else
    \setlength{\parindent}{0pt}
    \setlength{\parskip}{6pt plus 2pt minus 1pt}}
}{% if KOMA class
  \KOMAoptions{parskip=half}}
\makeatother
% Make \paragraph and \subparagraph free-standing
\makeatletter
\ifx\paragraph\undefined\else
  \let\oldparagraph\paragraph
  \renewcommand{\paragraph}{
    \@ifstar
      \xxxParagraphStar
      \xxxParagraphNoStar
  }
  \newcommand{\xxxParagraphStar}[1]{\oldparagraph*{#1}\mbox{}}
  \newcommand{\xxxParagraphNoStar}[1]{\oldparagraph{#1}\mbox{}}
\fi
\ifx\subparagraph\undefined\else
  \let\oldsubparagraph\subparagraph
  \renewcommand{\subparagraph}{
    \@ifstar
      \xxxSubParagraphStar
      \xxxSubParagraphNoStar
  }
  \newcommand{\xxxSubParagraphStar}[1]{\oldsubparagraph*{#1}\mbox{}}
  \newcommand{\xxxSubParagraphNoStar}[1]{\oldsubparagraph{#1}\mbox{}}
\fi
\makeatother


\usepackage{longtable,booktabs,array}
\usepackage{calc} % for calculating minipage widths
% Correct order of tables after \paragraph or \subparagraph
\usepackage{etoolbox}
\makeatletter
\patchcmd\longtable{\par}{\if@noskipsec\mbox{}\fi\par}{}{}
\makeatother
% Allow footnotes in longtable head/foot
\IfFileExists{footnotehyper.sty}{\usepackage{footnotehyper}}{\usepackage{footnote}}
\makesavenoteenv{longtable}
\usepackage{graphicx}
\makeatletter
\newsavebox\pandoc@box
\newcommand*\pandocbounded[1]{% scales image to fit in text height/width
  \sbox\pandoc@box{#1}%
  \Gscale@div\@tempa{\textheight}{\dimexpr\ht\pandoc@box+\dp\pandoc@box\relax}%
  \Gscale@div\@tempb{\linewidth}{\wd\pandoc@box}%
  \ifdim\@tempb\p@<\@tempa\p@\let\@tempa\@tempb\fi% select the smaller of both
  \ifdim\@tempa\p@<\p@\scalebox{\@tempa}{\usebox\pandoc@box}%
  \else\usebox{\pandoc@box}%
  \fi%
}
% Set default figure placement to htbp
\def\fps@figure{htbp}
\makeatother


% definitions for citeproc citations
\NewDocumentCommand\citeproctext{}{}
\NewDocumentCommand\citeproc{mm}{%
  \begingroup\def\citeproctext{#2}\cite{#1}\endgroup}
\makeatletter
 % allow citations to break across lines
 \let\@cite@ofmt\@firstofone
 % avoid brackets around text for \cite:
 \def\@biblabel#1{}
 \def\@cite#1#2{{#1\if@tempswa , #2\fi}}
\makeatother
\newlength{\cslhangindent}
\setlength{\cslhangindent}{1.5em}
\newlength{\csllabelwidth}
\setlength{\csllabelwidth}{3em}
\newenvironment{CSLReferences}[2] % #1 hanging-indent, #2 entry-spacing
 {\begin{list}{}{%
  \setlength{\itemindent}{0pt}
  \setlength{\leftmargin}{0pt}
  \setlength{\parsep}{0pt}
  % turn on hanging indent if param 1 is 1
  \ifodd #1
   \setlength{\leftmargin}{\cslhangindent}
   \setlength{\itemindent}{-1\cslhangindent}
  \fi
  % set entry spacing
  \setlength{\itemsep}{#2\baselineskip}}}
 {\end{list}}
\usepackage{calc}
\newcommand{\CSLBlock}[1]{\hfill\break\parbox[t]{\linewidth}{\strut\ignorespaces#1\strut}}
\newcommand{\CSLLeftMargin}[1]{\parbox[t]{\csllabelwidth}{\strut#1\strut}}
\newcommand{\CSLRightInline}[1]{\parbox[t]{\linewidth - \csllabelwidth}{\strut#1\strut}}
\newcommand{\CSLIndent}[1]{\hspace{\cslhangindent}#1}



\setlength{\emergencystretch}{3em} % prevent overfull lines

\providecommand{\tightlist}{%
  \setlength{\itemsep}{0pt}\setlength{\parskip}{0pt}}



 


% Custom title page mimicking Ergo class style
% Add this to your Quarto YAML with:
% format:
%   pdf:
%     include-in-header:
%       - maketitle.tex

\makeatletter

% Redefine \maketitle to match Ergo style
\renewcommand\maketitle{
  \begingroup
  \setlength{\parskip}{0pt plus 0pt minus 0pt}
  \long\def\@makefntext##1{\parindent 1em\noindent
          \hb@xt@1.8em{%
             \hss\@textsuperscript{\normalfont\@thefnmark}}##1}%
  \newpage
  \null
  \noindent{}\begin{minipage}[l]{4in}
    \vskip 53.3pt plus 0pt minus 0pt
    {\LARGE \textsc{\@title}\par}% Title in large small caps
    \vskip 15pt plus 0pt minus 0pt
    {\Huge\MakeUppercase{\@author}\par}% Author in huge uppercase
    \vskip 5pt plus 0pt minus 0pt
    {\normalsize\textit{University of Michigan}\par}% Date in normal size italics (or affiliation)
  \end{minipage}
  \vskip 12pt plus 0pt minus 0pt
  \thispagestyle{plain}
  \endgroup
  \setcounter{footnote}{0}%
  \global\let\thanks\relax
  \global\let\@thanks\@empty
  \global\let\@date\@empty
  \global\let\title\relax
  \global\let\author\relax
  \global\let\date\relax
  \global\let\and\relax
}

% Redefine abstract to match Ergo style with margins
\renewenvironment{abstract}
 {\list{}{
    \setlength{\leftmargin}{.25in}
    \setlength{\rightmargin}{\leftmargin}
  }
  \item\relax
  \small}
 {\vskip -3pt plus 0pt minus 0pt\null\endlist}

% Font size adjustments to match Ergo
\renewcommand\normalsize{\@setfontsize\normalsize{10.5pt}{14pt}}
\renewcommand\small{\@setfontsize\small{9pt}{13pt}}
\renewcommand\LARGE{\@setfontsize\LARGE{18pt}{22pt}}
\renewcommand\Huge{\@setfontsize\Huge{9.5pt}{13.5pt}}

\makeatother
% Body text and heading styles mimicking Ergo class
% Add this to your Quarto YAML with:
% format:
%   pdf:
%     include-in-header:
%       - body-style.tex

\usepackage[automark]{scrlayer-scrpage}
\clearpairofpagestyles
\cehead{
  Brian Weatherson
  }
\cohead{
  Are Choices Binary?
  }
\ohead{\bfseries \pagemark}
\cfoot{}

\makeatletter

% Paragraph indentation and spacing
\setlength{\parindent}{.25in}
\setlength{\parskip}{0pt plus 0pt minus 0pt}

% Section numbering with periods and spacing
\renewcommand*{\@seccntformat}[1]{%
  \csname the#1\endcsname.$\:$%
}

% Section styles (matching Ergo)
\renewcommand\section{\@startsection{section}{1}{\z@}%
  {-4.6ex \@plus 0ex \@minus 0ex}%        % space before
  {2.3ex \@plus 0ex}%                      % space after
  {\normalfont\Large\bfseries}}            % Large, bold

\renewcommand\subsection{\@startsection{subsection}{2}{\z@}%
  {-2.3ex \@plus 0ex \@minus 0ex}%        % space before
  {2.3ex \@plus 0ex \@minus 0ex}%         % space after
  {\normalfont\large\itshape}}             % large, italic

\renewcommand\subsubsection{\@startsection{subsubsection}{3}{\z@}%
  {-2.3ex \@plus 0ex \@minus 0ex}%        % space before
  {2.3ex \@plus 0ex \@minus 0ex}%         % space after
  {\normalfont\large}}                     % large, normal

% Font sizes (matching Ergo)
\renewcommand\normalsize{\@setfontsize\normalsize{10.5pt}{14pt}}
\renewcommand\footnotesize{\@setfontsize\footnotesize{9pt}{11.5pt}}
\renewcommand\small{\@setfontsize\small{9pt}{13pt}}
\renewcommand\large{\@setfontsize\large{11.5pt}{12pt}}
\renewcommand\Large{\@setfontsize\Large{12pt}{14pt}}

% Quotation and quote environments with Ergo spacing
\renewenvironment{quotation}
  {\list{}{
    \setlength{\listparindent}{.25in}%
    \setlength{\leftmargin}{.25in}
    \setlength{\rightmargin}{\leftmargin}
    \setlength{\parsep}{0in plus 0in minus 0in}
    \item\relax
    \let\item\relax}
  }
  {\endlist}

\renewenvironment{quote}
  {\vskip 5pt%
   \list{}{
    \setlength{\listparindent}{.25in}
    \setlength{\leftmargin}{.25in}
    \setlength{\rightmargin}{\leftmargin}
    \setlength{\parsep}{0in plus 0in minus 0in}
    }
    \item\relax
    \let\item\relax}
  {\endlist\vskip 5pt}

% Footnote rule styling
\renewcommand{\footnoterule}{%
  \kern -3pt
  \hrule width 1in height .4pt
  \kern 2.5pt
}

% Non-superscript numerals for footnote text numbering
\renewcommand\@makefntext[1]{%
  \parindent .25in%
  \@thefnmark.~#1}%

% Figure and table captions with period separator
\renewcommand\@makecaption[2]{%
  \vskip\abovecaptionskip
  \sbox\@tempboxa{#1. #2}%
  \ifdim \wd\@tempboxa >\hsize
    #1. #2\par
  \else
    \global \@minipagefalse
    \hb@xt@\hsize{\hfil\box\@tempboxa\hfil}%
  \fi
  \vskip\belowcaptionskip}

% Enumerate environment with extra vertical space
\let\oldenumerate\enumerate
\let\endoldenumerate\endenumerate
\renewenvironment{enumerate}
  {\vskip 5pt\oldenumerate}
  {\endoldenumerate\vskip 5pt}

\let\olddescription\description
\let\endolddescription\enddescription
\renewenvironment{description}
  {\vskip 5pt\olddescription}
  {\endolddescription\vskip 5pt}

\let\olditemize\itemize
\let\endolditemize\enditemize
\renewenvironment{itemize}
  {\vskip 5pt\olditemize}
  {\endolditemize\vskip 5pt}

\newcommand*\NoIndentAfterEnv[1]{%
  \AfterEndEnvironment{#1}{\par\@afterindentfalse\@afterheading}}
\makeatother
\NoIndentAfterEnv{itemize}
\NoIndentAfterEnv{enumerate}
\NoIndentAfterEnv{description}
\NoIndentAfterEnv{quote}
\NoIndentAfterEnv{equation}
\NoIndentAfterEnv{longtable}

\makeatother
\setlength\heavyrulewidth{0ex}
\setlength\lightrulewidth{0ex}
\usepackage[lines=2]{lettrine}
\cehead{EWIP Draft}
\makeatletter
\@ifpackageloaded{caption}{}{\usepackage{caption}}
\AtBeginDocument{%
\ifdefined\contentsname
  \renewcommand*\contentsname{Table of contents}
\else
  \newcommand\contentsname{Table of contents}
\fi
\ifdefined\listfigurename
  \renewcommand*\listfigurename{List of Figures}
\else
  \newcommand\listfigurename{List of Figures}
\fi
\ifdefined\listtablename
  \renewcommand*\listtablename{List of Tables}
\else
  \newcommand\listtablename{List of Tables}
\fi
\ifdefined\figurename
  \renewcommand*\figurename{Figure}
\else
  \newcommand\figurename{Figure}
\fi
\ifdefined\tablename
  \renewcommand*\tablename{Table}
\else
  \newcommand\tablename{Table}
\fi
}
\@ifpackageloaded{float}{}{\usepackage{float}}
\floatstyle{ruled}
\@ifundefined{c@chapter}{\newfloat{codelisting}{h}{lop}}{\newfloat{codelisting}{h}{lop}[chapter]}
\floatname{codelisting}{Listing}
\newcommand*\listoflistings{\listof{codelisting}{List of Listings}}
\makeatother
\makeatletter
\makeatother
\makeatletter
\@ifpackageloaded{caption}{}{\usepackage{caption}}
\@ifpackageloaded{subcaption}{}{\usepackage{subcaption}}
\makeatother
\usepackage{bookmark}
\IfFileExists{xurl.sty}{\usepackage{xurl}}{} % add URL line breaks if available
\urlstyle{same}
\hypersetup{
  pdftitle={Are Choices Binary?},
  pdfauthor={Brian Weatherson},
  colorlinks=true,
  linkcolor={blue},
  filecolor={Maroon},
  citecolor={Blue},
  urlcolor={blue},
  pdfcreator={LaTeX via pandoc}}


\title{Are Choices Binary?}
\author{Brian Weatherson}
\date{2025-10-23}
\begin{document}
\maketitle
\begin{abstract}
There is a natural view of the relationship between preference and
choice: an option is choiceworthy if and only if no alternative is
strictly preferred to it. I argue against this view on two grounds.
First, it makes false predictions about which options are choiceworthy
in games and in multi-dimensional choice settings. Second, it conflates
two distinct attitudes: choiceworthiness, which is assessed ex ante, and
preference, which is assessed ex post. I explore the consequences of
rejecting this natural view, including how it simplifies the
relationship between game theory and decision theory, and how it
complicates debates about what Ruth Chang calls `parity' between
options.
\end{abstract}


\setstretch{1}
\lettrine{S}{ubjective} decision theory concerns norms about three
concepts: values of options, preferences between pairs of options, and
choices out of sets of options. A common assumption, often implicit, is
that norms about preferences are prior to norms on values and choices.
One way to put this assumption, following Amartya Sen
(\citeproc{ref-Sen1970sec}{{[}1970{]} 2017}), is that choice functions
are \emph{binary}; they are grounded in binary relations of preference
and indifference.

I'm going to argue against this for two reasons. First, preferences,
being binary comparisons, don't provide a rich enough base to ground all
the norms. Sometimes decision theorists need to take as primitive
comparisons the chooser (hereafter, Chooser) makes between larger sets
of options. Second, preference is an ex post notion, in a sense to be
made clearer starting in Section~\ref{sec-multieq}, while
choiceworthiness is an ex ante notion. And ex ante norms are not
grounded in ex post ones.

\section{Choice and Choiceworthiness}\label{sec-choice}

Choice gets less attention in philosophical decision theory than one
might expect. The focus is usually on either value (e.g., this has value
17 and that has value 12) or preference (e.g., this is preferable to
that). Standard presentations treat norms on choice as almost an
afterthought. After a long discussion of values, preferences, or both,
the typical theorist breezily says that the norm is to choose the most
valuable or most preferred option.

There is a long tradition in economics, going back to Paul Samuelson
(\citeproc{ref-Samuelson1938}{1938}) and Herman Chernoff
(\citeproc{ref-Chernoff1954}{1954}), of taking choice to be primary.
Some of this literature rested on largely behaviourist or positivist
assumptions: it was better to theorize with and about choice because,
unlike preferences or values, it was observable. The picture was not
dissimilar to this recently expressed view:

\begin{quote}
Standard economics does not address mental processes and, as a result,
economic abstractions are typically not appropriate for describing them.
(\citeproc{ref-GulPesendorfer2008}{Gul and Pesendorfer 2008, 24})
\end{quote}

That's not going to be my approach here. I'm going to start not with
observable choice dispositions, like the economists, or with choice
frequencies, as psychologists like R. Duncan Luce
(\citeproc{ref-Luce1959}{1959}) do, but with judgments about
choiceworthiness. In familiar terminology\footnote{See, for instance,
  Hansson and Grüne-Yanoff (\citeproc{ref-sep-preferences}{2024}), where
  I learned about the Gul and Pesendorfer quote.}, I'm taking a
mentalist approach not a behaviourist approach. Much of the formal work
on choice theory has been done by theorists from the behaviourist side,
and I'll inevitably draw from their work. But the most important source
I'll be using is someone much more sympathetic to mentalism: Amartya
Sen.~In particular, I'll draw heavily on his ``Collective Choice and
Social Welfare'' (\citeproc{ref-Sen1970sec}{Sen {[}1970{]} 2017}), and
also on the literature that grew out of that book.

I'm not going to take a stance on the metaphysics of choiceworthiness
judgments. I'll sometimes talk as if they are beliefs. However, if
someone wanted to maintain a sharp belief-desire split and hold that
choiceworthiness involves an interplay of the two (like preference), I
wouldn't object. The key assumption is that choiceworthiness ascriptions
involve the same kind of mental state as preference ascriptions. The
biggest difference is that choiceworthiness relates to sets of any size,
while preference relates to pairs.

\section{Values First?}\label{sec-values}

Both to clarify what kind of question I'm asking, and to set aside one
kind of answer, I'm going to start looking not at preferences or
choices, but at numerical values. At first glance, it might seem that
many decision theorists take values to ground the other norms. One
should prefer the more valuable. So we get theorists discussing
Newcomb's Problem largely by offering theories about the value of
uncertain outcomes (like taking both boxes) in terms of the values and
probabilities of those outcomes.

On second glance, though, there are at least four reasons it is
implausible that values are really what ground the other attitudes.

First, it's surprising to have a numerical measure like this not have a
unit. We sometimes say an outcome has value 17 \emph{utils}, but `util'
is a placeholder, not a real unit like kilograms or volts. This is
related to the next reason.

Second, the orthodox view is that these values are only defined up to a
positive affine transformation. If it's appropriate to represent Chooser
with value function \emph{v}, it's appropriate to represent them with
any value function \emph{f} where
\emph{f}(\emph{o})~=~\emph{av}(\emph{o})~+~\emph{b}, for positive
\emph{a}. This transformation is allowed because values merely reflect
Chooser's preferences over outcomes and lotteries, which the
transformation preserves. In other words, the transformation is allowed,
and the definedness claim is true, because values are grounded in
comparatives.

Third, it's not at all obvious why values should be anything like
numbers. Indeed, treating values as numbers creates problems in cases
involving infinite goods.\footnote{See, e.g., Nover and Hàjek
  (\citeproc{ref-Nover2004}{2004}), or Goodman and Lederman
  (\citeproc{ref-GoodmanLederemanArXiV}{2024}).} Why should values have
the topology of the reals rather than any number of other possible
topologies? Why aren't they, for example, quintuples of rational numbers
ordered lexicographically? If one takes preferences to be primary, and
generates utility functions via representation theorems, as in Ramsey
(\citeproc{ref-RamseyTruthProb}{1926}) or von Neumann and Morgenstern
(\citeproc{ref-vNM1944yaml}{1944}), there is a reason for why values
should be numbers. But if values are primitive, it seems like an
unanswered and I'd say unanswerable question.

Finally, there is something very strange about the idea of values that
are not in any way comparative. Value appears to be inherently
comparative, reflecting how options compare to their alternatives.

The argument of this section will not be, I suspect, particularly
contentious. It's a widespread view, if often implicit, that that
values, and norms on values, are ultimately grounded in comparatives.
What is going to be contentious is the claim that preferences can't do
the job of grounding values and preferences. Instead, that task will
have to be done by judgments of, and norms about, choiceworthiness.

\section{Coherence}\label{sec-coherence}

There is one other striking thing about the picture we get in von
Neumann and Morgenstern (\citeproc{ref-vNM1944yaml}{1944}), which I
think is still broadly endorsed by decision theorists. The aim is to put
norms on preference, and hence on values and choices. But these norms
are almost always defined in terms of other preferences. For example, if
one strictly prefers \emph{x} to \emph{y}, and \emph{y} to \emph{z}, one
should prefer \emph{x} to \emph{z}. It seems circular to ground
normative facts about preference in other preferences. It looks like the
grounding relation will be at least cyclic, and possibly intransitive.

There are two (related) responses to this circularity. One is to take
the view, which perhaps Hume held, that the only constraints on
preference are coherence constraints. Another response allows for
non-coherence constraints on preference---for example, it may be wrong
to prefer the world's destruction to one's finger getting
scratched---but treats these as part of a separate subject from decision
theory. The result is the same: decision theory largely is about what it
takes for various preferences to cohere with one another.

I don't particularly agree with this picture, but I'm going to accept
half of it for the purposes of this paper. That is, I'll assume it's not
the role of decision theory to criticize the person who prefers the
destruction of the world to the scratching of their finger. That person
either violates no norms, or violates a different kind of norm from
those of decision theory. Decision theory, on the latter view, takes
Chooser's preferences over ends as given and judges Chooser on how well
their instrumental preferences serve these preferences over ends.

I'll argue that even if decision theory is about coherence, it should be
about coherence between choiceworthiness judgments. So the central
question shifts. Instead of asking whether preferences should satisfy
transitivity or independence, we ask whether choiceworthiness judgments
should satisfy conditions like those described in
Section~\ref{sec-properties}. As we'll see, the questions about which
conditions are genuine coherence constraints on choiceworthiness are
tied up with the metaphysical question about the priority of preferences
and choices.

\section{Sen on Preference}\label{sec-sen}

The main binary relation Sen uses, which he denotes \emph{R}, is such
that \emph{xRy} means that Chooser either prefers \emph{x} to \emph{y},
or is indifferent between them. The first disjunct obtains if
¬\emph{yRx}; the second disjunct obtains if \emph{yRx}. (As he sometimes
puts it, these are the symmetric and asymmetric parts of the relation.)
We'll write these two disjuncts as \emph{xPy} and \emph{xIy}. Formally,
they are defined in terms of \emph{R} in (1) and (2). (Throughout, I'm
leaving off wide scope universal quantifiers over free variables.)

\begin{enumerate}
\def\labelenumi{(\arabic{enumi})}
\tightlist
\item
  \emph{xPy}~↔︎~(\emph{xRy}~∧~¬\emph{yRx})
\item
  \emph{xIy}~↔︎~(\emph{xRy}~∧~\emph{yRx})
\end{enumerate}

It is important not to conflate indifference with equality. It is not
assumed that \emph{I} is transitive; indeed Sen makes great use of
models where it is not. If we take transitivity to be part of the
definition of equality, it is misleading to gloss \emph{xRy} as that
\emph{x} is greater than \textbf{or equal} to \emph{y} (for Chooser).
For this reason I won't use ≿ to denote it, since that is most naturally
read as that \emph{x} is better than or equal to \emph{y}.\footnote{Indeed,
  I'll use \emph{x}~≿~\emph{y} to explicitly mean that \emph{x} is
  better than or equal to \emph{y}, where equality is understood in the
  sense of the next paragraph.}

Contemporary philosophy more commonly starts with \emph{P} and a fourth
relation \emph{E}, where \emph{xEy} means that \emph{x} and \emph{y} are
equally good. On this picture, both (1) and (2) are true, but the
explanatory direction in both cases is right-to-left. So \emph{xRy} just
is ¬\emph{yPx}, and then \emph{xIy} is still defined via (2). On the
version Sen uses, it's a little trickier to define \emph{E}, but (3) is
plausible.

\begin{enumerate}
\def\labelenumi{(\arabic{enumi})}
\setcounter{enumi}{2}
\tightlist
\item
  \emph{xEy}~↔︎~{[}(\emph{xRz}~↔︎~\emph{yRz})~∧~(\emph{zRx}~↔︎~\emph{zRy}){]}
\end{enumerate}

That is, two options are equally good iff they are substitutable in
other preference relations. Given all these results, we can show that
the following claims are all tightly connected.

\begin{enumerate}
\def\labelenumi{(\arabic{enumi})}
\setcounter{enumi}{3}
\tightlist
\item
  \emph{xPy}~∨~\emph{xEy}~∨~\emph{yPx}
\item
  (\emph{xPy}~∧~\emph{yIz})~→ \emph{xPz}
\item
  (\emph{xIy}~∧~\emph{yIz})~→ \emph{xIz}
\end{enumerate}

(4) is what Ruth Chang (\citeproc{ref-Chang2017}{2017}) calls the
trichotomy thesis. (5) is what Sen calls PI-transitivity, and (6) is
what he calls II-transitivity.

Sen makes very few assumptions about \emph{R}, but it will simplify our
discussion to start introducing some assumptions here.\footnote{He makes
  few assumptions because he was interested in exploring what
  assumptions about preference are crucial to the impossibility theorem
  that Arrow (\citeproc{ref-Arrow1951}{1951}) derives. He initially
  noticed that without (6), Arrow's theorem didn't go through. This
  turned out to be less significant than it seemed, because Allan
  Gibbard (\citeproc{ref-Gibbard2014}{2014}) proved that a very similar
  theorem can be proven even without (6). See Sen
  (\citeproc{ref-Sen1969}{1969}) for the original optimism that this
  might lead to an interesting way out of the Arrovian results, and Sen
  (\citeproc{ref-Sen1970sec}{{[}1970{]} 2017}) for a more pessimistic
  assessment in light of Gibbard's result. Sen reports that Gibbard
  originally proved his result in a term paper for a seminar at Harvard
  in 1969 that he co-taught with Arrow and Rawls. Much of what I'm
  saying in this paper can be connected in various ways to the
  literature on Arrow's impossibility theorem, but I won't draw out
  those connections here.} We'll assume that \emph{R} is reflexive,
everything is at least as good as itself, and that \emph{P} is
transitive. Sen (\citeproc{ref-Sen1970sec}{{[}1970{]} 2017, 66}) notes
that if \emph{P} is transitive and \emph{R} is `complete' in the sense
that \emph{xRy}~∨~\emph{yRx} holds for arbitrary \emph{x} and \emph{y},
then (5) and (6) are equivalent. It's also easy to show that given (3)
plus these assumptions, (4) and (6) are equivalent.\footnote{Proof:
  Assume (4) is false. So the right hand side of (3) is false. Without
  much loss of generality, assume that \emph{xRz}~∧~¬\emph{yRz}; the
  other cases all go much the same way. So all the disjuncts are false.
  From ¬\emph{xPy} and ¬\emph{yPx} we get \emph{yRx}~∧~\emph{xRy}, i.e.,
  \emph{xIy}. And \emph{xRz} implies \emph{zIx}. So we have a
  counterexample to II-transitivity, since \emph{zIx} and \emph{xIy},
  but since ¬\emph{yRz}, \emph{yIz} is false. So if (4) is false, (6) is
  false. In the other direction, assume we have a counterexample to (6),
  i.e., \emph{xIy} and \emph{yIz} but not \emph{xIz}. From \emph{xIy} we
  immediately get that the two outer disjuncts of (4) are false. From
  \emph{yIz} we get \emph{yRz} and \emph{zRy}. So if \emph{xEy}, (3)
  implies that \emph{xRz} and \emph{zRx}, i.e., \emph{xIz}. But we
  assumed that ¬\emph{xIz}. So all three disjuncts of (4) are false.
  That is, if (6) fails, so does (4), completing the proof that they are
  equivalent.}

What should we call the principle (4)? The terminology around here gets
potentially confusing. If we define \emph{x}~≿~\emph{y} to just mean the
disjunction \emph{xPy}~∨~\emph{xEy}, and assume that \emph{E} is
symmetric, then (4) is equivalent to
\emph{x}~≿~\emph{y}~∨~\emph{y}~≿~\emph{x}. That's what Gustafsson
(\citeproc{ref-Gustafsson2025}{forthcoming}) calls \emph{completeness},
and I feel that's often what philosophers understand by `completeness'.
In his economic work, Sen (\citeproc{ref-Sen1970sec}{{[}1970{]} 2017})
uses the term `completeness' for a slightly different property of
preference relations, namely \emph{xRy}~∨~\emph{yRx}. In both cases the
claim is that some preference relation is guaranteed to hold in one
direction or other. The issue is whether that relation is the
disjunction of \emph{P} and \emph{E}, or the disjunction of \emph{P} and
\emph{I}.

Both of these notions are useful to have. There has been a huge amount
of literature on (4), i.e., \emph{x}~≿~\emph{y}~∨~\emph{y}~≿~\emph{x},
less so on \emph{xRy}~∨~\emph{yRx}. But the latter is useful because
various interesting possibilities open up both in social choice theory
and in the relationship between preference and choice, if it is
dropped.\footnote{On the latter, see Bradley
  (\citeproc{ref-Bradley2015}{2015}).}

While philosophers widely discuss (4), the economics literature
discusses it less often under that formulation. That literature does
contain several works discussing (6), starting with important works by
Wallace E. Armstrong (\citeproc{ref-Armstrong1939}{1939},
\citeproc{ref-Armstrong1948}{1948}, \citeproc{ref-Armstrong1950}{1950}).
In most of those works it is assumed that \emph{P} is transitive and
that \emph{xRy}~∨~\emph{yRx}, so (4) and (6) are equivalent. But the
different focus leads to more terminological confusion.

More generally, using `completeness' for either \emph{x} ≿ \emph{y} ∨
\emph{y} ≿ \emph{x} or \emph{xRy} ∨ \emph{yRx} is potentially confusing,
since readers may not know which property the author intends. So I'll
follow Chang (\citeproc{ref-Chang2017}{2017}) and say that relations
satisfying (4) are \emph{trichotomous}, and use \emph{definedness} for
\emph{xRy}~∨~\emph{yRx}.

\section{Properties of Choice Functions}\label{sec-properties}

In philosophy, we're sufficiently familiar with properties of preference
relations (transitivity, reflexivity, acyclicity, etc.) that these terms
don't need defining. We're mostly less familiar with properties of
choice functions. A choice function takes a set of options as input and
returns a non-empty subset of that set as output. The elements of the
output are the choiceworthy members of the original set. So
\emph{C}(\{\emph{a},~\emph{b},~\emph{c},~\emph{d}\})~=~\{\emph{a},~\emph{c}\}
means that if Chooser has to pick from \emph{a}, \emph{b}, \emph{c} and
\emph{d}, then \emph{a} and \emph{c} are choiceworthy, and \emph{b} and
\emph{d} are not.

I'll present six important properties that choice functions may have.
The first four are discussed in some detail by Sen
(\citeproc{ref-Sen1970sec}{{[}1970{]} 2017}), and I'll use his
terminology for them. The fifth is due to Aizerman and Malishevski
(\citeproc{ref-Aizerman1981}{1981}), and is usually named after
Aizerman. The sixth is discussed by Blair et al.
(\citeproc{ref-Blair1976}{1976}).

\begin{description}
\tightlist
\item[Property α]
(\emph{x}~∈~\emph{C}(\emph{S})~∧~\emph{x}~∈~\emph{T}~∧~\emph{T}~⊆~\emph{S})~→
\emph{x}~∈~\emph{C}(\emph{T})
\end{description}

That is, if \emph{x} is choiceworthy in a larger set, it remains
choiceworthy in any smaller set containing it. This is sometimes called
the \emph{Chernoff condition}, after Herman Chernoff
(\citeproc{ref-Chernoff1954}{1954}), and sometimes called
\emph{contraction consistency}.

\begin{description}
\tightlist
\item[Property β]
(\emph{x}~∈~\emph{C}(\emph{T})~∧~\emph{y}~∈~\emph{C}(\emph{T})~∧~\emph{T}~⊆~\emph{S})~→
(\emph{x}~∈~\emph{C}(\emph{S}) ↔ \emph{y}~∈~\emph{C}(\emph{S}))
\end{description}

That is, if \emph{x} and \emph{y} are both choiceworthy in a smaller
set, then in any larger set they are either both choiceworthy or neither
is. Intuitively, if \emph{x} and \emph{y} are both choiceworthy
together, then anything better than \emph{x} is also better than
\emph{y}.

\begin{description}
\tightlist
\item[Property γ]
(\emph{x}~∈~\emph{C}(\emph{S})~∧~\emph{x}~∈~\emph{C}(\emph{T}))~→
(\emph{x}~∈~\emph{C}(\emph{S}~∪~\emph{T}))
\end{description}

That is, if \emph{x} is choiceworthy in two sets, it is choiceworthy in
their union. This is sometimes called \emph{expansion}, e.g., by Hervé
Moulin (\citeproc{ref-Moulin1985}{1985}).

\begin{description}
\tightlist
\item[Property δ]
(\emph{x}~∈~\emph{C}(\emph{T})~∧~\emph{y}~∈~\emph{C}(\emph{T})~∧~\emph{T}~⊆~\emph{S})~→
(\{\emph{y}\} ≠ \emph{C}(\emph{S}))
\end{description}

This is a weakening of β. It says that if \emph{x} and \emph{y} are both
choiceworthy in the smaller set, then after options are added, it can't
be that \emph{y} alone is choiceworthy. If \emph{x} is not choiceworthy
in the larger set, that's because some other option, not \emph{y}, is
chosen in place of it.

\begin{description}
\tightlist
\item[Property Aiz]
(\emph{C}(\emph{S})~⊆~\emph{T}~∧~\emph{T}~⊆~\emph{S})~→
\emph{C}(\emph{T})~⊆~\emph{C}(\emph{S})
\end{description}

That is, if the smaller set contains all of the choiceworthy members of
the larger set, then no option is choiceworthy in the smaller set but
not the larger set. If \emph{x} is an unchoiceworthy member of \emph{S},
then it can only become choiceworthy by deleting choiceworthy members of
\emph{S}, not other unchoiceworthy ones.

\begin{description}
\tightlist
\item[Path Independence]
\emph{C}(\emph{S}~∪~\emph{T}) =
\emph{C}(\emph{C}(\emph{S})~∪~\emph{C}(\emph{T}))
\end{description}

The choiceworthy options from a union of two sets equal the choiceworthy
options from the union of each set's choiceworthy members. This is a
kind of independence of irrelevant alternatives principle; the presence
or absence of unchoiceworthy members of \emph{S} and \emph{T} doesn't
affect what should be chosen from \emph{S} ∪ \emph{T}.

I'll describe the effects of these properties in more detail in
subsequent sections.

\section{Property α}\label{sec-alpha}

This is the most commonly used constraint on choice functions, and it
seems intuitive. If \emph{x} is choiceworthy from a larger set, deleting
unchosen options shouldn't make it unchoiceworthy. Sen
(\citeproc{ref-Sen1970sec}{{[}1970{]} 2017, 323--26}) discusses two
possible counterexamples.

One is where the presence of options on the menu gives Chooser relevant
information. If the only two options are having tea with a particular
friend or staying home, Chooser will choose tea. But if the option of
taking cocaine with that friend is added, Chooser will stay home. The
natural thing to say here is that when one gets new information,
\emph{C} changes, so there isn't really a single \emph{C} here which
violates α.\footnote{For a quick argument for that, if Chooser learns
  the only options are tea and staying home because the friend has just
  run out of cocaine, they might still stay home.}

The more interesting case is where the value Chooser puts on options is
dependent on what options are available. So imagine Chooser prefers more
cake to less, but does not want to take the last slice. If the available
options are zero slices or one slice of cake, Chooser will choose zero.
But if two slices of cake is an option, Chooser will choose one, again
violating α.

This is a trickier case, and the natural thing to say is that Chooser
doesn't really have the same options in the two cases. Taking the last
slice of cake isn't the same thing as taking one slice when two are
available. But this move has costs. In particular, it makes it hard to
say that \emph{C} should be defined for any set of options. It doesn't
clearly make sense to ask Chooser to pick between \emph{taking one
slice, which is the last}, and \emph{taking three slices when five are
available}.

Still, I'll set these individuation issues aside and assume, following
most theorists, that α constrains coherent choice functions and that
choice functions are defined over arbitrary sets of options.

\section{Assumptions}\label{sec-assumptions}

I've said a few times I'm assuming this or that, so it's a good time to
put in one place the assumptions I'm making. These aren't intended to
stack the deck in my favour; if any of these assumptions are false,
their falsity makes the view that choices are not binary (a) more
plausible, but (b) harder to state. Anyway, here's what has been
assumed.

\begin{enumerate}
\def\labelenumi{(\roman{enumi})}
\tightlist
\item
  \emph{P} is transitive, i.e., \emph{xPy}~∧~\emph{yPz}~→ \emph{xPz}
\item
  \emph{R} is `defined', i.e., \emph{xRy}~∨~\emph{yRx}.
\item
  \emph{R} is reflexive, i.e., \emph{xRx}.
\item
  \emph{C} is non-empty, i.e., \emph{C}(\emph{S}) ≠ ∅.
\item
  \emph{C} is defined everywhere, i.e., there is a universe of options
  \emph{U} and all subsets of \emph{U} are in the domain of \emph{C}.
\item
  \emph{C} satisfies α.
\item
  The universe \emph{U} of options, that \emph{S} is a subset of, and
  \emph{x} is drawn from, is finite.
\end{enumerate}

In Section~\ref{sec-alpha} we saw one reason to reject (v), namely that
we might want to individuate options in terms of what else is available.
The cases we'll discuss in Section~\ref{sec-mixed} provide another, but
rather than explore that, I'll follow standard practice and assume (v)
throughout this paper.

When \emph{R} satisfies (i)-(iii), I'll follow Luce
(\citeproc{ref-Luce1956}{1956}) and call it a \emph{semiorder}. When it
also satisfies trichotomy, i.e., (4), I'll call it a \emph{weak order}.

\section{Defining Binariness}\label{sec-defining}

With these seven assumptions on board, it's easy to state what it is for
a choice function to be binary.

First, we'll define an inversion function \emph{B} (for binary) that
maps preference relations to choice functions, and vice-versa. Both are
sets of ordered pairs, which we'll define directly. I'll assume that
there is a universe \emph{U} of options and every option and set of
options is drawn from it.

If the input to \emph{B} is a preference relation \emph{R}:

\begin{enumerate}
\def\labelenumi{(\arabic{enumi})}
\setcounter{enumi}{6}
\tightlist
\item
  \emph{B}(\emph{R}) = \{⟨\emph{S}, \emph{x}⟩:
  ∀\emph{y}(\emph{y}~∈~\emph{S}~→ \emph{xRy})\}
\end{enumerate}

That is, \emph{B}(\emph{R}) is the choice function that, for any set
\emph{S}, selects what Sen calls `maximal' members---those members to
which nothing is strictly preferred.\footnote{Hansson
  (\citeproc{ref-Hansson2009}{2009}) calls this the `liberal
  maximisation' rule. He contrasts it with five other rules, which are
  distinct in general but equivalent given \emph{R} is a semiorder.}

If the input to \emph{B} is a choice function \emph{C}:

\begin{enumerate}
\def\labelenumi{(\arabic{enumi})}
\setcounter{enumi}{7}
\tightlist
\item
  \emph{B}(\emph{C}) = \{⟨\emph{x}, \emph{y}⟩:
  \emph{x}~∈~\emph{C}(\{\emph{x}, \emph{y}\})\}
\end{enumerate}

That is, \emph{B}(\emph{C}) is the preference relation stating that in
pairwise choices, an element is preferred only if it could be chosen
from the pair. Sen (\citeproc{ref-Sen1970sec}{{[}1970{]} 2017, 319})
calls these functions `basic binary', but the distinction marked by
`basic' doesn't matter given that \emph{R} is a semiorder and α holds.
Since we're assuming α, we'll use the simpler version.

A choice function \emph{C} is \textbf{binary} iff (9) holds:\footnote{Sen
  calls these functions `basic binary', but the distinction he's drawing
  attention to by adding `basic' doesn't make a difference given
  \emph{R} is a semiorder and α.}

\begin{enumerate}
\def\labelenumi{(\arabic{enumi})}
\setcounter{enumi}{8}
\tightlist
\item
  \emph{C} = \emph{B}(\emph{B}(\emph{C}))
\end{enumerate}

Converting \emph{C} into a preference relation and back yields the same
function. This will fail if the part of \emph{C} which concerns choice
from menus with three or more members contains extra information than
just the restriction of \emph{C} to pair sets.

The core claim of this paper is that there are coherent choice functions
which are not binary. A related claim is that a plausible pair of
coherence constraints that you can state using \emph{B} do not in fact
hold. The constraints require that \emph{C} = \emph{B}(\emph{R}) and
\emph{R} = \emph{B}(\emph{C}).

\section{Property β}\label{property-ux3b2}

If we start with choice functions, the definition of \emph{E} in (3) is
too simple. A better definition is in (10).

\begin{enumerate}
\def\labelenumi{(\arabic{enumi})}
\setcounter{enumi}{9}
\tightlist
\item
  \emph{xEy} ↔ {[}∀\emph{S}(\{\emph{x}, \emph{y}\}~⊆~\emph{S}~→
  (\emph{x}~∈~\emph{C}(\emph{S}) ↔ \emph{y}~∈~\emph{C}(\emph{S}))){]}
\end{enumerate}

That is, \emph{x} and \emph{y} are equal iff one is never chosen when
the other is not.\footnote{Without α, this is too weak, since it doesn't
  entail that \emph{x} and \emph{y} are intersubstitutable in general.
  But we won't worry about that.} Given this notion of equality, there
is an intuitive gloss on β: Two options are both choiceworthy iff they
are equal.\footnote{This gloss also assumes α.}

If choice functions are grounded in numerical values, then β follows
naturally. Assume there is some function \emph{v} from options to
numbers, and \emph{o} is choiceworthy iff \emph{v}(\emph{o}) is maximal.
Then if \emph{x} and \emph{y} are both choiceworthy, they must have the
same value. In any set, they will either both be choiceworthy (if no
alternative is more valuable) or neither will be (if some alternative is
more valuable).

More generally, given the assumptions from
Section~\ref{sec-assumptions}, \emph{C} satisfies β iff
\emph{B}(\emph{C}) is trichotomous. This is equivalent to
\emph{B}(\emph{C}) satisfying II-transitivity. Unsurprisingly, the two
historically significant cases of intuitive counterexamples to
II-transitivity also generate intuitive counterexamples to β.

The first example involves distinct but indistinguishable
options.\footnote{The idea that humans can't distinguish similar options
  is important in Fechner (\citeproc{ref-Fechner1860}{1860}), a work
  which is discussed in Beiser (\citeproc{ref-sep-fechner}{2024}). The
  earliest connection I've found between this and indifference being
  intransitive is in Armstrong (\citeproc{ref-Armstrong1939}{1939}).
  Armstrong's example is rather confusing; the one I'll use here is
  based on Luce (\citeproc{ref-Luce1956}{1956}).} Assume that Chooser
prefers more sugar in their coffee to less, but can only tell two
options apart if they differ by 10 grains of sugar or more. Now consider
these three options:

\begin{quote}
\emph{x} = Coffee with 100 grains of sugar.\\
\emph{y} = Coffee with 106 grains of sugar.\\
\emph{z} = Coffee with 112 grains of sugar.
\end{quote}

This is said to be a counterexample to II-transitivity because Chooser
is indifferent between \emph{x} and \emph{y}, and between \emph{y} and
\emph{z}, but strictly prefers \emph{z} to \emph{x}. It's also a
counterexample to β. Chooser would choose either from \emph{x} and
\emph{y}, but when \emph{z} is added, \emph{y} is still choiceworthy but
\emph{x} is not.

This example was historically important, but it's rarely discussed in
the contemporary philosophical literature. It could be because
philosophers were convinced by the argument by Delia Graff Fara
(\citeproc{ref-Fara2001}{2001}) that phenomenal indistinguishability is
in fact transitive. But it was widely discussed in economics, especially
after Luce (\citeproc{ref-Luce1956}{1956},
\citeproc{ref-Luce1959}{1959}) used similar examples to argue that
preferences form a semiorder

The cases that were more important in the philosophical literature are
what Chang (\citeproc{ref-Chang1997}{1997}) calls `small improvement'
cases. The earliest example of this form I know is from Luce and Raiffa
(\citeproc{ref-LuceRaiffa1957}{1957}).\footnote{I haven't found a case
  like this in Luce's sole-authored works, and indeed Debreu
  (\citeproc{ref-Debreu1960}{1960}) notes that a related case raises
  problems for one of the central assumptions of Luce
  (\citeproc{ref-Luce1959}{1959}).} In their notation,
\emph{P}(\emph{x},\emph{y}) represents the probability that Chooser will
select \emph{x} when \emph{x} and \emph{y} are both available.

\begin{quote}
Suppose that \emph{a} and \emph{b} are two alternatives of roughly
comparable value to some person, e.g., trips from New York City to Paris
and to Rome. Let \emph{c} be alternative a plus \$20 and \emph{d} be
alternative \emph{b} plus \$20. Clearly, in general \emph{P}(\emph{a},
\emph{c}) = 0 and \emph{P}(\emph{b}, \emph{d}) = 0. It also seems
perfectly plausible that for some people \emph{P}(\emph{b}, \emph{c})
\textgreater{} 0 and \emph{P}(\emph{a}, \emph{d}) \textgreater{} 0, in
which event \emph{a} and \emph{b} are not comparable, and so axiom 2
{[}i.e., (4){]} is violated. (\citeproc{ref-LuceRaiffa1957}{Luce and
Raiffa 1957, 375})
\end{quote}

An example with the same structure, involving a boy, a bicycle, and a
bell, is discussed by Lehrer and Wagner
(\citeproc{ref-LehrerWagner1985}{1985}), and mistakenly attributed to
Armstrong (\citeproc{ref-Armstrong1939}{1939}).\footnote{Many authors
  subsequently made the same attribution; if you want to see some
  examples, search for the word `bicycle' among the citations of
  Armstrong's paper on Google Scholar.}

The usual way these cases are discussed, starting with Luce and Raiffa,
is that they violate a certain kind of comparability. For example, Luce
and Raiffa say there is a sense in which the two holidays are `not
comparable'. I want to resist this reading. The core intuition in small
improvement cases is that β fails. Chooser would choose either option
from \{\emph{a}, \emph{b}\}, but if \emph{a}\textsuperscript{+} is added
as an option, \emph{a} becomes unchoiceworthy. If we add the assumption
that \emph{R} = \emph{B}(\emph{C}), then it does follow that trichotomy
fails, and there is a sense in which they are incomparable. But without
that assumption, it's consistent to say that these are counterexamples
to β but not to trichotomy. We'll return to this point in
Section~\ref{sec-dorr}.

\section{Properties γ and δ}\label{sec-gamma}

Assume \emph{R} does not satisfy trichotomy, but is a semiorder, and
\emph{C} = \emph{B}(\emph{R}). Then β will fail, but γ and δ will hold.
Conversely, for any \emph{C} where γ and δ hold, there is a semiorder
\emph{R} such that \emph{C} = \emph{B}(\emph{R})
(\citeproc{ref-Sen1970sec}{Sen {[}1970{]} 2017, 320}). We're not going
to be very interested in δ, but we will be very interested in γ.

The reason γ holds when \emph{R} is a semiorder and \emph{C} =
\emph{B}(\emph{R}) is instructive. If \emph{x} is choiceworthy among
\emph{S}, then nothing in \emph{S} is better than \emph{x}. Similarly,
if \emph{x} is choiceworthy among \emph{T}, then nothing in \emph{T} is
better than \emph{x}. So nothing in \emph{S}~∪~\emph{T} is better than
\emph{x}. So \emph{x} is choiceworthy among \emph{S}~∪~\emph{T}.

Conversely, if there are cases where \emph{C} should not satisfy γ, then
we'll have an argument that \emph{C} should not be grounded in some
semiorder \emph{R}. Showing that there are such cases will be one of the
main tasks of the rest of this paper.

We had two kinds of counterexamples to β, but only one of them will be
relevant here. I don't think there are any intuitive counterexamples to
γ that start with Fechner-style reflections on the intransitivity of
indifference. But there are going to be variations on the bicycle and
bell example that generate intuitive counterexamples to γ. We'll come
back to these in Section~\ref{sec-badcomp}.

It is common to say that when \emph{C} = \emph{B}(\emph{R}) for some
semiorder \emph{R}, \emph{C} is \textbf{rationalizable}, and when
\emph{R} is a partial order, \emph{C} is \textbf{rationalizable by a
partial order}. I find this terminology tendentious - why should
semiorders be the only things that can make \emph{C} rational? And as
we'll see in Section~\ref{sec-games}, it conflicts with the notion of a
choice being \emph{rationalizable} in game theory. But it's a common
enough terminology that I wanted to mention it here.

\section{Aizerman's Property}\label{sec-aiz}

The property Aiz is not particularly intuitive. Fortunately, it turns
out to be equivalent, given our assumptions, to one that is: Path
Independence.\footnote{Unless stated otherwise, the results in this
  section, along with proofs, can be found in the helpful survey by
  Moulin (\citeproc{ref-Moulin1985}{1985}).} Path Independence says that
to find what's choiceworthy from a union of sets, you only have to
consider which options are choiceworthy in the smaller sets.

Note that this isn't just saying that the options choiceworthy from the
union are choiceworthy from one of the members. That is implied by α.
What it says is that the presence or absence of unchosen options from
\emph{S} and \emph{T} doesn't affect which options are choiceworthy from
the union.

There is a very natural kind of model where α and Path Independence
hold, but β and γ do not. Let \emph{O} be a set of total orderings of
\emph{U}. (A total ordering is a relation \emph{R} such that
\emph{xRy}~∨~\emph{yRx}~∨~\emph{x}=\emph{y}.) Further, let
\emph{C}(\emph{S}) be the set of Pareto optimal members of \emph{S}
relative to those orderings. That is, it is the set of members of
\emph{S} such that no other member of \emph{S} is better according to
every member of the set of orderings. Every such model satisfies α and
Path Independence, but some violate β and γ.

There is a further relation between Path Independence and these models.
For any \emph{C} that satisfies α and Path Independence, there is some
set of total orderings over \emph{S} such that \emph{C}(\emph{S}) is the
set of Pareto optimal members of \emph{S} according to that set of
orderings.

It might seem strange after all the talk of weak orderings and
semiorders that we're now using total orders. Given that \emph{U} is
finite, this follows naturally from well-known properties of finite
semiorders. Any semiorder with a finite domain (and hence any such weak
order) can be represented by a set of total orders: \emph{x} is strictly
preferred to \emph{y} in the semiorder iff it is preferred in all the
total orders. (In fact we can put a sharp cap on how many such orders
there must be.) So being Pareto optimal relative to some set of
semiorders (or weak orders) is equivalent to being Pareto optimal
relative to a larger set of total orders.

If \emph{C} is determined by a set of orders in this way, it is said to
be \textbf{pseudorationalizable}. These choice functions are not always
binary. Consider one simple example, where \emph{U} is
\{\emph{x},~\emph{y},~\emph{z}\},
\emph{C}(\emph{U})~=~\{\emph{x},~\emph{z}\}, and for any other \emph{S},
\emph{C}(\emph{S})~=~\emph{S}. That is the choice function determined by
the pair of orderings \emph{x}~≻~\emph{y}~≻~\emph{z}~and
\emph{z}~≻~\emph{y}~≻~\emph{x}. This satisfies α, δ and Aiz, but not β
or γ. And it isn't binary. \emph{B}(\emph{C}) is the universal relation
\emph{R}, since whenever \emph{S} is a pair set,
\emph{C}(\emph{S})~=~\emph{S}.

The approach discussed here is obviously similar to what is called the
``multi-utility'' approach to representing incomplete preferences (Evren
and Ok (\citeproc{ref-EvrenOk2011}{2011})). In this approach we find a
set of utility functions and say \emph{xRy} iff
\emph{u}(\emph{x})~≿~\emph{u}(\emph{y}) for all \emph{u} in the set.

I'm going to argue over subsequent sections that this is a coherent
choice function, and hence not all coherent choice functions are binary.

\section{Preference and Trade}\label{sec-trade}

Let's go back to why we might have wanted to focus on binary preference
relations. One reason is that preference might naturally explain trade.
If Chooser trades a cow for some magic beans, it's natural to explain
that by saying Chooser preferred the magic beans to the cow.

In the monetary economies we live in, very little trade involves barter.
Most trade involves money, which is primarily valuable instrumentally.
If Chooser buys some shoes for \$100, we could say that Chooser prefers
the shoes to the money. But that doesn't seem like the full story, since
there is a reason Chooser values the money as they do. The deeper point
is that Chooser's money creates a budget constraint, and Chooser judges
the shoes to be the best use of that \$100 among available options. That
is, it seems more informative to describe Chooser as choosing the shoes
from the menu of things that cost \$100 than to describe them as
preferring the shoes to the money. In general, choiceworthiness seems
more relevant to explaining market behaviour in monetary economies than
preference, unless choiceworthiness can be defined in terms of
preference. So let's turn to some reasons to think that it cannot.

\section{Degenerate Games}\label{sec-games}

Say a two-player game is \textbf{degenerate} iff the payoffs to one of
the players are constant in all outcomes of the game. For convenience,
assume Column is the player with constant payoffs. So
Table~\ref{tbl-upmid}, Table~\ref{tbl-middown} and
Table~\ref{tbl-allthree} are degenerate games.

\begin{longtable}[]{@{}rcc@{}}
\caption{A degenerate two-option game}\label{tbl-upmid}\tabularnewline
\toprule\noalign{}
& Left & Right \\
\midrule\noalign{}
\endfirsthead
\toprule\noalign{}
& Left & Right \\
\midrule\noalign{}
\endhead
\bottomrule\noalign{}
\endlastfoot
Up & 10,0 & 0,0 \\
Middle & 1,0 & 1,0 \\
\end{longtable}

\begin{longtable}[]{@{}rcc@{}}
\caption{Another degenerate two-option
game}\label{tbl-middown}\tabularnewline
\toprule\noalign{}
& Left & Right \\
\midrule\noalign{}
\endfirsthead
\toprule\noalign{}
& Left & Right \\
\midrule\noalign{}
\endhead
\bottomrule\noalign{}
\endlastfoot
Middle & 1,0 & 1,0 \\
Down & 0,0 & 10,0 \\
\end{longtable}

\begin{longtable}[]{@{}rcc@{}}
\caption{A degenerate three-option
game}\label{tbl-allthree}\tabularnewline
\toprule\noalign{}
& Left & Right \\
\midrule\noalign{}
\endfirsthead
\toprule\noalign{}
& Left & Right \\
\midrule\noalign{}
\endhead
\bottomrule\noalign{}
\endlastfoot
Up & 10,0 & 0,0 \\
Middle & 1,0 & 1,0 \\
Down & 0,0 & 10,0 \\
\end{longtable}

Start with the following two assumptions, which seem fairly plausible
for games like these.

\begin{itemize}
\tightlist
\item
  If a move is part of a Nash equilibrium, it is choiceworthy.
\item
  A move is choiceworthy only if there is some probability distribution
  over the other player's moves such that the move maximizes expected
  utility given that distribution.\footnote{The notion of a
    rationalizable choice, in the sense of B. Douglas Bernheim
    (\citeproc{ref-Bernheim1984}{1984}) and David Pearce
    (\citeproc{ref-Pearce1984}{1984}), slightly strengthens this
    constraint. A choice is rationalizable iff it maximizes expected
    utility relative to a probability assignment that only gives
    positive probability to the other player, or players, making
    rationalizable choices. That's circular as stated, but one can
    remove the circularity at the cost of making the definition somewhat
    less intuitive.}
\end{itemize}

In degenerate games, these necessary and sufficient conditions for
choiceworthiness coincide, but in general they are rather
different.\footnote{Since rationalizability is between these notions, it
  also coincides with them for degenerate games.}

In Table~\ref{tbl-upmid} and Table~\ref{tbl-middown}, both options are
choiceworthy by this standard. The Nash equilibria of
Table~\ref{tbl-upmid} are Up-Left and Middle-Right, while the Nash
equilibria of Table~\ref{tbl-middown} are Middle-Left and Down-Right. So
each option is choiceworthy for each player in each game. But in
Table~\ref{tbl-allthree}, the only choiceworthy options for Row are Up
and Down. Whatever probability Row assigns to Left or Right, Middle will
not maximize expected utility. So this is a counterexample to γ. Middle
is choiceworthy from \{Up, Middle\} and from \{Middle, Down\}, but not
from their union.

It follows immediately from Lemma 3 in Pearce
(\citeproc{ref-Pearce1984}{1984}) that in degenerate games, a choice
satisfies those conditions for being choiceworthy iff it is not strictly
dominated, where this includes being dominated by mixed strategies. (In
Table~\ref{tbl-allthree}, Middle is strictly dominated by the 50/50
mixture of Up and Down.) This means that the choices will satisfy Path
Independence. An option is not dominated by the options in
\emph{S}~∪~\emph{T} iff it is not dominated by the undominated options
in \emph{S}~∪~\emph{T}, i.e., by the options in
\emph{C}(\emph{S})~∪~\emph{C}(\emph{T}). So removing options that are
not choiceworthy in \emph{S} and \emph{T} from the union doesn't change
what is undominated, i.e., choiceworthy.

This last paragraph is the start of a pattern in the examples that
follow. Although I'll be arguing against γ, I won't be arguing against
Aiz/Path Independence. I'm not going to offer anything like a conclusive
argument for Aiz, but the pattern suggests adding Aiz to α yields the
fundamental constraints on coherent choiceworthiness.

\section{Choice Under Uncertainty}\label{sec-uncertainty}

Luce and Raiffa (\citeproc{ref-LuceRaiffa1957}{1957}) discuss what they
call choices under `uncertainty', by which they mean choices where
Chooser cannot assign probabilities to the states. Martin Peterson
(\citeproc{ref-Peterson2017}{2017}) calls these choices under
`ignorance'. None of the proposed decisive rules for choice under
uncertainty/ignorance are particularly compelling; all lead to very
strange outcomes.

The best approach, in my opinion, is to treat these choices like
degenerate games. Indeed, degenerate games are really a paradigm of
choice under ignorance; Row has no reason to assign any particular
probability to Column's choice. Further, what the game theory textbooks
say about degenerate games seems fairly plausible; any undominated
option is choiceworthy. The same goes for choices under ignorance; any
undominated option is choiceworthy.

If that's right, then the three examples from Section~\ref{sec-games}
can be repurposed as examples of choice under ignorance, replacing Left
and Right with \emph{p} and ¬\emph{p}, and the same analysis will hold.
Again γ will fail, because Middle is choiceworthy when there is one
other option, but not when there are two. So there's no binary
comparison of Middle with the other two options that explains the facts
about what is choiceworthy in the three cases.

\section{Multiple Equilibria}\label{sec-multieq}

This is a decision theory paper, so we need to introduce a demon who can
reliably predict Chooser's choices. We'll start with a version of what
Skyrms (\citeproc{ref-Skyrms1982}{1982}) calls `Nice Demon'; we'll just
call them Demon. In Table~\ref{tbl-nice-demon}, Chooser selects Up or
Down, and Demon either predicts Up (PU), or predicts Down (PD). Whatever
Chooser does, Demon is very likely to have predicted correctly.

\begin{longtable}[]{@{}rcc@{}}
\caption{First version of Nice
Demon}\label{tbl-nice-demon}\tabularnewline
\toprule\noalign{}
& PU & PD \\
\midrule\noalign{}
\endfirsthead
\toprule\noalign{}
& PU & PD \\
\midrule\noalign{}
\endhead
\bottomrule\noalign{}
\endlastfoot
Up & 6 & 0 \\
Down & 0 & 4 \\
\end{longtable}

Jack Spencer (\citeproc{ref-Spencer2023}{2023}) argues against views,
like the one defended by Dmitri Gallow
(\citeproc{ref-Gallow2020}{2020}), which say only Up is choiceworthy in
Table~\ref{tbl-nice-demon}.\footnote{Evidential decision theory also
  says that only Up is choiceworthy in Table~\ref{tbl-nice-demon}.
  What's distinctive about Gallow's position is that it says only Up is
  choiceworthy here, although only two-boxing is choiceworthy in the
  original Newcomb Problem.} Spencer's argument relies on a simple
principle. If Chooser plans to play Down, then Chooser knows Down will
have the best return, and it's not irrational to make a choice if one
knows, when one makes it, that the choice will have the best return.
Gallow (\citeproc{ref-Gallow2024}{2024}) replies that in some cases,
especially those involving high stakes, it isn't always rational to do
what one knows will produce the best return. Since I think it's a
fundamental feature of knowledge that it does rationalize action in the
way Spencer suggests (Weatherson
(\citeproc{ref-WeathersonKAHIS}{2024})), I think Spencer has the better
of this exchange. So I'll follow him and assume that both options are
choiceworthy in games like Table~\ref{tbl-nice-demon} where multiple
options are self-verifying.\footnote{The version of causal decision
  theory set out by David Lewis (\citeproc{ref-Lewis1981bn}{1981}) has a
  more complicated verdict on Table~\ref{tbl-nice-demon}. It says the
  uniquely rational choice is whatever choice Chooser believes they will
  make. This isn't the view I'm assuming, since I take both options to
  be permissible. But it's very close to the picture I present, without
  endorsing, in Section~\ref{sec-dorr}. On that view, once Chooser
  commits to a choice, it becomes the only permissible one, so the view
  isn't far from Lewis's.}

Now add a third option, Exit, which has a guaranteed return of 1. So the
game table looks like Table~\ref{tbl-nice-demon-exit}.

\begin{longtable}[]{@{}rcc@{}}
\caption{Second version of Nice
Demon}\label{tbl-nice-demon-exit}\tabularnewline
\toprule\noalign{}
& PU & PD \\
\midrule\noalign{}
\endfirsthead
\toprule\noalign{}
& PU & PD \\
\midrule\noalign{}
\endhead
\bottomrule\noalign{}
\endlastfoot
Up & 6 & 0 \\
Exit & 1 & 1 \\
Down & 0 & 4 \\
\end{longtable}

In Table~\ref{tbl-nice-demon-exit}, Exit is not choiceworthy. Whatever
credences Chooser has about what Demon has done, it is better in
expectation to choose one of Up or Down. But if either Up or Down were
unavailable, Exit would be a permissible choice. This follows from the
earlier assumption that any self-verifying choice in cases like
Table~\ref{tbl-nice-demon} is choiceworthy. So like in
Section~\ref{sec-games} and Section~\ref{sec-uncertainty}, we have a
counterexample to γ.

There is an important general lesson from this case. What makes an
option choiceworthy in cases like this is that it is utility maximizing
\emph{once it is chosen}. We'll turn next to a more dramatic
illustration of this point.

\section{Mixed Strategies}\label{sec-mixed}

Demon has stopped being nice, and now wants to play Rock-Paper-Scissors
with Chooser. Given Demon's powers, this could go badly for Chooser.
Happily, Chooser can choose to randomize\footnote{I'm not going to argue
  for this here, but I think it is part of being ideally practically
  rational that one is able to randomize, just like it is part of being
  ideally practically rational that one can make calculations
  costlessly.}, and all Demon can predict is the probability that
Chooser's random process will come up with any option. So the best thing
for Chooser to do is to pick one of the three options at random. That
alone will maximize expected utility conditional on being chosen.

Not coincidentally, the only Nash equilibrium of Rock-Paper-Scissors is
for each player to randomize. Nash equilibria are arguably the only
sensible strategies if one assumes that every other player has
Demon-like abilities to detect what one is doing. But it's a
long-running puzzle in game theory how it can be uniquely rational to
randomize. Why can choosing randomly be better than choosing one of the
things one is randomizing between? To turn this rhetorical question into
an argument, note that the following three principles are inconsistent.

\begin{enumerate}
\def\labelenumi{(\arabic{enumi})}
\setcounter{enumi}{10}
\tightlist
\item
  randomizing is the only choiceworthy strategy in Rock-Paper-Scissors.
\item
  If only one choice is choiceworthy, it is rationally preferred to all
  other choices.
\item
  It is irrational to prefer a random mixture of some choices to every
  one of the choices.
\end{enumerate}

Since (11) is a standard view in game theory, (12) is a standard view in
choice theory, and (13) is a standard view in preference-based decision
theory, it is a little disconcerting to see they are inconsistent.

The example in Section~\ref{sec-multieq} shows how to steer through this
trilemma. Choiceworthiness is fundamentally an ex ante notion, and
preference is fundamentally an ex post notion. The reason Spencer's view
about Table~\ref{tbl-nice-demon} is right is not that the rational
chooser is indifferent between Up and Down. Rather, choosers don't have
preferences between these choices until they have chosen, and once they
choose, they prefer their choice.

Similarly in Rock-Paper-Scissors (especially against a Demon), what's
true is that prior to deciding, the only rational choice is to
randomize. Before the choice, there is no such thing as Chooser's
preferences over the options; what I earlier called definedness fails.
Once one has chosen, one should be strictly indifferent between the
options; they each have the same expected utility. In particular, one
should not prefer to re-randomize rather than put into effect the result
of the random process.

So we should reject (12) in its most natural interpretation. randomizing
in Rock-Paper-Scissors is the only choiceworthy option, but until a
choice is made, Chooser simply shouldn't have preferences over these
options.

Most of the arguments in this paper against the binariness of choice
turn on counterexamples to γ, but this is a distinct argument.
Sometimes, as in Rock-Paper-Scissors, there are grounds for rational
choice, but no grounds for rational preference. The only preferences
that would ground the choice would violate (13). So rational
choiceworthiness cannot be grounded in rational preference.

This is the deepest reason why \emph{C} = \emph{B}(\emph{R}) must be
wrong; \emph{C} and \emph{R} are fundamentally about different kinds of
attitudes. \emph{C} is about what is rational prior to making a choice,
\emph{R} is a constraint that must be satisfied once a choice is made.
Outside of Newcomb-like cases, this distinction won't often matter, but
it is another reason that the equation fails.

\section{Multiple Attributes and Decisiveness}\label{sec-sartre}

Sartre (\citeproc{ref-Sartre1946}{1946/2007}) has a famous example of a
young man (we'll call him Pierre) caught between two imperatives. The
actual example Sartre gives is complicated in interesting ways, but
we'll work with a very simple version of the story. Pierre lives in
occupied France during WWII, and feels torn between his duty to care for
his ailing mother, and his duty to fight for his occupied country. What
should he do?

The case is extremely underdescribed, but the following verdicts have
seemed plausible to many people. First, Pierre can rationally, and
morally, choose either option. Caring for his mother and fighting Nazis
are both noble goals, and it's fine to pursue either. Second, and this
follows from the first and the fact that the case is underdescribed, the
options are not equally good. After all, a small improvement to either
would not break a tie between them.

A third intuition is more contentious, but perhaps plausible: there is
something wrong about Pierre going back and forth between the two
choices; he should make a choice and stick to it. On this view, there is
something intrinsically good about settling on a choice and sticking to
it. What makes this intuition less than fully clear is that oscillation
would be practically bad in any realistic version of the case. He could
spend the whole war travelling between England and France as he changes
his mind on where he should be, and that would be bad. The intuition I
have in mind is that there is something good about taking a stance and
committing to it, even outside of the practical costs of changing one's
mind.\footnote{As Moss (\citeproc{ref-Moss2015}{2015}) points out, it is
  less clear this is intrinsically bad if there is more time between the
  reconsiderations; it makes more sense to change one's mind than to
  rapidly flip-flop. Conversely, if Pierre resembles the young Thomas
  Schelling (as discussed by Holton (\citeproc{ref-Holton1999}{1999})),
  firmly committing to one plan and then another over the course of
  successive nights, he's doing something wrong even if it has no
  practical downsides.}

A very simple model for these intuitions is that Pierre's situation is
surprisingly like the person playing Table~\ref{tbl-nice-demon}. There
are two options here, and either is acceptable, but once one is chosen,
it becomes the preferred one. There are two good values here, caring for
family and caring for country, and Pierre's fundamental choice is to
adopt one of these as his value. As Chang
(\citeproc{ref-Chang2024}{2024}) puts it, he chooses to put `his whole
self' behind one of those values. Once that choice is made, his
preferences and his actions follow naturally.

It is easy enough to reject the third intuition. Perhaps Pierre could
rationally, as Chang (\citeproc{ref-Chang2024}{2024}) puts it,
\emph{drift} into one choice. Perhaps he could adopt one path and
sometime later rationally regret his choice, because the other value
strikes his later self as more important. Still, if one holds all three
intuitions, toy models like Table~\ref{tbl-nice-demon} capture a
surprising amount of what's going on with Pierre. And those models are
incompatible with choiceworthiness being binary.

\section{β and Incompleteness}\label{sec-dorr}

The Pierre case in Section~\ref{sec-sartre} is, even without the third
intuition, a straightforward counterexample to β. Just to spell it out,

\begin{quote}
\emph{x} = Help mother\\
\emph{y} = Fight Nazis\\
\emph{z} = Help mother plus one extra ration book
\end{quote}

If the choices are \emph{x} and \emph{y}, either is acceptable. If the
choices are \emph{x}, \emph{y} and \emph{z}, \emph{x} alone is
unchoiceworthy. So β fails.

This is the Small Improvement argument, and it is often thought to be an
argument against completeness, i.e., (4). The point of this section is
to explain why the argument against (4) might fail, even though the
argument against β succeeds.

Imagine someone were convinced by the arguments in Dorr, Nebel, and
Zuehl (\citeproc{ref-DorrEtAl2023}{2023}) that (4) must be true. Their
arguments turn on semantic properties of comparatives; they claim that
since \emph{R} is a comparative, it must be trichotomous, as all
comparatives are.\footnote{For what it's worth, I think this argument
  fails because of the case of `stronger' in logic. They address this
  case, but I don't think their response works. But it would be a huge
  digression to follow that thread through.} Now it would be very
strange if the semantics of comparatives in natural language entailed
that some intuitions about three-way choice had to be false. And in fact
these claims about semantics are consistent with β failing in cases like
Pierre's.\footnote{I'm focussing here on the argument in Dorr, Nebel,
  and Zuehl (\citeproc{ref-DorrEtAl2023}{2023}), but a similar response
  works if someone is convinced of (4) by the argument in Broome
  (\citeproc{ref-Broome1997}{1997}).}

The following view is coherent, and its coherence shows that Dorr et
al.'s view of comparatives does not entail β.

\begin{itemize}
\tightlist
\item
  β fails in cases like Pierre's.
\item
  An option is choiceworthy for Pierre iff no option is
  \emph{determinately} preferred to it.
\item
  To make a choice, Pierre must determine which value is really his.
\item
  Once he does that, his preferences will satisfy (4).
\item
  Before he does that, there are two possibilities. One is that his
  preferences aren't even defined over these options, so asking which
  option is preferred is like asking whether the number 7 is taller,
  shorter, or the same height as justice. Another is that it is vague
  what Pierre's preferences are, but any resolution of the vagueness
  makes (4) true. This latter option fits nicely with the idea that
  \emph{C} should satisfy Aiz, since it should be determined by a set of
  orderings, each of them the possible precisifications, or
  determinations, of his current state.\footnote{This view of preference
    mirrors the view of credence defended by Carr
    (\citeproc{ref-Carr2020}{2020}).}
\end{itemize}

One objection to this view is that it seems to imply that Pierre could
rationally choose one option while he prefers, but does not
determinately prefer, another. But this isn't what the view implies.
Once Pierre makes the choice, he must, if he is rational, determine that
his preferences match it. Preference, as I argued in
Section~\ref{sec-mixed}, is fundamentally an ex post notion. That is,
(14) is true, while (15) is false.

\begin{enumerate}
\def\labelenumi{(\arabic{enumi})}
\setcounter{enumi}{13}
\tightlist
\item
  Once Pierre chooses an option, there must not be some other option he
  strictly prefers to it.
\item
  Before an option is choiceworthy for Pierre, it must be better or just
  as good (given his preferences) as any other option.
\end{enumerate}

What (14) says is that if Pierre is rational, he will choose what he
prefers. That does not mean, as (15) claims, that rational choices are
those that are preferred prior to the choice. Choice, on this picture,
is prior to preference, both analytically and, in this case, causally.

When I say this is coherent, I don't mean to half-heartedly say that it
is correct. My preferred view is that Pierre could rationally drift (in
Chang's sense) into either option, and if he does, (4) would fail even
ex post. All I mean to argue here is that the case against β doesn't
turn on this view, and one can coherently reject β while endorsing
trichotomy.

\section{Bad Compromises}\label{sec-badcomp}

My version of the Pierre example was very simple, but it allows for some
interesting complications. As stated, you might think Pierre isn't
thinking through his choices well enough. He should join the local
resistance, so he can stay close enough to his mother to help, while
also fighting the Nazis.

But maybe that's a terrible option. We can easily imagine that the
resistance is either so useless that it does practically nothing, or so
good at recruiting that it has little useful work. It's just as easy to
imagine that it creates busywork that dramatically reduces how much he
can care for his mother, while not doing much to help the war effort. At
risk of trivialising the issue, we can imagine that Pierre's options
look like this, where the two columns represent how much each option
respects/promotes the relevant value.

\begin{longtable}[]{@{}rcc@{}}
\caption{Pierre's options, including
resistance}\label{tbl-pierreresist}\tabularnewline
\toprule\noalign{}
& Caring value & Fighting value \\
\midrule\noalign{}
\endfirsthead
\toprule\noalign{}
& Caring value & Fighting value \\
\midrule\noalign{}
\endhead
\bottomrule\noalign{}
\endlastfoot
Stay Home & 10 & 0 \\
Join resistance & 1 & 1 \\
Join Free French & 0 & 10 \\
\end{longtable}

Intuitively, if these are Pierre's options he should not join the
resistance; it's almost the worst of both worlds. We don't want to rest
on an intuition, and I'll argue for the view that Pierre should shun the
resistance in the next two sections. If it is irrational to join the
resistance in Table~\ref{tbl-pierreresist}, this is another
counterexample to γ. To see this, consider these two variants of the
case.

\begin{quote}
\textbf{No exit}: While Pierre is deliberating, he hears that the
options for getting to the Free French have been decisively cut off. (In
the original, he worries this might happen.) Now his only two options
are to stay home, or to join the resistance.
\end{quote}

\begin{quote}
\textbf{Promise}: Pierre's brother Jean is fighting the Nazis. Pierre
has promised Jean that if Jean is killed, Pierre will take up the fight
in some way. Sadly, Jean is killed, and Pierre regards this promise as
binding. Now his only options are to join the resistance, or join the
Free French.
\end{quote}

In either case, joining the resistance seems choiceworthy. In both
cases, it is the option that does best of the remaining choices on one
of the criteria. Pierre could decide he endorses that criterion as his
own, and acts accordingly. So the resistance is choiceworthy amongst
either pair of options, but not amongst their union.

\section{Levi and Sen}\label{sec-levisen}

In \emph{Hard Choices}, Isaac Levi (\citeproc{ref-Levi1986}{1986})
defended a view where the choiceworthy options are only those that
maximize value on some resolution of the incompleteness in the agent's
values. (He doesn't use this language, but so far this is similar to the
multi-utility approach discussed in Section~\ref{sec-aiz}.) Levi also
had views about what further constraints there should be on choice, so
he did not defend the view I've been discussing where any option that is
maximal on any resolution is choiceworthy. But his views are still
relevant here, because this requirement that a choice be maximal on a
resolution meant that he was committed to γ failing, and choices not
being binary.

A central example which he uses, and which Sen
(\citeproc{ref-Sen2004}{2004}) picks up on, involves an executive
looking to hire a secretary. I'll follow Sen and call the executive
Ms.~Jones.\footnote{I've also changed the secretaries' names.} She is
looking for a secretary with good typing skills and good stenography
skills. (This is the 1980s.) We'll conceive of these skills, a bit
arbitrarily, as distinct values. There are three candidates: Jack,
Danny, and Luke, and their value on each measure is given in
Table~\ref{tbl-secretaries}. (I've slightly adjusted the numbers to
match the earlier examples.)

\begin{longtable}[]{@{}rcc@{}}
\caption{Three candidates for a secretarial
position}\label{tbl-secretaries}\tabularnewline
\toprule\noalign{}
& Typing Skill & Stenography Skill \\
\midrule\noalign{}
\endfirsthead
\toprule\noalign{}
& Typing Skill & Stenography Skill \\
\midrule\noalign{}
\endhead
\bottomrule\noalign{}
\endlastfoot
Jack & 10 & 0 \\
Danny & 1 & 1 \\
Luke & 0 & 10 \\
\end{longtable}

Levi argues that if the numbers are like this, and Danny is barely
better than the worst of the other two on each metric, he is not
choiceworthy. That's true even though he might be choiceworthy if one or
the other candidates was unavailable.

Sen argues that the right choice theory in this case should be
``inarticulate'', and say that any of the three is choiceworthy. He
responds to the intuition Levi presents with a dilemma.

On the first horn, we understand these numbers as representing an
objective measure of the skills of the candidates at each of the tasks.
As Sen points out, it's easy to imagine situations where someone who is
not abysmal at either half of the job is more useful than someone who is
an expert on one, and abysmal on the other.

On the other horn, we measure the ``importance''
(\citeproc{ref-Sen2004}{Sen 2004, 53}) of each skill for the task at
hand. Sen argues, by analogy with the difficulty in establishing a
social welfare function out of the welfare of each individual, that
there will be no way to do this. Let's turn to how we might go about it.

\section{Lotteries, Choices, and Values}\label{sec-lotteries}

In this section I'll offer a response, on Levi's behalf, to Sen's
dilemma. The example will draw heavily on recent work by Harvey Lederman
(\citeproc{ref-Lederman2025}{forthcoming}).\footnote{See also Lederman
  (\citeproc{ref-Lederman2023}{2023}), and Tarsney, Lederman, and Spears
  (\citeproc{ref-LedermanEtAl2025}{forthcoming}). This is not to say
  Lederman would endorse anything like this response; as we'll see in
  the next section he is sympathetic to Sen's view that the
  multi-utility approach is mistaken. But every principle I'll use here
  is discussed, one way or another, in his work, and I've drawn heavily
  on that discussion in what follows.} We'll start by imagining that
Ms.~Jones might have a choice not of secretaries, but of agencies, and
she has a reasonable credal distribution over the skills of the people a
particular agency might send. That is, each choice of agency will be a
choice of a \emph{lottery}, where she doesn't choose a package of
skills, but a probability distribution over some outcomes, where each
outcome is a secretary with a numerical skill on each attribute.

To set this up, I'll need three new bits of notation. I'll write
\emph{Lxy} for a lottery that has equal chance of returning outcomes
\emph{x} and \emph{y}, where these might be secretaries or further
lotteries. I'll write \emph{x}~=~\emph{y} for what I've previously
written as \emph{xEy}. It means that \emph{x} and \emph{y} are equally
good by Ms.~Jones's lights. It's perhaps suboptimal to introduce new
notation for an old concept, but the mix of L's and E's became hard to
read. Finally, I'll write ⟨\emph{x}, \emph{y}⟩ for a secretary with
skill \emph{x} at typing and skill \emph{y} at stenography.

If the numbers in Table~\ref{tbl-secretaries} measure importance, then
Ms.~Jones's preferences should satisfy (a special case of) what Lederman
calls \textbf{Unidimensional Expectations}.

\begin{description}
\tightlist
\item[Unidimensional Expectations (UE)]
~\newline~ \emph{L}⟨\emph{x}\textsubscript{1},
\emph{y}⟩⟨\emph{x}\textsubscript{2}, \emph{y}⟩ =
⟨(\emph{x}\textsubscript{1} + x\textsubscript{2})/2, \emph{y}⟩

\emph{L}⟨\emph{x}, \emph{y}\textsubscript{1}⟩⟨\emph{x},
\emph{y}\textsubscript{2}⟩ = ⟨\emph{x}, (\emph{y}\textsubscript{1} +
\emph{y}\textsubscript{2})/2⟩
\end{description}

That is, the value of a lottery where the possible outcomes agree on one
dimension also has the same value on that dimension, and has the
expected value of the other dimension. If the lottery does not involve
value conflict, old-fashioned expected value maximisation is the way to
go.

This is enough to rule out the example Sen has in mind on the first
horn, where a secretary with skill 1 on either dimension is more than
half as valuable as a secretary with skill 10. But it's not enough to
say that Ms.~Jones should not hire Danny. Unidimensional Expectations is
consistent with one resolution of his indeterminate value being that the
value of ⟨\emph{x},\emph{y}⟩ is \emph{xy}. If that's a permissible
resolution, then there will be a resolution on which Danny is maximally
valuable. But there are further principles that do rule out Danny. The
most intuitive argument I know uses the following.

\begin{description}
\tightlist
\item[Substitution of Identicals (SI)]
If \emph{x} = \emph{y}, then \emph{Lxz} = \emph{Lyz} and \emph{Lzx} =
\emph{Lzy}.
\item[No Trade-Off (NT)]
\emph{L}⟨\emph{x}, \emph{x}⟩⟨\emph{y}, \emph{y}⟩ = ⟨(\emph{x} +
\emph{y})/2,(\emph{x} + \emph{y})/2⟩
\item[Rearrangement of Outcomes (RO)]
\emph{L}(\emph{Lxy})(\emph{Lvw}) = \emph{L}(\emph{Lxw})(\emph{Lvy})
\item[Weak Independence (WI)]
If \emph{x} = \emph{Lxy} then \emph{x} = \emph{y}
\end{description}

Substitution of Identicals follows naturally from the idea that the
outcomes are truly equal, so it doesn't matter whether a lottery has an
outcome that results in one or the other. No Trade-Off, like
Unidimensional Expectations, says that when we are just considering
strictly better and worse options, so there are no relevant
complications about resolving indeterminacy, we're back in the land of
expected utility maximisation. Rearrangement of Outcomes follows from
the fact that the compound lotteries on either side of the identity sign
each have probability 1/4 of returning one of those four outcomes. And
Weak Independence, which is probably the most contentious of the lot,
says that if \emph{y} is not exactly as good as \emph{x}, then a 1/2
chance of \emph{y} should not be exactly as good as \emph{x}. Given
these principles, we can argue as follows.

\begin{longtable}[]{@{}rll@{}}
\toprule\noalign{}
\endhead
\bottomrule\noalign{}
\endlastfoot
1. & ⟨5, 5⟩ = \emph{L}⟨10, 5⟩⟨0,5⟩ & UD \\
2. & ⟨10, 5⟩ = \emph{L}⟨10, 10⟩⟨10,0⟩ & UD \\
3. & ⟨0, 5⟩ = \emph{L}⟨0, 10⟩⟨0,0⟩ & UD \\
4. & ⟨5, 5⟩ = \emph{L}(\emph{L}⟨10, 10⟩⟨10,0⟩)⟨0,5⟩ & 1,3 SI \\
5. & ⟨5, 5⟩ = \emph{L}(\emph{L}⟨10, 10⟩⟨10,0⟩)(\emph{L}⟨0, 10⟩⟨0,0⟩) &
2, 4 SI \\
6. & ⟨5, 5⟩ = \emph{L}(\emph{L}⟨10, 10⟩⟨0,0⟩)(\emph{L}⟨0, 10⟩⟨10,0⟩) & 5
RO \\
7. & ⟨5, 5⟩ = \emph{L}⟨10, 10⟩⟨0,0⟩ & NT \\
8. & ⟨5, 5⟩ = \emph{L}⟨5, 5⟩(\emph{L}⟨0, 10⟩⟨10,0⟩) & 6, 7 SI \\
9. & ⟨5, 5⟩ = \emph{L}⟨0, 10⟩⟨10,0⟩ & 8 WI \\
\end{longtable}

To complete the argument, we need two plausible principles. First,
\emph{x} is not choiceworthy from \emph{S} when \emph{x} is not
choiceworthy from the set consisting of \emph{x} and \emph{Lyz}, for
\emph{y}, \emph{z} in \emph{S}. Second, \emph{x} is not choiceworthy
from \emph{S} when an option that is strictly better on every dimension
is in \emph{S}.

There are several assumptions here, and any one of them could be the
subject of a whole paper. But they each seem very plausible, and they
show how we can ground the intuition that the multi-utility approach is
on the right track, and sometimes an option is not choiceworthy because
it is close to the worst option on every dimension we care about.

\section{Negative Dominance}\label{sec-negdom}

Harvey Lederman (\citeproc{ref-Lederman2025}{forthcoming}) notes that
the picture I've developed, where Danny is unchoiceworthy, appears to
violate a plausible principle which he calls \textbf{Negative
Dominance}. Lederman gives a few versions of this principle, the
following is the version most relevant to the case.\footnote{In all the
  quotes I'll change the names and example to match the one I'm using.}

\begin{description}
\tightlist
\item[Negative Dominance (Goodness)]
If one game of chance is better for {[}Ms.~Jones{]} than another, some
prize in the first game is better for her than some prize in the second.
(\citeproc{ref-Lederman2025}{Lederman forthcoming, 13})
\end{description}

The main application of this is to reject the idea that
\emph{L}(Jack)(Luke) is better than Danny.

I said the picture from previous sections `appears to violate' Negative
Dominance because one more assumption is needed to generate the tension.
The assumption is that when Chooser has only two options, preference and
choiceworthiness are closely related. In particular, \emph{xPy} iff
\emph{C}(\{\emph{x}, \emph{y}\})~=~\{\emph{x}\}. I've been arguing
against principles linking preference and choice throughout this paper,
but I've mostly accepted that they are tightly connected when there are
only two options on the menu. Still, without some assumption linking
choice and preference, the picture in the previous sections implies
nothing about preference, hence it says nothing that contradicts
Negative Dominance. For now, let's assume \emph{xPy} iff
\emph{C}(\{\emph{x}, \emph{y}\})~=~\{\emph{x}\}, and I'll return at the
end to what happens without that assumption.

Negative Dominance seems like a very plausible principle if (and I think
only if) one thinks that the role of decision theory is to come up with
coherence constraints on \emph{preferences}. Centering decision theory
on preference follows naturally from the idea that values and choices
are ultimately grounded in preferences. As Lederman argues, preferences
about lotteries have to be grounded in something about their prizes, and
if preferences are fundamental, presumably they have to be grounded in
preferences over their prizes. The key response I'm making is that if
choiceworthiness is prior to preference, this last step doesn't follow.

So when Lederman says,

\begin{quote}
a strict preference for one game of chance over another must be
explained by a strict preference for one of the \emph{prizes} of the
first, by comparison to one of the prizes of the second
(\citeproc{ref-Lederman2025}{Lederman forthcoming, 18}, emphasis in
original)
\end{quote}

we should question the uses of `one'. What's true is that attitudes
towards games of chance should be somehow explained by attitudes towards
the prizes, but these attitudes need not be \emph{preferences}. For
instance, the fact that \emph{L}(Jack)(Luke) is better than Danny could
be grounded in the fact that Danny is not choiceworthy from \{Jack,
Danny, Luke\}.

But there are more complicated cases where Lederman's challenge of how
to properly ground attitudes to lotteries is more pressing. The
following case mixes Levi's case with the main example in Tarsney,
Lederman, and Spears (\citeproc{ref-LedermanEtAl2025}{forthcoming}).
Ms.~Jones is now trying to hire a programmer, and she has four
candidates, each of which has the skills in the four languages she cares
about shown in Table~\ref{tbl-programmers}.

\begin{longtable}[]{@{}rcccc@{}}
\caption{Four programmers, and their
skills}\label{tbl-programmers}\tabularnewline
\toprule\noalign{}
& Python & Java & C & Ruby \\
\midrule\noalign{}
\endfirsthead
\toprule\noalign{}
& Python & Java & C & Ruby \\
\midrule\noalign{}
\endhead
\bottomrule\noalign{}
\endlastfoot
Jane & 6 & 6 & 0 & 0 \\
Dolly & 0 & 0 & 6 & 6 \\
Lily & 5 & 0 & 5 & 0 \\
Suzy & 0 & 5 & 0 & 5 \\
\end{longtable}

When the menu consists of any set of the programmers, all the options
are choiceworthy. But Ms.~Jones strictly prefers \emph{L}(Jane)(Dolly)
to \emph{L}(Lily)(Suzy), since the former lottery is better in
expectation on all four dimensions. Lederman is right that this needs to
be explained, that it should be explained in terms of evaluative
features of the prizes (i.e., the programmers), and if the explanation
uses expected value, we should explain why expected value matters. No
explanation in terms of the choiceworthiness of some options will work
to explain why one lottery is strictly better than another.

The fact to be explained is that when \emph{L}(Jane)(Dolly) and
\emph{L}(Lily)(Suzy) are available, only the former is choiceworthy.
Here's how I explain it:

\begin{enumerate}
\def\labelenumi{\arabic{enumi}.}
\tightlist
\item
  Ms.~Jones has four values, and it is indeterminate how they should be
  balanced. This means both that she hasn't decided how to balance them,
  and maybe it is unnecessary, or even inadvisable, to balance them.
\item
  Given Unidimensional Expectations and No Trade-Off (as discussed in
  Section~\ref{sec-lotteries}), the only permissible balancings are
  linear mixtures of the values.
\item
  Given the result from Pearce (\citeproc{ref-Pearce1984}{1984})
  discussed in Section~\ref{sec-games} (his Lemma 3), a lottery is best
  on no linear resolution of the indecision in point 1 iff some
  available lottery over other choices is better in expectation on every
  value.
\item
  A lottery is choiceworthy from a menu of other lotteries (or options)
  iff it is optimal on some permissible resolution of this indecision.
\end{enumerate}

If \emph{L}(Lily)(Suzy) were choiceworthy, by 4 it would have to be best
on some resolution of Ms.~Jones's values. By 1 and 2, this means that it
is best on some linear mixture of these values. By 3, that means it is
better in expectation on one of these values. But it is not; on all four
dimensions \emph{L}(Jane)(Dolly) has expected value 3, and
\emph{L}(Lily)(Suzy) has expected value 2.5. Even though Ms.~Jones has
not resolved the indeterminacy in her values, the fact that any
resolution would mean she prefers the first lottery is enough reason to
prefer the first lottery.

In short, the focus on expected values comes not from any particular
importance on expectations as such, but from the thought that
permissible reactions to indeterminacy in values are constrained by
permissible reactions to resolutions of that indeterminacy, combined
with (a) constraints on resolutions like Unidimensional Expectations and
No Trade-Off, and (b) Pearce's result linking expected value to linear
mixtures of values.

I'll end with two related objections to this reasoning. One comes from
Jamie Dreier (\citeproc{ref-Dreier2022}{2022}). He is discussing which
options are better and worse, not more or less preferred, but a
translated version of this objection has bite. He writes,

\begin{quote}
To say,``Right, but no matter how we got rid of parity {[}i.e., the
denial of trichotomy{]} in favor of the usual relations, this prospect
would turn out to be better than that prospect, so really it must just
\emph{be} better,'' just seems to be a non sequitur. It's as though,
having heard that the number six has no spatial location at all, someone
replied, ``Right, but no matter how we assigned it a spatial location,
it would be reachable from Tucson in some finite amount of time by a
light wave, so we can conclude that it just is reachable from Tucson in
a finite amount of time.'' (\citeproc{ref-Dreier2022}{Dreier 2022}: 124)
\end{quote}

Relatedly, we might ask why the fact that Ms.~Jones would prefer one
lottery to another on any way of balancing her values should mean that
she has that preference now. Why should features of some other value
function, one not her own, constrain what she now values?\footnote{Compare
  the objection to supervaluationism in Fodor and Lepore
  (\citeproc{ref-Fodor1996}{1996}).}

My reply is that choice functions are meant to play a certain role, they
are meant to guide action. But they can't do this on their own. If
\emph{x} and \emph{y} are both in \emph{C}(\emph{S}), and Chooser must
choose from \emph{S}, Chooser needs something more. Chooser needs a
\emph{plan}. Ideally she would have what Gibbard
(\citeproc{ref-Gibbard2003}{2003}) calls a \emph{hyperplan}. A hyperplan
\emph{H} is a function that takes a menu of options \emph{S}, and
returns a member of \emph{S}. It is a very plausible constraint on
\emph{C} such that there is some coherent \emph{H} such that for all
\emph{S}, \emph{H}(\emph{S})~∈~\emph{C}(\emph{S}). I conjecture, though
I don't have a complete proof of this, that there are plausible
constraints on \emph{H}, similar to Unidimensional Expectations and No
Trade-Off, which imply that \emph{H} is coherent only if there is some
utility function such that \emph{H}(\emph{S}) is an element of \emph{S}
with maximal expected utility. That's far from a complete argument, but
it's the direction I think a reply to Dreier's good objection should
take.

The other objection also comes from Dreier's work. He notes that
principles like that some choice is choiceworthy iff it is optimal on
some permissible resolution of the incompleteness in the values are
plausible if we view incompleteness in value as a kind of indeterminacy.
So the explanation of Ms.~Jones's choice dispositions I gave in
Section~\ref{sec-negdom} goes through very smoothly if we say it is
indeterminate how she values different skills.

On this way of thinking, it's natural to reject the assumption with
which I started this section, i.e., \emph{xPy} iff \emph{C}(\{\emph{x},
\emph{y}\})~=~\{\emph{x}\}. What's more plausibly true is that \emph{x}
is \textbf{determinately} preferred to \emph{y} iff \emph{C}(\{\emph{x},
\emph{y}\})~=~\{\emph{x}\}. But if \emph{C}(\{\emph{x},
\emph{y}\})~=~\{\emph{x}, \emph{y}\}, then it is indeterminate what
preferences Ms.~Jones has.

This leads back naturally to positions similar to the one discussed in
Section~\ref{sec-dorr}. For example, we could say that preferences are
defined not in terms of choice functions, but in terms of hyperplans.
For a chooser with choice function \emph{C} and hyperplan \emph{H}, they
prefer \emph{x} to \emph{y} iff \emph{H}(\{\emph{x},
\emph{y}\})~=~\emph{x}. For more normal choosers, who have not
determined a hyperplan, it is indeterminate what their preferences are;
the preferences don't exist until (that part of) the hyperplan does.

On this view Ms.~Jones does not violate Negative Dominance because it
isn't true that she is indifferent between the four programmers. Rather,
it is indeterminate which of them she prefers. What's true of her
preferences is simply what is true on all ways of turning her choice
function into a hyperplan. That will include that her preferences are
trichotomous.\footnote{Indeed, the way I've set things up, her
  preferences satisfy \emph{xPy}~∨~\emph{yRx}; there are various ways to
  allow for equality in this framework, and hence preserving trichotomy
  while rejecting \emph{xPy}~∨~\emph{yRx}.}

The striking thing is that, like in Section~\ref{sec-dorr}, we can say
all this while saying all the things which trichotomy was supposed to
rule out. Ms.~Jones can rationally select any of the programmers, and
this remains true even if any of them improves by ε on any skill, but
she cannot rationally choose \emph{L}(Lily)(Suzy) when
\emph{L}(Jane)(Dolly) is available. The resulting position is one where
preferences are largely disconnected from rational choice.

The position set out in the last two paragraphs is not a compulsory
consequence of the arguments in the rest of the paper. We could simply
reject Negative Dominance (Goodness) and say that Ms.~Jones is indeed
indifferent between the programmers. But we, as theorists, have some
options here. Once we say that preferences are analytically posterior to
choiceworthiness judgments, there are a lot of ways to understand what
preferences are. Some important arguments, like Lederman's, are not
quite arguments against a particular view of choice, but the conjunction
of that view with a way of understanding preference. It could be we're
best off simply understanding preference a different way.

\section{Conclusion}\label{sec-conclusion}

This paper has been ultimately about the grounding of facts about
rational choice. I've been mostly concerned to argue against a popular,
if largely implicit, view: rational choice is grounded in rational
preference. If Chooser wants a holiday, and is choosing where to go,
which destinations are rationally choiceworthy is grounded in Chooser's
(rational) preferences over pairs of choices. I've rejected this
primarily for two reasons:

\begin{enumerate}
\def\labelenumi{\arabic{enumi}.}
\tightlist
\item
  As argued in Section~\ref{sec-multieq} and Section~\ref{sec-mixed},
  choiceworthiness is an ex ante concept, and preference is an ex post
  concept, and hence choiceworthiness is analytically prior to
  preference, so should not be grounded in preference.
\item
  Any choice function that violates γ cannot be generated from a
  preference relation, and there are many reasons for endorsing choice
  functions which violate γ.
\end{enumerate}

If rational choiceworthiness is not grounded in preferences, what is it
grounded in? There are two natural options here.

A subjectivist theory, as flagged in Section~\ref{sec-coherence}, says
that norms on choiceworthiness are just coherence norms. In particular,
the key norms are α and Aiz. What makes this choiceworthiness judgment
rational just is that it fits well with the other choiceworthiness
judgments. There are tricky metaphysical questions about just how
instances of coherence norms are grounded, and whether it will mean we
have to give up widely accepted principles like that grounding is
acyclic. But these questions aren't different in kind from ones that
we'd face on the more mainstream view that decision theory is largely
about preferences, and norms on preference are coherence norms.

A more objectivist view is also possible, and I'll end by just stating
it. The world is full of values, many of them. All of these values are
orderings (or perhaps semiorderings) of outcomes. An option is
rationally choiceworthy iff it does best, given Chooser's evidence, on
some permissible mixture of some of these values. If all goes well, the
values can be measured numerically, and the permissible mixtures are
linear mixtures, but as noted in Section~\ref{sec-values}, we need some
good arguments about why they should be numerical. If they are, the
argument in Section~\ref{sec-lotteries} should generalize to argue that
the permissible mixtures are linear mixtures. Whether those last two
sentences work or not, the picture is that the rationality of a choice
is grounded in something external to the agent, i.e., values in the
world.

I'm going to leave arguments about which of these views is correct, or
which positions between them might be preferable to another day. All I
hoped to do in this final section is to sketch ways in which the core
thesis of the paper, that the rationality of choices is prior to the
rationality of preferences, could be true.

\section*{References}\label{references}
\addcontentsline{toc}{section}{References}

\phantomsection\label{refs}
\begin{CSLReferences}{1}{0}
\bibitem[\citeproctext]{ref-Aizerman1981}
Aizerman, M., and A. Malishevski. 1981. {``General Theory of Best
Variants Choice: Some Aspects.''} \emph{IEEE Transactions on Automatic
Control} 26 (5): 1030--40.
\url{https://doi.org/10.1109/TAC.1981.1102777}.

\bibitem[\citeproctext]{ref-Armstrong1939}
Armstrong, W. E. 1939. {``The Determinateness of the Utility
Function.''} \emph{The Economic Journal} 49 (195): 453--67.
\url{https://doi.org/10.2307/2224802}.

\bibitem[\citeproctext]{ref-Armstrong1948}
---------. 1948. {``Uncertainty and the Utility Function.''} \emph{The
Economic Journal} 58 (229): 1--10.
\url{https://doi.org/10.2307/2226342}.

\bibitem[\citeproctext]{ref-Armstrong1950}
---------. 1950. {``A Note on the Theory of Consumer's Behaviour.''}
\emph{Oxford Economic Papers} 2 (1): 119--22.
\url{https://doi.org/10.1093/oxfordjournals.oep.a041384}.

\bibitem[\citeproctext]{ref-Arrow1951}
Arrow, Kenneth J. 1951. \emph{Social Choice and Individual Values}. New
York: John Wiley \& Sons.

\bibitem[\citeproctext]{ref-sep-fechner}
Beiser, Frederick C. 2024. {``{Gustav Theodor Fechner}.''} In \emph{The
{Stanford} Encyclopedia of Philosophy}, edited by Edward N. Zalta and
Uri Nodelman, {S}ummer 2024.
\url{https://plato.stanford.edu/archives/sum2024/entries/fechner/};
Metaphysics Research Lab, Stanford University.

\bibitem[\citeproctext]{ref-Bernheim1984}
Bernheim, B. Douglas. 1984. {``Rationalizable Strategic Behavior.''}
\emph{Econometrica} 52 (4): 1007--28.
\url{https://doi.org/10.2307/1911196}.

\bibitem[\citeproctext]{ref-Blair1976}
Blair, Douglas H, George Bordes, Jerry S Kelly, and Kotaro Suzumura.
1976. {``Impossibility Theorems Without Collective Rationality.''}
\emph{Journal of Economic Theory} 11 (3): 361--79.
\url{https://doi.org/10.1016/0022-0531(76)90047-8}.

\bibitem[\citeproctext]{ref-Bradley2015}
Bradley, Richard. 2015. {``A Note on Incompleteness, Transitivity and
Suzumura Consistency.''} In \emph{Individual and Collective Choice and
Social Welfare: Essays in Honor of Nick Baigent}, edited by Constanze
Binder, Giulio Codognato, Miriam Teschl, and Yongsheng Xu, 31--47.
Berlin: Springer.

\bibitem[\citeproctext]{ref-Broome1997}
Broome, John. 1997. {``Is Incommensurability Vagueness?''} In
\emph{Incommensurability, Comparability and Practical Reason}, edited by
Ruth Chang, 67--89. Cambridge, MA: Harvard University Press.

\bibitem[\citeproctext]{ref-Carr2020}
Carr, Jennifer Rose. 2020. {``Imprecise Evidence Without Imprecise
Credences.''} \emph{Philosophical Studies} 177 (9): 2735--58.
\url{https://doi.org/10.1007/s11098-019-01336-7}.

\bibitem[\citeproctext]{ref-Chang1997}
Chang, Ruth. 1997. {``Introduction.''} In \emph{Incommensurability,
Incomparability and Practical Reason.}, edited by Ruth Chang, 1--34.
Cambridge, MA: Harvard University Press.

\bibitem[\citeproctext]{ref-Chang2017}
---------. 2017. {``Hard Choices.''} \emph{Journal of the American
Philosophical Association} 3 (1): 1--21.
\url{https://doi.org/10.1017/apa.2017.7}.

\bibitem[\citeproctext]{ref-Chang2024}
---------. 2024. {``What's so Hard about Hard Choices?''} \emph{Erasmus
Journal for Philosophy and Economics} 17 (1): 272--86.
\url{https://doi.org/10.23941/ejpe.v17i1.872}.

\bibitem[\citeproctext]{ref-Chernoff1954}
Chernoff, Herman. 1954. {``Rational Selection of Decision Functions.''}
\emph{Econometrica} 22 (4): 422--43.
\url{https://doi.org/10.2307/1907435}.

\bibitem[\citeproctext]{ref-Debreu1960}
Debreu, Gerard. 1960. {``Review of \emph{Individual Choice Behavior: A
Theoretical Analysis}, by {R. Duncan Luce}.''} \emph{American Economic
Review} 50 (1): 186--88.

\bibitem[\citeproctext]{ref-DorrEtAl2023}
Dorr, Cian, Jacob M. Nebel, and Jake Zuehl. 2023. {``The Case for
Comparability.''} \emph{Noûs} 57 (2): 414--53.
\url{https://doi.org/10.1111/nous.12407}.

\bibitem[\citeproctext]{ref-Dreier2022}
Dreier, Jamie. 2022. {``Blessed Lives, Bright Prospects, Incomplete
Orderings.''} \emph{Oxford Studies in Normative Ethics} 12: 105--26.
\url{https://doi.org/10.1093/oso/9780192868886.003.0006}.

\bibitem[\citeproctext]{ref-EvrenOk2011}
Evren, Özgür, and Efe A. Ok. 2011. {``On the Multi-Utility
Representation of Preference Relations.''} \emph{Journal of Mathematical
Economics} 47 (4): 554--63.
\url{https://doi.org/10.1016/j.jmateco.2011.07.003}.

\bibitem[\citeproctext]{ref-Fara2001}
Fara, Delia Graff. 2001. {``Phenomenal Continua and the Sorites.''}
\emph{Mind} 110 (440): 905--36.
\url{https://doi.org/10.1093/mind/110.440.905}.

\bibitem[\citeproctext]{ref-Fechner1860}
Fechner, Gustav. 1860. \emph{Elemente Der Psychophysik}. Leipzig:
Breitkopf und H{ä}rtel.

\bibitem[\citeproctext]{ref-Fodor1996}
Fodor, Jerry A., and Ernest Lepore. 1996. {``What Cannot Be Valuated
Cannot Be Valuated, and It Cannot Be Supervaluated Either.''}
\emph{Journal of Philosophy} 93 (10): 516--35.
\url{https://doi.org/10.5840/jphil1996931013}.

\bibitem[\citeproctext]{ref-Gallow2020}
Gallow, J. Dmitri. 2020. {``The Causal Decision Theorist's Guide to
Managing the News.''} \emph{The Journal of Philosophy} 117 (3): 117--49.
\url{https://doi.org/10.5840/jphil202011739}.

\bibitem[\citeproctext]{ref-Gallow2024}
---------. 2024. {``It Can Be Irrational to Knowingly Choose the
Best.''} \emph{Australasian Journal of Philosophy} 103 (2): 540--46.
\url{https://doi.org/10.1080/00048402.2024.2310197}.

\bibitem[\citeproctext]{ref-Gibbard2003}
Gibbard, Allan. 2003. \emph{Thinking How to Live}. Cambridge, MA:
Harvard University Press.

\bibitem[\citeproctext]{ref-Gibbard2014}
---------. 2014. {``Social Choice and the Arrow Conditions.''}
\emph{Economics and Philosophy} 30 (3): 269--84.
\url{https://doi.org/10.1017/S026626711400025X}.

\bibitem[\citeproctext]{ref-GoodmanLederemanArXiV}
Goodman, Jeremy, and Harvey Lederman. 2024. {``Maximal Social Welfare
Relations on Infinite Populations Satisfying Permutation Invariance.''}
\url{https://arxiv.org/abs/arXiv:2408.05851}.

\bibitem[\citeproctext]{ref-GulPesendorfer2008}
Gul, Faruk, and Wolfgang Pesendorfer. 2008. {``The Case for Mindless
Economics.''} In \emph{Foundations of Positive and Normative Economics},
edited by Andrew Caplin and Andrew Schotter, 2--40. Oxford: Oxford
University Press.
\url{https://doi.org/10.1093/acprof:oso/9780195328318.003.0001}.

\bibitem[\citeproctext]{ref-Gustafsson2025}
Gustafsson, Johan E. forthcoming. {``A Behavioural Money-Pump Argument
for Completeness.''} \emph{Theory and Decision}, forthcoming.
\url{https://doi.org/10.1007/s11238-025-10025-3}.

\bibitem[\citeproctext]{ref-Hansson2009}
Hansson, Sven Ove. 2009. {``Preference-Based Choice Functions: A
Generalized Approach.''} \emph{Synthese} 171 (2): 257--69.
\url{https://doi.org/10.1007/s11229-009-9650-5}.

\bibitem[\citeproctext]{ref-sep-preferences}
Hansson, Sven Ove, and Till Grüne-Yanoff. 2024. {``{Preferences}.''} In
\emph{The {Stanford} Encyclopedia of Philosophy}, edited by Edward N.
Zalta and Uri Nodelman, {W}inter 2024.
\url{https://plato.stanford.edu/archives/win2024/entries/preferences/};
Metaphysics Research Lab, Stanford University.

\bibitem[\citeproctext]{ref-Holton1999}
Holton, Richard. 1999. {``Intention and Weakness of Will.''} \emph{The
Journal of Philosophy} 96 (5): 241--62.
\url{https://doi.org/10.2307/2564667}.

\bibitem[\citeproctext]{ref-Lederman2025}
Lederman, Harvey. forthcoming. {``Of Marbles and Matchsticks.''}
\emph{Oxford Studies in Epistemology}, forthcoming.

\bibitem[\citeproctext]{ref-Lederman2023}
---------. 2023. {``Incompleteness, Independence, and Negative
Dominance.''} \url{https://philpapers.org/archive/LEDIIA.pdf}.

\bibitem[\citeproctext]{ref-LehrerWagner1985}
Lehrer, Keith, and Carl Wagner. 1985. {``Intransitive Indifference: The
Semi-Order Problem.''} \emph{Synthese} 65: 249--56.
\url{https://doi.org/10.1007/BF00869302}.

\bibitem[\citeproctext]{ref-Levi1986}
Levi, Isaac. 1986. \emph{Hard Choices}. Cambridge: Cambridge University
Press.

\bibitem[\citeproctext]{ref-Lewis1981bn}
Lewis, David. 1981. {``Causal Decision Theory.''} \emph{Australasian
Journal of Philosophy} 59 (1): 5--30.
\url{https://doi.org/10.1080/00048408112340011}.

\bibitem[\citeproctext]{ref-Luce1956}
Luce, R. Duncan. 1956. {``Semiorders and a Theory of Utility
Discrimination.''} \emph{Econometrica} 24 (2): 178--91.
\url{https://doi.org/10.2307/1905751}.

\bibitem[\citeproctext]{ref-Luce1959}
---------. 1959. \emph{Individual Choice Behavior: A Theoretical
Analysis}. New York: Wiley.

\bibitem[\citeproctext]{ref-LuceRaiffa1957}
Luce, R. Duncan, and Howard Raiffa. 1957. \emph{Games and Decisions:
Introduction and Critical Survey}. New York: Wiley.

\bibitem[\citeproctext]{ref-Moss2015}
Moss, Sarah. 2015. {``Time-Slice Epistemology and Action Under
Indeterminacy.''} \emph{Oxford Studies in Epistemology} 5: 172--94.
\url{https://doi.org/10.1093/acprof:oso/9780198722762.003.0006}.

\bibitem[\citeproctext]{ref-Moulin1985}
Moulin, Hervé. 1985. {``Choice Functions over a Finite Set: A
Summary.''} \emph{Social Choice and Welfare} 2 (2): 147--60.
\url{https://doi.org/10.1007/BF00437315}.

\bibitem[\citeproctext]{ref-vNM1944yaml}
Neumann, John von, and Oskar Morgenstern. 1944. \emph{Theory of Games
and Economic Behavior}. Princeton, NJ: Princeton University Press.

\bibitem[\citeproctext]{ref-Nover2004}
Nover, Harris, and Alan Hàjek. 2004. {``Vexing Expectations.''}
\emph{Mind} 113 (450): 237--49.
\url{https://doi.org/10.1093/mind/113.450.237}.

\bibitem[\citeproctext]{ref-Pearce1984}
Pearce, David G. 1984. {``Rationalizable Strategic Behavior and the
Problem of Perfection.''} \emph{Econometrica} 52 (4): 1029--50.
\url{https://doi.org/10.2307/1911197}.

\bibitem[\citeproctext]{ref-Peterson2017}
Peterson, Martin. 2017. \emph{An Introduction to Decision Theory}.
Second. Cambridge: Cambridge University Press.
\url{https://doi.org/10.1017/9781316585061}.

\bibitem[\citeproctext]{ref-RamseyTruthProb}
Ramsey, Frank. 1926. {``Truth and Probability.''} In \emph{Philosophical
Papers}, edited by D. H. Mellor, 52--94. Cambridge: Cambridge University
Press.

\bibitem[\citeproctext]{ref-Samuelson1938}
Samuelson, Paul A. 1938. {``A Note on the Pure Theory of Consumer's
Behaviour.''} \emph{Econometrica} 5 (17): 61--71.
\url{https://doi.org/10.2307/2548836}.

\bibitem[\citeproctext]{ref-Sartre1946}
Sartre, Jean-Paul. 1946/2007. {``Existentialism Is a Humanism.''} In
\emph{Existentialism Is a Humanism}, translated by Annie Cohen-Solal,
17--72. New Haven: Yale University Press.

\bibitem[\citeproctext]{ref-Sen1969}
Sen, Amartya. 1969. {``Quasi-Transitivity, Rational Choice and
Collective Decisions.''} \emph{The Review of Economic Studies} 36 (3):
381--93. \url{https://doi.org/10.2307/2296434}.

\bibitem[\citeproctext]{ref-Sen2004}
---------. 2004. {``Incompleteness and Reasoned Choice.''}
\emph{Synthese} 140 (1-2): 43--59.
\url{https://doi.org/10.1023/B:SYNT.0000029940.51537.b3}.

\bibitem[\citeproctext]{ref-Sen1970sec}
---------. (1970) 2017. \emph{Collective Choice and Social Welfare:} An
expanded edition. Cambridge, MA: Harvard University Press.
\url{https://doi.org/10.4159/9780674974616}.

\bibitem[\citeproctext]{ref-Skyrms1982}
Skyrms, Brian. 1982. {``Causal Decision Theory.''} \emph{Journal of
Philosophy} 79 (11): 695--711. \url{https://doi.org/10.2307/2026547}.

\bibitem[\citeproctext]{ref-Spencer2023}
Spencer, Jack. 2023. {``Can It Be Irrational to Knowingly Choose the
Best?''} \emph{Australasian Journal of Philosophy} 101 (1): 128--39.
\url{https://doi.org/10.1080/00048402.2021.1958880}.

\bibitem[\citeproctext]{ref-LedermanEtAl2025}
Tarsney, Christian, Harvey Lederman, and Dean Spears. forthcoming. {``A
Dominance Argument Against Incompleteness.''} \emph{Philosophical
Review}, forthcoming. \url{https://doi.org/10.48550/arXiv.2403.17641}.

\bibitem[\citeproctext]{ref-WeathersonKAHIS}
Weatherson, Brian. 2024. \emph{Knowledge: A Human Interest Story}.
Cambridge: Open Book Publishers.
\url{https://doi.org/10.11647/obp.0425}.

\end{CSLReferences}




\end{document}
