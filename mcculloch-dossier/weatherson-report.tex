% Options for packages loaded elsewhere
\PassOptionsToPackage{unicode,hidelinks}{hyperref}
\PassOptionsToPackage{hyphens}{url}
\PassOptionsToPackage{dvipsnames,svgnames,x11names}{xcolor}
%
\documentclass[
  letterpaper,
  DIV=11,
  numbers=noendperiod]{scrartcl}

\usepackage{amsmath,amssymb}
\usepackage{setspace}
\usepackage{iftex}
\ifPDFTeX
  \usepackage[T1]{fontenc}
  \usepackage[utf8]{inputenc}
  \usepackage{textcomp} % provide euro and other symbols
\else % if luatex or xetex
  \usepackage{unicode-math}
  \defaultfontfeatures{Scale=MatchLowercase}
  \defaultfontfeatures[\rmfamily]{Ligatures=TeX,Scale=1}
\fi
\usepackage{lmodern}
\ifPDFTeX\else  
    % xetex/luatex font selection
  \setmainfont[]{Lato}
\fi
% Use upquote if available, for straight quotes in verbatim environments
\IfFileExists{upquote.sty}{\usepackage{upquote}}{}
\IfFileExists{microtype.sty}{% use microtype if available
  \usepackage[]{microtype}
  \UseMicrotypeSet[protrusion]{basicmath} % disable protrusion for tt fonts
}{}
\makeatletter
\@ifundefined{KOMAClassName}{% if non-KOMA class
  \IfFileExists{parskip.sty}{%
    \usepackage{parskip}
  }{% else
    \setlength{\parindent}{0pt}
    \setlength{\parskip}{6pt plus 2pt minus 1pt}}
}{% if KOMA class
  \KOMAoptions{parskip=half}}
\makeatother
\usepackage{xcolor}
\usepackage[margin=1.4in]{geometry}
\setlength{\emergencystretch}{3em} % prevent overfull lines
\setcounter{secnumdepth}{-\maxdimen} % remove section numbering
% Make \paragraph and \subparagraph free-standing
\ifx\paragraph\undefined\else
  \let\oldparagraph\paragraph
  \renewcommand{\paragraph}[1]{\oldparagraph{#1}\mbox{}}
\fi
\ifx\subparagraph\undefined\else
  \let\oldsubparagraph\subparagraph
  \renewcommand{\subparagraph}[1]{\oldsubparagraph{#1}\mbox{}}
\fi


\providecommand{\tightlist}{%
  \setlength{\itemsep}{0pt}\setlength{\parskip}{0pt}}\usepackage{longtable,booktabs,array}
\usepackage{calc} % for calculating minipage widths
% Correct order of tables after \paragraph or \subparagraph
\usepackage{etoolbox}
\makeatletter
\patchcmd\longtable{\par}{\if@noskipsec\mbox{}\fi\par}{}{}
\makeatother
% Allow footnotes in longtable head/foot
\IfFileExists{footnotehyper.sty}{\usepackage{footnotehyper}}{\usepackage{footnote}}
\makesavenoteenv{longtable}
\usepackage{graphicx}
\makeatletter
\def\maxwidth{\ifdim\Gin@nat@width>\linewidth\linewidth\else\Gin@nat@width\fi}
\def\maxheight{\ifdim\Gin@nat@height>\textheight\textheight\else\Gin@nat@height\fi}
\makeatother
% Scale images if necessary, so that they will not overflow the page
% margins by default, and it is still possible to overwrite the defaults
% using explicit options in \includegraphics[width, height, ...]{}
\setkeys{Gin}{width=\maxwidth,height=\maxheight,keepaspectratio}
% Set default figure placement to htbp
\makeatletter
\def\fps@figure{htbp}
\makeatother
\newlength{\cslhangindent}
\setlength{\cslhangindent}{1.5em}
\newlength{\csllabelwidth}
\setlength{\csllabelwidth}{3em}
\newlength{\cslentryspacingunit} % times entry-spacing
\setlength{\cslentryspacingunit}{\parskip}
\newenvironment{CSLReferences}[2] % #1 hanging-ident, #2 entry spacing
 {% don't indent paragraphs
  \setlength{\parindent}{0pt}
  % turn on hanging indent if param 1 is 1
  \ifodd #1
  \let\oldpar\par
  \def\par{\hangindent=\cslhangindent\oldpar}
  \fi
  % set entry spacing
  \setlength{\parskip}{#2\cslentryspacingunit}
 }%
 {}
\usepackage{calc}
\newcommand{\CSLBlock}[1]{#1\hfill\break}
\newcommand{\CSLLeftMargin}[1]{\parbox[t]{\csllabelwidth}{#1}}
\newcommand{\CSLRightInline}[1]{\parbox[t]{\linewidth - \csllabelwidth}{#1}\break}
\newcommand{\CSLIndent}[1]{\hspace{\cslhangindent}#1}

\KOMAoption{captions}{tableheading}
\makeatletter
\makeatother
\makeatletter
\makeatother
\makeatletter
\@ifpackageloaded{caption}{}{\usepackage{caption}}
\AtBeginDocument{%
\ifdefined\contentsname
  \renewcommand*\contentsname{Table of contents}
\else
  \newcommand\contentsname{Table of contents}
\fi
\ifdefined\listfigurename
  \renewcommand*\listfigurename{List of Figures}
\else
  \newcommand\listfigurename{List of Figures}
\fi
\ifdefined\listtablename
  \renewcommand*\listtablename{List of Tables}
\else
  \newcommand\listtablename{List of Tables}
\fi
\ifdefined\figurename
  \renewcommand*\figurename{Figure}
\else
  \newcommand\figurename{Figure}
\fi
\ifdefined\tablename
  \renewcommand*\tablename{Table}
\else
  \newcommand\tablename{Table}
\fi
}
\@ifpackageloaded{float}{}{\usepackage{float}}
\floatstyle{ruled}
\@ifundefined{c@chapter}{\newfloat{codelisting}{h}{lop}}{\newfloat{codelisting}{h}{lop}[chapter]}
\floatname{codelisting}{Listing}
\newcommand*\listoflistings{\listof{codelisting}{List of Listings}}
\makeatother
\makeatletter
\@ifpackageloaded{caption}{}{\usepackage{caption}}
\@ifpackageloaded{subcaption}{}{\usepackage{subcaption}}
\makeatother
\makeatletter
\@ifpackageloaded{tcolorbox}{}{\usepackage[skins,breakable]{tcolorbox}}
\makeatother
\makeatletter
\@ifundefined{shadecolor}{\definecolor{shadecolor}{rgb}{.97, .97, .97}}
\makeatother
\makeatletter
\makeatother
\makeatletter
\makeatother
\ifLuaTeX
  \usepackage{selnolig}  % disable illegal ligatures
\fi
\IfFileExists{bookmark.sty}{\usepackage{bookmark}}{\usepackage{hyperref}}
\IfFileExists{xurl.sty}{\usepackage{xurl}}{} % add URL line breaks if available
\urlstyle{same} % disable monospaced font for URLs
\hypersetup{
  pdftitle={Cameron McCulloch Dossier Report},
  pdfauthor={Brian Weatherson},
  colorlinks=true,
  linkcolor={black},
  filecolor={Maroon},
  citecolor={Blue},
  urlcolor={Blue},
  pdfcreator={LaTeX via pandoc}}

\title{Cameron McCulloch Dossier Report}
\author{Brian Weatherson}
\date{2023-04-17}

\begin{document}
\maketitle
\ifdefined\Shaded\renewenvironment{Shaded}{\begin{tcolorbox}[borderline west={3pt}{0pt}{shadecolor}, sharp corners, frame hidden, interior hidden, breakable, boxrule=0pt, enhanced]}{\end{tcolorbox}}\fi

\setstretch{1.2}
\hypertarget{thesis-plan}{%
\subsection{Thesis Plan}\label{thesis-plan}}

Cameron plans to write a thesis on privacy. As is fairly common, the
thesis consists of a set of papers. As is less common, the thesis is in
two `parts'. The first part is about the relationship between privacy
and knowledge. The second part is about the relationship between privacy
and property. Since one of the papers from the first part is the sample
chapter, I'll say much more about this below. So I'll mostly comment
here on the connections to property.

The first thing to say is that the thesis plan does the thing that is,
in my mind, sufficient for passing at this stage. It identifies some
important questions, and makes feasible promises for how the thesis will
make progress on them, and those promises are such that, if fulfilled,
we'd have an excellent thesis. So the top line is that I think it's good
enough to move on to the defense, and then to writing. And if it does
carry out the task of connecting privacy both to property law/rights,
and to epistemology, it could be a really really \emph{great} thesis. So
I definitely want to encourage passing the thesis, and moving forward.

That said, the plan (and the sample chapter) felt rather repetitive in
places. I feel I saw that quote from Goldman 1999 at least three times,
for example. And the fifth chapter (the second chapter on property)
looks like it is going to spend a lot of time on papers that were at the
center of the sample chapter.

There is a problem here that I don't know the general solution to. In a
thesis styled as a continuous manuscript, one can simply refer to
earlier discussions of a point, in a way that's not so easy in a set of
papers. And if the papers are to appear in journals, it helps for them
to be self-standing, and that implies a degree of repetition. But sooner
or later, you want someone (e.g., someone on a hiring committee) to take
a serious look at the body of work you're producing. And at that point
excess repetition can hurt. As I said, I don't have a general answer
here. But my gut feeling is that as things stand, this leans too far in
the direction of making everything self-standing, and hence in having
too much get repeated across multiple chapters.

The other thing is that there is a hard choice to be made about how much
of a deep dive into the property literature that the thesis needs to
make. It might be a big task to do this. The Merrill and Smith papers
that are mentioned have between them thousands of citations. But it
actually kind of matters to the thesis whether something like what they
say is right. If the 20th century view they oppose - that property is a
disunified bundle of rights - turns out to be correct, that creates some
complications for the unifying project of the thesis as a whole.
(Jonathan Sarnoff has some work in progress arguing that Merrill and
Smith are mistaken, and the 20th century view is closer to the truth,
and his arguments seemed persuasive to me.) So there are some big
questions here. But I'm not sure given the timeline that learning a lot
about the nuances of property jurisprudence is going to be the optimal
use of time.

\hypertarget{sample-chapter}{%
\subsection{Sample Chapter}\label{sample-chapter}}

The sample chapter argues that privacy is distinctively epistemic - in
particular, one loses privacy if and only if one becomes more `known'.
This is a striking claim, and one that seems worth defending. And some
of the replies to existing critiques of it seem on point. But I felt
that at a lot of points, it needed much more engagement with the
philosophical literature, or, perhaps better, more careful treatment of
difficult matters that were lurking just off screen.

This is, I think, a slightly odd paper for us to have as a sample
chapter. It's not a lit review, so it isn't a rule violation. But it is
more review-like than we normally see. While there is a sketch of a
positive argument for the view at the start of the chapter, the bulk of
it consists of replies to objections to it. And the objections come from
a special issue of \emph{Episteme}, and an unpublished ms. My guess is
that if something like this were to be a writing sample, it would be
much better to focus on a positive argument for the view. (But perhaps
it isn't meant as that, in which case that doesn't matter.)

As I mentioned, I thought the paper at a few places needed to engage
much more with some questions that were just below the surface. I'll
focus here on three of them, though maybe we'll talk about more at the
defense.

\hypertarget{knowledge-first}{%
\subsubsection{Knowledge First}\label{knowledge-first}}

The positive view here has a lot in common with the knowledge first
program associated with Timothy Williamson
(\protect\hyperlink{ref-Williamson2000}{2000}). That was especially true
around page 15, when Cameron is arguing that even if some things that
don't look epistemic are privacy breaches, they might still be
consistent with his thesis because they are connected in the right way
to knowledge. And a lot of the argumentation at the start of the paper
about the centrality of knowledge talk to privacy talk, and the
usefulness of having a unified account, felt very Williamsonian. I
wasn't sure by the end of the paper exactly what the objection to a JTB
theory of privacy would be, but I suspected it would be a lot like
Williamson's objection to a JTB theory of evidence.

But obviously the knowledge first program is a Big Deal in recent
epistemology. To the extent that this project consists of extending the
program to privacy, it would probably be worth engaging with some of its
critics. The papers in Greenough, Pritchard, and Williamson
(\protect\hyperlink{ref-Greenough2009-GREWOK}{2009}) are useful places
to start, especially the Goldman
(\protect\hyperlink{ref-Goldman2009}{2009}) paper on evidence. I've only
skimmed the book on knowledge first by Aidan McGlynn
(\protect\hyperlink{ref-McGlynn2014-MCGKF}{2014}), but it could be
useful. But what really jumped out of some of the discussions was the
possibility that different notions of knowledge being central were being
run together. Carrie Jenkins and Jonathan Ichikawa
(\protect\hyperlink{ref-Ichikawa2017-ICHOPK}{2017}) have a nice
discussion of the different things that one might mean by `knowledge
first', and I thought it could be helpful in sorting out exactly what
way knowledge is meant to be central to privacy.

\hypertarget{who-knows}{%
\subsubsection{Who Knows?}\label{who-knows}}

I was a little shocked by the reference on page 17 to ``other papers
that spring to mind''. That didn't exactly seem like academic writing.
And I thought there was a much bigger question here about exactly who
the knowers are.

For one thing, it isn't obvious at all that something is known iff it is
known by someone, or even by some group. Alexander Bird
(\protect\hyperlink{ref-Bird2010}{2010}) has some nice examples where he
claims something is known by `the scientific community' without any one
person knowing it. But you might worry (I in fact do worry) that the
scientific community is too diffuse a thing to really have knowledge.
What's true, in Bird's case, is that a fact is known without being known
by anyone. (The simplest examples are ones where a fact is recorded in
the right place such that anyone who cared to look would find it, but
all the people who ever looked it up, or who discovered it and stored it
in the right place, have died.)

Now maybe that's ok. Maybe the fact that some fact about one is known
does diminish one's privacy, even if it isn't known by anyone. At least,
this isn't an implausible claim. But it seems important to understanding
the core slogan of the paper, and it felt like it got left out here.

There is also a bit of a literature on how knowledge is graded. Nick
Treanor (\protect\hyperlink{ref-Treanor2013-TRETMO}{2013}) has a paper
(which, to be honest, I didn't love) on how to measure how much someone
knows. And Carlotta Pavese
(\protect\hyperlink{ref-Pavese2017-PAVKAG}{2017}) has a paper that uses
a tool similar to what is used here to generate gradability from what
looks like an ungraded source. Some of this should also be noted in the
thesis.

There are some technical problems around here which I'm sure are
solvable, but do need solving at some point. For example, if someone
knows one fact about A, then does some simple logical deductions
(or-introduction with random propositions, and-introduction with random
other pieces of knowledge) do they diminish A's privacy? I would think
not, but as it stands I think the view implies that they do.

\hypertarget{about-aboutness}{%
\subsubsection{About Aboutness}\label{about-aboutness}}

First, a caveat. There are several really hard questions lurking around
here, any one of which could make a perfectly good thesis topic in
itself. It would \emph{not} be a good idea to go into too much depth on
any of them. But that said, the breezy way in which questions about
aboutness was glossed over, especially on page 11, seemed off-putting.
Here are some practical and theoretical questions that it would be good
to at least have views about, even if they aren't all in the thesis.

Romeo loves Juliet. That's about Romeo, for sure. Is it about Juliet?
That's much harder. On the one hand, it doesn't seem to violate, or even
diminish, Juliet's privacy to learn this. (Imagine Juliet is a rock star
and Romeo is a fan she's never heard of.) On the other hand, it seems
deeply equivalent to \emph{Juliet is loved by Romeo}, and that looks
like it is about Juliet. (The `deep equivalence' here is not just
necessary equivalence; it seems like it is stating the same Tractarian
fact.)

Let's try an even harder case - \emph{Romeo dreamed that Juliet was
flying}. Again, clearly about Romeo. But is it about Juliet? I don't
think so, but I think it's a little hard to say why. If it is about
Juliet, then presumably if Romeo knows that Romeo dreamed that Juliet
was flying, that diminishes Juliet's privacy. And that seems maybe a
little implausible. This also seems relevant to the Tattoo Dream case in
the paper, the discussion of which I found a little hard to follow.

Absurdly hard case. Here's a true thing that I know (at least on sunny
days): \emph{Either the sky is blue or Juliet is a Seahawks fan}. Is
that about Juliet? I think the answer is no. I mean it better be no for
this thesis to hold. Do I really diminish Juliet's privacy when I look
out the window, see the sky is blue, and do or-introduction? That
doesn't seem plausible. But a theory of aboutness that gets this case
right is not exactly easy to generate. Both the most natural theory of
aboutness that takes the bearers of aboutness to be unstructured
entities (like Lewis's), and the simplest theory that takes the bearers
of aboutness to be structured propositions, both make it turn out to be
about Juliet.

At this point one might think the thing to do would be to have a theory
of aboutness, and use it to resolve the hard cases. I want to very
strongly urge against that course of action. A theory of aboutness is
\emph{at least} a thesis sized project. It might be worth looking at the
theories developed by Lewis (\protect\hyperlink{ref-Lewis1988e}{1988})
(building on Lewis (\protect\hyperlink{ref-Lewis1982c}{1982})), Yablo
(\protect\hyperlink{ref-Yablo2014-YABA}{2014}), and the survey of more
recent views in section 4.2 of Berto and Nolan
(\protect\hyperlink{ref-sep-hyperintensionality}{2021}). But don't
expect to get any clear answers to the big picture question of what
aboutness really is.

One thing to keep in the back of one's mind is that while the thesis
doesn't presuppose, or need, any particular theory of aboutness, it does
presuppose something about the architecture of such a theory. That is,
the thesis presupposes that the things that are known are the things
that have aboutness properties. That's not implausible, but it isn't
completely obvious. And at times some of the language of the thesis
seems to presuppose that these are the same kind of things as (a) facts,
and (b) information. Now (a) isn't too surprising - it's nearly
universally accepted that facts are things that can be known. But (b) is
much trickier. It's somewhat plausible that information is intensional
(i.e., is a set of possible worlds), but aboutness is hyperintensional.
That is to say, contra Lewis's best efforts, there is no thing that a
set of possible worlds is about. So strictly speaking, there is no
information about me, because the things that are information are the
wrong kind of thing to bear aboutness relations to individuals. If
that's right, it probably just means that some wording needs changing
here or there. (And even if no information is strictly speaking about
anyone or anything, there must be some informal way of capturing the
ordinary way of using the phrase `information about'.) But it brings up
how tricky this whole area is. To speak a little metaphorically, the ice
gets very thin very fast around here. I don't think the right reaction
to that is to do careful measurements of the exact thickness of the ice;
but it is a reason to watch your footing.

\hypertarget{references}{%
\subsubsection*{References}\label{references}}
\addcontentsline{toc}{subsubsection}{References}

\hypertarget{refs}{}
\begin{CSLReferences}{1}{0}
\leavevmode\vadjust pre{\hypertarget{ref-sep-hyperintensionality}{}}%
Berto, Francesco, and Daniel Nolan. 2021. {``{Hyperintensionality}.''}
In \emph{The {Stanford} Encyclopedia of Philosophy}, edited by Edward N.
Zalta, {S}ummer 2021.
\url{https://plato.stanford.edu/archives/sum2021/entries/hyperintensionality/};
Metaphysics Research Lab, Stanford University.

\leavevmode\vadjust pre{\hypertarget{ref-Bird2010}{}}%
Bird, Alexander. 2010. {``Social Knowing: The Social Sense of
'Scientific Knowledge'.''} \emph{Philosophical Perspectives} 24 (1):
23--56. \url{https://doi.org/10.1111/j.1520-8583.2010.00184.x}.

\leavevmode\vadjust pre{\hypertarget{ref-Goldman2009}{}}%
Goldman, Alvin. 2009. {``Williamson on Knowledge and Evidence.''} In
\emph{{Williamson on Knowledge}}, 73--91. Oxford: Oxford University
Press.

\leavevmode\vadjust pre{\hypertarget{ref-Greenough2009-GREWOK}{}}%
Greenough, Patrick, Duncan Pritchard, and Timothy Williamson. 2009.
\emph{{Williamson on Knowledge}}. Oxford University Press.

\leavevmode\vadjust pre{\hypertarget{ref-Ichikawa2017-ICHOPK}{}}%
Ichikawa, Jonathan, and C. S. I. Jenkins. 2017. {``On Putting Knowledge
'First'.''} In \emph{Knowledge First: Approaches in Epistemology and
Mind}, edited by Joseph Adam Carter, Emma C. Gordon, and Benjamin
Jarvis, 113--31. Oxford University Press.

\leavevmode\vadjust pre{\hypertarget{ref-Lewis1982c}{}}%
Lewis, David. 1982. {``Logic for Equivocators.''} \emph{No{û}s} 16 (3):
431--41. \url{https://doi.org/10.1017/cbo9780511625237.009}.

\leavevmode\vadjust pre{\hypertarget{ref-Lewis1988e}{}}%
---------. 1988. {``Statements Partly about Observation.''}
\emph{Philosophical Papers} 17 (1): 1--31.
\url{https://doi.org/10.1080/05568648809506282}.

\leavevmode\vadjust pre{\hypertarget{ref-McGlynn2014-MCGKF}{}}%
McGlynn, Aidan. 2014. \emph{Knowledge First?} New York, NY: Palgrave
Macmillian.

\leavevmode\vadjust pre{\hypertarget{ref-Pavese2017-PAVKAG}{}}%
Pavese, Carlotta. 2017. {``Know-How and Gradability.''}
\emph{Philosophical Review} 126 (3): 345--83.
\url{https://doi.org/10.1215/00318108-3878493}.

\leavevmode\vadjust pre{\hypertarget{ref-Treanor2013-TRETMO}{}}%
Treanor, Nick. 2013. {``The Measure of Knowledge.''} \emph{Noûs} 47 (3):
577--601. \url{https://doi.org/10.1111/j.1468-0068.2011.00854.x}.

\leavevmode\vadjust pre{\hypertarget{ref-Williamson2000}{}}%
Williamson, Timothy. 2000. \emph{{Knowledge and its Limits}}. Oxford
University Press.

\leavevmode\vadjust pre{\hypertarget{ref-Yablo2014-YABA}{}}%
Yablo, Stephen. 2014. \emph{Aboutness}. Oxford: Princeton University
Press.

\end{CSLReferences}



\end{document}
